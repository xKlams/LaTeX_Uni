\documentclass[12px]{article}

\title{Lezione 4 Fisica Generale 1}
\date{2024-10-07}
\author{Federico De Sisti}

\input{../../setup.tex}

\begin{document}
	\maketitle
	\newpage
	\section{Parte mancante da recuperare}
	\section{Moto Circolare}
$\overrightarrow{v}(t) = \frac{d \overrightarrow{r}(t)}{dt}$ \\
$\overrightarrow{a}(t) = \frac{d \overrightarrow{v}(t)}{dt}$\\
Per una qualunque funzione derivabile possiamo scrivere
\[
	\frac{df(x)}{dx} = u(x)
.\] 
\[ f(x) = \int u(x) dx
.\]
\[
	f(x) = f(x_0) + \int_{x_0}^x v(x) dx
.\] 
Analogo è il ragionamento per i vettori
\[
	\overrightarrow{r}(t) = \overrightarrow{r}(t_0) + \int_{t_0}^t \overrightarrow{u}(t') dt'  \Leftrightarrow \begin{cases}
		x(t) = x(t_0) + \int_{t_0}^t v_x(t')dt'\\
		y(t) = y(t_0) + \int_{t_0}^t v_y(t')dt'\\
		z(t) = z(t_0) + \int_{t_0}^t v_z(t')dt'\\
	\end{cases}
.\] 
\[
	\overrightarrow{v}(t) = \overrightarrow{v}(t_0) + \int_{t_0}^t \overrightarrow{a}(t') dt'
.\] 
Anche questa vale per 3 equazioni scalari
\section{Moto rettilineo uniforme}
$v = costante = 5 m/s$\\
 $x(0) = 2m$\\
  $s(5) = ?$\\ 
  \[
	  x(t) = x(t_0) + \int_{t_0}^t v(t')dt' \ \Rightarrow \ 2 + \int_0^5 5dt' = 27 m
  .\] 
  \textbf{Generalizzazione}\\
  \[
	  v(t) = v_0 + \int_0^t adt'= v_0 + at
  .\] 
  \[
  x(t) = x_0 + \int_0^t (v_0 + at')dt' = x_0 + v_0 t+ \frac 1 2 at^2
  .\] 
  Espressione del moto uniformemente accelerato\\
  \newpage\ \\
  \textbf{Esercizio}\\
  $a = cost$\\
  Se il punto è in $x_1$ la sua velocità è $v_1$\\
  Se il punto è in $x_2$ la sua velocità è $v_2$ \\
  Trova l'accelerazione\\
  \textbf{Svolgimento}\\
  \[
	  x(t) = x(0) + v(0)t + \frac 1 2 at^2\ \ \ v(t) = v(0) + at
  .\] 
  \[
  x_1 = x(0) + v(0)t_1 + \frac 1 2 at_1^2\ \ \ v_1 = v(0) + at_1
  .\] 
  \[
  x_2 = x(0) + v(0)t_2 + \frac 1 2 at_2^2\ \ \ v_2 = v(0) + at_2
  .\] 
  Scelgo da dove inizio a contare:\\
  $x(t=0) = x_1$\\
  $v(t=0) = v_1$
  \[
	  x(t_2) = x_2 = x_1 + v_1t_2 + \frac 1 2 a t_1^2
  .\] 
  \[
  v(t_2) = v_2 = v_1 + at_2
  .\] 
  \[
	  t_2 = \frac{v_2 - v_1}{a}
  .\] 
\[
x_2 = x_1 + v_1\frac{v_2-v_1}{ a} + \frac 1 2 \cancel a\frac{(v_2-v_1)^2}{a^{\cancel 2}}

.\] 
\[
x_2 = x_1 + \frac{\cancel v_1v_2 - v_1^2}{a} + \frac 1 2 \frac {v_2^2 + v_1^2 \cancel{- 2v_1v_2}}{a}
.\] 
\[
	\Rightarrow x_2 = x_1 + \frac {v_2^2 - v_1^2}{2a}
.\] 
\[
	a = \frac 1 2 \frac{v_2^2-v_1^2}{x_2-x_1}
.\] 
	
	
\end{document}
