\documentclass[12px]{article}

\title{Lezione 4 Fisica Generale 1}
\date{2024-10-07}
\author{Federico De Sisti}

\usepackage{amsmath}
\usepackage{amsthm}
\usepackage{mdframed}
\usepackage{amssymb}
\usepackage{nicematrix}
\usepackage{amsfonts}
\usepackage{tcolorbox}
\tcbuselibrary{theorems}
\usepackage{xcolor}
\usepackage{cancel}

\newtheoremstyle{break}
  {1px}{1px}%
  {\itshape}{}%
  {\bfseries}{}%
  {\newline}{}%
\theoremstyle{break}
\newtheorem{theo}{Teorema}
\theoremstyle{break}
\newtheorem{lemma}{Lemma}
\theoremstyle{break}
\newtheorem{defin}{Definizione}
\theoremstyle{break}
\newtheorem{propo}{Proposizione}
\theoremstyle{break}
\newtheorem*{dimo}{Dimostrazione}
\theoremstyle{break}
\newtheorem*{es}{Esempio}

\newenvironment{dimo}
  {\begin{dimostrazione}}
  {\hfill\square\end{dimostrazione}}

\newenvironment{teo}
{\begin{mdframed}[linecolor=red, backgroundcolor=red!10]\begin{theo}}
  {\end{theo}\end{mdframed}}

\newenvironment{nome}
{\begin{mdframed}[linecolor=green, backgroundcolor=green!10]\begin{nomen}}
  {\end{nomen}\end{mdframed}}

\newenvironment{prop}
{\begin{mdframed}[linecolor=red, backgroundcolor=red!10]\begin{propo}}
  {\end{propo}\end{mdframed}}

\newenvironment{defi}
{\begin{mdframed}[linecolor=orange, backgroundcolor=orange!10]\begin{defin}}
  {\end{defin}\end{mdframed}}

\newenvironment{lemm}
{\begin{mdframed}[linecolor=red, backgroundcolor=red!10]\begin{lemma}}
  {\end{lemma}\end{mdframed}}

\newcommand{\icol}[1]{% inline column vector
  \left(\begin{smallmatrix}#1\end{smallmatrix}\right)%
}

\newcommand{\irow}[1]{% inline row vector
  \begin{smallmatrix}(#1)\end{smallmatrix}%
}

\newcommand{\matrice}[1]{% inline column vector
  \begin{pmatrix}#1\end{pmatrix}%
}

\newcommand{\C}{\mathbb{C}}
\newcommand{\K}{\mathbb{K}}
\newcommand{\R}{\mathbb{R}}


\begin{document}
	\maketitle
	\newpage
	\section{Ripasso scorsa lezione}
	Accelerazione ha due componenti:\\
	$\overrightarrow{a_t}$ tangente al vettore velocità\\
	$\overrightarrow{a_n}$ normale al vettore velocità\\
%TODO add image of curve
	\[
		ds \simeq dr = \sqrt{dx^2 + dy^2}
	.\]
	\[
		\diff s t  = \sqrt{\left(\diff x t\right)^2 + \left( \diff y t\right) ^2
	.\] 
	\[
	ds = dt\sqrt{ \left(\diff x t \right)^2 + \left(\diff y t \right)^2
	.\] 
	\[
		\int_{S(0)}^{S(t)}ds = \int^t_0 dt\sqrt{ \left(\diff x t \right)^2 + \left(\diff y t \right)^2}
	.\] 
	\[
		S(t) = \int^t_0dt\sqrt{ \left(\diff x t \right)^2 + \left(\diff y t \right)^2}
	.\] 
	%TODO includi disegno
	\[
	\begin{cases}
		x(\theta) = r \cos\theta\\
		y(\theta) = r\sin\theta
	\end{cases}
	.\] 
	\[
		\int^{S(\theta)}_{S(\theta = 0) = 0}ds = \int^\theta_0 rd\theta = r\theta
	.\] 
	\[
	S(\theta) = \theta r\\
	.\] 
	Voglio adesso  parametrizzare in funzione del tempo\\
	\[
	\begin{cases}
		x(t) = r\cos(\theta(t))\\
		y(t) = r\sin(\theta(t))
	\end{cases} \Rightarrow \begin{cases}
		\diff x t = -r\sin(\theta(t))\theta'(t)\\
		\diff y t = r\cos(\theta(t))\theta'(t)
	\end{cases}
	.\] 
	\[
	ds = dt\sqrt{ \left(\diff x t \right)^2 + \left(\diff y t \right)^2
	.\] 
	Scrivendo $\theta'(t) = \diff {\theta(t)} t$ otteniamo
	 \[
	ds = dt r \left| \diff \theta t \right| \Leftrightarrow	v(t) = \diff r t = r\omega (t)
	.\] 
	\[
	\overrightarrow{v} = \overrightarrow{\omega}\times \overrightarrow{r}
	.\] 
	\[
	\overrightarrow{a} = \overrightarrow{a_t} + \overrightarrow{a_c}
	.\] 
	\[
		\overrightarrow{a_t} = \diff {v(t)}{t}\hat v(t)
	.\] 
	\[
	\overrightarrow{a_c} = \overrightarrow{\omega}(t)\times \overrightarrow{v}(t) = \overrightarrow{\omega}(t)\times \left( \overrightarrow{\omega}(t)\times \overrightarrow{r}(t) \right)
	.\] 
	\section{Moto Circolare Uniforme}
$v(t) = v = cost \Leftrightarrow \omega(t) = \omega = cost$\\
\begin{cases}
	x(t) = r\cos(\omega t)\\
	y(t) = r\sin(\omega t)
\end{cases}
$x^2(t) = y^2(t) = r^2 = cost$\\
 \begin{cases}
	v_x(t) = -r\omega \sin(\omega t)\\
	v_y(t) = r\omega \cos(\omega t)
\end{cases}\\
\[
	|v(t)| = \sqrt{v_x^2(t) + v_t^2(t)} = \sqrt{r^2\omega^2\sin^2(\omega t) + r^2\omega^2\cos^2(\omega t)} = r\omega 
.\] 
Quindi la velocità non dipende dal tempo\\
\textbf{Accelerazione:}\\
\begin{cases}
	a_x(t) = -r\omega^2\cos(\omega t)\\
	a_y*t( = -r\omega^2\sin(\omega t)
\end{cases}
\[
	|a(t)| = \sqrt{a^2_x(t) + a_y^2(t)} = r\omega^2
.\] 
Che risulta coinncidere solo con $|a_c(t)|$ poichè la componente  $a_n(t)$ non è presente nel moto circolare uniforme
\begin{defi}[Periodo]
	Tempo impiegato da un punto materiale a percorrere l'intera circonferenza
	\[
		v(t) = \diff s t = \frac{2\pi r} T
	.\] 
	Con v(t) costante (caso moto circolare uniforme)
	\[
		T = \frac {2\pi r}{v} = \frac {2\pi r}{\omega r} = \frac {2\pi} \omega
	.\] 
\end{defi}
\begin{defi}[Frequenza]
	\[
		f = \frac 1 T = \frac \omega {2\pi}
	.\] 
\end{defi}
\textbf{Nota sulle unità di misura}\\
\begin{cases}
	\omega: \frac {rad} s\\
	f: s^{-1}
\end{cases}
	\section{Moto Circolare}
$\overrightarrow{v}(t) = \frac{d \overrightarrow{r}(t)}{dt}$ \\
$\overrightarrow{a}(t) = \frac{d \overrightarrow{v}(t)}{dt}$\\
Per una qualunque funzione derivabile possiamo scrivere
\[
	\frac{df(x)}{dx} = u(x)
.\] 
\[ f(x) = \int u(x) dx
.\]
\[
	f(x) = f(x_0) + \int_{x_0}^x v(x) dx
.\] 
Analogo è il ragionamento per i vettori
\[
	\overrightarrow{r}(t) = \overrightarrow{r}(t_0) + \int_{t_0}^t \overrightarrow{u}(t') dt'  \Leftrightarrow \begin{cases}
		x(t) = x(t_0) + \int_{t_0}^t v_x(t')dt'\\
		y(t) = y(t_0) + \int_{t_0}^t v_y(t')dt'\\
		z(t) = z(t_0) + \int_{t_0}^t v_z(t')dt'\\
	\end{cases}
.\] 
\[
	\overrightarrow{v}(t) = \overrightarrow{v}(t_0) + \int_{t_0}^t \overrightarrow{a}(t') dt'
.\] 
Anche questa vale per 3 equazioni scalari
\section{Moto rettilineo uniforme}
$v = costante = 5 m/s$\\
 $x(0) = 2m$\\
  $s(5) = ?$\\ 
  \[
	  x(t) = x(t_0) + \int_{t_0}^t v(t')dt' \ \Rightarrow \ 2 + \int_0^5 5dt' = 27 m
  .\] 
  \textbf{Generalizzazione}\\
  \[
	  v(t) = v_0 + \int_0^t adt'= v_0 + at
  .\] 
  \[
  x(t) = x_0 + \int_0^t (v_0 + at')dt' = x_0 + v_0 t+ \frac 1 2 at^2
  .\] 
  Espressione del moto uniformemente accelerato\\
  \newpage\ \\
  \textbf{Esercizio}\\
  $a = cost$\\
  Se il punto è in $x_1$ la sua velocità è $v_1$\\
  Se il punto è in $x_2$ la sua velocità è $v_2$ \\
  Trova l'accelerazione\\
  \textbf{Svolgimento}\\
  \[
	  x(t) = x(0) + v(0)t + \frac 1 2 at^2\ \ \ v(t) = v(0) + at
  .\] 
  \[
  x_1 = x(0) + v(0)t_1 + \frac 1 2 at_1^2\ \ \ v_1 = v(0) + at_1
  .\] 
  \[
  x_2 = x(0) + v(0)t_2 + \frac 1 2 at_2^2\ \ \ v_2 = v(0) + at_2
  .\] 
  Scelgo da dove inizio a contare:\\
  $x(t=0) = x_1$\\
  $v(t=0) = v_1$
  \[
	  x(t_2) = x_2 = x_1 + v_1t_2 + \frac 1 2 a t_1^2
  .\] 
  \[
  v(t_2) = v_2 = v_1 + at_2
  .\] 
  \[
	  t_2 = \frac{v_2 - v_1}{a}
  .\] 
\[
x_2 = x_1 + v_1\frac{v_2-v_1}{ a} + \frac 1 2 \cancel a\frac{(v_2-v_1)^2}{a^{\cancel 2}}

.\] 
\[
x_2 = x_1 + \frac{\cancel v_1v_2 - v_1^2}{a} + \frac 1 2 \frac {v_2^2 + v_1^2 \cancel{- 2v_1v_2}}{a}
.\] 
\[
	\Rightarrow x_2 = x_1 + \frac {v_2^2 - v_1^2}{2a}
.\] 
\[
	a = \frac 1 2 \frac{v_2^2-v_1^2}{x_2-x_1}
.\] 
	
	
\end{document}
