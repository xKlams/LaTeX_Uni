\documentclass[12px]{article}

\title{Lezione 6 Fisica Generale I}
\date{2024-10-11}
\author{Federico De Sisti}

\usepackage{amsmath}
\usepackage{amsthm}
\usepackage{mdframed}
\usepackage{amssymb}
\usepackage{nicematrix}
\usepackage{amsfonts}
\usepackage{tcolorbox}
\tcbuselibrary{theorems}
\usepackage{xcolor}
\usepackage{cancel}

\newtheoremstyle{break}
  {1px}{1px}%
  {\itshape}{}%
  {\bfseries}{}%
  {\newline}{}%
\theoremstyle{break}
\newtheorem{theo}{Teorema}
\theoremstyle{break}
\newtheorem{lemma}{Lemma}
\theoremstyle{break}
\newtheorem{defin}{Definizione}
\theoremstyle{break}
\newtheorem{propo}{Proposizione}
\theoremstyle{break}
\newtheorem*{dimo}{Dimostrazione}
\theoremstyle{break}
\newtheorem*{es}{Esempio}

\newenvironment{dimo}
  {\begin{dimostrazione}}
  {\hfill\square\end{dimostrazione}}

\newenvironment{teo}
{\begin{mdframed}[linecolor=red, backgroundcolor=red!10]\begin{theo}}
  {\end{theo}\end{mdframed}}

\newenvironment{nome}
{\begin{mdframed}[linecolor=green, backgroundcolor=green!10]\begin{nomen}}
  {\end{nomen}\end{mdframed}}

\newenvironment{prop}
{\begin{mdframed}[linecolor=red, backgroundcolor=red!10]\begin{propo}}
  {\end{propo}\end{mdframed}}

\newenvironment{defi}
{\begin{mdframed}[linecolor=orange, backgroundcolor=orange!10]\begin{defin}}
  {\end{defin}\end{mdframed}}

\newenvironment{lemm}
{\begin{mdframed}[linecolor=red, backgroundcolor=red!10]\begin{lemma}}
  {\end{lemma}\end{mdframed}}

\newcommand{\icol}[1]{% inline column vector
  \left(\begin{smallmatrix}#1\end{smallmatrix}\right)%
}

\newcommand{\irow}[1]{% inline row vector
  \begin{smallmatrix}(#1)\end{smallmatrix}%
}

\newcommand{\matrice}[1]{% inline column vector
  \begin{pmatrix}#1\end{pmatrix}%
}

\newcommand{\C}{\mathbb{C}}
\newcommand{\K}{\mathbb{K}}
\newcommand{\R}{\mathbb{R}}


\begin{document}
	\maketitle
	\newpage
	\section{Moto del proiettile}
%TODO immagine
	\[\begin{cases}
		x(t) = v_xt\\
		y(y) = h + v_yt - \frac 12 g t^2
	\end{cases} \Rightarrow t = \frac {x}{v_x}
\]
	$z(t) = const = 0$\\
	Eliminando il parametro dal sistema otteniamo:\\
	$ y = h + v_y \frac {x}{v_x} - \frac 12 g \frac {x^2}{(v_x)^2}$
	\begin{teo}[Del grande cazzo]
		per trovare il massimo di una curva tocca fare la derivata 
		\[
		8=========D
		.\] 
	\end{teo}
	\[
		\diff y x = \frac {v_y}{v_{0x}} = \frac {yx}{(v_{0y})} = 0 \Rightarrow x = \frac {v_{0x} v_{0y}}{g} = \frac{ v_0^2\cos\theta\sin\theta} {g}
	.\] 
	\[
		y_{max} = h + \frac {v_{0y}}{v_{0x}}\frac {v_{0x}v_{0y}}{g} - \frac 12 g \frac {v_{0x}^2 v_{0y}^2}{v_{0x}^2}
	.\] 
	\[
		y_{max}=h + \frac 12 g v_{0y}^2
	.\] 
	\[
		t_{hmax} = \frac {v_{0y}} g
	.\] 
	
	\[
	\sin(\alpha + \beta) = \sin(\alpha)\cos(\beta) + \sin(\beta)\cos(\alpha)
	.\] 
	ponendo $\alpha=\beta = \theta$
	 \[
	\sin(2\theta) = 2\sin(\theta)\cos(\theta)
	.\] 
	\[
		\max_\theta x(\theta) = \max_\theta \frac{\sin(2\theta)} g \Rightarrow \theta = \frac \pi 4
	\]\[
	\begin{cases}
		v_x(t) = v_{0x}\\
		v_y(t) = v_{0y} - gt
	\end{cases}\]
	\[
		v_y(t_{max}) = 0 = v_{0y} - gt_h{max} \Rightarrow t_{hmax} = \frac { v_{0y}} g
	.\] 
	\[
		y_max = y(t_{hmax}) = h + \frac {v_{0y}^2} g - \frac 12 g \left(\frac {v_{0y}}g \right)^2 = h + \frac 12 \frac{v_{0y}^2} g
	.\] 
	\textbf{Esercizio}\\
	%TODO DISGENO
	appena sparo la scimmia cade dal ramo, il mio proiettile colpisce la scimmia?\\
	\textbf{Svolgimento}\\
	Scrivo le equazioni del moto per il proiettile e la scimmia\\
	\[
	\begin{cases}
		x_p(t) = v_{0x}t\\
		y_p(t) = v_{0y}t - \frac 12 gt^2
	\end{cases}
\begin{cases}
	x_s(t) = d\\
	y_s(t) = h - \frac 12 gt^2
\end{cases}
\]
\[
x_p(t) = v_{0x}t => t_d = \frac d{v_{0x}}\\
y_p(t_p) = v_{0y}\frac d {v_{0x}} - \frac 12 g \frac {d^2}{v_{0x}}
.\] 
\[
\begin{cases}
	y_p(t_d) = v_{0y}\frac d{v_{0x}} - \frac 12 g \frac {d^2}{v_{0x}^2}\\
	y_s(t_d) = h - \frac 12 g\frac {d^2}{v_{0x}^2}
\end{cases}
.\] 
troviamo $t_g = \frac {2v_{0y}}g$\\
Ci sono delle condizioni sul fatto che la scimmia colpisca il proiettile\\
\textbf{Esercizio per casa}\\
cosa succede nello stesso problema se non parto da 0, ma da un'altezza $h$, per colpire la scimmia è una buona strategia avere una velocità che punta la scimmia\\
\textbf{In quale punto l'accelerazione normale è massima}\\
Nel massimo
\[
	r = \frac {v^2} {a_n} \Rightarrow r = \frac {v_{0x}^2}{g} = \frac{v_0^2\cos\theta}g
.\] 
\[
\begin{cases}
	a_t(t)=g\sin\gamma(t)\\
	a_n(t)=g\cos\gamma(t)
\end{cases}
.\] 
$\gamma$ è l'angolo che la velocità forma con l'asse x e che la normale forma con l'asse y\\
\textbf{fai esercizi 212 210 fine capitolo 2}\\
\section{Sistemi di riferimento in movimento}
\subsection{Orientamento degli assi orientati nello stesso modo, l'origine ha un moto traslatorio}
%TODO aggiungi disegno decisamente
Indicheremo con l'apice i vettori dal secondo sistema di riferimento\\
\[
	\overrightarrow{r}(t) = \overrightarrow{r}_{O'}(t) + \overrightarrow{r}'(t)
.\] 
\[
	x(t)\hat i + y(t) \hat j + z(t) \hat k = x_{O'}(t)\hat i + y_{O'}(t)\hat j + z_{O'}(t)\hat k + x'(t)\hat i ' + y'(t)\hat j ' + z'(t)\hat k '
.\] 
qui ho fatto le foto, sarebbe da trascrivere il tutto\\
\textbf{guarda trasformaszioni galileane pagina 77}
\end{document}
