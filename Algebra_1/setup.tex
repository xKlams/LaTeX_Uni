\usepackage{amsmath}
\usepackage{amsthm}
\usepackage{mdframed}
\usepackage{amssymb}
\usepackage{nicematrix}
\usepackage{amsfonts}
\usepackage{tcolorbox}
\tcbuselibrary{theorems}
\usepackage{xcolor}
\usepackage{cancel}
\usepackage{graphicx}
\usepackage{rotating}
\usepackage{color}
\usepackage{soul}
\usepackage{imakeidx}
\usepackage{wrapfig}
\usepackage{blindtext}
\usepackage{tikz}
\usepackage{fourier}
\usepackage{marvosym}
\usepackage{hyperref}
\usepackage{mathrsfs}
\usepackage{unicode-math}
\usepackage[mathscr]{euscript}

\hypersetup{
    colorlinks,
    citecolor=black,
    filecolor=black,
    linkcolor=black,
    urlcolor=black
}

\graphicspath{ {./images/} }
\newtheoremstyle{break}
  {1px}{1px}%
  {\itshape}{}%
  {\bfseries}{}%
  {\newline}{}%
\theoremstyle{break}
\newtheorem{theo}{Teorema}
\theoremstyle{break}
\newtheorem{lemma}{Lemma}
\theoremstyle{break}
\newtheorem{nomen}{Nomenclatura}
\theoremstyle{break}
\newtheorem{notaz}{Notazione}
\theoremstyle{break}
\newtheorem{defin}{Definizione}
\theoremstyle{break}
\newtheorem{corollario}{Corollario}
\theoremstyle{break}
\newtheorem{propo}{Proposizione}
\theoremstyle{break}
\newtheorem*{dimo}{Dimostrazione}
\theoremstyle{break}
\newtheorem*{es}{Esempio}

\newenvironment{dimo}
  {\begin{dimostrazione}}
  {\text{ }\hfill\square\newline\end{dimostrazione}}

\newenvironment{teo}
{\begin{mdframed}[linecolor=red, backgroundcolor=red!10]\begin{theo}}
  {\end{theo}\end{mdframed}}

\newenvironment{coro}
{\begin{mdframed}[linecolor=red, backgroundcolor=red!10]\begin{corollario}}
  {\end{corollario}\end{mdframed}}

\newenvironment{nome}
{\begin{mdframed}[linecolor=green, backgroundcolor=green!10]\begin{nomen}}
  {\end{nomen}\end{mdframed}}

\newenvironment{nota}
{\begin{mdframed}[linecolor=green, backgroundcolor=green!10]\begin{notaz}}
  {\end{notaz}\end{mdframed}}

\newenvironment{prop}
{\begin{mdframed}[linecolor=red, backgroundcolor=red!10]\begin{propo}}
  {\end{propo}\end{mdframed}}

\newenvironment{defi}
{\begin{mdframed}[linecolor=orange, backgroundcolor=orange!10]\begin{defin}}
  {\end{defin}\end{mdframed}}

\newenvironment{lemm}
{\begin{mdframed}[linecolor=red, backgroundcolor=red!10]\begin{lemma}}
  {\end{lemma}\end{mdframed}}

\newcommand{\icol}[1]{% inline column vector
  \left(\begin{smallmatrix}#1\end{smallmatrix}\right)%
}

\newcommand{\irow}[1]{% inline row vector
  \begin{smallmatrix}(#1)\end{smallmatrix}%
}

\newcommand{\matrice}[1]{% inline column vector
  \begin{pmatrix}#1\end{pmatrix}%
}
\newcommand{\cmatrice}[1]{% inline column vector
  \begin{bmatrix}#1\end{bmatrix}%
}
\newcommand{\storto}[1]{ \begin{sideways}#1\end{sideways}}

\newcommand{\C}{\mathbb{C}}
\newcommand{\K}{\mathbb{K}}
\newcommand{\R}{\mathbb{R}}
\newcommand{\A}{\mathbb{A}}
\newcommand{\pro}{\mathbb{P}}
\newcommand{\normale}{\trianglelefteq}
\newcommand{\bigslant}[2]{{\raisebox{.2em}{$#1$}\left/\raisebox{-.2em}{$#2$}\right.}}
