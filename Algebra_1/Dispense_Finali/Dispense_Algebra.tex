\documentclass[12px]{article}

\title{Dispense Algebra I\\[5px] \normalsize Prima parte del corso}
\date{2024-12-05}
\author{Federico De Sisti}

\usepackage{amsmath}
\usepackage{amsthm}
\usepackage{mdframed}
\usepackage{amssymb}
\usepackage{nicematrix}
\usepackage{amsfonts}
\usepackage{tcolorbox}
\tcbuselibrary{theorems}
\usepackage{xcolor}
\usepackage{cancel}

\newtheoremstyle{break}
  {1px}{1px}%
  {\itshape}{}%
  {\bfseries}{}%
  {\newline}{}%
\theoremstyle{break}
\newtheorem{theo}{Teorema}
\theoremstyle{break}
\newtheorem{lemma}{Lemma}
\theoremstyle{break}
\newtheorem{defin}{Definizione}
\theoremstyle{break}
\newtheorem{propo}{Proposizione}
\theoremstyle{break}
\newtheorem*{dimo}{Dimostrazione}
\theoremstyle{break}
\newtheorem*{es}{Esempio}

\newenvironment{dimo}
  {\begin{dimostrazione}}
  {\hfill\square\end{dimostrazione}}

\newenvironment{teo}
{\begin{mdframed}[linecolor=red, backgroundcolor=red!10]\begin{theo}}
  {\end{theo}\end{mdframed}}

\newenvironment{nome}
{\begin{mdframed}[linecolor=green, backgroundcolor=green!10]\begin{nomen}}
  {\end{nomen}\end{mdframed}}

\newenvironment{prop}
{\begin{mdframed}[linecolor=red, backgroundcolor=red!10]\begin{propo}}
  {\end{propo}\end{mdframed}}

\newenvironment{defi}
{\begin{mdframed}[linecolor=orange, backgroundcolor=orange!10]\begin{defin}}
  {\end{defin}\end{mdframed}}

\newenvironment{lemm}
{\begin{mdframed}[linecolor=red, backgroundcolor=red!10]\begin{lemma}}
  {\end{lemma}\end{mdframed}}

\newcommand{\icol}[1]{% inline column vector
  \left(\begin{smallmatrix}#1\end{smallmatrix}\right)%
}

\newcommand{\irow}[1]{% inline row vector
  \begin{smallmatrix}(#1)\end{smallmatrix}%
}

\newcommand{\matrice}[1]{% inline column vector
  \begin{pmatrix}#1\end{pmatrix}%
}

\newcommand{\C}{\mathbb{C}}
\newcommand{\K}{\mathbb{K}}
\newcommand{\R}{\mathbb{R}}


\begin{document}
	\maketitle
	\newpage

	\tableofcontents
	\newpage


	\section{Preambolo}
	Siamo giunti ad un nuovo capitolo della nostra vita, talvolta per andare avanti bisogna guardarsi un po'indietro, ed è per questo che propongo ora una lista delle passate mogli di Alberto Agostinelli, la quale anima è più pura rispetto a quella del passato corso di Geometria I, ma non abbastanza per non essere citato in questo testo:\\[10px]
Alessia\\
Beatrice\\
Camilla\\
Chiara\\
Elisa\\
Federica\\
Francesca\\
Gaia\\
Giulia\\
Ilaria\\
Laura\\
Martina\\
Nicole\\
Sara\\
Sofia\\
Valentina\\
Vittoria\\
Eleonora\\
\vfill \ \\
Speciali ringraziameti vanno fatti ad Andrea Duca, magnifico sulle parallele
\newpage

	\section{Cosa c'è su e-leaning di Francesco Mazzini}
	Date appelli\\
	Esercizi settimanali\\
	All'esame ti chiedono due esercizi delle schede scelti a caso\\
	Ci sono 2 esoneri (primo 17 dicembre) (secondo ?? maggio)\\
\textbf{Libri}\\
M. Artin Algebra\\
IN. Hernstein: Algebra (difficile)\\
\section{Gruppi}
\begin{defi}[Gruppo]
	Un gruppo è un dato di un insieme $G$ con un'operazione $\cdot$ tali che:\\
	1) L'operazione è associativa\\
	\[f \cdot (g\codt h) = (f\cdot g)\cdot h \ \ \ \ \ \forall f, g, h\in G\]
	2) Esistenza elemento neutro
	\[
		\exists e\in G \text{    tale che     } g\cdot e = e\cdot g = g \ \ \ \forall g\in G
	.\] 
	3) esistenza degli inversi
	\[
		\forall g\in G \ \ \ \ \ \exists \ \ \ \ g^{-1}\in G \ \ \text{ tale che    } g^{-1}\cdot g = g\cdot g^{-1} = e
	.\] 
\end{defi}
\begin{nome}[notazione]
	$(G,\cdot)$
	dato  $g\in G$ denotiamo con: \\
	1) g^0 = e\\
	2) g^1 = g\\
	3) g^n = g\cdot\ldots\cdot g
	4) g^{-n} = (g^{-1})^n

\end{nome}
\textbf{Osservazione:}\\
Con questa notazione:\\
\[
	(g^n)^m = g^{nm}
\] 
\[
	g^n \cdot g^m = g^{n + m}
\] 
\textbf{Esempi}\\
1) $(\mathbb Z, + ), (\mathbb Q, + ), (\R, +)$\\
2)  $GL_n(\K) = \lbrace{A\in Mat_{n\times n}(\K) | det(A) \neq 0\rbrace$ con prodotto\\
	3) $SL_n(\K) = \lbrace{A\in Mat_{nn}(\K) | det(A) = 1\rbrace$\\
4) $X$ insieme\\
\text{ } \hspace{90px} $S_X = \lbrace$ funzioni  $X \rightarrow X$ invertibili$\rbrace$\\
\textbf{Speciale}
Se $X = \lbrace 1,\ldots, n\rbrace}$\\
Allora chiamiamo 
 \[
S_n = S_X
.\] 
(è i lgruppo di permutazioni su $n$ elementi)\\
Si chiama gruppo simmetrico
\begin{defi}[Gruppo diedrale]
	$n\geq 3$ Consideriamo l'n-agono regoalre nel piano (3-agono, triangolo)\\
	 $D_n$ è l'insieme delle simmetrie del piano che preservano l' $n$-agono\\
	 Si chiama gruppo diedrale, l'operazione è la composizione
\end{defi}
\textbf{Esempio:}\\
Per $n=3$ abbiamo  $D_3$\\
\textbf{TODO INSERISCI DISEGNO gruppo diedrale}\\
\textbf{Esercizio}\\
Determina gli inversi e tutti i possibili prodotti degli elementi di $D_3$
\begin{defi}[Gruppo Abeliano]
	$(G,\codt)$ gruppo si dice Abeliano se l'operazione è commutativa \[f\cdot g = g\cdot f)\]
\end{defi}
\begin{defi}[Gruppo finito]
	$(G,\cdot)$ gruppo si dice finito se la sua cardinalità è finita \[|G| < +\infty\]
\end{defi}
\begin{defi}[Ordine del gruppo]
	L$(G,\cdot)$ gruppo, l'ordine di $G$ è $|G|$
\end{defi}
\begin{defi}[Ordine di un elemento]
	$ord(g) = \min \lbrace{n\in \mathbb N | g^n = e\rbrace$\\
		se $\cancel\exists n\in \mathbb N$  tale che $ g^n = e$\ \ \  poniamo  \ \ $ord(g) = +\infty$
\end{defi}
\begin{defi}[Gruppo ciclico]
	$n\geq 3$ consideriamo  $C_n$ l'insieme delle isometrie del piano che preservano l' $n$-agono e preservano l'orientazione, questo si chiama gruppo cicliclo
\end{defi}
\textbf{Esempio}\\
Nel caso di $n = 3$ abbiamo solamente 3 elementi: identità, e le due rotazioni (ordine dispari)
\textbf{Esercizi}\\
1) si dimostri che l'elemento neutro in un gruppo è unico\\
2) si dimostri che ogni elemento in un gruppo ammette un unico elemento inverso\\
per casa\\
1) Trvoare un'applicazione biunivoca $S_3 \rightarrow D_3$\\
2) Dimostrare che non esiste un'applicazione biunivoca $S_4 \rightarrow D_4$\\
3) Dimostrare che i seguenti nkn sono gruppi\\
$\cdot Mat_{n\times n} (\K)$ con prodotto riche per colonne\\
 $\codt GL(\K)$ con somma tra matrici\\
 $\codt \mathbb Z \mathbb Q \R con il prodotto$\\
  \begin{prop}
	  $(G, \cdot)$ gruppi finito, Allora ogni elemento ah ordine finito
 \end{prop}
 \begin{dimo}
 	$g\in G$ Considero il sottoinsieme
	 \[
	A = \lbrace g, g^2, g^3, \ldots\rbrace\subseteq G
	.\] 
	quindi $|A|<+\infty \Rightarrow \exists s,t\in \mathbb N, s> t$  tali che \[
	g^s = g^t
	.\] 
	Moltiplico per $g^{-t}$ a destra
	\[
		g^s = g^t \ \ \ \Rightarrow  \ \ \ g^s\cdot g^{-t} = g^t\cdot g^{-t} \ \ \ \Rightarrow \ \ \ g^{s-t} = e
	.\] 
	Quindi $n = s-t\geq 1$ e $g^n = e $ $ \Rightarrow ord(g) \leq n < +\infty $
 \end{dimo}
 \begin{defi}[Sottogruppo]
 	$(G, \cdot)$ gruppo $H\subseteq$ G sottosinsieme, si dice che H è un sottogruppo se $(H,\cdot)$ è un gruppo.\\
	In tal caso scriveremo $H \leq G$
 \end{defi}
 \textbf{Osservazione}\\
 $(G,\cdot)$ gruppo,  $G\subseteq G$ sottoinsieme allora  $H\leq G$se H è chiuso rispettto a $\cdot$ \\e  $H$ è chiuso rispetto agli inversi \\(se  $g, h\in G \Rightarrow g\cdot h\in H$ e se $h\in H \Rightarrow h^{-1}\in H)$

 \begin{prop}
 	$(G, \cdot)$ gruppo $H\subseteq G$ sottoinsieme con  $|H| < +\infty$ Allora:\\
	1)  $H \leq G$ se e solo se  $H$ è chiuso rispetto a $\cdot$
 \end{prop}
 \begin{dimo}
 	$( \Rightarrow )$ ovvia\\
	$ ( \Leftarrow )$ basta dimostrare che $H$ è chiuso rispetto alĺ inverso ovvero\\
	se $|H| < +\infty$\\
	e  H chiuso rispetto a  $\cdot$\\
	Allora  $H$ è chiuso rispetto agli inversi\\
	Sia $h\in H$\\
	$A = \lbrace h, h^2, h^3,\ldots\rbrace\subseteq H$\\
	Allora  $|A|<\infty $\\
	Ragionando come prima deduciamo  $ord(h) < + \infty$
	 \[
		 h\cdot h^{ord(h) -1} = h^{ord(h) -1}\cdot h = e
	.\] 
	Quindi $h^{-1} = h^{ord(h)-1} = h\cdot \ldots\cdot h\in H \Rightarrow h^{-1}\in H$
 \end{dimo}
 \textbf{Esempi}\\
 1)$C_n\leq D_n$\\
 2)  $SL_n(\K)\leq GL_n(\K)$\\
 3) $(G, \cdot)$ gruppo $g\in G$ 
 \[
 <g> = \lbrace g^n \in G| n\in \mathbb Z\rbrace
 .\] 
 Allora  $<g>\leq G$\\
  \textbf{Congruenze}\\
  $(G,\cdot)$ gruppo $H\leq G$\\
   \begin{defi}
  	$f,g\in G$ si dicono congruenti modulo  $H$ se 
	 \[
		 f^{-1}\codt g\in H
	.\] 
	In tal caso scriveremo 
	\[
	f\equiv g \ \ \ mod \ \ \ H
	.\] 
  \end{defi}
  \textbf{Esercizio}\\
  Dimostrare che al congruenza modulo $H$ definisce una relazione di equivalenza su  $G$\\
  \textbf{Suggerimento}\\
  $(f^{-1}\cdot g)^{-1} = g^{-1}\cdot (f^{-1})^{-1} = g^{-1}\cdot f\\$
  e $H$ è chiuso rispetto agli inversi\\
  \textbf{Esercizi:}\\
  $(G,\cdot)$ è un gruppo $H\leq G$ Allora la classe di equivalenza di $g\in G$ modulo  $H$ è il sottoinsieme
   \[
   gH = \lbrace g\cdot h | h\in H\rbrace
  .\] 
  C'è una classe di equivalenza speciale in $G$ data da 
  \[
  e\cdot H = H
  .\] 
  l'unica ad essere un sottogruppo\\
\hline \ \\
Dimostrare che esiste un'applicazione biunivoca tra $H \rightarrow gH\ \ \ \forall g\in G$
\subsection{Classi di equivalenza}
	\begin{nota}
		Sia $(G,\cdot)$ un gruppo e sia $H\leq G$ un sottogruppo:\\
	\begin{itemize}
		\item Dato $g\in G$ il sottoinsieme  $gH$ si chiama laterale sinistro o "classe laterale sinistra"
		\item L'insieme dei laterali sinistri si indica con $G/H$
	\end{itemize}
	\end{nota}
	\textbf{Esempi importanti}\\
	$(G,\cdot) = (\mathbb Z, +)$
	$H = (m) = \lbrace a m | a\in \mathbb Z\rbrace$ con m fissato\\
	 $G/H = \mathbb Z/(m)$ \\\
	 \textbf{Attenzione}\\
	 potete definire $f = g$ mod $H$ tramite la condizione  $f\cdot g^{-1}$ \\
	 Le due definizioni non sono equivalenti [La chiameremo congruenza destra]\\
	 \begin{nota}
	L'insieme delle classi di equivalenza destra si indica con
	\[
		H\backslash G
	.\] 
	 \end{nota}
	 \begin{defi}
	 	Gli elementi di $G/H$ si chiamano laterali sinistri, quelli di  $H\backslash G$ si chiamano laterali destri
	 \end{defi}
	 \textbf{Esercizio:}\\
	 $(G,\cdot)$ gruppo\\
	 $H\leq G \ \ g\in G$ fissato\\
	 Allora il laterale sinistro a cui appartiene  $g$ è\\ \[
	 gH = \lbrace g\cdot h | h\in H\rbrace
	 .\] 
	 \textbf{Soluzione}\\
	 fisso $f\in G$ e osserviamo che  \[
		 g\equiv f \text{ mod } H
	 .\] 
	 Se e solo se $g^{-1}\cdot f\in H$.\\
	 Questo è equivalente a 
	  \[
		  \exists h\in H \text{ tale che } g^{-1}\cdot f=h
	 .\] 
	 ovvero
	 \[
		 \exists h\in H \text{ tale che } f = g\cdot h
	 .\] 
	 \ \\ \hline \ \\ 
 \textbf{Esercizio}\\
 $H\leq G$\\
 Allora  $|G/H| = |H\backslash G|$ \\
 \textbf{Soluzione}\\
 Basta eseguire un'applicazione biunivoca tra i due insiemi
 \begin{defi}
 	$(G,\cdot)$ gruppo $H\leq G$ si dice sottogruppo normale se  $gH = Hg \ \ \ \forall g\in G$
 \end{defi}
 \textbf{Esempio}\\
 $G=S_3$ ricordo che  $S_3$ è il gruppo di permutazioni dell'insimee $\lbrace 1,2,3\rbrace$\\
 Quali sono gli elementi di  $S_3$?\\
 \[
	 \matrice{1&2&3\\1&3&2} = (2,3)
 \] 
 \[
	 \matrice{1&2&3\\2&3&1} = (3,2,1)
 \] 
scambio il 3 con l'uno , il 2 con il 2\\


\begin{aligend}
	&(2,3,1)\\
	&(1,3)\\
	&(1,2)\\
	&Id
\end{aligend}
\[
H_1 = <(1,2)> = \lbrace id, (1,2)\rbrace
.\] 
\[
H_2=<(3,2,1)> = \lbrace id, (3,2,1),(2,3,1)\rbrace
.\] 
\textbf{Esempio (Nuova) Notazione}\\
 $G=S_3$ ricordo che  $S_3$ è il gruppo di permutazioni dell'insimee $\lbrace 1,2,3\rbrace$\\
 Gli elementi sono $S_3 = \{Id, (12), (13),(23),(231),(321)\}$\\
 Per la composizione usiamo la seguente notazione:\\
 Per esempio leggiamo $(123)(231) = (213)$, quindi leggiamo i cicli da destra  e permutiamo verso destra, quindi 2 va in 3, (mi sposto nel primo ciclo) 3 va in 1, quindi ottengo 2 va in 1, finisce quidni con 1 e controllo dove va, nel ciclo di destra, quindi 1 va in 2, cambio cilco, 2 va in 3, ottengo quindi (213)\\
\textbf{Esercizio}\\
Dimostrare che $H_1\leq S_3$ \underline{non} è normale, mentre $H_2\leq S_3$ è normale
\begin{nota}
	Se $H\leq G$ è normale scriveremo
	 \[
	H \trianglelefteq G
	.\] 
\end{nota}
\textbf{Esercizio}\\
$H\leq G$ sottogruppo dimostrare che l'applicazione  \begin{*aligend}&\phi:H \rightarrow gH\\
	&g \rightarrow g\cdot h\end{*aligend}\\
	\textbf{Soluzione}\\
	$ \phi$ è suriettiva per definizione di $gH$\\
	è anche iniettiva infatti se  $h_1,h_1\in H$ soddisfano 
	\[
		gh_1 = gh_2 \ \ 
	.\] 
	allora $h_1 = h_2$ (per la legge di cancellazione)\\
	\textbf{Ossercazione}\\
	$(G,\cdot)$ gruppo\\
	$H\leq G$ Allora
	 \[
	|gH| = |Hg| \ \ \forall g\in G
	.\] 
	anche se $gH\neq Hg$ poiché hanno entrambi la stessa cardinalità di  $H$\\
	Inoltre tutti i laterali sinistri (e destri) hanno la stessa cardinalità\\
	 \begin{defi}
		 $(G,\cdot)$ gruppo, $H\leq G$ l'indice di  $H$ in $G$ è 
		  \[
			  [G:H] = |G/H|
		 .\] 
		 dove $|G/H|$ è il numero di classi laterali sinistre
	\end{defi}
	\textbf{Osservazione}\\
	$H\leq G$ sottogruppo\\
	Se  $G$ è abeliano allora  $H\leq G$\\
	Il viceversa è falso! Possono esistere sottogruppi normali in gruppi non abeliani\\
	\begin{prop}
		$(G,\cdot)$ gruppo $H\leq G$ allora
		 \[
			 |G| = [G:H]|H|
		.\] 
	\end{prop}
	\begin{dimo}
		Basta ricordare che la cardinalità di ciascun laterale sinistro è pari a $|H|$
	\end{dimo}
	\textbf{Osservazione}\\
	$\displaystyle H\subseteq G => [G:H] = \frac {|G|}{|H|}$\\
	\begin{teo}[Lagrange]
		$(G,\cdot)$ gruppo $H\leq G$ Allora l'ordine di  $H$ divide l'ordine di  $G$
	\end{teo}
	\begin{dimo}
		Dall'osservazione segue $\displaystyle\frac{|G|}{|H|} = [G:H]\in \mathbb N$
	\end{dimo}
	\begin{coro}
		$(G,\cdot)$ gruppo di ordine primo (ovvero $|G| = p$ con $p$ primo)\\
		Allora $G$ non contiene sottogruppi non banali (tutto il gruppo o il gruppo minimale)
	\end{coro}
	\begin{dimo}
		Sia $H\leq G$ allora per Lagrange abbiamo
		 \[
			 |H| \text{ divide } p
		.\] 
		$ \Rightarrow |H| = 1$ quindi  $H = \lbrace e \rbrace$\\
		oppure 
		$ \Rightarrow |H| = p$ quindi $H=H$

	\end{dimo}
	\begin{coro}
		$(G,\cdot)$ gruppo (finito)\\
		Dato $g\in G $ si ha  $ord(g)$ divide l'ordine di $G$
		
	\end{coro}
	\begin{dimo}
		Dato $g\in G$ considero\\
		\[
			<g> = \lbrace e, g, g^2, \ldots, g^{n-1}\rbrace
		\]
		\[
		|<g>| = ord(g)
		.\] 
		La tesi segue ora da Lagrange
	\end{dimo}
	\subsection{Operazioni fra sottogruppi}
	\begin{prop}
		$(G,\cdot)$ gruppo $H,K \leq G$\\
		Allora  $H\cap K\leq G$
	\end{prop}
	\begin{dimo}
		$H\cap K$ è chiuso rispetto all'operazione e agli inversi poiché sia $H$ che $K$ che lo sono 
	\end{dimo}
	\textbf{Esercizio}\\
	Esibire due sottogruppi $H,J\leq G$ tali che  $H\cup K$ non è un gruppo\\
	\newpage
	\begin{defi}
		Dati $H,K\leq G$ definiamo il \underline {sottoinsieme}\\
		 \[
		HK = \lbrace h\cdot k | h\in H, k\in K\rbrace
		.\] 
		\textbf{Attenzione} non è necessariamente un sottogruppo
	\end{defi}
	\textbf{Esercizio}\\
	Dimostrare che $HK$ è un sottogruppo, di $G$ se e solo se 
	\[
	HK = KH
	.\] 
	\textbf{Soluzione}\\
	Supponiamo che $HK$ sia un sottogruppo
	\[
		HK = (HK)^{-1} = \lbrace (h\cdot k)^{-1} | h\in H, k\in K\rbrace = K^{-1}H^{-1} = KH
	.\] 
	Viceversa supponiamo che $HK = KH$\\
	1) Dimostro che  $KH$ è chiuso rispetto all'operazione.\\
	$h_1\codt k_1\in HK$ e $h_2\cdot k_2\in HK$\\
\[
	(h_1\cdot k_1)\cdot (h_2\cdot k_2) = h_1\cdot (k_1\cdot h_2)\cdot k_2 = h_1\cdot h_3\cdot k_3\cdot k_2 = (h_1\cdot h_3)\cdot(k_3\cdot  k_1)
.\] 
2) $HK$ è chiuso rispetto agli inversi
\[
	h\cdot k\in HK \leadsto (h\cdot k)^{-1} = k^{-1}\cdot h^{-1} = h_4\cdot k_4\in HK
.\] 
\begin{defi}[Sottogruppo generato da un sottoinsieme]
	$(G,\cdot)$ gruppo $X\subseteq G$ sottoinsieme\\
	Il sottogruppo generato da $X$ è
	\[
		<X> = \bigcap_{H\leq G, X\subseteq H} H
	.\] 
\end{defi}
\begin{nota}
	$\cdot H,K\leq G$\\
	 \[
	<H,K> := <H\cup K>
	.\] \\
	$\cdot g_1,\ldtos,g_n\in G$\\
	\[
	<g_1,\ldots, g_n> := <\lbrace{g_1,\ldots, g_n\rbrace >
	.\] 
\end{nota}
\textbf{Caso Speciale}\\
$(G,\cdot) = (\mathbb Z, +)\ \ \  m\in \mathbb Z$\\
$(m) := <m>$
\subsection{Sottogruppi di $(\mathbb Z, +)$}
\textbf{Ricordo}\\
dato $a\in \mathbb Z$ si ha  $(a)\leq \mathbb Z$\\
 \textbf{Obbiettivo}\\
 non esisotno altri sottogruppi
 \begin{teo}
 	$H\leq \mathbb Z$ allora esiste $m\in \mathbb Z$ tale che  $H = (m)$
 \end{teo}
 \begin{dimo}
 	Distinguiamo due casi:\\
	1) $H= (0)$ finito\\
	2) $H\neq(0)$ allora H contiene (almeno) un intero positivo, Definiamo
	\[
		m := \min\lbrace n\in\mathbb Z | n\geq 1, n\in H\rbrace
	.\] 
	Vogliamo verificare che $H = (m)$ Sicureamente $(m)\subseteq H$ poich\e H\leq \mathbb Z\\
	Viceversa supponiamo che  $\exists n\in Hx(n)$.\\
	Allora
	 \[
		 n = qm - r \text{ per qualche } q\in \mathbb Z \ \ 0 < r < m
	.\] 
	$ \rightarrow r = n - qm \in H$\\
	Ma $r>0, r<m$ quindi otteniamo l'assurdo per minimalità di m

 \end{dimo}
 \begin{prop}
 	$a,b\in \mathbb Z$, Allora:\\
	$1) (a)\cap (b) = (n)$ dove $m := mcm\lbrace a,b\rbrace$\\
	$2) (a) + (b) = (d)$ dove  $d := MCD\lbrace a, b \rbrace$

 \end{prop}
 \textbf{Osservazione}\\ 
 $(a)+(b)$ è della forma $HK$ con $ H = (a)$ e $K = (b)$\\
 inoltre $(a) + (b)\leq \mathbb Z$ poich\e $(\mathbb Z, +)$ è abeliano
 \begin{dimo}
	 $1)(a)\cap (b)$ è il sottogruppo dei multipli di $a$ e di $b$\\
	 Dunque $(a)\cap(b) = (m)$ \\
	 $2)a+b\leq \mathbb Z \Rightarrow (a) + (b) = (d')$ per teorema\\
	 Dobbiamo verificare che $d' = d$ \\
	 \[
		 (d) = (a)+(b)\supseteq (a) \Rightarrow d'|a ( d'\text{ divide }a)
	 .\] 
	 \[ \Rightarrow  \begin{cases}
	 	d'|a\\
		d'|b
	\end{cases} \Rightarrow d'\leq d\]\\
	 $d'\in (a) + (b) \Rightarrow \exists h,k\in \mathbb Z$ tale che $d' = ha + kb$\\
	 Dunque:\\
	 \[
	  \begin{cases}
	  	d|a\\
		d|b
	  \end{cases} \Rightarrow d|d' => d \leq d'\]
	  Allora $d=d'$
 \end{dimo}
 \subsection{Gruppi $D_n$ e $C_n$}
  \textbf{Ricordo}\\
  $n\geq 3$\\
  Fissiamo un  $n-agono$ \\
  $D_n = \lbrace$isometrie che preservano l'n-agono$\rbrace$\\
   $C_n  = \lbrace$isometrie che preserrvano l'n-agono e l'orientazione$\rbrace$\\
 \begin{teo}
 	$n\geq 3$ Allora\\
	\[|D_n| = 2n\]
\[|C_n| = n\]
 \end{teo}
 \begin{dimo}
	 Fissiamo un lato l dell'$n$-agono. Un'isometria $ \varphi\in D_n$ è univocamente determinata dall'immagine di $ \varphi(l)$\\
	 Ho $n$ scelte per il lato e per ogniuna di queste ho 2 scelte per le orientazione (mando il lato in se stesso? in quello dopo? in quello dopo ancora?, posso anche invertire la sua orientazione, i successivi lati vengono definiti da dove viene mandato il primo)\\
	 se non scegliamo l'orientazione, ci rimane il gruppo ciclico, e ciò conclude la dimostrazione
 \end{dimo}
 \textbf{Osservazione}\\
 La dimostrazione prova che
 \[
  C_n = <\rho>
 .\]
 dove $\rho$ è la rotazione di angolo $\frac {2\pi}{n}$ attorno al centro dell'$n$-agono\\
 Infatti $\rho\in C_n \Rightarrow <\rho>\subseteq C_n$ ma l'ordine di questa rotazione è $n$
  \[
 |<\rho>| = ord(\rho) = n = |C_n| => C_n = <\rho>
 .\] 
 \textbf{Osservazione}\\
 Dalla dimostrazione segue che $D_n$ è costituito da $n$ rotazioni \\(della forma $\rho^i \ \ i\in\lbrace 1,\ldots, n\rbrace$ \\
 e $n$ riflessioni\\
 \begin{prop}
 	$n\geq 3$ Allora:\\
	 $1)D_n = <\rho,\sigma >$\\
	 Dove  $\sigma$ è una rotazione qualsiasi  $(\sigma \in D_n\setminus C_n)$ \\
	 $2)\rho^i \sigma = \sigma \rho^{n-i}$ 
 \end{prop}
 \begin{dimo}
 	1)Sicuramente $<\rho,\sigma>\subseteq D_n$\\
	$H = <\rho> = \lbrace{Id, \rho, \rho^2, \ldots,\rho^{n-1}\rbrace $\\
	  $K = <\sigma> = \lbrace Id, \sigma \rbrace$ \\
	  $H\cap K = \lbrace Id\rbrace$
	   \[
		   |KH| = \frac{|H||K|}{|H\cap K|} = 2n
	   .\] 
	   $ \Rightarrow HK \subseteq D_n$ (In particolare HK è sottogruppo)
	   $ \Rightarrow D_n = HK = <\rho, \sigma>$\\
	   $\rho\sigma$ \underline{non} preserva l'orientazione\\
	   $ \Rightarrow \rho^i\sigma$ è riflessione\\
	   $ \Rightarrow ord(\rho^i\sigma) = 2$\\
	   $ \Rightarrow \rho^i\sigma\rho^i\sigma = Id$ \\
	   $ \Rightarrow \rho^i\sigma\rho^i=\sigma$ \\
	   $ \Rightarrow \sigma\rho^i = \rho^{n-1}\sigma$
 \end{dimo}
	\begin{prop}[Caratterizazzione dei sottogruppi normali]
		$(G,\cdot)$ gruppo, $N\leq G$\\ 
		Le seguenti sono equivalenti:\\
		$1) gNg^{-1}\subseteq N \ \ \ \forall g\in G$\\
		$2) gNg^{-1} = N \ \ \ \forall g\in G$\\
		$3) N\normale G$\\ 
		 $4)$ L'operazione $G/N\times G/N \rightarrow G/N$ \\
		 è ben posta \ \ \ \ \ \ \ $(fN,gN) \rightarrow fgN$ \\
		 o equivalentemente $N\backslash G\times n\backslash G \rightarrow n\backslash G$\\
		 \text{} \hspace{90px} $(Nf,Ng) \rightarrow Nfg$

	\end{prop}
	\begin{dimo}
		$1 \rightarrow 2$\\
		Verifichiamo che $N\subseteq gNg^{-1}$\\
		Dato che  $n\in N \Rightarrow n = g(g^{-1}ng)g^{-1}$ basta dimostrare che $g^{-1}ng\in N$\\
		D'altra parte  $g^{-1}ng\in g^{-1}Ng\subseteq N$ (per ipotesi 1)\\
		$2 \rightarrow 3$\\
		$\forall g\in G \ \ \forall n\in N$\\
		$gng^{-1}\in N$ (per ipotesi 2)
		\[
		\begin{cases}
			gn\in Ng\\
			ng^{-1}\in g^{-1}N
		\end{cases} \Rightarrow \begin{cases}
			gN\subseteq Ng (1)\\
			Ng^{-1} \subseteq g^{-1}N (2)
		\end{cases}
		.\] 
		Il che è equivalente a dire che $gN = Ng$ la prima condizione mi dice $G/N\subseteq G/N$ e la seconda dell'arbitrarietà di  $g$\\
		 $G/N\subseteq G/N$\\
		 3  \rightarrow 4\\
		 Dati $f$ e $g\in G$ abbiamo
		 \[
			 (Nf)(Ng) = (fN)(Ng) = fNg = (fN)g = (Nf)g = Nfg
		 .\] 
		 $4 \rightarrow 1$\\
		 Per ipotesi 4 $(Nf)(Ng)=Nfg \ \ \forall f,g\in G$ quindi 
		  \[
		 nfn'g\in Nfg \ \ \ \forall n,n'\in N
		 .\] 
		 dall'arbitrarietà di g, scelgo $g=f^{-1}$, quindi 
		 \[
			 nfn'f^{-1}\in N \ \ \forall f\in G .\] 
			 Moltiplico (a sinistra) per $n^{-1}$ e ottengo\\
			 \[
				 fn'f^{-1}\in N \ \ \forall f\in G
			 .\] 
			 Dall'arbitrarietà di $n'$ otteniamo  $fNf^{-1}\subseteq N\ \ \forall f\in G$ che è la condizione (1) \\
	\end{dimo}
	\textbf{Osservazione}\\
	 $(G,\cdot)$ gruppo, la proposizione ci dice che un sottogruppo $H\leqG$ è normale se e solo se l'operazione indotta su $G/H$ è ben definita 
	 \begin{teo}
	 	$(G,\cdot)$ gruppo $N\trianglelefteq G$\\
		Allora $(G/N,\cdot)$ è un gruppo (detto gruppo quoziente)
	 \end{teo}
	 \begin{dimo}
	 	Associatività, ovvia\\
		elemento neutro : $N=Ne$\\
		elemento inverso di  $Ng$ è $Ng^{-1}$ \ \  $\forall g\in G$
	 \end{dimo}
	 \textbf{Osservazione}\\
	 $(G,\cdot)$ gruppo e $H\leq G$ t.c.  $[G:H] = 2$ Allora  $H\trianglelefteq G$\\
	 Infatti esistono solo due laterali sinistri o destri: H, G/H\\
	 \textbf{Osservazione}\\
	 $(G,\cdot)$ gruppo abeliano $\Rightarrow$ ogni sottogruppo è normale\\
	 \textbf{\underline{Non} vale sempre il viceversa}\\
	 \textbf{Esempio}\\
	 Dimostrare che $Q = \lbrace \pm 1, \pm i,\pm j,\pm k\rbrace$\\ 
	 è un gruppo (rispetto al prodotto)
	 non abeliano in cui però tutti i sottogruppi sono normali\\
	  \textbf{Prodotti:}\\
	  $i^2 = k^2 = j^2 = -1$\\
	   $ij = k \ \ jk = i \ \ ki = j$\\
	    $ji = -k \ \ kh = -i \ \ ik = -j$

	   \subsection{Omomorfismi tra gruppi}
	 \begin{defi}
		Siano $(G_1,\cdot)$ e $(G_2,*)$ gruppi\\
		Sia $ \varphi$ un'applicazione\\
		$
		\varphi: G_1  \rightarrow G_2$ si dice omomorfismo se:
		\[
		\varphi(g\cdot f) = \varphi(g)* \varphi(f) \ \ \ \forall g,f\in G_1
		.\] 
		
	\end{defi}
	\textbf{Osservazione}\\
	Graficamente $ \varphi$ è un omomorfismo se
	
\begin{tikzcd}
	(g, f) \arrow[d, ] & G_1 \times G_1 \arrow[r, "\cdot"] \arrow[d, " \varphi \times \varphi"'] & G_1 \arrow[d, " \varphi"] &  \color{red}(g,f) \arrow[r,red] & \color{red}g\cdot f\arrow[d, red]\\
	(\varphi(g), \varphi(f)) & G_2 \times G_2 \arrow[r, "*"] & G_2 & {} & \color{red}\varphi(g\cdot f)
\end{tikzcd}\\
\textbf{Esempi:}\\
$(\R, +)$ gruppo additivo reali\\
$(\R_{> 0}, \cdot)$ gruppo moltiplicativo reali positivi\\
\textbf{Allora}\\
$exp: \R \rightarrow \R_{>0}$\\
\text{}\ \ \ \ $x \rightarrow e^x$\\
è un omomorfismo infatti: $\forall x,y\in \R$\\
 \[
	 e^{x+y} = e^x\cdot e^y
.\] 
\textbf{Esempio}\\
$ln:\R_{>0} \rightarrow \R$\\
\text{}\ \ \ \ \ $x \rightarrow ln(x)$\\
è un omomorfismo, infatti $\ln (x\cdot y) = \ln(x) + \ln(y) \ \ \ \forall x,y\in\R_{>0}$\\
 \textbf{Osservazione:}\\
 $l^0 = 1 \ \ ln(1) = 0$\\
  $0$ è l'elemento neutro in $(\R, +)$\\
  $1$ è l'elemento neutro in  $(\R_{>0},\cdot)$\\
  \textbf{Osservazione:}\\
  $e^{-x} = \frac{1}{e^x}$\\
  Inverso di  $x$ in $(\R, +)$ \\
  è invero di $e^x$ in  $(\R_{>0},\cdot)$\\
   $\ln(\frac 1 x) = -\ln(x)$\\
   \textbf{Esercizio}\\
   $ \varphi : G_1 \rightarrow G_2$ omomorfismo. Dimostrare\\
   $1) \varphi(e_1) = e_2$\\
   $2) \varphi(g^{-1}) = \varphi(g)^{-1} \ \ \forall g\in G_1$\\
   \textbf{Soluzione:}\\
   $ \varphi (e_1) = \varphi(e_1\cdot e_2) = \varphi(e_1)* \varphi(e_1)$\\
   moltiplico per $ \varphi(e_1)^{-1}$\\
   $ \Rightarrow e_2 = \varphi(e_1)^{-1}* \varphi(e_1) = \varphi(e_1)^{-1}*( \varphi(e_1)* \varphi(e_1)) = \varphi(e_1)$ \\
   \textbf{Esempio}\\
   $(G,\cdot)$ gruppo, $N\trianglelefteq G$\\
   Allora\\
   $\pi:G \rightarrow G/N$\\
   \text{} \ \ \ $g \rightarrow gN$\\
   è un omomorfismo\\
   \textbf{Esempio}\\
   $det: GL_n (\K) \rightarrow \K^*$ \\
   dove $\K$ campo\\
    $\K^* = \K \setminus \lbrace 0 \rbrace$ è un gruppo rispetto l prodotto\\
    allora  $det$ è un omomorfismo\\
    infatti:
     \[
    \forall A,B\in GL_n(\K) \ \ \ \ det(AB) = det(A)det(B)
    .\] 
    in particoalre:\\
    $det(Id) = 1$\\
    $det(A^{-1}) = \frac 1 {det(A)} \ \ \ \forall A\in GL_n(\K)$
     \begin{defi}
    	$ \varphi: G_1 \rightarrow G_2$ omomorfismo\\
	il nucleo di $ \varphi$ è $ker( \varphi) := \lbrace g\in G_1| \varphi(g) = e\rbrace$\\
	L'immagine di $\phi$ è \\
	$Im ( \varphi) = \lbrace{h\in H_2 | \exists g\in G_1: \varphi (g) = h \rbrace$
    \end{defi}
    \textbf{Esercizio:}\\
    $\varphi: G_1 \rightarrow G_2 $ omomorfismo\\
    Allora $ker( \varphi)\trianglelefteq G_1)$\\
    \textbf{Soluzione}\\
    Chiamo $H:ker ( \varphi )$\\
    vorrei verificare che $g Hg^{-1}\subseteq H \ \ \forall g\in G_1$\\
    scegliamo $h\in H$ (ovvero $ \varphi(g) = e_2)$\\
    $ \Rightarrow  \varphi(ghg^{-1}) = \varphi(g) \varphi(h) \varphi(g^{-1})$ = per esercizio = $ \varphi(g) \varphi(h) \varphi(g)^{-1} = e_2$\\
    $ \Rightarrow ghg^{-1}\in H\forall h\in H, \forall g\in G \Rightarrow gHg^{-1}\subseteq H$\\
    \textbf{Osservazione}\\
    $(G,\cdot)$ gruppo, $H\leq G$. Allora  $H\trinaglelefteq G$ se e solo se esiste  $ \varphi: G_1 \rightarrow G_2$ omomorfismo tale che $H=ker( \varphi)$\\
    \begin{dimo}
    	Resta solo l'implicazione $ \Rightarrow $\\
	Sia $H\trianglelefteq G$. considero l'omomorfismo\\
$	\pi: G \rightarrow G/H\\
\text{}\ \ \ g \rightarrow gH$\\
chi è $ker(\pi)$\\
 ker(\pi) = \lbrace g\in G | gH = H\rbrace = \lbrace g\in G | g\in H\rbrace = H$
    \end{dimo}\\
     \textbf{Esempio}\\
     $det: GL_n(\K) \rightarrow K^*$\\
     $ker(det) := \lbrace A\in GL_n(\K) | det(A) = 1\rbrace = SL_n(\K)\\
     quindi\\
     $SL_n(\K)\trianglelefteq GL_n (\K)$\\
     \textbf{Esercizio}\\
     $(G,\cdot)$ gruppo  $g\in G$ fissato\\
      $ \varphi: \mathbb Z \rightarrow G$ \\
     $ \text{} \ \ \ n \rightarrow g^n$\\
     è un omomorfismo\\
     determinare $ker \varphi$ e $Im \varphi$\\
     \textbf{Esercizio}\\
     Sia $  \varphi: G_1 \rightarrow G_2$ omomorfismo\\
     $1)$ Se $H_1 \leq G_1 \Rightarrow \varphi(H_1)\leq G_2$ \\
     se $H_1 \trianglelefteq G_1 \Rightarrow \varphi(H_1)\trianglelefteq \varphi (G_1)$\\
     $2)$ Se $H_3 \leq G_2 \Rightarrow \varphi^{-1}(H_2)\leq G_1$ \\
     se $H_1 \trianglelefteq G_2 \Rightarrow \varphi^{-1}(H_2)\trianglelefteq \varphi (G_1)$\\
	\textbf{Soluzione}\\
	1)Se $H_1\subseteq G_1$ dimostriamo che $ \varphi(H_1)\trianglelefteq \varphi(G_1)$\\
	Verifichiamo che\\
	\[
		f \varphi(H_1) f^{-1}\subseteq \varphi(H_1) \ \ \forall f\in (G_1)
	.\] 
	Quindi basta dimostrare che\\
	$\forall h\in H_1 \ \ \forall g\in G_1$ abbiamo\\
	$ \varphi(g) \varphi(h) \varphi(g)^{-1}\in \varphi(H_1)$\\
	Questo è equivalente a richiedere che\\
	\[
		\varphi(g\cdot h\cdot g^{-1}) \varphi(H_1)
	.\] 
	Ma $ghg^{-1}\in gH_1g^{-1}=H_1$ dato che $H_1\trianglelefteq G_1$\\
	\begin{gather*}
		\exists \tilde h\in H_1 \text { t.c } g\cdot h\cdot g^{-1} = \tilde h\\
		\varphi(ghg^{-1}) = \varphi(\tilde h)\in \varphi(H_1)
	\end{gather*}
	2) Se $H_2\trianglelefteq G_2$ dimostriamo che $ \varphi^{-1}(H_2)\trianglelefteq G_1$\\
	Ho due omomorfismi,\\
	li compongo:
	\[
		\psi: G_1 \xrightarrow[ \varphi]{} G_2 \xrightarrow[\pi]{} G_2/H_2
	.\] 
	Studia il $\ker (\psi)$\\
	 $\ker(\psi):= \lbrace g\in G_1 | \psi (g) = H_2\rbrace = \lbrace g\in G_1 | \varphi(g)H_2 = H_2\rbrace$\\
	 $ker (\psi) = \lbrace g\in G | \varphi(g)\in H_2 \rbrace = \varphi^{-1}(H_2)$\\
	 Quindi $ \varphi^{-1}(H_2)$ è il nucleo di un omomorfismo $\psi : G_1 \rightarrow G_2/H_2$ e dunque $ \varphi^{-1}(H_2)\trianglelefteq G_1$\\
     %lezione 4
	\textbf{Esercizio}\\
	Sia $ \varphi: G_1 \rightarrow G_2$ omomorfismo dei gruppi\\
	$ker \varphi = \lbrace g\in G_1 | \varphi(g) = e_2\rbrace$\\
	Dimostrare che\\
	$ \varphi$ è iniettivo $ \Leftrightarrow ker( \varphi) = \lbrace e_1 \rbrace$ \\
	soluzione:\\
	supponiamo che $ker ( \varphi) = \lbrace e_1\rbrace$\\
	Allora dati $g,h \in G_1$ t.c $ \varphi (g) = \varphi(h)$\\
	dobbiamo mostrare che $g=h$\\
	moltiplico per  $ \varphi(h)^{-1}$\\
	\begin{gather*}
		\Rightarrow \varphi(h)^{-1} * \varphi(g) = e_2\\
		\Rightarrow \varphi(h^{-1})* \varphi(g) = e_2\\
		\Rightarrow \varphi(h^{-1}\cdot g) = e_2\\
	 \Rightarrow h^{-1}\cdot g\in ker \varphi\\
	 \Rightarrow h^{-1}\cdot g = e_1\\
	 \Rightarrow g = h\\
	\end{gather*}
	Il viceversa è lasciato  al lettore come esercizio\\
	 \textbf{Osservazione:}\\
	 Se $ \varphi:G_1 \rightarrow G_2$\\
	 omomorfismo di gruppi\\
	 $H_2 = \lbrace e_2 \rbrace \trianglelefteq G_2$\\
	 l'esercizio $(2)$ ci dice che  $\ker( \varphi) = \varphi^{-1}(\lbrace e_2\rbrace)\trianglelefteq G_1$\\
	 \textbf{Osservazione}\\
	 Dalla parte (1) segue che\\
	 $H_1\leq G_1 \Rightarrow \varphi(H_1)\leq G_2$ \\
	 Quindi  se scelgo $H_1=G_1\leq G_1$\\
	 $ \Rightarrow Im( \varphi) = \varphi(G_1)\leq G_2$ 
	 \begin{lemm}
	 	
	 $(G,\cdot)$ gruppo\\
	 $N\trianglelefteq G, H\trianglelefteq G$ sottogruppi normali\\
	  $\pi: G \rightarrow G/N$\\
	  Allora $\pi(H) = \pi(HN)$\\
	 \end{lemm}
	  \begin{dimo}
	  	$H\subseteq HN$ poiché $e\in N$ ogni elemento di  $H$ lo scrivo come lui stesso e $ \Rightarrow \pi(H)\subseteq \pi(HN)$ \\
		Viceversa dimostriamo che $\pi(HN) \subseteq \pi(H)$\\
		infatti:\\
		 $\forall h\in H \ \ \ \forall n\in N$\\
		  $\pi(hn)=\pi(h)\pi(n)$ (omomorfismo)\\
		   $n\in N$\\
		    $ \Rightarrow \pi(n) = N \rightarrow \pi(h)\pi(e) = \pi(ne)$ \\
		    \pi(e) = N \ \ \ \ =\pi(n)\in \pi H
	  \end{dimo}
	  \begin{lemm}
	  	$(G,\cdot)$ gruppo\\
		$\cdot H\trianglelefteq G$\\
		$\cdot N\trianglelefteq G$\\
		$\cdot \pi \rightarrow G/N$\\
		Allora:\\
		$1)\pi^{-1}(\pi(H))=HN$\\
		$2)$ se $N\subseteq H \rightarrow\pi^{-1}(\pi(H)) = N$\\
		$3) \bar H\leq G/N \rightarrow \pi(\pi^{-1}(\bar H)) = \bar H$

	  \end{lemm}
	  \begin{dimo}[1]
		  $\pi^{-1}(\pi(H)) = ?$\\
		  osserviamo che dal lemma 1\\
		   $\pi(H) = \pi(HN) = HN$\\
		   dato che  $\pi(hn) = \pi(h)\pi(n) = hn$\\
		   $ \Rightarrow \pi^{-1}(\pi(H)) = \pi^{-1}(\pi(HN)) = \pi^{-1}(HN)\supseteq HN$ \\
		   Resta da verificare che $\pi^{-1}(\pi(H))\subseteq HN$\\
		    \begin{gather*}
			    \pi^{-1}(\pi(H)):=\lbrace g\in G|\pi(g)\in \pi(H)\rbrace\\
			    =\lbrace g\in G|\exists g\in H:\pi(g) = \pi(h)\rbrace\\
			    =\lbrace g\in G| \exists h\in H: \pi(h)^{-1}\pi(g)=N\rbrace \text{ N= elemento neutro in G}\\
			    =\lbrace g\in G|\exists h\in H: \pi(hg) = N\rbrace\\
			    =\lbrace g\in G | \exists h\in H: h^{-1}g\in N\rbrace\\
			    =\lbrace g\in G | \exists h\in H: g\in hNŕbrace \subseteq HN
		   \end{gather*}
		   segue (1)
	  \end{dimo}
	  \begin{dimo}[2]
	  	È un caso particolare del punto 1, infatti se
		\[
		N\subset H \Rightarrow HN = H
		.\] 
	  \end{dimo}
	  \begin{dimo}[3]
	  	Segue dal fatto che $\pi$ è un omomorfismo suriettivo 
		
		 \[
			 \pi(\pi^{-1}(\bar H))=\pi (G)\cap \bar H = \bar H
		.\] 
	  \end{dimo}
	  \begin{teo}
		$(G,\cdot)$, $n\trianglelefteq G$\\
		Allora esistono due corrispondenze biunivoche
		 \begin{center}
		 	
			 $\lbrace \text{sottogruppi } H\leq G \ \ t.c. \ \ N\supseteq H\rbrace \rightarrow \{\text{sottogruppi di } G/N\rbrace\\
			 H \rightarrow \pi (H)\\
			 \pi^{-1} \leftarrow \bar H\\
			 \lbrace \text{ sottogruppi normali } H\trianglelefteq G \ t.c \ N\subseteq H\rbrace \rightarrow \lbrace \text{ sottogruppi normali } G/N \rbrace\\
			 H \rightarrow \pi(H)\\
			 \pi^{-1}(\bar H) \rightarrow \bar H$
		 \end{center}
	  \end{teo}
	  \begin{dimo}
		  Il lemma 2 (punti 2 e 3) garantisce che le due applicazioni $H \rightarrow \pi(H)$ $\pi^{-1}(H) \rightarrow\bar H$\\
		  sono una l'inversa dell'altra
	  \end{dimo}
	  \textbf{Osservazione:}\\
	  Per la seconda corrispondenza osserviamo che per la suriettività di $\pi$ e l'esercizio di oggi \\
	  \[
	  H\trianglelefteq G \rightarrow \pi (H)\trianglelefteq G/N
	  .\] 
	  \begin{teo}[Teorema di omomorfismo]
	  	$ \varphi:G_1 \rightarrow G_2$ omomorfismo\\
		$\cdot N\trianglelefteq G_1$\\
		$\pi:G_1 \rightarrow G/N$\\
		Allora:\\
		1) esiste unico omomorfismo\\
		$\bar\varphiL G/N \rightarrow G_2$\\
		t.c. $ \bar  \varphi\circ \pi = \varphi$
		\begin{tikzcd}
G_1 \arrow[r, "\varphi"] \arrow[d, "\pi"] & G_2  \\
G_1 / N \arrow[ur, dashed, "\exists ! \bar\varphi"]
\end{tikzcd} \\
2) $Im (\bar \varphi) = Im ( \varphi)$\\
3) $ \bar \varphi$ è iniettivo \Leftrightarrow $ker\varphi = N$
	  \end{teo}
	  \begin{dimo}
	  	La condizione $\bar \varphi \cdot \pi = \varphi$\\
		Significa\\
		$\forall g\in G_1$ si ha \\
		$\bar \varphi\cdot \pi(g) = \varphi(g)$\\
		ovvero\\
		$ \bar \varphi(gN) = \varphi(g)$\\
		Dobbiamo verificare:\\
		$\cdot $ Unicità (segue da $\bar \varphi\cdot \pi = \varphi)$\\
		$\cdot \bar \varphi$  è ben definita\\
		$\cdot \bar \varphi$ è un omomorfismo\\
		significa che se $gN=fN$ per qualche $g,f\in G_1$, allora $ \varphi(g) = \varphi(f)$\\
		Verifichiamo:\\
		$gN = fN \rightarrow g\equiv f mod N$\\
		$ \Rightarrow \exists n\in N$ t.c. $g^{-1}f = n$\\
		 $ \Rightarrow f=gn \Rightarrow \varphi(f) = \varphi(gn)$ \\
		 $ \Rightarrow \varphi(f) = \varphi(g) \varphi(n) = \varphi(g)$ \\
		 dato che $ \varphi(n) = e_2$ ovvero $N\subseteq\ker \varphi$\\
		 \textbf{Mostriamo adesso che $\bar \varphi$ è un omomorfismo}\\
		 Significa che $\forall f,g\in G$\\
		  \[
		 \bar \varphi((fN)\cdot (gN)) = \bar \varphi(fN)\cdot \bar\varphi(gN)
		 .\] 
		 Per definizione\\
		 \[
		 \bar\varphi ((fN)(gN)) = \bar \varphi(fgN) = \varphi(fg) = \varphi(f) \varphi(g)
		 .\] 
		 $2) \bar \varphi\circ \pi = \varphi$\\
		 dalla suriettività del $\pi$ segue che  $Im(\bar  \varphi) = Im ( \varphi)$\\
		 $3) \bar \varphi$ è iniettivo $ \Leftrightarrow ker \bar \varphi = \lbrace N\rbrace$\\
		 $ker\bar \varphi = \lbrace gN\in G_1/N | \bar \varphi(gN) = e_2\rbrace$\\
		 $=\lbrace gN\in G_1/N | \varphi(g) = e_2\rbrace$\\
		 $=\lbrace gN\in G_1/N | g\in ker( \varphi)\rbrace$
	  \end{dimo}
	  \begin{coro}
	  	$(G,\cdot)$, $N\trianglelefteq G$\\
		Allora esiste una corrispondenza biunivoca\\
		\begin{center}
			$
			 \lbrace \text{omomorfismi } \varphi:G \rightarrow G' \ t.c. \ N\subseteq ker( \varphi)\rbrace \rightarrow \lbrace \text{omomorfismi }G/N \rightarrow G'\rbrace\\
			 \varphi \rightarrow \bar \varphi\\
			 \bar\circ \pi \leftarrow \bar \varphi
			 $
		\end{center}
	  \end{coro}
	  \begin{dimo}
		  basta osservare che\\
		  dato $\bar \varphi: G/N \rightarrow G'$ la composizione\\
		  $\bar \varphi\circ \pi: G \rightarrow G'$ è un omomorfismo\\
		  tale che $ker(\bar \varphi\circ \pi)\supseteq N$\\
		  segue $\pi(N) = N$ che è l'elemento neutro di  $G/N$\\
		   $ \Rightarrow \bar \varphi\circ \pi (N) = e'$ che è l'elemento neutro  di $G'$\\
		   \end{dimo}
		    \begin{defi}
		   	$ \varphi: G_1 \rightarrow G_2$\\
			omomorfismo si dice isomorfismo se è invertibile\\

		   \end{defi}
		   \begin{teo}[Primo teorema di isomorfismo]
		   	$ \varphi:G_1 \rightarrow G_2$\\
			Allora:\\
			$Im( \varphi) \cong G_1/ker( \varphi)$\\
			Dove $\cong$ (isomorfo) significa che esiste un isomorfismo tra i due gruppi
		   \end{teo}\\
		   \begin{dimo}
		   \[
\begin{tikzcd}
G_1 \arrow[r, "\varphi"] \arrow[d, "\pi"] & G_2 \\
G_1 / N \arrow[ur, "\exists ! \overline{\varphi}"]
\end{tikzcd}
\]	
scelgo $N-ker \varphi$\\
il teorema di isomorfismo fornisce un omomorfismo iniettivo
\[
\bar\varphi: G_1/\ker \varphi \rightarrow G_2
.\] 
Allora mi restringo all'immagine di $\bar \varphi$ così diventa suriettiva\\
\[
G/ker \varphi \cong  Im ( \bar\varphi) \cong Im( \varphi)
.\] 
la prima tramite $\bar \varphi$ la seconda per il teorema di isomorfismo\\
\end{dimo}
\textbf{Applicazione:}\\
det: $GL_n(\K) \rightarrow (\K^*,\cdot) = (\K\setminus\{0\},\cdot)$ \\
$\ker(det) = SL_n(\K)$ matrici con det 1\\
$ \Rightarrow GL_n(\K)/SL_n(\K) \cong (\K^*,\cdot)$
		   %lezione 5
		   \newpage
	\subsection{Teoremi di isomorfismo}
	\begin{teo}[Secondo teorema di isomorfismo]
		$(G,\cdot)$ gruppo\\
		$H,N\normale G$ tali che $N\subseteq H$ Allora
		 \begin{enumerate}
			 \item$ H/M\normale G/N $
			 \item $G/N/H/N \cong G/H$
		 \end{enumerate}
	\end{teo}
	\begin{dimo}
	\begin{tikzcd}
		G \arrow[r, "\varphi = \pi_H"] \arrow[d,"\pi"]  & G/H \\
G / N \arrow[ur, dashed, "\exists ! \overline{\varphi}"]
\end{tikzcd}
$\pi_H$ proiezione sul quoziente H\\
$N\subseteq H = ker ( \varphi)$ \\
Inoltre $Im( \bar\varphi) = Im( \varphi) = G/H$\\
\textbf{Idea:} applicare il primo teorema di isomorfismo\\
\underline{suriettiva} $\bar \varphi: G/N \rightarrow G/H$\\
basta quindi dimostrare che $ker (\bar \varphi) = H/N$\\
Studiamo
\[
ker(\bar \varphi) = \lbrace gN\in G/N | \bar \varphi(gN) = H\rbrace
.\] 
\[
	\{gN\in G/N | gH = H\}
.\] 
\[
	\{gN\in G/N | g\in H\} = H/N
.\] 
	\end{dimo}
	\begin{coro}
		In $(\mathbb Z, +)$ gruppo abeliano\\$a,n\in \mathbb Z$ interi non nulli\\
		Denotiamo con
		\[
			[a] = a + (n) \in \mathbb Z/(n) = \{ [0],[1],[2],\ldots,[n-1]\}
		.\] 
		Allora $ord_{\mathbb Z/(n)}([a]) = \frac n {MCD(a,n)}$
	\end{coro}
	\textbf{Nota:}\\
	se $MCD(n,a) = 1$ allora a genera il gruppo ciclico  $\mathbb Z/(n)$
	 \begin{dimo}
		Consideriamo $G=\mathbb Z \ \ H = (a) + (n) \ \ N= (n)$\\
		Dal II Teorema di isomorfismo
		 \[
			 \bigslant{\mathbb Z/(n)}{([a])
			 }\cong\bigslant{\mathbb Z/(n)}{(a) + (n)/(n)} \cong \bigslant {G/N}{H/N}\cong G/N \cong \mathbb Z/(MCD(a,n))
		.\] 
	\end{dimo}
	Confrontiamo le cardinalità\\
	\[
	MCD(a,n) = |\mathbb Z/(MCD(a,n))|
	.\] 
	\[
		= |\bigslant{\mathbb Z/ (n)}{([a])}|

	.\] 
	\[
		\frac {|Z/(n)|}{([a])} = \frac n {ord([a])}
	.\] 
	\[
		ord([a]) = \frac n {MCD(a,n)}
	.\] 
	\begin{lemm}
		$a,bg\in \mahbb Z$ non nulli\\
		tali che  $a|b$ (allora  $(b)\subseteq (a)$ \\
		Allora\\
		\[
		|(a)/(b)| = \frac ba
		.\] 
	\end{lemm}
	\begin{dimo}
		Studiamo $(a)/(b)$\\
		Per definizione è l'insieme dei laterali\\
		 \[
			 (a)/(b) = \{ta + (b) | t\in \mathbb Z\}
		.\] 
		dobbiamo capire quanti laterali \underline{distinti} esistono\\
		Dati $t,s\in \mathbb Z$ tali che\\
		\[
		ta + (b) = sa + (b)
		.\] 
		\[
		\Leftrightarrow ta \equiv sa \ mod(b)
		.\] 
		\[
		\Leftrightarrow -ta + sa \in(b)
		.\] 
		Allora\\
		\[
			(a)/(b) = \{ta + (b)|tt\in\{1,\ldots,\frac ba\}\}
		.\] 
	\end{dimo}
	\newpage
	\begin{teo}[III teorema di isomorfismo]
		$(G,\cdot)$ gruppo\\
		\begin{itemize}
			\item $N\trianglelefteq G$\\
			\item  $H\leq G$
		\end{itemize}
		Allora
		\begin{enumerate}
			\item$H\cap N\trianglelefteq H$\\
			\item  $\bigslant H {H\cap N}\cong HN/N$
		\end{enumerate}
	\end{teo}
		\begin{dimo}
			$\pi_N:G \rightarrow G/N$\\
			$g \rightarrow gN$\\
			consideriamo la restrizione\\
			\begin{gather*}
				\pi_N|_H:H \rightarrow G/H\\
				h \rightarrow hN\\
				ker (\pi_N|_H) = \{h\in H| \pi_N|_H(h) = N\}\\
					       =\{h\in H|hN = N\}\\
					       =\{h\in H| h\in N\}\\
					       =H\cap N
			\end{gather*}\\
			Deduciamo che $H\cap N\trianglelefteq N$\\
			Idea: Applicare il I teorema di isomorfismo all'omomorfismo
			 \[
			\varphi=\pi_N|_H:H \rightarrow G/N
			.\] 
			Avremo $Im( \varphi)\cong H/ker( \varphi) = H/H\cap N$\\
			Studiamo $Im( \varphi)$\\
	\[
	Im( \varphi) = Im( \pi_N|_H) = \pi_N(H) = \pi_N(HN) = HN/N
	.\] 
	Il penultimo passaggio deriva da un lemma già visto a lezione
		\end{dimo}
		\newpage
		\begin{coro}
			$a,b\in \mathbb Z$ non nulli\\
			Allora $mcm(a,b) = \frac {ab} {MCD(a,b)}$
		\end{coro}
		\begin{dimo}
			$G = \mathbb Z$ \\
			$H = (a)$\\
			 $N= (b)$\\
			  $H+N = (MCD(a,b))$\\
			   $H\cap N = (mcm(a,b))$\\
		Dal III teorema di isomorfismo 
		 \[
			 \bigslant{(a)}{(mcm(a,b))}\cong \bigslant H {H\cap N}\cong \bigslant {HN} N \cong \bigslant{(MCD(a,b))}{(b)}
		.\] 
		Confrontiamo la cardinalità\\
		Per il lemma\\
		\[
			\frac {mcm(a,b)}{a} = |\biglsant {(a)}{(mcm(a,b))}| = |\bigslant {(MCD(a,b))}{(b)} = \frac {b}{MCD(a,b)}
		.\] 
		Quindi 
		\[
			mcm(a,b) = \frac {ab}{MCD(a,b)}
		.\] 
		\end{dimo}
		\subsection{Classificazione di gruppi di ordine "piccolo" a meno di isomorfismo}
		\textbf{Ordine 1}\\
		Se $|G| = 1 \ \ \Rightarrow  \ \ G = \{e\}$\\
		\textbf{Ordine p primo:}\\
		Abbiamo mostrato che se $|G|=p$ allora  $G$ non ammette sottogruppi non banali\\
		Sia $g\in G$ tale che $g\neq e$
		$ \Rightarrow ord(g) = p \ \Rightarrow G = <g>$ \\
		\begin{gather*}
			\varphi: G \rightarrow G_p = <p>\\
			g \rightarrow p
		\end{gather*}\\
		\textbf{Obiettivo:} classificare a meno di isomorfismo i gruppi di ordine 4 e di ordine 6\\
		\newpage
		\begin{defi}[Klein,1884]
			Il gruppo di Klein, $K_4$ è il gruppo delle isometrie del piano che preservano un rettangolo fissato.
		\end{defi}
		\textbf{Esercizio}\\
		Verificare che $K_4 = \{id, \rho,\sigma, \rho\sigma\}$\\
		dove $\rho$ = rotazione di angolo $\pi$\\
		e dove  $\sigma$ = riflessione rispetto ad un lato\\
		\textbf{Osservazione}\\ tutti gli elementi in $K_4$ hanno ordine $\leq 2$ Quindi  $K_4\neq C_4$\\
		\begin{nota}
			Dato che $K_4 = <\rho,\sigma>$\\
			denoteremo anche
			\[
				K_4 = D_2 \text{ (gruppo diedrale)}
			.\] 
		\end{nota}
		\textbf{Esercizio}\\
		$(G,\cdot)$ gruppo in cui ogni elemento ha ordine $\leq 2$ (equivalentemente ogni elemento è inverso di se stesso)\\
		1) Dimostrare che $G$ è abeliano\\
		2) Se $|G| = 4$ dimostrare che  $G \cong K_4$\\
		\textbf{Svolgimento}
		1) Dati $f,g\in G$\\
		$fg = (fg)^{-1} = g^{-1}f^{-1} = gf$\\
		2) Sia $|G| = 4$\\
		Scelgo  $g,f\in G$ distinti tali che \begin{cases}
			g\neq e\\
			f\neq e
		\end{cases}\\
		Considero $H = <g,h>$ \\
		Per Lagrange\\
		$H \geq 3$\\
		$ \Rightarrow H = H\\
		\Rightarrow G = \{e,f,g,fg\}$\\
		abeliano\\
		Costruisco l'isomorfismo esplicito con $K_4$\\
		\begin{gather*}
			\varphi:G \rightarrow K_4 = <\rho,\sigma>\\
			e \rightarrow e\\
			f \rightarrow\rho\\
			g \rightarrow\sigma\\
			fg \rightarrow\rho\sigma
		\end{gather*}
		che è chiaramente biunivoca ed è un omomorfismo $ \Rightarrow \varphi$ è un isomorfismo
		%lezione 6
		
		\newpage
\subsection{Teoremi sulla cardinalità dei gruppi}
	\begin{teo}
		
	$(G,\cdot)$ gruppo. Se $|G| = 6$ allora\\
	 $G\cong C_6$ (abeliano) oppure $G\cong D_3$ (non abeliano)
	\end{teo}
	\begin{dimo}
		Se $G$ contiene un elemento di ordine 6 allora $G\cong C_6$\\
		Se invece $G$ non contiene elementi di  ordine 6, per l'esercizio (2) esistono elementi $r,s\in G$ t.c. $ord(r) = 3$ e $ord(s) = 2$\\
		Definisco:\\
		\[
			G:=<r>=\{e,r,r^2\} \ \ \ \ k:=<s>=\{e,s\}
		.\] 
		\[
			H\cap K = \{e\}
		.\] 
		\[
			|HK| = \frac{|H||K|}{|H\cap K |} = 6 = |KH|
		.\] 
		$ \Rightarrow HK = G = KH$ \\
		\textbf{Esplicitamente:}\\
		$HK = \{e,r,r^2,s,rs,r^2s\}$\\
		$KH = \{e,r,r^2,s,sr,sr^2\}$\\
		 \textbf{Dobbiamo considerare 2 casi:}\\
		 I caso:  $rs = sr$\\
		 studiamo  $ord(rs)$\\
		  $(rs)^2 = r^2s^2 = r^2\neq e \Rightarrow ord(rs)\neq 2$ \\
		  $(rs)^3 = r^3s^3 = s^3 = s\neq e$\\
		  \textbf{Per Lagrange}\\
		   necessariamente $ord(rs) = 6$\\
		    $ \Rightarrow G$ è cicliclo $ \Rightarrow  $ Assurdo\\
		    II caso:
		    \begin{cases}
		    	rs = sr^2\\
			r^2s = sr
		    \end{cases}\\
		    Costruiamo l'isomorfismo\\
		    \begin{gather*}
		    	G \rightarrow D_3:=<\rho,\sigma>\\
			e \rightarrow Id\\
			r \rightarrow \rho\\
			r^2 \rightarrow \rho^2\\
			s \rightarrow \sigma\\
			sr \rightarrow \sigma\rho
		    \end{gather*}
	\end{dimo}
	\begin{defi}
		Dato un gruppo $(G,\cdot)$ il reticolo dei sottogruppi $T_G$ è un grafo definito come\\
		\begin{itemize}
			\item esiste un vertice in $T_G$ per ogni sottogruppo $H\leq G$ \\
			\item esiste un lato $H_1$ \---- $H_2$ se e solo se $H_1\subseteq H_2$ \\e $\cancel\exists K\leq G$ t.c. $H_1\subset K\subset H_2$
		\end{itemize}
	\end{defi}
	\textbf{Esempio:}\\
	$T_{D_4}$\\
	Ricordiamo che $D_4 = <\sigma,\rho>  \ \ |D_4|=8$\\
	studiamo i sottogruppi di $D_4$\\
	\textbf{ordine 1:} L'unico sottogruppo è $H=\{e\}$\\
	 \textbf{ordine 2:} Sono tutti e soli quelli generati da un elemento di ordine $2$ in $D_4$
	 \begin{center}
	 \begin{tikzpicture}
    % Nodes
    \node (A) at (0,0) {$\langle \sigma \rangle$};
    \node (B) at (2,0) {$\langle \sigma \rho \rangle$};
    \node (C) at (4,0) {$\langle \rho^2 \rangle$};
    \node (D) at (6,0) {$\langle \sigma \rho^2 \rangle$};
    \node (E) at (8,0) {$\langle \sigma \rho^3 \rangle$};
    \node (F) at (4,-1) {$\{e\}$}$

    % Arrows
    \draw[-] (A) -- (F);
    \draw[-] (B) -- (F);
    \draw[-] (C) -- (F);
    \draw[-] (D) -- (F);
    \draw[-] (E) -- (F);
\end{tikzpicture}
	 \end{center}
	 \textbf{ordine 4:} per la classificazione sono ciclici $(C_4)$ oppure di Klein $(K_4)$ altre al ciclico $<p>$ esistono altri sottogruppi\\
	 \[
		 \langle \rho^2,\sigma\rangle = \{e, \sigma,\rho^2,\sigma\rho^2\}
	 .\] 
	 \[
		 \rangle\rho^2, \sigma\rho\rangle = \{e,\sigma\rho, \rho^2,\sigma\rho^3\}
	 .\]
	 Ordine 8: $D_4$\\
	 \begin{center}
	 \begin{tikzpicture}
    % Nodes
    \node (A) at (4,0) {$\langle \sigma, \rho \rangle$};
    \node (B) at (2,-2) {$\langle \sigma\rho^2 \rangle$};
    \node (C) at (4,-2) {$\langle \rho \rangle$};
    \node (D) at (6,-2) {$\langle \sigma, \rho^2 \rangle$};
    \node (E) at (0,-4) {$\langle \sigma\rho^3 \rangle$};
    \node (F) at (2,-4) {$\langle \sigma\rho \rangle$};
    \node (G) at (4,-4) {$\langle \rho^2 \rangle$};
    \node (H) at (6,-4) {$\langle \sigma\rho^2 \rangle$};
    \node (I) at (8,-4) {$\langle \sigma \rangle$};
    \node (J) at (4,-6) {$\lbrace e \rbrace$};


    % Arrows
    \draw[-] (A) -- (B);
    \draw[-] (A) -- (C);
    \draw[-] (A) -- (D);
    \draw[-] (B) -- (E);
    \draw[-] (B) -- (F);
    \draw[-] (B) -- (G);
    \draw[-] (C) -- (G);
    \draw[-] (D) -- (G);
    \draw[-] (D) -- (H);
    \draw[-] (D) -- (I);
    \draw[-] (E) -- (J);
    \draw[-] (F) -- (J);
    \draw[-] (G) -- (J);
    \draw[-] (H) -- (J);
    \draw[-] (I) -- (J);

\end{tikzpicture}
	 \end{center}
	 \textbf{Esempio:}\\
	 $G=D_4\\
	 N=<\rho^2>\trianglelefteq G$\\
	 Vogliamo $T_{G/N}$\\
	 studiamo $G/N = D_4/<rho^2>$\\
	 $\displaystyle|G/N| = [G:N] = \frac {|G|}{|N|} = \frac 82 = 4$\\
	 chi sono i laterali?\\
	 $IdN = N <\rho^2> = \{Id,\rho^2\}$\\
	 $\rho N = \{\rho, \rho ^3\}$\\
	 $\sigma N = \{\sigma, \sigma \rho^2\}$\\
	 $\sigma\rho N = \{\sigma\rho, \sigma\rho^3\}$\\
	  \textbf{Ricordo:}\\
	  Abbiamo una corrispondenza biunivoca tr ai sottogruppi di $G/N$ e i sottogruppi di $G$ contenenti $N$.\\
	  \begin{center}
	  	
	  \begin{tikzpicture}
    \node (A) at (4,0) {$G/N$};
    \node (B) at (2,-2) {$\langle\sigma\rho N\rangle$};
    \node (C) at (4,-2) {$\langle\rho N\rangle$};
    \node (D) at (6,-2) {$\langle\sigma N\rangle$};
    \node (E) at (2,-3) {$\{\sigma\rho N, N\}$};
    \node (F) at (4,-3) {$\{\rho N, N\}$};
    \node (G) at (6,-3) {$\{N, \sigma N\}$};
    \node (H) at (4,-5) {$\{N\}$};
    
    \draw[-] (A) -- (B);
    \draw[-] (A) -- (C);
    \draw[-] (A) -- (D);
    \path (B)--(E) node[midway]{\storto =};
    \path (C)--(F) node[midway]{\storto =};
    \path (D)--(G) node[midway]{\storto =};
    \draw[-] (E) -- (H);
    \draw[-] (F) -- (H);
    \draw[-] (G) -- (H);
	  \end{tikzpicture}
	  \end{center}
	  \textbf{Obiettivo: studiare $S_n$}\\
	  \textbf{Ricordo:}\\
	  $X:=\{1,\ldots,n\}$\\
	  $S_n:=S_X= \{$ applicazioni biunivoche $X \rightarrow X\}$\\
	  $S_n$ gruppo di permutazioni\\
	   \textbf{Osservazione:}\\
	   $|S_n| = n!$\\
	   \textbf{Osservazione:}\\
	   se $n=3 \rightarrow |S_3| = 6$\\
	   $ \Rightarrow S_3 \cong D_3$ \\
	   \textbf{Osservazione}\\
	   $S_n\cong D_n \ \ \forall n\geq 4$\\
	   Infatti  $n! > 2n \ \ \forall n\geq 4$ 
	   \subsection{notazioni in $s_n$}\\
	   \begin{aligend*}
	   	\sigma = (1 2 3)(4 7)\\
	   	\tau = (23456)\\
		\sigma\tau=\sigma\circ\tau = (123)(46)(23456)(12)(36)(45)\\
		\tau\circ\sigma = (23456)(123)(46) = (13)(24)(56)

	   \end{aligend*}
	   \begin{lemm}
		   Data $\sigma\in S_n$ allora  $\sigma$ partizione $X = \{1,\ldots, n\}$ in sottoinsiemi permutati ciclicamente e disgiunti tra loro
	   \end{lemm}
	   \begin{dimo}
	   	Definiamo la relazione d'equivalenza $i\sim j \Leftrightarrow \exists k\in \mathbb Z \ t.c. \ \sigma^k(i) = j$\\
		È una relazione d'equivalenza!\\
		\textbf{studiamo le classi di equivalenza}\\
		fissato $i\in X$\\
		la sua clase
		 \[
			 X_i = \{\sigma^k(i)|k\in \mathbb Z\}\subseteq X
		.\] 
	quindi $\exists \ \ k_1,k_2\in \mathbb Z$ distinti t.c. $\sigma^{k_1}(i) = \sigma^{k_2}(i)$\\
	\begin{aligned*}
		\Rightarrow i = \sigma^{k_2-k_1}(i)\\
		\Rightarrow m:=min\{k\in\mathbb Z_{>0}|\sigma^k(i) = i\}\\
		\Rightarrow X_i = \{i,\sigma(i),\sigma^2(i),\ldots,\sigma^{n-1}(i)\}
	\end{aligned*}
	   \end{dimo}
	\begin{prop}
		Data $\sigma\in S_n$, allora  $\sigma$ può essere rappresentata come composizione di cicli disgiunti
	\end{prop}
	\textbf{Obiettivo:}
		Definire un omomorfismo
		\[
			sgn: S_n \rightarrow (\{\pm 1\}, \cdot)
		.\] 
		Questo ci permetterà di definire  il sottogruppo  alterno $A_n\trianglelefteq S_n$\\
		$A_n:=ker(sgn)$
		\newpage
	\begin{nota}
		Dato un polinomio\\
		$f\in \mathbb Q[x_1,\ldots, x_n]$\\
		e data $\sigma\in S_n$\\
		Definiamo
		 \[
			 f^\sigma (x_1,\ldots,x_n) := f(x_{\sigma(1),\ldots,x_{\sigma(n)\}
		.\]
		Ci sta un polinomio speciale:\\
		\begin{itemize}
			\item $\Delta (x_1,\ldots, x_n) = \prod_{1\leq i< j\leq n}(x_i - x_j)$
				\item $\Delta^\sigma (x_1,\ldots, x_n) = \prod_{1\leq i< j\leq n} (x_{\sigma (i)} - x_{\sigma (j)})$
		\end{itemize}
	\end{nota}
	\begin{defi}
		$\sigma\in S_n$\\
		$sgn(\sigma) := \frac {\Delta^\sigma}\Delta\in\{\pm 1\}$\\
	\end{defi}
		 \textbf{Osservazione}\\
		 $sgn: S_n \rightarrow \{\pm 1\}$ \\
		 è un omomorfismo\\
		 \begin{dimo}
		 	In generale\\
			$(f^\sigma)^\tau = f^{\sigma\tau}$\\
			 $(fg)^\sigma = f^\sigma g^\sigma$\\
			 $\displaystyle sgn(\sigma\tau)=\frac {\Delta^{\delta\tau}}\Delta = \frac {((\Delta^\sigma)^\tau)} \Delta = \frac {\Delta^\sigma}{\Delta} \frac {(\Delta^\sigma)^\tau}{\Delta^\sigma} = sgn(\sigma)\frac{\Delta^\tau}\Delta = sgn(\delta)sgn(\tau)$
		 \end{dimo}
		 % lezione 7
		 \subsection{Il sottogruppo alterno}

\begin{defi}
Sia $n \in \mathbb{Z}$ un intero positivo. Il \emph{sottogruppo alterno} $A_n \trianglelefteq S_n$ è definito da
\[
A_n := \{\sigma \in S_n \mid \operatorname{sgn}(\sigma) = 1 \}.
\]
Una permutazione $\sigma \in S_n$ si dice \emph{pari} se $\sigma \in A_n$ e si dice \emph{dispari} altrimenti.
\end{defi}
\textbf{Osservazione}\\
Dal momento che $\operatorname{sgn}: S_n \to \{\pm 1\}$ è un omomorfismo di gruppi per l'osservazione precedente,
abbiamo che $A_n = \ker(\operatorname{sgn}) \leq S_n$, ed è un sottogruppo normale (Esercizio passato).

\begin{prop}
Sia $n \geq 2$ un intero. Allora:
\begin{itemize}
    \item $[S_n : A_n] = 2$,
    \item $[H : A_n \cap H] = 2$ per ogni sottogruppo $H \leq S_n$ tale che $H \not\subseteq A_n$.
\end{itemize}
\end{prop}

\begin{dimo}
Chiaramente è sufficiente dimostrare la seconda parte dell'enunciato. Sia dunque $H \leq S_n$.
Due permutazioni $\sigma, \tau \in S_n$ sono congruenti modulo $A_n$ se e solo se $\sigma^{-1}\tau \in A_n$,
ovvero se e solo se
\[
\operatorname{sgn}(\sigma)\operatorname{sgn}(\tau) = 1,
\]
dove abbiamo sfruttato l'osservazione anche per dedurre $\operatorname{sgn}(\sigma^{-1}) = \operatorname{sgn}(\sigma)$.
Pertanto esistono solo due laterali sinistri dati da
\[
H \cap A_n \quad \text{e} \quad H \setminus (H \cap A_n) = \{\sigma \in S_n \mid \operatorname{sgn}(\sigma) = -1\}.
\]
\end{dimo}
\textbf{Esercizio 7.4} (Sottogruppi di $A_4$).\\ Consideriamo il sottogruppo alterno $A_4 \trianglelefteq S_4$.
\begin{enumerate}
    \item Determinare tutti gli elementi di $A_4$.
    \item Dimostrare che il sottoinsieme $V := \{\operatorname{id}, (12)(34), (13)(24), (14)(23)\} \subseteq A_4$ è un sottogruppo di $A_4$ isomorfo al gruppo di Klein $K_4$.
    \item Dimostrare che $A_4$ non contiene sottogruppi di ordine 6.
\end{enumerate}
\textbf{Soluzione.} Procediamo per passi.
\begin{enumerate}
    \item[1.] Dalla Proposizione 7.3 segue che $\lvert A_4 \rvert = 12$ poiché $\lvert S_4 \rvert = 4! = 24$. I suoi elementi sono:
    \begin{itemize}
        \item ordine 1: $\operatorname{id}$,
        \item ordine 2: $(12)(34), (13)(24), (14)(23)$,
        \item ordine 3: $(123), (132), (124), (142), (134), (143), (234), (243)$.
    \end{itemize}
    Il fatto che gli otto 3-cicli siano elementi di $A_4$ segue dall'Esercizio 6.21.

    \item[2.] Osserviamo che $V$ è chiuso rispetto all'operazione poiché
	    \begin{gather*}
    (12)(34) \cdot (13)(24) = (14)(23), \\
    (13)(24) \cdot (14)(23) = (12)(34),\\
    (14)(23) \cdot (12)(34) = (13)(24).
    \end{gather*}
    Dunque $V$ è un sottogruppo. Inoltre, gli elementi di $V$ hanno tutti ordine 1 o 2. Dalla classificazione dei gruppi di ordine 4 si ha dunque $V \cong K_4$ (si veda l'Esercizio 6.4).

    \item[3.] Dal momento che $A_4$ non contiene elementi di ordine 6 non può contenere sottogruppi isomorfi a $C_6$. Dunque un sottogruppo $H \leq A_4$ di ordine 6 è necessariamente isomorfo a $D_3$ (si veda Teorema 6.7); pertanto $H$ contiene 3 elementi di ordine 2 e due elementi di ordine 3. Ne segue che $V \subset H$, da cui l'assurdo per il Teorema 2.14 di Lagrange.
\end{enumerate}
\textbf{Esercizio 7.5} (Sottogruppi di $S_4$). Consideriamo il gruppo simmetrico $S_4$.
\begin{enumerate}
    \item[1.] Determinare il numero di sottogruppi di $S_4$ di ordine 2 e di ordine 3.
    \item[2.] Determinare tutti i sottogruppi di $S_4$ non ciclici e di ordine 4.
    \item[3.] Determinare tutti i sottogruppi di $S_4$ di ordine 6.
    \item[4.] Determinare tutti i sottogruppi di $S_4$ di ordine 8.
    \item[5.] Determinare tutti i sottogruppi di $S_4$ di ordine 12.
\end{enumerate}
\textbf{Soluzione.} Procediamo per punti.
\begin{enumerate}
	\item[1.] Ogni sottogruppo di ordine 2 è generato da un elemento di ordine 2; pertanto è sufficiente contare gli elementi di ordine 2 in $S_4$. Abbiamo: $ {4 \choose 2} = 6$ trasposizioni e i 3 elementi non banali del sottogruppo $V \subseteq A_4$ (si veda l'Esercizio 7.4). In totale esistono dunque 9 sottogruppi di ordine 2 in $S_4$.

	Ogni sottogruppo di ordine 3 è generato da un elemento di ordine 3; pertanto è sufficiente contare gli elementi di ordine 3 in $S_4$, ovvero i $3$-cicli. Questi sono $2 \cdot {4 \choose 3} = 8$, ed esistono 8 sottogruppi di ordine 3 in $S_4$.

\item[2.] Sia $H \leq S_4$ un sottogruppo non ciclico tale che $\lvert H \rvert = 4$. Ora, se $H \leq A_4$ abbiamo necessario che $H = V$, poiché $V$ contiene tutti gli elementi di $A_4$ di ordine divide 4. Se invece $H \not\leq A_4$, allora abbiamo $\lvert H \cap A_4 \rvert = 2$ per la Proposizione 7.3. Ne deduciamo che $H$ contiene un solo prodotto di trasposizioni disgiunte $(i_1 i_2)(i_3 i_4)$ con $\{i_1,i_2,i_3,i_4\} = \{1,2,3,4\}$ Ne segue necessariamente che \[H = \{ \operatorname{id}, (i_1 i_2), (i_3 i_4), (i_1 i_2)(i_3 i_4)\}.\]
	poichè il prodotto di un'altra traposizione con l'elemento $(i_1i_2)(i_3i_4)$ fornisce come risultato un $4$-ciclo. Concludiamo che i sottogruppi non ciclici di oridne $4$ di $S_4$ sono
	\[
		H_1 = \{id, (12),(34),(12)(34)\}\quad H_2 = \{id, (13),(24),(13)(24)\}
	.\] 
	\[
	H_3 = \{id, (14),(23),(14)(23)\}\quad V = \{id, (12)(34),(13)(24),(14)(23)\}
	.\] 

    \item[3.] Dal momento che $S_4$ non contiene elementi di ordine 6, non può contenere sottogruppi isomorfi a $C_6$. Dunque un sottogruppo $H \leq S_4$ di ordine 6 è necessariamente isomorfo a $D_3$ (si veda il Teorema 6.7); pertanto $H$ contiene 3 elementi di ordine 2 e 2 elementi di ordine di 3, ovvero $H$ contiene necessariamente due $3$-cicli che saranno uno l’inverso dell’altro.
Ora, ricordiamo che dall’Esercizio 7.4 segue che $A_4$ non contiene sottogruppi di ordine 6, dunque 
$\lvert H \cap A_4 \rvert = 3$ per la Proposizione 7.3. Ne deduciamo che i restanti tre elementi di ordine 2 in $H$ devono necessariamente avere segno $-1$, dunque sono trasposizioni.

Si noti infine che la scelta delle trasposizioni da inserire nel sottogruppo $H$ è univocamente determinata dai 3-cicli contenuti in $H$. Infatti, se $(i_1 i_2 i_3) \in H$ allora il prodotto 
\[
	(i_1i_4)(i_1 i_2 i_3) = (i_1 i_2 i_3 i_4)
\]
fornisce un 4-ciclo in $H$, contraddicendo il Corollario 2.15. Ne segue che se $(i_1 i_2 i_3) \in H$, allora
\[
H = \{\operatorname{id}, (i_1 i_2), (i_1 i_3), (i_2 i_3), (i_1 i_2 i_3), (i_1 i_3 i_2)\}.
\]

I sottogruppi di ordine 6 in $S_4$ sono allora:
\[
H_1 = \{\operatorname{id}, (12), (13), (23), (123), (132)\}, \quad
H_2 = \{\operatorname{id}, (12), (14), (24), (124), (142)\},
\]
\[
H_3 = \{\operatorname{id}, (13), (14), (34), (134), (143)\}, \quad
H_4 = \{\operatorname{id}, (23), (24), (34), (234), (243)\}.
\]

    \item[4.] Sia $H \leq S_4$ un sottogruppo di ordine 8. Dal momento che $8 \nmid 12$, dalla Proposizione 7.3 deduciamo che $\lvert H \cap A_4 \rvert = 4$. Dall’Esercizio 7.4 segue dunque che $V = H \cap A_4$, poiché i 3-cicli contenuti in $A_4$ hanno ordine 3 e non possono dunque appartenere ad $H$ per il Corollario 2.15.

    Supponiamo ora che $H$ contenga una trasposizione $(i_1 i_2) \in H$. Abbiamo
    \[
    (i_1 i_2)(i_1 i_2)(i_3 i_4) = (i_3 i_4) \in H,
    \]
    e
    \[
    (i_1 i_2)(i_1 i_3)(i_2 i_4) = (i_1i_3i_2i_4) \in H,
    \]
    da cui si deduce che 
    \[
    H = V \cup \{(i_1 i_2), (i_3 i_4), (i_1 i_3 i_2 i_4), (i_1 i_4 i_2 i_3)\}.
    \]

    D’altra parte, assumendo che $H$ contiene un 4-ciclo, si deduce facilmente che $H$ contiene una trasposizione e dunque $H$ è ancora della forma precedente. Concludiamo che i sottogruppi di ordine 8 di $S_4$ sono:
    \[
    H_1 = \{\operatorname{id}, (12)(34), (13)(24), (14)(23), (12), (34), (1324), (1423)\},
    \]
    \[
    H_2 = \{\operatorname{id}, (12)(34), (13)(24), (14)(23), (13), (24), (1234), (1432)\},
    \]
    \[
    H_3 = \{\operatorname{id}, (12)(34), (13)(24), (14)(23), (14), (23), (1234), (1342)\}.
    \]

    \item[5.] Sia $H \leq S_4$ un sottogruppo di ordine 12. Dalla Proposizione 7.3 deduciamo che 
    \[
    \lvert H \cap A_4 \rvert = 6 \quad \text{oppure} \quad \lvert H \cap A_4 \rvert = 12.
    \]
    Dall’Esercizio 7.4 segue dunque che $H = A_4$.
\end{enumerate}


		 %lezione 8
		 \newpage
	\subsection{Prodotto diretto tra gruppi}
	\begin{defi}
		Siano $(G_1,\cdot)$, $(G_2, *)$ gruppi il loro prodotto diretto risulta l'insieme $(G_1\times G_2)$ dotato dell'operazione:
		\[
			(g_1,g_2)\cdot(f_1,f_2) = (g_1\cdot f_1, g_2 * f_2) \ \ \forall g_1,f_1\in G_1, \ \ \forall g_2,f_2\in G_2
		.\]
		e lo indichiamo con $(G_1\times G_2)$
	\end{defi}
	\begin{prop}
		$(G_1\times G_2, \cdot)$ è un gruppo
	\end{prop}
	\begin{dimo}
		L'associatività segue da quella di $\cdot$ e  $*$ l'elemento neutro è  $(e_1,e_2)$\\
		l'inverso di $(g,f)$ con  $g\in G_1$ e $f\in G_2$ risulta $(g^{-1},f^{-1})$
	\end{dimo}
	\textbf{Esercizio}\\
	$(G_1, \cdot)$ e  $(G_2,*)$ gruppi\\
	Dimostrare:
	1) $|G_1\times G_2| = |G_1||G_2|$\\
	2) $G_1\times G_2$ è abeliano se e solo se $G_1$ e $G_2$ sono entrambi abeliani\\
	3) Dati due sottogruppi $H\leq G_1$ e $K\leq G_2 \Rightarrow H\times K\leq G_1\times G_2$ \\
	4) Dati $H\trianglelefteq G_1$ e $K\trianglelefteq G_2 \Rightarrow H\times K \trianglelefteq G_1\times G_2$ \\
	5) Dati  $H\trianglelefteq G_1$ e $K\trianglelefteq G_2$\\
	\[
		G_1/H \times G_2/H \cong \bigslant{G_1\times G_2}{H\times K}
	.\] 
	\begin{dimo}[4,5]
	\[
\begin{tikzcd}
G_1 \times G_2 \arrow[r, " \varphi"] \arrow[d] & \frac{G_1}{H} \times \frac{G_2}{K} \\
\frac{(G_1 \times G_2)}{ker \varphi} \arrow[ur, dotted, "\exists ! \bar \varphi"]
\end{tikzcd}
\]

dove 
\[
\varphi(g_1, g_2) = \left( g_1 H, g_2 K \right)
\]
	Dal primo teorema di isomorfismo
	\[
		Im \varphi \cong \frac {G_1\times G_2}{ker \varphi}
	.\] 
	$\cdot \varphi$ suriettiva poichè $\pi_H e \pi_K$ sono suriettive\\
	\begin{aligned*}
	$\cdot$ $\ ker \varphi = \{(g_1,g_2)\in G_1\times G_2| \varphi(g_1,g_2) = (H,K)\} \\= \{(g_1,g_2) | g_1H = H \ e \ g_2K = K\}$ \\
	$\{(g_1,g_2) | g_1\in H,g_2\in K\} = H\times K$
	\end{aligned*}\\
	quindi $H\times K\trianglelefteq G_1\times G_2$
	\[
		\frac{G_1\times G_2}{H\times K}\cong G_1/H\times G_2/K
	.\] 
	\end{dimo}
	\textbf{Esercizio (importante)}\\
	$(G_1,\cdot)$ e $(G_2,*)$ gruppi\\
	$H,K\normale G_1\times G_2$ tali che $H\cap K = \{\tilde e\}$ dove  $\tilde e  = (e_1,e_2)$\\
	Dimostrare che ogni elemento di $H$ commuta con ogni elemento di K.\\
	\textbf{Dimsotrazione}\\Consideriamo $h\in H, k\in K$ e verifichiamo che  $hk = kh$\\
	 \textbf{Idea:}\\
	 Dimostrare che $hkh^{-1}k^{-1} = e$\\
	 Data l'ipotesi  $H\cap K = \{e\}$ è sufficiente dimostrare che  $hkh^{-1}k^{-1}\in H\cap K$
	  \textbf{Sfruttare la normalità di H e K}\\
	  \textbf{Esercizio}\\
	  $(G_1,\cdot)$, $(G_2,*)$ gruppi
	  \[
		  H:= G_1\times \{e_2\} = \{(g,e_2)|g\in G_1\}\leq G_1\times G_2\\
	  .\] 
	  \[
		  H:= e_1\times G_2\} = \{(e_1,g)|g\in G_2\}\leq G_1\times G_2\\
	  .\] 
	  Verificare che $H$ e $K$ soddisfano le ipotesi dell'esercizio precedente
	  \begin{defi}
	  	$(G,\cdot)$ gruppo $H,K\leq G$\\
		Diremo che  $G$ è \\
		Prodotto diretto interno di $H$ e $K$ se:\\
		1) $H,K\normale G$\\
		2)  $H\cap K = \{e\}$\\
		3)  $HK = G$
	  \end{defi}
	  \begin{teo}
	  	$(G,\cdot)$ gruppo\\
		1) Se $G$ è un prodotto diretto interno di $H,K\leq G$ allora  $G\cong H\times K$\\
		2) Se  $G\cong G_1\times G_2$ allora esistono $H,K\leq G$ tali che $G$ sia prodotto diretto interno di $H$ e $K$ e inoltre $H\cong G_1, K\cong G_2$
	  \end{teo}
	  \begin{dimo}[1]
	  	$\psi: H\times K \rightarrow G$\\
		$\ \ \ \ (h,k) \rightarrow hk$\\
		Dobbiamo verificare che $\psi$ sia isomorfismo\\
		1)$\psi$ è suriettiva perchè ogni elemento di $G$ si scrive come $hk$ quindi $Im(\psi) = G$\\
		2)È anche iniettiva infatti se  $\psi(g_1,k_1) = \psi(h_2,k_1)$
		\begin{gather*}
			\Rightarrow h_1k_1 = h_2k_2\\
			\Rightarrow h_2^{-1}h_1k_1 = k_2\\
			\Rightarrow h_2^{-1}h_1 = k_2k_1^{-1}\in H\cap K = \{e\}\\
			\Rightarrow \begin{cases}
				h_2^{-1}h_1 = e\\
				k_2k_1^{-1} = e
			\end{cases} \Rightarrow (h_1,k_1) = (h_2,k_2)\\
			\Rightarrow \psi \text{ iniettiva}
		\end{gather*}\\
		Bisogna in fine dimostrare che $\psi$ è un omomorfismo, ovvero che\\
		\[
		\psi(h_1h_2,k_1k_2) = \psi(h_1,k_1)\psi(h_2,k_2)
		.\] 
		dunque
		\[
		\psi(h_1h_2,k_1k_2) = h_1h_2k_1k_2 = h_1(h_2k_1)k_2 = h_1(k_1h_2)k_2 = \psi(h_1,k_1)\psi(h_2,k_2)
		.\] 
		Ricordando che tutti gli elementi di $H$ commutano con quelli di $K$
	  \end{dimo}
	  \begin{dimo}[2]
	  	Per ipotesi esiste un isomorfismo
		$ \varphi: G_1\times G_2 \rightarrow G$\\
		$\ \ \ (g_1,g_2) \rightarrow \varphi(g_1,g_2)$\\
		considero\\
		$H:= \varphi(G_1,\{e_2\})$\\
		$K:= \varphi(\{e_1\}\times G_2)$ \\
		Abbiamo visto che\\
		$\cdot G_1\times \{e_2\}\trianglelefteq G_1\times G_2 \rightarrow H\normale G$\\
		$\cdot \{e_1\}\times G_2\normale G_1\times G_2 \rightarrow K\normale G$ \\
		\[
			H\cap K = \varphi((G_1\times\{e_2\})\cap(\{e_1\}\times G_2)) = \{e\}
		.\] \\
		\[
			HK = \varphi((G_1\times \{e_2\})(\{e_1\}\times G_2)) = G
	.\]
	Le opportune restrizioni di $ \varphi$ forniscono gli isomorfismi
	\[
		H\cong G_1\times \{e_2\}\cong G_1
	.\] 
	\[
		K\cong \{e_1\}\times G_2\cong G_2
	.\] 

	  \end{dimo}
\textbf{Esempio:}\\
Siano $n,m\in\mathbb Z_{>0}$ t.c.\\
$MCD(n,m) = 1$\\
Consideriamo  $C_{nm} = <p>$\\
dove  $ord(p) = nm$\\
Considero
 \[
H = <\rho^m> \ \ \ K = <\rho^n>
.\] 
$|H| = ord(\rho^m) = n$\\
$|K| = ord(\rho^n) = m$\\
Verifichiamo che
 \[
	 C_{nm} \cong H\times K
.\] 
Dobbiamo mostrare:
\begin{enumerate}
	\item $H,K\normlae C_{nm}$\\
	\item $H\cap K = \{Id\}$\\
	\item $HK = C_{nm}$
	
\end{enumerate}\\
1) $C_{nm}$ abeliano, quindi $H,K\nomrale C_{nm}$\\
2) $H\cap K = ?$\\
sia $\rho^h\in H\cap K$\\
Allora
 \[
\begin{cases}
	\rho^h = (\rho^m)^{t_1}\\
	\rho^h = (\rho^h)^{t_2}
\end{cases} \ \ \ \ \ \begin{cases}
	m|h\\
	n|h
\end{cases}
.\] 
Ma $h\geq mcm(m,n) = mn \Rightarrow  h = mn \Rightarrow \rho^h = Id \Rightarrow H\cap K = \{Id\}$
\[
	|HK| = \frac{|H||K|}{|H\cap K|} = \frac {nm}1
.\] 
$ \Rightarrow HK$ è tutto chiuso quindi è $C_{nm}$\\
\begin{defi}[Automorfismo]
	$(G,\cdot)$ gruppo\\
	Un automorfismo di $G$ è un isomorfismo  $ \varphi:G \rightarrow G$
\end{defi}
\textbf{Osservazione}\\
$(G,\cdot)$ gruppo\\
$ \Rightarrow Aut(G) = \{\text{automorfismi di } G\}\\$
è un gruppo (rispetto alla composizione)\\
\textbf{Esempio:}\\
$(G,\cdot)$ gruppo\\
Fissato $g\in G$ definiamo\\
 \begin{aligned}
	 I_g: &G \rightarrow G\\
	      &f \rightarrow gfg^{-1}
\end{aligned}\\
$I_g$ si dice automorfismo interno\\
$Int(G) = \{\text{automorfismi interni di } G\}$ \\
\begin{prop}
	$Int(G)\normale Aut(G)$
\end{prop}
\begin{dimo}
	$If_G = I_e\in Int(G)$\\
	dato  $g\in G$ allora\\
	\[
		I_{g^{-1}} = I_g^{-1} \rightarrow \begin{cases}
			I_g\in Aut(G)\\
			Int(G) \text{ è chiuso rispetto agli inversi}
		\end{cases}
	.\] 
	$I_{g_2}\cdot I_{g_1}(f) = g_3g_2fg_2^{-1}g_2^{-1}
	= (g_2 g_1)f(g_2g_1)^{-1} = I_{g_2g_1}(f)$\\
	$I_{g_2}\cdot I_{g_1} = I_{g_2g_1}$\\
	quindi $Int(G)$ è chiuso rispetto alla composizione\\
	Quindi $Int(G)\leq Aut(G)$\\
	Basta verificare che:\\
	$ \varphi\circ Int(G)\circ \varphi^{-1}\subseteq Int(G)  \ \ \forall \varphi\in Aut(G)$\\
	ovvero dato $g\in G$\\
	 \[
		 \varphi\circ I_g\circ \varphi^{-1}\in Int(G)
	.\] 
	$\forall f\in G$\\
	\begin{aligend}
		 &\varphi\circ I_g \circ \varphi^{-1}(f) = \varphi(g \varphi^{-1}(f) g^{-1}) = \\
		 &\varphi(g) \varphi( \varphi^{-1}(f)) \varphi(g^{-1}) = \\
		 &= \varphi(g) f \varphi(g) =\\
		 & = I_{ \varphi(g)}(f)\\
		 &\Rightarrow \varphi\circ I_g \circ \varphi^{-1} = I_{ \varphi(g)}\in Int(G)
	\end{aligend}
\end{dimo}
\begin{defi}[Centro di un gruppo]
	$(G,\cdot)$ gruppo\\
	Il centro di $G$ è 
	\[
		Z(G):=\{g\in G| gf = fg \ \ \forall f\in G\}
	.\] 
\end{defi}
	\textbf{Osservazione}\\
	$Z(G)\normale G$\\
	 \textbf{Osservazione:}\\
	 $(G,\cdot)$ gruppo\\
	 Definiamo un omomorfismo\\
	 \begin{aligned}
		 \varphi: \ &G \rightarrow Int(G)\\
			& g \rightarrow I_g
	 \end{aligned}\\
		 \cdot $ \varphi$ è suriettiva\\
		 \cdot $ \varphi$ è omomorfismo\\
	 $ \varphi(g_2g_1)= \varphi(g_2) \varphi(g_1)$\\
	 $I_{g_2g_1} = I_{g_2}\circ I_{g_1}$\\
	 Chi è il $ker( \varphi)$\\
	 \begin{aligned}
		 ker( \varphi) &=  \{ g\in G | \varphi(g) = Id\} = \\
				 &= \{ g\in G | I_g = Id\} = \\
				 & = \{g\in G | \forall f\in G : I_g(f) = Id(f)\} = \\
				 & = \{g\in G | \forall f\in G : gfg^{-1} = f \} = Z (G)
	 \end{aligned}\\
	 Dal I teorema di isomorfismo si ha che
	 \[
	 Int(G) \cong G/Z(G)
	 .\] 
	 \subsection{Prodotto semidiretto}\\
	 Consideriamo due gruppi\\
	 $(N,\cdot)$ e $(H,*)$\\
	 Fissiamo un omomorfismo\\
	  \begin{aligned}
		  \phi: &H \rightarrow Aut(N)\\
			& h \rightarrow \o_n
	 \end{aligned}
	 \begin{defi}[Prodotto semidiretto]
	il prodotto semidiretto di $N$ e $H$ tramite $\o$ è l'insieme $N\times H$ dotato dell'operazione
	 \[
		 (n_1,h_1)\cdot (n_2,h_{2}) = (n_1\cdot \o_{h_1}(n_2), h_1*h_2)
	.\] 
	$\forall n_1,n_2\in N \ \ \ \ \forall h_1,h_2\in H$
\end{defi}
\begin{nota}
Indichiamo il prodotto semidiretto tra $N$ e $H$ con il simbolo $N\rtimes_{\o}$$H$
\end{nota}
\begin{prop}
	$N\rtimes_{\o} H$ è un gruppo
\end{prop}
\begin{dimo}
	Dato $(n,h)\in N\rtimes_{\o} H$\\
	l'inverso è dato da  $(\o_{h^{-1}}(n^{-1}),h^{-1})$
\end{dimo}
\newpage
\begin{defi}
	$(G,\cdot)$ gruppo\\
	$N,H\leq G$ Diremo che\\
	 $G$ è prodotto semidiretto interno di $N$ e $H$ se\\
	 \begin{itemize}[noitemsep]
		 \item $N\normale G$\\
		 \item $N\cap H = \{e\}$\\
		 \item $NH = G$
	 \end{itemize}
\end{defi}
\textbf{Esempio}\\
$D_n = <\rho, \sigma> \ \ N = <\rho> \normale D_n$\\
 $ H = <\sigma > \leq D_n$. Allora $D_n$ è prodotto semidiretto interno di $N$ e $H$\\
\textbf{Osservazione:}\\
$h_1\in H, \ \ \o_{h_1}\in Aut(N) \ \ \o_{h_1}(n_2)\in N$\\
\textbf{Esempio}\\
Scegliendo\\
\begin{aligned}
	\o : &H \rightarrow Aut(N)\\
	     & h \rightarrow \o_h
\end{aligned}\\
con $\o_n := Id_N \ \ \forall h\in H$\\
Abbiamo:
 \[
	 (n_1,h_1)\cdot (n_2,h_2) = (n_1\cdot n_2, h_1 * h_2)
.\] 
Quindi il prodotto diretto è un caso particolare del prodotto semidiretto

\subsection{Prodotto semidiretto interno:}
Un gruppo \( G \) si dice \textit{prodotto semidiretto interno} di \( N \) e \( H \leq G \) se:
\begin{enumerate}
    \item \( N \trianglelefteq G \),
    \item \( N \cap H = \{ e \} \),
    \item \( N H = G \).
\end{enumerate}
\textbf{Esercizio}\\
Sia $\o : H \rightarrow Aut(N)$ un omomorfismo\\
Dimostrare:\\
$1) |N\rtimes_{\o} H | = |N||H|$\\
$2) N\rtimes_{\o} H$ è abeliano $ \Leftrightarrow N,H$ abeliani \\
$3) \tilde H \leq H, \tilde N \leq N$ (sottogruppo caratteristico)\\
 \[
\tilde N \semi \tilde H := \{ (n,h)\in N\semi H | 
	n\in \tilde N,
	n\in \tilde H
 \}
.\] 
è un sottogruppo di $N \semi H$\\
\begin{defi}[Sottogruppo caratteristico]
	$\tilde N\leq N$ sottogruppo caratteristico se \\$ \varphi(n)\in \tilde N\ \ \forall n\in \rilde N\ \ \ \forall \varphi\in Aut(N)$
\end{defi}
\begin{teo}
Sia \( G \) un gruppo. \\1) Se \( G \) è prodotto semidiretto di \( N \) e \( H \leq G \), allora esiste un omomorfismo \( \o : H \to \text{Aut}(N) \) tale che $G\cong N\semi H$\\
2) Se  $G\cong \tilde N\semi \tilde H$ allora esistono $N,h\leq G$ t.c.
\begin{itemize}
	\setlength\itemsep{-1em}
	\item $G$ sia prodotto semidiretto interno di $N$ e $H$ \\
	\item $N\cong \tilde N, h\cong \tilde H$
\end{itemize}
\end{teo}
\begin{dimo}[1]
	Definiamo l'applicazione\\
	\begin{aligned}
		\o :& H \rightarrow Aut(N)\\
		    &	h \rightarrow \o_n
	\end{aligned}\\
	dove $\o_{h}(n) := (hnh^{-1})\in hNh^{-1} = N \ \ \forall n\in N$\\
	Dato che abbiamo assunto  $N$ normale\\
	Abbiamo verificato la volta scorsa che è un omomorfismo.\\
	Definiamo l'applicazione\\
	\begin{aligned}
		\psi: &N\semi H \rightarrow G\\
		      &(n,h) \rightarrow nh
	\end{aligned}\\
	$\psi$ è suriettiva poiché $N\cdot H = G$ \\
	$\psi$ è iniettiva poichè\\
	\begin{gather*}
		n_1h_1 = n_2h_2 \rightarrow n_2^{-1}h_1 = h_2h_1^{-1}\in H\cap N = \{e\}\\ 
		\Rightarrow \begin{cases}
			n_2^{-1} n_1 = e\\
			h_2h_1^{-1} = e
		\end{cases} \rightarrow (n_1,h_1) = (n_2,h_2)
	\end{gather*}
	\textbf{$\psi$ è omomorfismo:}\\
	\begin{aligned}
	&\psi((n_1,h_1)\cdot(n_2,h_2)) =\\
	&=\psi((n_1\o_{h_1}(n_2),h_1h_2))\\
	& =n_1\o_{h_1}(n_2)h_1h_2\\ 
	&= n_1h_1n_2h^{-1}_1h_1h_2 = \psi(n_1,h_1)\cdot\psi(n_2,h_2)
		
	\end{aligned}\\
	Omomorfismo biunivoco
\end{dimo}
\newpage
	\begin{dimo}[2]
		Dato un isomorfismo\\
		$\psi :\tilde N\semi\tilde H \rightarrow G $\\
		definiamo:\\
		$N := \psi (\tilde N\semi \{e_{\tilde H}\})\normale G$\\
		$H:=\psi(\{e_n\}\semi \tilde H)$\\
		Osserviamo che:\\
		$\cdot \tilde N \cong \tilde N\semi \{e_{\tilde H}\}\cong N$\\ $\cdot \tilde H\cong \{e_{\tilde H\}\semi \tilde H\cong H$\\
			$\cdot N\cap H = \{e\}$\\
			 $\cdot NH = e$ \\(analogo alla dimostrazione per prodtto diretto)
	\end{dimo}
	\section{Numeri primi e aritmetica}
	\begin{defi}[Numero primo]
		Un intero $\rho > 1$ si dice primo se  $\forall a,b\mathbb Z$
		 \[
			 \rho|ab \rightarrow \rho|a \text{ oppure } \rho|b
		.\] 
	\end{defi}
	\begin{defi}[Numero irriducibile]
		Un intero $\rho>1$ si dice irriducibile se i suoi unici divisori positivi\\ sono  $1$ e $\rho$
	\end{defi}
	\textbf{Esercizio:}\\
	Dimostrare che $\rho$ è primo $\Leftrightarrow$ è irriducibile
	\begin{teo}[Fondamentale dell'aritmetica]
		$n>1$ intero. Allora $n$ si scrive in modo unico come
		\[
			n = \rho_1^{k_1}\cdot\ldots\cdot \rho_r^{k_r} \ \ \ \ \ \text{ (forma canonica)}
		\] 
		dove $k_i>0 \ \ \forall i\in\{1,\ldots,r\}$\\
		e $\rho_1<\rho_2<\ldots<\rho_r$\\
		e $\rho_i$ è primo  $\forall i\in \{1,\ldots,r\}$
	\end{teo}
	\begin{teo}
		$\rho$ primo. Allora\\
		$\sqrt \rho $ è irrazionale (ovvero $\sqrt\rho\ni \mathbb Q$)
	\end{teo}
	\begin{dimo}[Per assurdo]
		$\exists a, b\in \mathbb Z$ t.c.  $\sqrt\rho = \frac a b$ con $MCD(a,b) = 1$\\
		Allora:\\
		 $(a) + (b) = (MCD(a,b)) = (1)$\\
		  $\rightarrow 1\in (a)) + (b)\\
		  \exists r,s,\in\mathbb,$ t.c. $1 = ra + sb$ (identità di Bezout)\\
		  ora:  $ \begin{cases}
		  	a = \sqrt \rho b \\ b\rho = a\sqrt\rho
		  \end{cases}$ \\
		  Quindi:
		  $\sqrt \rho = \rho\cdot 1 = \sqrt \rho\cdot (ra + sb)\\
		  (\sqrt\rho a)r + (\sqrt\rho b)s$\\
		  =  $\rho b r + as\in \mathbb Z$\\
		   $ \Rightarrow \sqrt\rho\in\mathbb Z$ quindi $\sqrt\rho$ è un intero che divide $\rho$ e $1<\sqrt\rho<\rho$
	\end{dimo}
	\begin{teo}[Euclide]
		Esistono infiniti numeri primi
	\end{teo}
	\begin{dimo}
		Supponiamo per assurdo che $\exists$ un numero finito di primi $\rho_1,\ldots, \rho_r$\\
		Definiamo:
		$N:= (\rho_1 \cdot\ldots\cdot\rho_r) + 1 > 1$\\
		$ \Rightarrow\exists \rho_k$ primo tale che $\rho_k | N$\\
		 $ \Rightarrow \begin{cases}
		 	\rho_k|N\\
			\rho_k|N-1
		 \end{cases} \Rightarrow \rho_k|N-(N-1) \Rightarrow \rho_k | 1$, assurdo
	\end{dimo}
	\begin{defi}[Primi di Euclide]
		Sia $\rho$ primo
		\[
			\rho^\# := \left(\prod_{q\in\rho, q \text{ primo}} q \right) + 1
		.\] 
		$\rho^\#+1$ si dice numero di Euclide
	\end{defi}
	\textbf{Congettura}\\
	Esistono infiniti primi di euclide

\subsection{Svolgimento esercizi}
	\textbf{Ossercazione:}\\
	Quali sono gli elementi di oridne $21$ in $S_{13}$?\\
	Ricordo che in $S_4$, gli elementi $(12)(34), (13)(24), (14)(23)$ hanno ordine 2\\
	gli elementi di ordine 21 sono $(3-ciclo)(7-ciclo)$ sono $\frac {13!}{126}$\\
	$(3-ciclo)(3-ciclo)(7-ciclo)$ sono  $\frac{13!}{126}$ \\
	Nelle note del corso trovi soluzioni degli esercizi\\
	%TODO aggiugni la roba
	\textbf{Esercizi:}\\
	1) $(\Z, +)\ \ \ Aut(\Z) = ?$ \\
	\textbf{Osservazione}\\
	Per definire un omomorfismo è sufficiente definirlo sui generatori.\\
	Se inoltre vogliamo un automorfismo $ \phi(1)$ deve generare $\Z \Rightarrow  \phi(1) = 1 $ o $-1$\\
	$ \Rightarrow Aut(\Z) = \{Id, -Id\}\cong C_2$ \\
	2) Dimostrare che $D_n\cong C_n\semi C_2$ dovea\\  \begin{aligned}
		\phi: &C_2 \rightarrow Aut(C_n)  = <\rho> \ \ \text{ e } \phi(\rho) = \rho^{-1}\\
		      &\sigma \rightarrow\phi_\sigma
	\end{aligned}
SOluzione:\\
$N = <\rho> \normale D_n \hfill [D_n:N] = 2$\\
 $H:= <\sigma>\leq D_n$\\
 Verifichiamo che  $D_n$ è prodotto semidiretto interno di $N$ e $H$\\
 $\cdot N\cap H = \{Id\}$\\
 $\cdot |NH| = \frac {|N||H|}{|N\cap H|} = 2n \Rightarrow  NH =  D_n$ \\
 Ora dal teorema segue che $D_n = N\semi H$\\
 dove  \begin{aligned}
	 \phi: &H \rightarrow Aut(N)\hfill H = \{Id,\sigma\}\\ 
	       &h \rightarrow h
 \end{aligned}\\
 $\cdot\phi_{Id} = Id_N$\\
 $\cdot_\sigma(\rho) = \sigma\rho\sigma^{-1} = \sigma\rho\sigma = \sigma\sigma\rho^{n-1} = p^{n-1} = p^{-1}$\\
 dove abbiamo usato il fatto che $\rho^i\sigma = \sigma\rho^{n-i}$\\
 Infine  $H\cong C_2 \ \ N\cong C_n$\\
 \textbf{Osservazione}\\
 Se avessimo scelto  \begin{aligned}
	 \phi: &C_2 \rightarrow Aut(C_n)\\
	       &\sigma \rightarrow \phi_\sigma
 \end{aligned}\\
 con $\phi_\sigma (\rho) = \rho$ Avremmo  $C_n\semi C_2= C_n\times C_2$ è abeliano $ \Rightarrow $ non isomorfo a $D_n \ \ \forall n\geq 3$\\
  $3) G = C_5\cong \Z/(5) \ \\ \ Aut(C_5)$\\
  Cerchiamo le immagini di $ \varphi(\rho)$\\
  $\Z/(5) = \{[0],[1],[2],[3],[4]\}$ ricordo che  $ord_{\Z/(n)}([a]) = \frac {n}{MCD(a,n)}$\\
  $ ord([1]) = ord([2]) = ord([3]) = ord([4])$\\
  $ord([0]) = 1 \Rightarrow |Aut(\Z/(5))| = 4$ \\
  \textbf{Osservazione} \\
  In generale denotiamo con $U_n$ il gruppo delle classi in  $\Z/(n) \ \ U_n = \{[a]\in\Z/(n) | MCD(a,n) = 1\}$\\
  \textbf{Esercizio}\\
  $U_n$ è un gruppo rispetto \begin{aligned}
	  U_n&\times U_n \rightarrow U_n\\
	  ([a&],[b]) \rightarrow[a\cdot b]
  \end{aligned}\\
  Si dice gruppo degli invertibili\\
  \textbf{Esercizi}|\
  $\cdot Aut(C_n) \cong U_n$\\
   $\cdot Aut(K_4) = S_3$\\
   \begin{teo}[Cinese del resto]
	   $C_{nm}\cong C_n\times C_m$ per ogni coppia di interi tale che  $MCD(m,n) = 1$
\end{teo}
\begin{dimo}
	Già dimostrato
\end{dimo}
\begin{teo}[Piccolo teorema di Fermat]
	$p$ primo, $a\geq 1$  $MCD(a,p) = 1 \Rightarrow a^{p-1}\equiv_p 1$
\end{teo}
\begin{dimo}
	$A:=\{a,2a,3a,\ldots,(p-1)a\}$ sono $p-1$ interi.\\
	Mi chiedo se $[ra] = [sa]$ in $\Z/(p)$\\
	Sappiamo che esistono  $1\leq r < s \leq p-1$ tali che  $[ra] = [sa]$ in $\Z/(p) \Rightarrow $ Assurdo poiché \\
	$[r]\neq [s]$ in $\Z/(p)$\\
	Quindi le classi  definite dagli elementi di $A$ sono tutte distinte e non banali $ \Rightarrow $ $\{[a],[2a], \ldots, [(p-1)a]\} = \{[1],[2],\ldots,[p-1]\}$\\
	Consideriamo il prodotto  $a\cdot 2a\cdot\ldots\cdot (p-1)a\eqiuv_p 1\cdot 2 \cdot 3 \cdot \ldots \cdot (p-1)$\\
	$ \Rightarrow a^{p-1}(p-1)!\equiv_p (p-1)$ Dato che $MCD(p,(p_1)) = 1$ abbiamo $a^{p-1}\equiv_p 1$
\end{dimo}
	 \begin{coro}
		$A\geq 1 $ p primo $ \Rightarrow ap \equiv_p a$ \\
	\end{coro}
	\begin{dimo}
		Se $MCD(a,p) = 1 $ segue dal piccolo teorema di Fermat\\
		$\cdot$ Se  $MCD(a,p)\neq 1 \ \Rightarrow  \ p|a \Rightarrow [a] = [0] $  in $\Z/(p) \Rightarrow a^p\equiv_p 0\equiv_p a$ \\
	\end{dimo}
	\textbf{Obbiettivo}\\
	Cosa succede al piccolo teorema di Fermat senza $p$ primo?
	\begin{defi}[Funzione di Eulero]
		$n\geq 1 \Rightarrow \phi(n) = |U_n|$ ovvero $\phi(n)$ è il numero di interi positivi minori o uguali ad  $n$ coprimi con $n$
	\end{defi}
	\textbf{Esempio}\\
	$p$ primo $ \Rightarrow $  $\phi(p) = p-1$;  $\phi (1) = 1$\\
	 \textbf{Esercizio}\\
	 Mostrare che se $ p $ è primo $ \Rightarrow  \phi (p^k) = p^k - p^{k-1} = p^{k-1}(p-1)$ \\
	 \textbf{Soluzione:}\\
	 $MCD(a,p) = 1 \Leftrightarrow p | a$ Quindi gli elementi inclusi sono $p,2p,3p,\ldots, (p^{k-1})p $ tutti i multipli di $p \leq p^k$\\
	 Sono  $ p^{k-1}$ elementi $ \Rightarrow \phi (p^k) = p^k - p^{k-1}$ \\
	 \newpage
	 \begin{defi}
		 Una funzione $f:\Z_{>0} \rightarrow\Z$ si dice moltiplicativa se $f(n\cdot m) = f(n)\cdot f(m)$ se $MCD(n,m) = 1$
	 \end{defi}
	 \textbf{Obiettivo}\\
	 Dimostrare che $\phi$ è moltiplicativa\\
	 \textbf{Esercizio}\\
	 $a,b,c\in \Z$\\
	  $MCD(a,b,c) = 1 \Leftrightarrow \begin{cases}
	  	MCD(a,b) = 1\\
		MCD(a,c) = 1
	  \end{cases}$
	  \begin{teo}[Eulero]
		  $\phi : \Z_{>0} \rightarrow \Z$ di Eulero è moltiplicativa
	  \end{teo}
	  \begin{dimo}
	  	Per il teorema cinese del resto se $MCD(n,m) = 1 \Rightarrow  $\\ \begin{aligned}
			$\psi : &\Z/(nm) \rightarrow\Z(n)\times \Z(m)\\
				&[a]_{nm} \rightarrow([a]_n,[a]_m)$
	  	\end{aligned}
		Consideriamo la restrizione $\psi|_{U_{nm}}:U_{nm} \rightarrow Z/(n)\times \Z/(m)$ è una funzione iniettiva\\
		Studiamo $Im(\psi|_{U_{nm}}) = U_n\times U_m$\hfill (esercizio)\\
		$ \Rightarrow |U_{nm}| = |U_n|\times |U_m| \Rightarrow \phi(nm) = \phi(n)\phi(m)$
	  \end{dimo}
	  \textbf{Osservazione} (Formula generale)\\
	  	$n>1$ intero con fattorizzazione canonica
		\[
			n = p_1^{k_1}\cdot\ldots\cdot p_r^{k_r} \Rightarrow \phi(n) = \phi(p_1^{k_1})\cdot\ldots\cdot\phi(p_r^{k_r})
		\]
	\[
		= (p_1^k_1- p_1^{k_1-1})\cdot\ldots\cdot(p_r^{k_r}-p_r^{k_r-1}) = \left(1-\frac {1}{p_1^{k_1}} \right) \cdot\ldots\cdot \left(1 - \frac {1}{p^{k_r}_r} \right)
		.\] 
		\textbf{Osservazione}\\
		$|Aut(C_n)| = \phi(n)$ dato che  $Aut(\Z/(n))\cong U_n$\\
		 \textbf{Osservazione}\\
		 Se $n\geq 2 \Rightarrow \phi(n)$ è pari\\
		 \begin{itemize}
			 \item Se $ n= 2^k \Rightarrow \phi(n) = 2^k - 2^{k-1} = 2^{k-1}$
			 \item $n > 2 \Rightarrow k > 1 \Rightarrow k-1 > 0 \Rightarrow 2 | \phi(n)$ 
			 \item Se $n\neq 2^k \Rightarrow \exists p$ primo e dispari tale che $p | n \Rightarrow n = p_1^{k_1}\cdot\ldots\cdot p_r^{k_r}$  con $p = p_j$ per qualche  $j\leq r \Rightarrow \phi(p_j^{k_j})|\phi(n)$. Ma $\phi(p_j^{k_j})= p^{k_j}-p^{k_j-1} = p^{k_j-1}(p-1)$\\
				 $ \Rightarrow (p_j - 1_ | \phi (p_j^{k_j}). $Ma $p_j$ è dispari $ \Rightarrow 2 | (p_j-1) \Rightarrow 2 | \phi(n)$
		 	
		 \end{itemize}
		 \begin{teo}[Waring 1770, Lagrange 1771]
		 	Se $p$ è un numero primo $  \Rightarrow (p-1)! \equiv_p (p-1)$\\
		 \end{teo}
\begin{dimo}
	Studiare le soluzioni di $x^2 - 1 \equiv_p 0$\\
	$(x^2-1)\equiv_p (x-1)(x+1)\equiv_p 0$\\
	Quindi $x-1\equiv_p 0$ opprure  $x + 1\equiv_p 0$\\
	ovvero deduciamo che gli unici elemnti in  $U_p$ di ordine $\leq 2 $ sono $[1]$ e $[p-1]$\\
	Nel prodotto  $[(p-1)!]$ compaiono tutti gli elementi di $U_p \Rightarrow $   ogni elemento di $U_p$ diverso da  $1$ e $p-1$ oppure con il suo inverso ("moltiplicativo")
\end{dimo}
\subsection{funzione di Eulero}



	%lezione successiv
	$\phi: \mathbb Z_{>0} \rightarrow \mathbb Z$\\
	$\text{} \ \ \ \ n \rightarrow |U_n|$\\
	\textbf{Ricordo:}\\
	$\phi(1) = 1$\\
	 $\phi(\rho) = \rho -1$\\
	 $\phi(\rho^k) = \rho^k - \rho ^{k-1}$\\
	 $\phi(n\cdot m) = \phi(n)\phi(m)$\ \ \ \ se  $MCD(n,m) = 1$\\
	  \begin{lemm}
	 	$n>1, a\in\Z$ t.c.  $MCD(n,a) = 1$\\
		sia  $\{a_1,\ldots,a_{\phi(n)}\}$ l'insieme dei numeri positivi minori di $n$ coprimi con n distinti fra loro.\\
		Allora $\{[a_1],\ldots,[a_{\phi(n)}] = \{[aa_1],\ldots, [aa_{\phi(n)}]\}$ (Classi in $Z/(n)$)
	 \end{lemm}
	 \begin{dimo}
		 Basta verificare che gli elementi delle classi $[aa_i] \ \ \forall \ 0<i<\phi(n)$\\
		 Siano tutte distinte tra loro e  $aa_i$ sia coprimo con n  $\forall \ 0<i<\phi(n)$ \\
		 Se per assurdo $[aa_i] = [aa_j] \ \ i\neq j \Rightarrow aa_i\equiv aa_j\  mod(n) \Rightarrow a\equiv a_j \ mod(n)$ Assurdo perché $1\leq a_i, a_j< n$ per ipotesi e dunque  $a_i-a_j\not\in (n)$\\
		 \begin{cases}
		  $MCD(a,n) = 1 \\ MCD(a_i,n) = 1$
		 \end{cases} $\Rightarrow MCD(a,a_i) = 1\\$
	 \end{dimo}
	 \begin{teo}[Eulero 1760]
	 	$n>1 ,a\in\Z$ tale che $MCD(a,n) = 1$\\
		Allora
		 \[
			 a^{\phi(n)}\equiv 1\  mod(n)
		.\] 
	 \end{teo}
	 \textbf{Nota}\\
	 Se $n$ è primo ritroviamo il piccolo teorema di Fermat
	 \begin{dimo}
	 	Considero la situazione del lemma:\\
		$A = \{a_1,\ldots,a_{\phi(n)}\}$\\
			Insieme degli interi positivi minori di n e coprimi con n distinti tra loro\\
			Dal lemma segue che
			\[
				a_1\cdot\ldots\cdot a_{\phi(n)}\equiv (aa_1)\cdot\ldots\cdot (aa_{\phi(n)}) \ mod(n)
			.\] 
			\[
				\equiv a^{\phi(n)}\cdot a_1\cdot\ldots\cdot a_{\phi(n)} \ mod(n)
			.\] 
			Dal momento che $MCD(a_i,n) = 1$\\
			abbiamo:  $1\equiv a^{\phi(n)}\ mod(n)$
	 \end{dimo}
	 \textbf{Esempio}\\
	 Se volessi calcolare le ultime $3$ cifre di $2024^{2025}$ Studiamo la congruenza  \[x\equiv 2024^{2025} \ mod(1000)\]
	 È equivalente al sistema (Teorema cinese del resto):\[
	 \begin{cases}
	 x\equiv 2024^{2025} \ mod(2^3)\\
	 x\equiv 2024^{2025} \ mod(5^3)
 \end{cases}\]
 Alternativamente mi accorgo che la prima equazione è equivalente a
 \[
	 x\equiv 24^{2025} \ mod(1000)
 .\] 
 $\phi(1000) = \phi(2^3)\phi(5^3) = (2^3 - 2^2)(5^3-5^2) = 400$\\
 $ \Rightarrow 24^{400} \equiv 1 mod(n)$\\
 Ma questo implica che la congruenza che devo studiare è:
 \[
 \Rightarrow x\equiv 24^{2025} mod(1000)
 .\] 
 \[
 \Rightarrow \begin{cases}
	 x\equiv 24^{2025}\ mod(8)\\
	 x\equiv 24^{2025}\ mod(125)
 \end{cases} \Rightarrow \begin{cases}
 	x\equiv 0\ mod(8)\\
	x\equiv 24^{2025} \ mod(125)
 \end{cases}
 .\] 
 Dove nell'ultimo passaggio abbiamo utilizzato il fatto che $8|24$ e $24^{\phi(125)}\equiv 24^{100}\equiv 1\ mod(125)$\\
 Alla fine dovremmo ricostruire la soluzione in $\Z/(1000)$ che sarà unica per il teorema cinese del resto\newpage
 \subsection{Teorema cinese del resto}
 \textbf{Problema}\\
 Dato un sistema di congruenze\\
 \begin{cases}
 	x\equiv = a_1\ mod(n_1)\\
	\vdots\\
	x\equiv = a_r\ mod(n_r)\\
 \end{cases}\\
 con $MCD(n_i,n_j) = 1 \ \ \forall i\neq j$  \\
 Come ricostruire l'unica soluzione $[\bar x]\in \Z/(n_1\cdot\ldots\cdot n_r)$\\ $\bar x\equiv a_i\ mod(n_i)\ \forall i\in \{1,\ldots,r\}
  $\\
  \textbf{Idea}\\
  Definiamo:\\
  $n: = n_1\cdot n_r$\\
  $N_i := \frac n {n_i}$\\
  $\bar  x := a_1N_1^{\phi(n_1)} + \ldots + a_rN_r^{\phi(n_r)}$\\
  Ora $\bar x\equiv a_i N^{\phi(n)}\ mod(n) \Rightarrow \bar x = a_i mod(n_i) \ \ \forall i$ 
  \begin{teo}[TCR]
	Dato il sistema\\
	\begin{cases}
		x\equiv a_1 \ mod(n)\\
		\ldots\ome
		x\equiv a_r \ mod(n_r)\\
	\end{cases}\\
	con $MCD(n_i,n_j) = 1 \ \ \forall i\neq j$\\
	Allora esiste un'unica classe  $[\bar x]\in \Z/(n_1\cdot\ldots\cdot n_r)$ tale che\\
	$\bar x\equiv a_i\ mod(n_i) \ \ \forall i\in \{1,\ldots, r\}$\\
\end{teo}
	\begin{dimo}[Alternativa al teorema di Eulero]
		$n:= n_1\codt\ldots\cdot n_r$\\
		$N_i = \frac n {n_i}$\\
		$\bar x :=  a_1N_1m_1+\ldots a_rN_rm_r\\$
		dove gli $m_i$ sono univocamente determinati dalla condizione
		$N_im_i\equiv 1 mod(n_i)$\\
		Infatti  \[
		\bar x\equiv a_iN_im_i \ mod(n_i) \Rightarrow \bar x\equiv a_i mod(n_i)
		.\] 
		Osserviamo che $MCD(N_i,n_i) = 1$ Per ipotesi\\
		Quindi $[N_i]\in U_{n_i}$ e  $[m_i]$ è l'unico inverso di $[N_i]$ in  $U_{n_i}$
	\end{dimo}
	\textbf{Osservazione}\\
	Per risolvere i sistemi di congruenze "basta" saper trovare gli inversi degli elementi in gruppi $U_{n_i}$\\
	 \textbf{Esercizi dalle schede}\\
	 \textbf{Esercizio (Gauss)}\\
	 Dato un intero $n > 1$ dimostrare che  $n = \sum_{d | n} \phi (d)$ (somma di tutti i divisori positivi di $n$
\begin{dimo}
	$S_d : = \{m\in \Z | MCD(m,n) = d, 1 \leq m\leq n\}$\\
	Osserviamo che \\
	$\{1,\ldots,n\} = \bigcup_dS_d$\\
	 $ \Rightarrow  n= \sum_{d | n} | S_d|$\\
 $MCD(m,n) = d \ \Leftrightarrow MCD(\frac md, \frac nd) = 1$\\
 Quindi $|S_d| = \phi(\frac nd)$\\
 $n = \sum_{d|n}|S_d| = \sum_{d|n}\phi(\frac nd) = \sum_{d|n}\phi(d)$\\
\end{dimo}
 \textbf{Esempio}\\
 $n = 15$\\
 Voglio ripetere la dimostrazione per ottenere  $15 = \sum_{d|15}\phi(d)$\\
 $S_1 = \{1,2, 4, 7, 8, 11, 13 , 14\} \Rightarrow \phi(15/1) = 8$\\
 $S_3 = \{3, 6, 9, 12\} \Rightarrow \phi(15/3) = 4$\\
 $S_5 = \{5, 10\$ \Rightarrow \phi(15/5) = 2$\\
	 $S_{15} = \{15\} \Rightarrow \phi ( 15/15) = 1$\\
\textbf{Esempio}\\
$n.1$ Allora la somma di tutti gli interi positivi minori di $n$ coprimi con $n$ vale $\frac 12n\phi(n)\in \Z$
\begin{dimo}
	Chiamiamo $a_1,\ldots, a_{\phi(n)}$ tali interi:\\
	Studio $\sum_{i=1}^{\phi(n)}a_i$\\
	Osserviamo che  $MCD(a,n)=1 \Leftrightarrow MCD(n-a_i, n) = 1$ \\
	Quindi\\
	$\{a_1,\ldots, a_{\phi(n)}\} = \{n - a_1,\ldots n - a_{\phi(n)}\}$\\
	$ \Rightarrow \sum^{\phi(n)}_{i=1}a_i = \sum^{\phi(n)}_{i=1}(n-a_i) = n\phi(n) - \sum^{\phi(n)}_{i=1}a_i \Rightarrow 2 \sum^{\phi(n)}_{i=1}a_i = n \phi(n)$
\end{dimo}
\subsection{Teorema di Wilson/Lagrange}
Ricordo
\begin{teo}[Wilson]
	$p$ primo. Allora\\
	$(p-1)! \equiv (p-1) \ mod(p)$
\end{teo}
\begin{teo}[Lagrange]
	$m > 1 $ intero tale che\\
	$(m-1)! \equiv (m-1) \ mod(m)$\\
	Allora $m$ è primo
\end{teo}
\begin{dimo}
	Per assurdo, se $m$ non è primo allora esiste un intero $d|m$ tale che   $1<d<m$\\
	Osserviamo che:\\
	 $d < m \Rightarrow d | (m-1)!$ \\
	 dall'ipotesi segue che
	  \[
	 m | (m-1)! + 1
	 .\] 
	 $ \Rightarrow d| (m + 1) ! + 1$ \\
	 Quindi 
	 \begin{cases}
	 	d | (m-1)!\\
		d|(m-1)!+ 1
	 \end{cases}
	 $=>d | 1$ che è un assurdo 
\end{dimo}
% Lezione 12 (TODO guarda se le lezioni sono tutte)
\subsection{Divisione Euclidea}
	\begin{teo}
		$a,b\in\Z$ con  $b\neq 0$ allora $\exists q,r\in \Z$ tale che \\
		$	\cdot a = qb+r\\
			\cdot 0\leq r< |b|$
	\end{teo}
	\begin{dimo}
		Procediamo per passi\\
		$1) a,b\in \Z_{>0}$
		 \[
			 A = \{k\in\Z | kb>a\}
		.\] 
		Osserviamo che $A\neq \emptyset$\\
		Infatti  $(a+1)b=ab+b>ab\geq a \Rightarrow a + \in A$ \\
		Per il principio del buon ordinamento di $\mathbb N$
		 \[
		 \Rightarrow \exists m := min\{k\inA\}\in\Z^+
		.\] 
		Definiamo 
		\[
		q:=m-1\in \Z^+
		.\] 
		$q\not\in A$ e $q+1\in A$\\
		 $qb\leq a < (q+1)b=qb + b$ \\
		 $ \Rightarrow 0\leq a - qb < b$ \\
		 Definiamo $r = a - qb$ e otteniamo:\\
		  $0\leq r < b\\
		  a = qb + r$\\
		  2)  $a\in\Z\ b > 0$\\
		  Se  $a\geq 0 \ ($ok per 1)\\
		  Se  $a < 0 \Rightarrow - a > 0$ \\
		  $ \Rightarrow -a = qb + r$ con $0\leq r< b$\\
		   $ \Rightarrow a = (-q)b - r$ \\
		   Se $r = 0$ abbiamo finito \\
		   Se invece $0 < r < b$\\
		   definiamo  $r' = b-r \Rightarrow 0 < r' < b$ \\
		   $a = (-q)b - b + \frac{b-r}{r'}$\\
		    $ \Rightarrow a = (-q-1)b + r' = q'b + r'$\\
		    3) $a\in\Z, b < 0$\\
		     $ \Rightarrow -b > 0\\
		     a = q(-b) + r$ con $0\leq r<-b$\\
		      $ \Rightarrow a = (-q)b + r \ \ 0\leq r < |b|$
	\end{dimo}
	\newpage
	\subsection{Esercizi delle schede}
	\begin{cases}
		
	x\equiv 50\ mod(110)\\
	x \equiv 47 mod(73)
	\end{cases}\\
	Dal teorema cinese del resto sappiamo che esiste un'unica soluzione modulo il prodotto $mod(110 * 73) = mod(8030)$\\
	Come lo costruisco?\\
	$\bar x = 50 \cdot 73 \cdot m_1 + 47 \cdot 110 \cdot m_2$\\
	L'idea è di infilare al posto di $m_1$ l'inverso di $73\ mod(110)$
	\[
	\begin{cases}
		73\cdot m_1\equiv 1 \ (110)\\
110 \cdot m_2 \equiv 1 \ (73)
	\end{cases}
	.\] 
	Bisogna determinare $m_1, m_2$\\
	\textbf{Idea:} Sfruttare l´identità di Bezoit: $(n_1) + (n_2) = (MCD(n_1,n_2)) = (1)$\\
	obiettivo: $n_1\cdot e + n_2\cdot s = 1$\\
	Nel nostro caso cerco $110 \cdot r + 73\cdot s = 1$  \ \ $r,s\in\Z$\\
 Perché è importante
 $110\cdot r \equiv 1\ mod(73)$\\
  $73\cdot s\equiv 1 \ mod(110)$\\
Il nuovo obiettivo è determinare  $r,s$ \\
Procedo con la divisione euclidea tra $110$ e $73$\\
\begin{gather*}
	110 = 73 + 37\\
	73 = 2 \cdot 37 - 1\\
	\Rightarrow 1 = 2\cdot 37 - 73\\
	\Rightarrow 2\cdot(110-73)-73 = 1\\
	\Rightarrow 2\cdot 110 - 3\cdot 73
\end{gather*}
Quindi:\\
$1 = 2 \cdot 110 - 3\cdot 73$\\
da cui \\
 \begin{gather*}
	m_1 = -3\\
	m_2 = 2
\end{gather*}
 \[
\bar x \equiv 5-\cdot 73\cdot(-3) + 47\cdot 110 \cdot (2) \equiv -620 \ mod(8030)
.\] 
\pene\\
\textbf{Nuovo Esercizio}\\
\begin{cases}
	x\equiv_6 2\\
	x\equiv_{10} 3
\end{cases}
Non possiamo sfruttare il teorema cinese del resto \\
\begin{gather*}
	x\equiv_6 2\\
	\storto{ \Leftrightarrow}\\
	x = 2 + 6k \ \ k\in\Z\\
	\storto{ \Leftrightarrow}\\
	x = 2(1 + 3k)
\end{gather*}
\begin{gather*}
	x\equiv_{10} 3\\
	\storto \Leftrightarrow\\
	x = 3 + 10h \ \ h\in \Z\\
	\storto \Leftrightarrow\\
	x = 2(5h + 1) + 1
\end{gather*}
Dunque dalla prima congruenza segue
\[
x\equiv_2 0 
.\] 
dalla seconda
\[
x\equiv_2 1
.\] 
\pene \\
\textbf{Nuovo Esercizio}\\
\begin{cases}
	3x\equiv_{15} 6\\
	7x\equiv_9 2
\end{cases}\\
Non posso usare $TCB$ studio $3x\equiv_{15} 6$ \\
\begin{gather*}
	3x\equiv 6 + 15k\\
	\storto \Leftrightarrow\\
	3x = 3(2 + 5k)\\
	\storto \Leftrightarrow\\
	x = 2 + 5k
\end{gather*}
\begin{cases}
	x\equiv_5 2\\
	7x\equiv_9 2
\end{cases}\\
Ora $MCD(3,9) = 1$ Vorrei sfruttare TCR, per farlo dobbiamo eliminare i coefficienti\\
Noto che $7$ e $9$ sono coprimi $ \Rightarrow [7]\in U_9$ (invertibili modulo 9)\\
Cerchiamo l'inverso moltiplicativo di $[7]\in U_9$\\
ovvero cerco $s\in\Z$ tale che  $7s \equiv_9 1$\\
Utilizzo la divisione euclidea\\
\begin{gather*}
	9 = 7 + 2\\
	7 = 3\cdot 2 + 1\\
	\Rightarrow 1 = 7 - 3\cdot 2\\
	\Rightarrow 1 = 7 - 3\cdot (9 - 7)\\
	\Rightarrow 1 = 4\cdot 7 - 3\cdot 9
\end{gather*}
Quindi $s = 4$\\
 \begin{gather*}
 	7x\equiv_9 2\\
	\storto \Leftrightarrow\\
	4\cdot 7 \equiv_9 4\cdot 2\\
	\storto \Leftrightarrow\\
	x\equiv_9 8\\
 \end{gather*}
Il sistema è quindi equivalente a \\
\[\begin{cases}
	x\equiv_5 2\\
	x\equiv_9 8
\end{cases}\]
Applico TCR\\
La soluzione esiste ed è unica modulo $(45)$\\
Soluzione:
 \[
\bar x \equiv_{45} 2\cdot 9 \cdot m_1 - 1\cdot 5\cdot m_2
.\] 
Dove : \ \ 
\begin{cases}
	5m_2\equiv_9 1\\
	9m_1\equiv_5 1
\end{cases}
Divisione euclidea
\begin{gather*}
	9 = 5 + 4\\
	5 = 4 + 1\\
	1 = 5 - 4\\
	1 = 5 - (9 - 5)\\
	1 = 2\cdot 5 - 9
\end{gather*}
$ \Rightarrow m_2 = 2 \ \ m_1 = -1$ 
\[
\bar x \equiv_{45} -18 -10 \equiv_{45} -28
.\] 
\newpage
\section{Azioni di gruppi}
\begin{defi}
	Un'azione di un gruppo $(G,\cdot)$ su un insieme $X$ è un'applicazione 
	\\
	\begin{center}
		
	\begin{aligned}
		G&\times X \rightarrow X\\
		 &(g,x) \rightarrow g.x
	\end{aligned}
	\end{center}
	tale che\\
	1) $e.x = x$\\
	2)  $(f \cdot g).x = f(g.x) \ \ \ \forall f,g\in G \ \ \ \forall x\in X$
\end{defi}\\
\textbf{Esempi:}\\
1)$(G,*)$ gruppo scelgo $X = G$ agisce per moltiplicazione sinistra\\
\begin{center}
	\begin{aligned}
		&G\times X \rightarrow X\\
		&(g,x) = g\c* x
	\end{aligned}
\end{center}
2) $ G = S_n \ \ \ X = \{1,\ldots, n\}$\\
 \begin{center}
	\begin{aligend}
		&S_n\times X \rightarrow X\\
		(\sigma, x) \rightarrow \sigma (x)
	\end{aligend}
\end{center}
3) $n,m\in \Z^+$\\
$G := GL_n(\R)\times GL_n(\R)$\\
$X = Mat_{n,m}(\R)$\\
 \begin{center}
	\begin{aligned}
		&G\times X \rightarrow X\\
		&(AB, C) \rightarrow BCA^{-1}
	\end{aligned}
\end{center}\\
4) $G = GL_n(\R) \ \ X = \R^n$\\
\begin{center}
	
 \begin{aligend}
	&G\times X \rightarrow X\\
	&(A,v) \rightarrow Av
\end{aligend}
\end{center}
5) $G = GL_n(\R) \ \ \ X = Mat_{n,m}(\R)$\\
\begin{center}
	\begin{aligned}
		&G\times X \rightarrow X\\
		&(A,C) \rightarrow ACA^{-1}
	\end{aligned}
\end{center}
6) $(G,\cdot)$ gruppo $X = G$\\
 \begin{center}
	\begin{aligned}
		&G\times X \rightarrow X\\
		& (g,x) \rightarrow g*x*g^{-1}
	\end{aligned}
\end{center}
\begin{defi}
	Data un'azione di un gruppo $G$ su un insieme $X$ si dice transitiva se
	 \[
		 \forall x,y\in X \ \exist g\in G \text{ tale che } g.x = y
	.\] 
\end{defi}
\begin{defi}
	Un'azione si dice semplicemente transitiva se 
	\[
		\forall x,y\in X \ \ \exist g\in G\text{ tale che } g.x = y
	.\] 
\end{defi}
\textbf{Esercizio:}\\
1) Dimostrare che gli esempi dati sono azioni\\
2) stabilire quali degli esempi sono semplicemente transitivi, transitivi o nessuna delle due
\begin{nota}
	Scriveremo $G\curvearrowright X$ per indicare che il gruppo  $G$ agisce sull'inseme $X$
\end{nota}
\begin{defi}
	$G\agisce X$, Dato  $x\in X$ definiamo:\\
	$\cdot$ l'orbita di x come il sottoinsieme
	\[
		O_x = \{g.x|g\in G\}\subseteq X
	.\] 
	$\codt$ lo stabilizzatore di $x$ il sottogruppo:
	\[
		Stab_x = \{g\in G| g.x = x\}\subseteq G
	.\] 
\end{defi}
\textbf{Esercizio:}\\
Dimostra che lo stabilizzatore di ogni elemento è sempre un sottogruppo (non necessariamente normale\\
\textbf{Esercizio:}\\
Sia $G$ gruppo finito ($|G| < +\infty)$ con $G\agisce X$, per ogni  $x\in X$ si ha:\\
1) $|Stab_x|<+\infty$ \ \ (banale)\\
2)  $|O_x|<+\infty$\\
3)  $|G| = |O_x||Stab_x|$\\
\textbf{Suggerimento:}\\
2) Abbiamo un'applicazione suriettiva \\
\begin{aligned}
	&G \rightarrow O_x\\
	&g \rightarrow g.x
\end{aligned}\\
3) L'idea è di dimostrare che esiste una corrispondenza biunivoca fra gli elementi dell'orbita e i laterali sinistri dello stabilizzatore, poi concludete ricordando che $[G:Stab_x] = \frac {|G|}{|Stab_x|}$ (numero di laterali sinistri)\\
 \textbf{Idea}(per la corrispondenza biunivoca)\\
 Verificare che $\forall g,f\in G$
  \begin{gather*}
  g\equiv f mod(Stab_x)\\
  \storto \Leftrightarrow\\
  g.x = f.x
 \end{gather*}
 \begin{teo}[Cauchy]
 	Sia $G$ un gruppo finito, Sia $p$ primo tale che $p \ |\ |G|$\\
	Allora esistono (almeno)  $p - 1$ elementi di ordine  $p$ in G
 \end{teo}
 \begin{dimo}
 	1) In generale se $G\agisce X$ allora  $X$ è unione disgiunta di orbite\\
	Definiamo la relazione di equivalenza $\tilde$ su $X$ come
	 \[
		 x\tilde y \Leftrightarrow \exist g\in G \text{ tale che }g.x = y
	.\]  
	Basta dimostrare che è una relazione d'equivalenza\\
	2) $X = \{(g_1,\ldots, g_n)\in G\times\ldots\times G | g\cdot\ldots\cdot g_p = e\}$\\
Vogliamo definire un'azione del gruppo ciclico $C_p =<p>$ su $ X$
 \begin{center}
	\begin{aligned}
		&C_p\times X \rightarrow X\\
		&\rho.(g_1,\ldots,g_p) \rightarrow (g_2,g_3,\ldots, g_p,g_1)
	\end{aligned}
\end{center}
Verifichiamo che l'azione sia ben definita ovvero che\\ $\rho.(g_1,\ldots,g_p)\in X \ \ \ \forall (g_1,\ldots,g_p)\in X$
\[
	g_2\cdot\ldots\cdot g_pg_1 = (g_1^{-1}g_1)(g_2\cdot\ldots\cdot g_p)g_1 = g_1^{-1}(g_1\cdot\ldots\cdot g_p)g_1 = g_1^{-1}g_1 = e
.\] 
3) Studio $| X|$ abbiamo   $|X| = |G|^{p-1}$ infatti:\\
$\forall (g_1,\ldots, g_{p-1},g_p)\in X$ dove $_p = (g_1,\ldots,g_{p-1})^{-1} \Rightarrow$ in particolare $p | |X|$ \\
4)Studiamo le orbite dell'azione $C_p\agisce X$, Sappiamo che  $|C_p| = |O_x||Stab_x| \ \forall x\in X$\\
Quindi  $|O_x| = 1 \ \ \vee\ \  |O_x| = p$\\
5) Dato che $X$ è unione disgiunta di orbite e $p | |X|$\\
Allora il numero di orbite formate da  $(x)$ unico elemento è un multiplo di  $p$\\
6) Studio tali orbite\\
L'orbita  $O_{(g_1,\ldots,g_p)}$ è formata da un singolo elemento se e solo se\\ $g_1 = g_2=\ldots=g_p$
 \end{dimo}
 Dunque abbiamo una corrispondenza biunivoca 
 \[
	 \{O_x : \ |O_x| = 1\} \ \leftrightarrow \  \{g\in G | g^p = e\}
 .\] 
 Quindi $p$ divide $|\{g\in G|g^p = e\}| $\\
 d'ora in poi  $ A = \{g\in G | g^p = e\}$\\
  $7) A \neq\emptyset$ poiché  $e\in A$\\
 \[
	 A = \{e\}\cup \{g\in G | ord(g) = p\}
.\] 
Quindi modulo $(p)$ abbiamo
\[
	0\equiv_p 1 + |\{g\in G | ord(g) = 1\}|
.\] 
Quindi l'insieme di elementi di ordine $p$ in  $G$ è non uvoto e 
\[
	|\{g\inG|ord(g)=p\} \equiv_p p - 1
.\] 
Deduciamo 
\[
	|\{g\in G | ord(g) = p\} = kp-1\geq p-1\\
.\] 
con $k\in Z^+$
\subsection{Torniamo alle schede}
\begin{cases}
	3x\equiv_{15} 6\\
	21x\equiv_{49} 13
\end{cases}
La prima congruenza è equivalente a $x\equiv_5 2$\\
$MCD(21,49) = 7$\\
La seconda congruenza significa\\
 \[
 21x = 13 + 49k \ \ k\in \Z
 .\] 
 \begin{gather*}
 	21x - 49k = 13\\
	7(3x-7k) = 13\ \
 \end{gather*}
 \textbf{Osservazione:}\\
 Se $MCD(a,n) \not | b$\\
 allora 
  $ax\equiv_n b$ non ammette soluzioni\\
  Infatti:  $d = MCD(a,n)$ \\ con  $d\not | b$ allora\\
  con $d$ divide il membro di sinistra ma non quello di destra\\
  \textbf{Esercizio}\\
  $G$ gruppo $g\in G$ \ \  $ord(g) = n$\\
  Allora,  $g^h = g ^k$ se e solo se $h\equiv_n k$\\
   \textbf{Soluzione}\\
   Assumiamo che $g^h=g^k$ Divisione Euclidea\\
   $h-k=qn + r$ con $0\leq r< n$
    \begin{gather*}
	    g^h = g^k \Rightarrow g^{h-k} = e\\
	    \Rightarrow g^{qn+r}=e\\
	    \Rightarrow (g^n)^qg^r = e\\
	    \Rightarrow g^r = e
   	
   \end{gather*}
   Assurdo se $0<r<n$ \  $r = 0$\\
   $h-k=qn \Rightarrow h\equiv_n k$\\
   \textbf{Esercizio}\\
   per quali $n,m\in\Z$ si ha  $2^n + 2^m$ divisibile per $9$\\
    \textbf{Soluzione}\\
    Studio
    \begin{gather*}
    	2^n + 2^m\equiv_9 0\\
	\storto \Leftrightarrow
	2^n\equiv_9 -2^m\\
	\storto \Leftrightarrow\\
	2^{n-m} \equiv_9 -1\\
	\storto \Leftrightarrow\\
	2^{n-m} \equiv_9 8\\
	\storto \Leftrightarrow\\
	2^{n-m}\equiv_{9} 2^3
    \end{gather*}
    Sfruttiamo l'esercizio precedente con $G = U_9$ \\
    La congruenza è verificata se e solo se
    \[
	    n - m \equiv 3 \ mod(ord_{U_9}([2]))
    .\] 
    \begin{gather*}
    	2\\
	2^2 = 4\\
	2^3 = -1\\
	2^4 = -2\\
	2^5 = -4\\
	2^6 = 1
    \end{gather*}
    quindi $ord([2]) = 6$\\
    Soluzione: $n-m \equiv_6 3$
    %lezione 13 bubibabu

\subsection{Azione di coniugio}
	\begin{defi}
		Se $G$ gruppo e $a,b,g\in G$ tali che:
		 \[
			 a = gbg^{-1}
		.\] 
		diremo che $a,b$ sono coniugati
	\end{defi}
	\begin{defi}
		$G$ gruppo. Allora $G$ agisce su se stesso tramite l'azione di coniugio\\
		\begin{aligned}
			&G\times G \rightarrow G\\
			&g.f = gfg^{-1}
		\end{aligned}
	\end{defi}
	\textbf{Esercizio}\\
	Verificare che è un'azione\\
	\begin{teo}
		$G$ gruppo\\
		$1)$ elementi coniugati hanno lo stesso ordine\\
		$2)$ $|O_a| = [G:C(a)]$ dove\\
		$C(a):=\{g\in G|ga = ag\}\leq G$ (centralizzatore di  $a$)
		$3)$ equazione delle classi\\
		$\displaystyle|G| = |Z(G)| = \sum_{O_a\not\subseteq Z(G)}\frac{|G|}{|C(a)|}$
	\end{teo}
	\begin{dimo}
		1) Siano $a,b,g\in G$ tali che  $a = gbg^{-1}$ supponiamo che  $b^k = e \ \ \ k\in\Z$\\
		Allora $a^k = (gbg^{-1})\cdot\ldots\cdot(gbg^{-1}) = gb^kg^{-1} = e$\\
		Quindi $ord(a)\leq ord(b)$.\\
		Per simmetria b =  $g^{-1}ag \Rightarrow ord(b)\leq ord(a)$ \\
		Allora $ord(a) = ord(b)$\\
		2)Osserviamo che \\
		\begin{aligned}
			C(a)=& {g\in G|ga = ag\}\\
			=&\{g\in G | gag^{-1} = a\}\\
			=& Stab_a
		\end{aligned}
		Ricordiamo che :\\
		$|O_a|\cdot |Stab_a| = |G|$\\
		$ \Rightarrow |O_a| = \frac {|G|}{|Stab_a|} = [G : C(a)]$ \\
		3) se $a\in Z(G)$ allora  $O_a = \{a\}$ poiché\\
		$\forall g\in G$ si ha $g\codt a  = gag^{-1} = agg^{-1} = a$\\
		Ricordiamo che  $G$ ammette una partizione in $G$-orbite
		\[
		|G| = |Z(G)| + \sum_{O_a\not\subseteq Z(G)} |O_a|
		.\] 
		Dal punto (2) $ \Rightarrow  |O_a| = \frac{|G|}{|C(a)|}$ 
		\[
			|G| = |Z(G)| + \sum_{O_a\not\subseteq Z(G)}\frac{|G|}{|C(a)|}
		.\] 
		\textbf{Esempio} (dalla nuova scheda)\\
		$n\geq 3$ intero dispari  $G = D_n$\\
		$Z(D_n) = \{Id\}$ infatti  $\rho^i\sigma = \sigma \rho^{n-i}$\\
		Quindi\\
		$1) O_\sigma = \{Id\}$\\
		$2) O_{\rho^i} = ?$ \\
		Idea $|O_{\rho^i}| = [D_n : C(\rho^i)]$ \\
		$C(\rho') = \{\rho^i | i = 0,\ldots,n-1\}$ \\
		$ \Rightarrow |C(\rho^i)| \geq n$ \\
		Dato che $C(\rho^i)\leq D_n$ allora  $|C(\rho^i)| = n$ oppure  $|C(\rho^i) = 2n$\\
		Ma  $\sigma\rho^i = \rho^{n-i}\sigma\neq\rho^i\sigma \ \ \forall 0<i<n$\\
		 $ \Rightarrow |O(\rho^i)| = n$ \\
		 Quindi \\
		 $O_{\rho^i} = [D_n:C(\rho^i)] = \frac{2n}n = 2$ \\
		 Basta ora trovate un altro elemento coniugato a $\rho^i \ \ (0<i<n)$
		  \[
			  \sigma\rho^i\sigma^{-1} = \rho^{n-i}\sigma\sigma^{-1} = \rho^{n-i}
		 .\] 
		 quindi $O_{\rho^i} = \{\rho^i, \rho^{n-i}\} \ \ \forall 0< i< n$\\
		 $3) O_\sigma = \{\sigma, ?\}$\\
		 Studiamo $C(\sigma)$\\
		  $\sigma$ non commuta con $\rho^i \ \ \forall 0 < i  <n$\\
		  Se $\sigma$ commuta con $\sigma\rho^i \ \ con 0<i<n$\\
		  Allora  $\sigma$ commuta anche con il prodotto $\sigma(\sigma\rho^i) = \rho^i$ assurdo \lightning \\
		  $C(\sigma) = \{Id, \sigma\}$\\
		  Quindi  $|O|_\sigma| = [D_n:C(\sigma)] = \frac{2n}2 = n$\\
		  $ \Rightarrow O_\sigma = \{\sigma\rho^i | 0\leq i < n\}$\\
		  Equazione delle classi.\\
		  $|D_n| = |Z(D_n)| + \sum_{O_a\not\subseteq Z(D_n)}|O_a|$\\
		   \[
		  2n = 1 + 2 + \ldots + 2 + n
		  .\] 



	\end{dimo}
	\begin{teo}
		$G$ gruppo tale che $|G| = p^k \ \ p$ primo $k>0$.\\
		Allora:\\
		1) $Z(G)\neq \{e\}\\$
		2) $[G:Z(G)] \neq p$
	\end{teo}
	\begin{dimo}
		1) \textbf{IDEA} equazioni delle classi
		\[
			|G| - |Z(G)| = \sum_{O_a\not\subseteq Z(G)}\frac{|G|}{C(a)}
		.\] 
		modulo  $(p)$ avremmo
		 \[
		|Z(G)|\equiv_p 0 
		.\] 
		$|Z(G)| \neq 1 \Rightarrow Z(G) \neq \{e\}$ \\
		Attenzione, $\warning \frac {|G|}{C(a)} = 1 \Rightarrow C(a) = G \Rightarrow a\in Z(G) \Rightarrow O_a = \{a\}\subseteq Z(G)$\\
		Supponiamo per assurdo che\\
		$[G:Z(G)] = p$ \\
		$ \Rightarrow \frac{|G|}{|Z(G)|} = p \Rightarrow |Z(G)| = p^{k-1}$ \\
		Consideriamo $g\inG\setminus Z(G)$\\
		$ \Rightarrow C(g)\supseteq Z(G)\cup \{g\}$ \\
		$ \Rightarrow  \Rightarrow |C(g) = p^{k-1} + 1$ \\
		$ \Rightarrow |C(g)| = p^k \Rightarrow C(g) = G\\$ 
		$ \Rightarrow g\in Z(G)$ assurdo
	\end{dimo}
	\begin{coro}[Classificazione dei gruppi di oridine $p^2$]
		$G$ gruppo tale che $|G| = p^2$ con  $p$ primo.\\
		Allora $G\cong C_{p^2}$ oppure $G\cong C_p\times C_p$
	\end{coro}
	\begin{dimo}
		\textbf{IDEA CHIAVE} Se $|G| = p^2$ allora  $G$ è abeliano.\\
		Infatti dal teorema:\\
		$\cdot$ $Z(G)\neq\{e\}$\\
		$\cdot |Z(G)|\neq p$ perché avremmo $[G:Z(G)]=p$\\
		allora per Lagrnage\\
		$|Z(G)| = p^2 \Rightarrow Z(G) = G \Rightarrow  G$ abeliano\\
		Ora se $\exists g\in G$ tale che  $ord(g) = p^2$ allora  $G\sim G_{p^2}$\\
		Se invece non esistono elementi di ordine  $p^2$ allora tutti gli elementi  $(\neq e)$ in $G$ hanno ordine $p$\\
		Sia $h\in G$ tale che $h\neq e \Rightarrow H:=<h>$ con $|H| = p$\\
		Sia  $k\in G\setminus H$\\
		 $ \Rightarrow K := <k> \ \ con |K|  = p$ \\
		 Verifichiamo che $G$ è prodotto diretto interno di $H$ e $K$\\
		  $\cdot H\normale G$ e $K\normale G$ (poiché $G$ abeliano)\\
		  $H\cap K = \{e\}$ Infatti:\\
		 \[\begin{cases}
			H\cap K \eq K\\
			H\cap K\neq K
		\end{cases} \Rightarrow |H\cap K| = 1.\]
		$HK = ?$\\
		$|HK| = \frac{|H||K|}{|H\cap K|} = \frac{p^2}1 = p^2$\\
		 $ \Rightarrow HK = G$ \\
		 Allora $G\cong H\times K\cong C_p\times C_p$
		
	\end{dimo} 
	\textbf{Osservazione} (per $p = 2$ )\\
	$G = C_4$ oppure $G\cong K_4 \cong \Z/(2)\times\Z/(2)$\\
	 \textbf{Osservazione}\\
	 $p = 3$ $|G| = 0$ allora\\
	  $G\cong C_9$ oppure $G\cong C_3\times C_3$\\
	  %lezione 14

	\subsection{Classi coniugate in $S_n$}
	\begin{teo}[Fondamentale]
		Due permutazioni in $S_n$ sono coniugate se e solo se hanno la stessa struttura ciclica
	\end{teo}
	\begin{dimo}
	  	$\tau = (a_1,\ldots, a_n)\in S_n$ un $k$-ciclo $\sigma \in S_n$\\
		Studio ora  $\sigma \tau\sigma^{-1}$ e la sua azione sull'insieme  $\{\sigma(1),\ldots, \sigma(n)\}$\\
		Se  $\tau(j) = j$\\
		$ \Rightarrow \sigma \tau \sigma^{-1}(\sigma(j)) = \sigma\tau(j) = \sigma (j)$ \\
		Se $j = a_i$ per qualche $1\leq i\leq k \Rightarrow \tau (j) = \tau (a_i) = a_{i + 1 \ mod(k)}$ \\
		$\sigma\tau\sigma^{-1}(\sigma(a_i)) = \sigma\tau(a_i) = \sigma (a_{i + 1 \ mod(k)})$\\
		Allora \\
		$\sigma\tau\sigma^{-1} = (\sigma(a_1),\sigma(a_2),\ldots, \sigma(a_k))$\\
		 Da questo abbiamo dedotto che date $\sigma, \tau\in S_n$  qualsiasi, allora:\\
		 $\sigma\tau\sigma^{-1}$ ha la stessa struttura ciclica di $\tau$\\
		 $\cdot$ Vogliamo ora dimostrare il viceversa, ovvero: Date $\tau,\omega\in S_n$ vogliamo costruire  $\sigma$ tale che $\sigma\tau\sigma^{-1} = \omega$ ( $\tau,\omega$ con la stessa struttura ciclica)\\
		 Per ipotesi, $\tau = \tau_1\ldots tau_h$ e $\omega = \omega_1\ldots\omega_h$ dove $h\geq 1, \tau_i,\omega_i$ sono  $k_i-cicli$\\
		 Denotiamo  $\tau_i = (a_1^i\ldotsa^i_{k_i}), \omega = (b_1^i\ldots b_{k_i})$\\
		 Possiamo definire $\sigma$ esplicitamente\\
		 Infatti\\
		 $\sigma\tau_i\sigma^{-1} = (\sigma(a_1^i)\ldots\sigma(a_{k_i}^i)$\\
		 Quindi\\
		 Definiamo $\sigma := \{\sigma (a_j^i) = b_j^i \ \ \forall i\in\{1,\ldots, h\}, j\in \{1,\ldots,k\}, \sigma (t) = t $ se $t\neq a_j^i\}$\\
		 Allora $\sigma\tau_i\sigma^{-1} = \omega_i \ \ \forall i = h$\\
		 $ \Rightarrow \sigma\tau\sigma^{-1} = \sigma\tau_1\ldots\tau_h\sigma^{-1}$ \\
		 $= (\sigma\tau_1\sigma^{-1})\ldots(\sigma\tau_h\sigma^{-1})\\
		 =\omega_1\ldots\omega_h = \omega$
	\end{dimo}
	\textbf{Osservazione}\\
	Dato che la dimostrazione è costruttiva, è molto utile per risolvere gli esercizi.
	\subsection{Il gruppo p-Sylow}
	\textbf{Idea}\\
	Prendiamo un gruppo finito.\\
	Esistono sottogruppi di un dato ordine (divisore di $|G|$)?\\
	Il risultato parziale che abbiamo è dato dal Teorema di Cauchy:\\
	Se  $\exists p$ primo e divide  $|G|$, allora:\\
	 $\exists H\leq G$ t.c. $|H| = p$ \\
	 Sylow, va avanti secondo questo filone:\\\newpage
	 \begin{defi}
		 Sia $G$ gruppo finito, $p,r,m\in\Z_{>0} t.c.$\\
		  $\cdot |G| = p^r\cdot m$\\
		   $\cdot p$ primo \hfill (ogni gruppo finito ha queste caratteristiche)\\
		   $\cdot MCD(p,m) = 1$\\
	 Un sottogruppo di ordine $p^r$ in $G$ si chiama $p-Sylow$\\
	 L'insieme dei  $p-Sylow$ si denota con $Syl_p(G)$\\
	 \end{defi}
	 \begin{teo}[I Teorema di di Sylow  (1862-1872)]
	 	Se $G$ gruppo finito, $p$ primo che divide $|G|$, Allora:\\
		 $Syl_p(G)\neq \emptyset$
	 \end{teo}
	 \begin{dimo}
		 Sia $X:= \{S\subseteq G:\ |S| = p^r\}$\\
		 Definisco un azione\\
		  \begin{aligned}
			  G&\times X\rightarrow X\\
			  (&g,s) \rightarrow gS = \{gs|s\in S\}
		 \end{aligned}\\
		 Dalle osservazioni $ \Rightarrow p \not | \ |X|$ \\
		 D'altra parte, $x$ si decompone in $G$-orbite\\
		 Inoltre $|O_S|\cdot|Stab_S| = |G| = p^r\cdot m$\\
		  $ \Rightarrow \exists$ almeno un elemento $\underline S\in X$t.c.  $\underline S|\not\equiv_p 0$\\
		  Allora  
		   \[
			   \frac{|O_{\underline S}|}{|O_{\underline S}|}\cdot |Stab_{\underline S}| = \frac{p^r\cdot m}{|O_{\underline S}|}\in \Z
		  .\] 
		  Da cui segue che $|Stab_{\underline S}|\equiv_{p^r} 0$ \\
		  $p^r\leq |Stab_{\underline S}$\\
	  L'idea ora è di dimostrare che $Stab_{\underline S}\in Syl_p(G)$\\
	  Essendo uno stabilizzatore, è sicuramente un sottogruppo, quindi basta dimostrare che $|Stab_{\underline S}|\leq p^r$\\
	  \textbf{Osservazione/Esercizio}\\
	  $\exists$ applicazione iniettiva, $Stab_{\underline S} \rightarrow p$ definita fissando un elemento qualsiasi $\underline s\in S$ \\
	   \begin{aligned}
		   &Stab_{\underline S} \rightarrow \underline S\\
		   & g \rightarrow g\underline s
	  \end{aligned}\\
	  dimostrare che questa funzione è iniettiva, questo porta alla conclusione che $|Stab_{\underline S}|\leq |\underline S| = p^r$ dato che  $\underline S\in X$
	 \end{dimo}
	 \textbf{Esempio}\\
	 Sia $|G| = 12  = 2^2\cdot 3 = 3 \cdot 4$\\
	 Dal  I Teorema di Sylow segue:\\
	  $\cdot Syl_2(G)\neq 0 \Rightarrow \exists H\leq G : |H| = 4$ \\
	  $\cdot Syl_3(G)\neq 0 \Rightarrow \exists H\leq G : |H| = 3$ \\
	  \textbf{Osservazione}\\
	  $\cdot X = O_{S_1}\circ O_{S_2}\circ\ldots\cric O_{S_r}$\\
	  $ \Rightarrow |X| = \sum^r_{j=1} |O_{S_j}|$ Ma $|X|\not\equiv_p 0$\\
	  \textbf{Idea}\\
	  $G$ gruppo, $|G| = p^r \cdotm$\\
	  $MCD(p,m) = 1, p$ primo, $p,r,m\in\Z_{>0}$ \\
	  Per il I teorema sappiamo che $(1) Syl_p(G)\neq 0$.\\
	  il  II Teorema ci dirà che $(2)$ Tutti i $p-Sylow$ sono tra loro coniugati.\\
	  Il (3) ci dice che $ \rightarrow$ Un $p$-Sylow è normale se e solo se è l'unico $p$-Sylow.\\
	  Quanti sono i $p$-Sylow? Analogamente $n_p := |Syl_p(G)| = ?$\\
	  \begin{teo}[II Teorema di Sylow]
		  Dati $H_1, H_2\in Syl_p (G), \exists g\in G$ t.c. \ \ $g H_1 g^{-1} = H_2$\\
	  \end{teo}
	  \begin{dimo}
		  	L'enunciato è equivalente a dimostrare che la seguente azione è transitiva.
			\begin{center}
				\begin{aligned}
					G\times  Syl_p(G) &\rightarrow Syl_p (G)\\
					(g, H) \hspace{20px}&\rightarrow gHg^{-1}
				\end{aligned}
			\end{center}
			o equivalentemente, che esiste un'unica orbita.\\
			Per assurdo supponiamo che esistano due orbite distinte, $O_H^G$ e $O_K^G$.\\
			 \textbf{Passo 1}\\
			 Denotiamo con $Stab_H^G$ lo stabilizzatore di $H$ rispetto a questa azione
			 \[
			 \begin{cases}
				 |G| = |O^G_H|\cdot |Stab_H^G| = |O^G_H|\cdot[Stab_H^G:H]\cdot |H|\\
				 H\leq Stab_H^G
			 \end{cases}
			 .\] 
			 Quindi $p \not |  \ |O^G_H|$\\
			 \textbf{Passo 2}\\
			 Restringiamo l'azione\\
			 \begin{center}
			 	\begin{aligend}
					K\times O^G_H & \rightarrow O_H^G\\
					(k,S) & \rightarrowk S k^{-1}
			 	\end{aligend}
			 \end{center}\\
			 Rispetto a questa azione abbiamo orbite diverse.\\
			 In particolare\\
			 $|O_H^G| = O_{H_1}^K\cup\ldots\cup O_{H_r}^K$\\
			 \begin{aligned}
			 $	
			 \Rightarrow |O^G_H| &= \sum^r_{i=1} |O_{H_i}^K|\\
					     & = \sum^r_{i=1} \frac{|K|}{Stab_{H_i}^K}\\
					     & = \sum^r_{i=1} \frac{ p^r}{|Stab_{H_i}|}
			 $\end{aligned}\\
			 Dato che $p \not | |O_H^G|$ deduciamo che $\exists H_i$ t.c. \  $|O^K_{H_i}| = 1$\\
			 $ \Rightarrow 1 = |O^K_{H_i}|= [K:Stab_{H_i}^G]$ \\
			 Quindi $K$ stabilizza $H_i$\\
			 $ \Rightarrow kH_ik^{-1} = H_i \ \ \forall k\in K$\\
			 $ \Rightarrow KH_i = H_i K$ \\
			 \textbf{Passo 3}
			 $KH_i = H_i K \Rightarrow KH_i\leq G$ \\
			 $\displaystyle|KH_i| = \frac{|K|\cdot |H_i|}{|K\cap H_i|} = \frac {p^{2r}}{p^s}$ \ \ con  $s<r$ (poichè altrimenti $K = H_i$)\\
			 $|KH_i| = p^{2r-s} = p^{r + t}$ con $t > r$\\
			 Ma  $|KH_i|$ divide $|G| = p^r m$ per Lagrange (assurdo)

		\begin{coro}
			$p$ primo che divide $|G|$ allora  $H\in Syl(G)$ è normale se e solo se\\ $n_p = |Syl_p(G)| = 1$
		\end{coro}
		\textbf{Osservazione}\\
		è importante sapere se $n_p = 1$ perché l'esistenza di sottogruppi normali spesso permette di realizzare un gruppo come prodotto semidiretto
	\\
	\begin{teo}[III teorema di Sylow]
		$G$ gruppo finito
		\begin{itemize}
			\item $|G| = p^r m$
			\item $r,p,m\in \Z_{>0}$
			\item $p$ primo
			\item $MCD(p,m) = 1$
		\end{itemize}
		Allora:\\
		1) $n_P = [G:N_G(H)]$ dove  $H\in Syl_p(G)$ \\
		2) $m\equiv_{n_p} 0$\\
		3)  $n_p\equiv_p 1$
	\end{teo}
	Prima della dimostrazione vogliamo estendere la nozione di centralizzatore (o centralizzante)
	\newpage
	\begin{defi}[Normalizzatore]
		$G$ gruppo $S\subseteq G$ sottoinsieme\\
		1)Il centralizzatore di  $S$ in  $G$ è
		 \[
			 C(S) = \{g\in G| gs = sg \ \ \forall s\in S\}
		.\] 
		2) Il normalizzatore di $S$ in $G$ è 
		\[
			N_G(S) = \{g\in G | gS = Sg\}
		.\] 
	\end{defi}
	\textbf{Esercizio:}\\
	Dimostrare che\\
	1) Se $S\subseteq G \Rightarrow C(S) \leq G$ \\
	2) $S\subseteq G \Rightarrow N_G(S)\leq G$ \\
	3) $S\leq G \Rightarrow S\leq N_G(S)$ \\
	\begin{dimo}
		Considero l'azione\\
		$G\times Syl_p(G) \rightarrow Syl_p(G)$\\
		$(g,H) \rightarrow g.H := gHg^{-1}$ \\
		Allora $\forall H\in Syl_p(G)\\
		p^rm = |G| = [G:Stab_H]\cdot |Stab_H|$\\
		 $= [G:N_G(H)]\cdot |N_G(H)|$\hfill
		 (dato che $Stab_h = N_G(H)$)\\
		 $=[G:B_G(H)][N_G(H):H]|H|$\hfill
		 (dato che $H\leq N_G(H)$) \\
		 Deduciamo che $m = [G:N_G(H)]\cdot[N_G(H):H]$\\
		 Ora:\\
	 $n_p = |Syl_p(G)| = |O_H^G|$\hfill (II Teorema di Sylow)\\
	 $ = [G:Stab_H]$\\
	 Quindi abbiamo dimostrato $(1)$ e $(2)$\\
	 Resta da dimostrare $(3)$\\
	 di un fissato  $K\in Syl_p(G)$\\
	 \[
		 K\times Syl_p(G) \rightarrow Syl_p (G)
	 .\] 
	 \[
		 (k,H) \rightarrow k.H := kHk^{-1}
	 .\] 
	 Questa azione avrà $r + 1$ orbite (con $r \geq 0$)\\
	 $O_K^K, O_{H_1}^K,\ldots,O_{H_r}^K$\\
	 Abbiamo una decomposizione in orbite disgiunte
	 \[
		 Syl_p(G) = O_K^K\cup O_{H_1}^K\cup\ldots\cup O_{H_r}^K
	 .\] 
	 \[ \Rightarrow n_p = |Syl_p (G)| = |O_K^K| + \sum^r_{j=1}|O_{H_j}^K|\]
	 \[
		 =|O_K^K| + \sum^r_{j=1}[K:Syl_{H_j}^K]
	 .\] 
	 \[
		 = |O_K^K| + \sum^r_{j=1}[K:N_K(H_j)]
	 .\]
	 \textbf{Idea}\\
	 Basta ora verificare che
	 \begin{itemize}
		 \item $|O_K^K| = 1$
		 \item  $O_{H_j}^K\equiv_p 0 \ \ \ \forall 1\leq j\leq r$
	 \end{itemize}\\
	 Abbiamo:\\
	  \[
		  O_K^K = [K:N_K(K)] = 1
	 .\] 
	 Dato che $H\leq N_G(H)\leq G \Rightarrow N_K(K) = K$ 
	 \[
		 |O_{H_j}^K| = [K:N_K(H_j)] = \frac {|K|}{|N_K(H_{j})|} = \frac {p^r}{|N_K(H_j)|}
	 .\] 
	 dato che $K \in Syl_p(G)$\\
	 Quindi resta da escludere il caso  $N_K(H_j) = K$\\
	 Ma questo è equivalente a  $KH_j = H_j K$ 
	 \[
	 \Rightarrow \begin{cases}
	 	KH_j\leq G\\
		|KH_j| = \frac { |K||H_j|}{|K\cap H_j|} = \frac{p^{2r}}{p^{s_j}}
	 \end{cases}
	 .\] 
	 dove $0\leq s_j < r$ dato che  $H_j\neq K_j$\\
	 Ma  $p^{2r-s_j}\not | \ p^rm$ da cui l'assurdo per Lagrange
	\end{dimo}
	\subsection{Applicazioni di Sylow}
	Possiamo (ri)-dimostrare un vecchio risultato
	\begin{teo}[Cauchy]
		$G$ gruppo finito, $p$ primo che divide $|G|$ allora\\
		 \[
			 \Exists g\in G\text{ tale che } $ord(g) = p$
		.\] 
		
	\end{teo}
	\begin{dimo}
		Da Sylow I segue che esiste $H\in Syl_p(G)$ \\
		Scegliamo $h\in H$ tale che  $h\neq e$\\
		Ora  $ord(h) = p^s$ per qualche $s>0$\\
		Definiamo $f = h^{p^{s-1}}$\\
		 $f = h^{p^{s-1}}\neq e \Rightarrow ord(h)\neq 1$ \\
		 $f^p = (h^{p^{s-1}})^p = h^{p^s} = e \Rightarrow ord(f) = p$
	\end{dimo}
	\begin{teo}[Wilson]
		$p$ primo allora $(p-1)!\equiv_p p-1$
	\end{teo}
	\begin{dimo}
		Scelgo $G= S_p$ Studio $n_p$\\
		I  $p$-Sylow in $S_p$ hanno ordine $p$\\
		 $ \Rightarrow$ sono tutti i sottogruppi ciclici di ordine $p$ in $S_p$\\
		 $\cdot$ Gli unici elementi di ordine  $p$ in $S_p$ sono i $p$-cicli.\\
		 fissato il primo elemento, abbiamo $p-1$ scelte per il secondo, $p-2$ per il terzo e così via\\
		 quindi i $p$-cicli sono $ (p-1)!$\\
		 Quindi i sottogruppi di $S_p$ di ordine $p$ sono $\frac{(p-1)!}{(p-1)} = (p-2)!$ perché in ogni tale sottogruppo appaiono  $p-1$ $p$-cicli\\
		 $ \Rightarrow (p-2)! = n_p \equiv_p 1 \ \ \Rightarrow \ \ (p-1)! \equiv_p p-1$
	\end{dimo}
	\begin{teo}[Classificazione dei gruppi $pq$]
		$G$ gruppo finito, $p,q > 1$ tali che \\
		 $\cdot p,q$ primi\\
		 $\cdot p < 1$\\
		  $\cdot|G| = pq$ \\
		  Allora\\
		  1) Se $p\not | \ q-1$ allora  $G\cong C_{pq}$\\
		  2) Se  $p | q-1$ allora  $G\cong C_q\semi C_p$
	\end{teo}
	\begin{dimo}
		Studio $n_p$ \\
		\begin{cases}
			p = m \equiv_{nq} 0\\
			n_q \equiv_q 1
		\end{cases}\\
		$ \Rightarrow $ \begin{cases}
			n_q =1 \text { oppure } m_q = p\\
			\text{ seconda esclude } n_q = p \text{ perchè } p < q

		\end{cases}\\
		$ \Rightarrow n_q = 1$ \\
		$ \Rightarrow \exists ! Q\in Syl_p(G)$\\
		$ \Rightarrow Q\trianglelefteq G$ e $|G| = q \Rightarrow Q\cong C_q$\\
		Studio $n_p$ nel caso $p\not | q -1$\\
		 \begin{cases}
			 $q-m \equiv_{n_p} 0$\\
			  $n_p\equiv_p 1$
		\end{cases}
		$ \Rightarrow n_p = 1$ oppure $n_p = q$\\
		 $ \Rightarrow n_p\neq q$ perché\\
		 $q\not\equiv_p 1$ per ipotesi\\
		  $n_p = 1 \Rightarrow \exists ! P\in Syl_p(G)$ \\
		  $ \Rightarrow P\triangleleftwq G$ e $|P| = p \Rightarrow P\cong C_p$ \\
		  Ora abbiamo due sottogruppi normali $P,Q\trianglelefteq G$\\
		  tali che\\
	  $\cdot P\cap Q = \{e\}$  perchè  $|P\cap Q|$ divide sia  $|P| = p$ che $|Q| = q$\\
	  $\cdot PQ| = \frac{|P||Q|}{|P\capQ|} = pq = |G|$\\
$ \Rightarrow G\cong P\times Q\cong C_p\times C_q\cong C_{pq}$

	Resta il caso $p | q-1$\\
	 $\cdot \exists ! Q\in Syl_p(G) \leadsto Q\trianglelefteq G$\\
	  $\cdot \exists P\in Syl_p(G)\leadsto P\leq G$\\
	  Ora\\
	  $\cdot P\cap Q = \{ e\}$ come prima\\
	   $PQ = G$ come prima\\
	   Quindi $G$ prodotto semidiretto interno\\
	   $ \Rightarrow G\cong Q\semi P \Rightarrow C_q \semi C_p$ \\
	   per qualche omomorfismo $\phi :C_p \rightarrow Aut(C_q)$
	\end{dimo}
	\textbf{Esercizio:}\\
	Classificare i gruppi di ordine $2q$ con  $q>2$ primo
	  \end{dimo}
	\subsection{Ricordo:}
	\begin{teo}
		$p < q$ primi $G$ gruppo finito di ordine $pq$\\
		Allora:\\
		$\cdot$ se $p\not  | q + 1$ allora $G\cong C_{pq}$\\
		 $\cdot$ se $p | q + 1$ allora $ G\cong C_q\semi C_p$
	\end{teo}
	\textbf{Inserisci tabella fino ad ordine 9}
	\begin{coro}
		$q > 2$ primo, $G$ gruppo di ordine $2q$\\
		Allora $G\cong C_{2q}$ oppure $G\cong D_q$
	\end{coro}
	\begin{dimo}
		Dal teorema basta studiare gli omomorfisimi
		\begin{center}
		\begin{aligned}
			\phi:&C_2 \rightarrow Aut(G)\\
			     &s \rightarrow (\phi_s : r \rightarrow s)
			
		\end{aligned}
		\end{center}
		Affinchè $\phi$ sia un omomorfismo, dato che $ord_{C_2}(s) = 2$\\
		dobbiamo imporre che $ord_{Aut(G)}(\phi_s) \in \{1,2\}$\\
		Se è uguale a 1 $\phi_s = Id \Rightarrow \phi$ omomorfismo banale\\
		$ \Rightarrow $ il prodotto è diretto\\
		$ \Rightarrow G\cong C_q\times C_2\cong C_{2q}$ \\
		Nell'altro caso $ord_{Aut(G)}(\phi_s) = 2$\\
		$ \Rightarrow \phi_s\circ\phi_s = Id_{C_q} \Rightarrow \phi_s(\phi_s(r))=r$ \\
		$\phi_s(r^k) = r$\\
		$ \Rightarrow k^2\equiv_{ord_{C_1}(r)} 1 \Rightarrow k^2\equiv_q 1$ \\
		$ \Rightarrow (k-1)(k+1)\equiv_q 0$ \\
		$ \Rightarrow k\equiv_q \pm 1$ \\
		Se $k \equiv_q 1$\\
		$ \Rightarrow \phi_s = Id_{C_q} \Rightarrow G\equiv C_{2q}$ \\
		Se $k\equiv_q -1$\\
		$ \Rightarrow \phi_s(r) = r^{-1}$\\
		$ \Rightarrow G\cong C_q\semi C_2\cong D_q$ (già visto)

	\end{dimo}
	\subsection{Gruppi di ordine 12}
	Studiamo $G$ tramite i teoremi di Sylow\\
	$\cdot Syl_2(G)\neq \emptyset$\\
	 $\cdot Syl_3(G)\neq\emptyset$ \\
	 Dal Sylow III abbiamo
	 \[
	 \begin{cases}
	 	n_2 \equiv_2 1\\
		3\equiv_{n_2} 0
	 \end{cases}
	 .\] 
	 $ \Rightarrow n_2= 1$ oppure $n_2 = 3$\\
	 Dal Sylow II
	  \[
		  \begin{cases}
		  	
	 n_3\equiv_3 1\\
	 4\equiv_{n_3} 0
		  \end{cases}
	 .\] 
	 $n_3 = 1$ oppure $n_3 = 4$\\
	  \textbf{Osservazione:}\\
	  Esiste un sottogruppo normale in $G$
	  \begin{dimo}
	  	se $n_3 = 4$\\
		Allora in  $G$ esistono 4 sottogruppi di ordine 3\\
		Ognuno dei quali contenente due elementi di ordine 3.\\
		Quindi $G$ contiene 8 elementi di ordine 3.\\
		Quindi i restanti 3 elementi di ordine diverso da 3 formano necessariamente l'unico $2$-Sylow
	  \end{dimo}
	  \textbf{Esercizio:}\\
	  Se $|G| = 12$ e  $n_3 = 4$ allora esiste un omomorfismo iniettivo  $G \rightarrow S_4$\\
	  \textbf{Nota}\\
	  Da questo segue che $G\cong A_4$ perchè $A_4$ è l'unico sottogruppo di ordine $12$ in $S_4$
	  \begin{dimo}
	  	$G\times Syl_3(G) \rightarrow Syl_3(G)$\\
		$(g,H) \rightarrow g H g^{-1}$\\\
		$n_3 = 4$\\
		$ \Rightarrow Syl_3(G) = \{H_1,H_2,H_3,H_4\}$ \\
		Definiamo\\
		\begin{aligend}
			\psi : &G \rightarrow S_4\\
			       &g \rightarrow \tau_g
		\end{aligend}\\
		$\tau_g(i) = j \Leftrightarrow gHg^{-1} = H_j$ con $i\in\{1,2,3,4\}$ (Questa è l'idea da utilizzare negli esercizi delle schede\\
Verifiche:\\
$1) \psi$ è ben definita, Infatti $\tau_g$ è invertibile con inversa  $\tau_{G^{-1}}$\\
 $2) \ \psi $ è un omomorfismo, ovvero
 \[
 \psi(gf) = \psi(g)\psi(f)
 .\] 
 $\tau_{gf}(i) = j$ \\\begin{aligned}
	\Leftrightarrow&(gf)H(gf)^{-1} = H_j\\
	\Leftrightarrow& g(fHf^{-1})g^{-1} = H_j\\
	\Leftrightarrow& \tau_g(\tau_f(i)) = j
 \end{aligned}\\
 3) $\psi$ iniettiva\\
 supponiamo che $\tau_g = \tau_f$\\
 $gHg^{-1} = fHf^{-1} \ \ \forall H\in Syl_3(G)$\\
 $ \Rightarrow (f^{-1}g)H(f^{-1}g)^{-1} = H \ \ \forall H\in Syl_3(G)$ \\
 $ \Rightarrow f^{-1}g\in N_G(H) \ \ \forall H\in Syl_3(G)$ \\
 $ \displaystyle\Rightarrow f^{-1}g\in \bigcap_{H\in Syl_3(G)} N_G(H)$ \\
 $ \displaystyle\Rightarrow f^{-1}g\in \bigcap_{H\in Syl_3(G)} H = \{e\} \Rightarrow f^{-1}g = 3 \ \Rightarrow  \ f = g$\\
 Resta da verificare che $H = N_G(H)$ \\
 $4 = n_3 = [G:N_G(H)] \displaystyle = \frac { |G|}{|N_G(H)|} = \frac {12}{N_G(H)} \Rightarrow |N_G(H)| = 3$ \\
 Ma $H\leq N_G(H) \Rightarrow H = N_G(H)$
	  \end{dimo}
	  \subsection{Studiare gruppi di ordine 12 in cui $n_3 = 1$}
Da Sylow III Segue che $\exists ! Q\in Syl_3(G) \Rightarrow Q\normale G$ \\
Esiste in $G$ almeno un $2$-Sylow $P\leq G$\\
Ora:\\
 $\cdot G\normale G, \ \ P\leq G$\\
 $\cdot Q\cap P = \{e\}$ (perchè l' $MCD(|Q|,|P|) = 1$\\
 $\displaystyle\cdot |QP| = \frac{|Q||P|}{|Q\cap P|} = \frac{3\cdot 4}1 = 12$\\
  $ \Rightarrow QP = G$ \\
  Allora $G\cong Q\semi P$ per qualche\\
  $\phi: P \rightarrow Aut(Q)\cong C_2$\\
Quindi studiamo i possibili omomorfismi\\
$\phi : P \rightarrow Aut(C_3)\hfill$ se $P\cong C_4$\\
$C_4 = <\gamma> \ \ \ \ C_3 = <r>$\\
\begin{aligned}
	\phi: <&\gamma> \rightarrow Aut(C_3)\\
	      &\gamma \rightarrow (\phi_\gamma : r \rightarrow r^k \text{ con }$k \pm$ 1)
\end{aligned}
nel csao $k = 1$ abbiamo  $\phi$ banale\\
$ \Rightarrow$ prodotto diretto\\
$ \Rightarrow G\cong C_3\times C_4\cong C_{12}$ \\
nel caso $k = -1$\\
abbiamo  $G\cong C_3\semi C_4\cong Dic_3$\\
dove\\
\begin{aligned}
	\phi: &C_4 \rightarrow Aut(C_3)\\
	      &\gamma \rightarrow (\phi_\gamma:r \rightarrow r^{-1})
\end{aligned}\\
$P\cong K_4$ \\
\begin{aligned}
	$\phi : &K_4 \rightarrow Aut (C_3)$\\
	$&\{Id,a,b,ab\}$\\
	 &a \rightarrow (\phi_a : r \rightarrow r^{\pm 1})\\
	 &b \rightarrow (\phi_b : r \rightarrow r^{\pm 1})\\
	 &ab \rightarrow (\phi_{ab} : r \rightarrow r^{\pm 1})\\
\end{aligned}\\
Se $\phi$ è banale\\
$ \Rightarrow$ prodotto diretto\\
\begin{aligned}
	$\Rightarrow G&\cong C_3\times K_4$ \\
	$ &\cong C_3\times C_2\times C_2$ \\
	$ & \cong C_6\times C_2$
	
\end{aligned}\\
Se $\phi$ è non banale, a meno di rinominare gli elementi $\{a,b,ab\}$ avremo che \\ \begin{aligned}
	\text{    \hspace{20px}} &  $\phi_a r\rightarrow r$\\
	&$\phi_b r\rightarrow r^{-1}$\\
	&$\phi_{ab} r\rightarrow r^{-1}$
\end{aligned}
Grazie (!) a Esercizio 1 di scheda 7 tutti i restanti prodotti semidiretti sono isomorfi\\
$G\cong C_3\semi K_4\cong D_6$\\
Infatti $|D_6| = 12$\\
$D_6$ non è isomorfo ad alcuno dei precedenti casi\\
$1) C_2$ è ciclico\\
$2) C_6\times C_2$ è abeliano, ma non ciclico\\
$3) A_4$ unico caso in cui $n_3 = 4$\\
$4) Dic_3$ non è abeliano e contiene elementi di ordine 4\\
 $5) D_6$ non è abeliano e non contiene elementi di ordine 4 ($C_4)$ \\
 %TODO sei arrivato in ritardo! guarda le foto da RIcordo Lagrange (Forse no)
 \subsection{Radici primitive}
 \begin{defi}[Radice primitiva modulo (n)]
	 Un intero $a$ si definisce radice primitiva modulo (n) se $ord_{U_n}([a]) = \phi(n)$
 \end{defi}
 \textbf{Osservzaione:}\\
 Per teorema di Eulero\\
 \[
	 a^{\phi(n)}\equiv_n 1
 .\] 
 $ \Rightarrow ord_{U_n}([a]) = \phi(n)$ \\
 \textbf{Osservazione}\\ $a$ radice primitiva mod (n) significa che $U_n = <[a]>$ \\
 \textbf{Obiettivo} (Scheda 7)\\
 Dimostrare che se $ p > 1$ primo allora $\exists $ radice primitiva modulo $(p)$\\
  \textbf{Esempi}\\
  Non esistono radici primitive mod(8)\\
  Studio $U_8 = \{[1],[3],[5],[7]\}$
   \[
  \phi(8) = 2^3 - 2 ^2 = 4
  .\] 
  \begin{aligned}
	1^2\equiv_8  1\\
  	3^2\equiv_8  1\\
  	5^2\equiv_8  1\\
  	7^2\equiv_8  1
  \end{aligned}\\
  \textbf{Es(ercizio esempio)}\\
  $3$ è radice primitiva mod(7)\\
  \textbf{Svolgimento:}\\
  $3^1 \equiv_7 3$\\
  $3^2 \equiv_7 2$\\
  $3^3 \equiv_7 1$\\
  $3^4 \equiv_7 3$\\
  $3^5 \equiv_7 2$\\
  $3^6 \equiv_7 1$\\
  \hline \ \\
  $2$ è radice primitiva mod(9)\\
  \textbf{Da fare}\\
  \textbf{Esercizio}(Scheda 7)\\
  Dimostrare che \\
  $Aut(C_p)\cong C_{p-1}$\\
   \textbf{Soluzione}\\
   Sappiamo che\\
   $Aut(C_p)\cong U_p\cong C_{\phi(p)}\cong C_{p-1}$\\
   \textbf{Esercizio}\\
   $p$ primo\\
   $f(x) = a_nx^n + \ldots + a_1x + a_0$\\
   $f(x)\equiv_p 0$ ammette al più $p$ soluzioni distinte in  $\Z / (p)$\\
   \begin{dimo}
   	per induzione su $n$ \\
	se $n = 1$  $ \Rightarrow a_1x\equiv_p -a_0$ \\
	$ \Rightarrow x \equiv_p = -a\cdot a_1^{-1}$\hfill $a_1$ invertibile in $\Z/(p)$ per ipotesi \\
	$n > 1$ \\
	Se $f(x) \equiv_p 0$\\
	non ammette soluzioni ok \\
	Se invece a è soluzione dividiamo\\
	$f(x) = (x - a ) q(x) + r$\\
	 $ \Rightarrow f(x)\equiv_p (x-a)q(x) + r$ \\
	 Valuto in $a$:\\
	 $ \Rightarrow 0\equiv_p f(a) \equiv_p (a-a)q(a) + r$ \\
	 $ \Rightarrow f(x) \equiv_p (x - a)q(x)$\\
	 Sia  $b\not \equiv_p a$ tale che  $f(b)\equiv_p 0$\\
	  $0\equiv_p f(b) \equiv_p (b-a)q(b)$\\
	  $\Z/(p)$ dominio d'integrità\\
	  $q(b)\quiv_p 0$\\
	  Ma per induzione  $q(x)\equiv_p 0$\\
	  ammette al più  $n-1$ soluzioni distinte\\
	  $ \Rightarrow  f(x)\equiv_p 0$ ammette al più n soluzioni 

   \end{dimo}
	\subsection{Ricordo (Lagrange)}
	$f(x) = a_nx^n + \ldots + a_1 x + a_0\in \Z[x]$ tale che $a_n \not\equiv_p 0$ con  $p > 1$ primo \\
	Allora  $f(x)\equiv_p 0$ ammette al più  $n$ soluzioni
	\begin{coro}
	Dimostrare che se $p$ primo e $d | (p-1)$ allora  $x^d -1 \equiv_p 0$ ammette esattamente  $d $ soluzioni
\end{coro}
\begin{dimo}[Soluzione]
	Abbiamo che se $d  (p-1)$ allora $(x^d -1) | (x^{p-1} - 1)$\\
	$  \Rightarrow x^{p-1} = (x^d-1)f(x)$\\
	dove $f$ è di grado $(p-1-d)$ \\
	Ora  $x^{p-1}\equiv_p 1$ ammette  $p-1$ soluzioni distinte per il piccolo teorema di Fermat. Le soluzioni sono  $1,2,\ldots, p-1$\\
	Se una di tali soluzioni non risolve  $f(x)\equiv_p 0$ allora risolve  $x^d-1\equiv_p 0$ (Sto usando il fatto che  $\Z/(p)$ è un dominio d'integrità [prodotto commutativo e se il prodotto tra due numeri è 0 allora o uno o l'altro sono 0])\\
	Dato che $ f(x)\equiv_p 0$ ammette al più $p-1-d$ soluzioni distinte deduciamo che $x^d - 1\equiv_p$ ammette almeno $d = (p-1)-(p-1-d)$ soluzioni distinte in  $\Z/(p)$.\\
	D'altra parte per l'esercizio precedente ne ammette al più $d$, e quindi segue la tesi.
\end{dimo}
\begin{coro}[Esercizio]
	$p> 1$ primo, $d | (p-1)$ Allora, esistono esattamente $\phi(d)$ interi, distinti in $U_p$, di ordine $d$ in $U_p$
\end{coro}
\begin{dimo}[Soluzione]
	Introduco $S_d = \{k\in \Z | ord_{U_p}([k]) = d, \ \ 1\leq k\leq p-1\}$\\
	La tesi è equivalente a dimostrare che  $|S_d| = \phi(d)$\\
	Abbiamo una partizione  $\displaystyle\{1,\ldots, p-1\}=\bigcup_{d|p-1}S_d$\\
	Quindi $\displaystyle p-1 = \sum_{d|(p-1)}|S_d|$\\
	Ricordo:\\
	$n = \sum_{d|n} \phi(d)$ (esercizio delle vecchie schede)\\
	Scegliendo $n = p-1$ deduciamo\\
	$\displaystyle \sum_{d|p-1}|S_d| = \sum_{d|p-1} \phi(d)$\\
	Basta allora dimostrare che $|S_d|\leq \phi(d) \ \ \forall d|p-1$\\
	Se $S_d = \emptyset$ $ \Rightarrow |S_d| = 0\leq \phi(d)$ \\
	Se $S_d\neq \emptyset \ \Rightarrow \exists a\in S_d$\\
	$ \Rightarrow \{a, a^2, a^3, \ldots, a^d\}$ sono tutti distinti  $mod(p)$ infatti \\
	\begin{center}
		
	 \begin{aligned}
		 a^i &\equiv_p a^k \\
		     &\storto \Leftrightarrow\\
		 i &\equiv_d j
	 \end{aligned}\\
	\end{center}
	 Quindi $a,a^2,\ldots, a^n$ sono tutte e sole le soluzioni di  $x^d - 1\equiv_p 0$ Quindi gli elementi di ordine  $d$ in $U_p$ sono della forma $a^j$ per qualche $j\in \{1,\ldots, j\}$\\
	 Ma  $ord([a^j]) = \frac{d}{MCD(j,d)}$ (esercizio di una riga) \\
	 Quindi $|S_d| = \phi(d)$

\end{dimo}
\begin{coro}[Esercizio]
	$p > 1 $ primo:\\
	Allora esistono esattamente $ \phi(p-1)$ radici primitive distinte
\end{coro}
\begin{dimo}[Soluzione]
	Basta applicare l'esercizio precedente, scegliendo $d = p-1$
\end{dimo}
\hline\ \\
\textbf{Esercizio}\\
$p> 1$ primo\\
dimostrare che  $Aut(C_p)\cong C_{p-1}$\\
\textbf{Soluzione:}\\
Sappiamo che $Aut(C_p)\cong U_p\cong C_{p-1}$\\
Dove la prima congruenza la sappiamo da teoremi precedenti, la seconda viene data dal precedente corollario\\
\begin{congettura}[Gauss, 1801]
	Esistono infiniti primi per cui $10$ è una radice primitiva
\end{congettura}
\begin{congettura}[E. Artin, 1927]
	$a\in\Z$,  $a\neq \pm 1$\\
	Assumiamo che  $a$ non sia un quadrato perfetto, Allora esistono infiniti primi per cui $a $ è una radice prima
\end{congettura}
\textbf{Osservazione}\\
Oggi sappiamo che la congettura di Artin è vera per infiniti interi $a$, ma non è noto quali
\textbf{Esercizio:}
$p > 1$ primo\\
Sia  $a = x^2$ con  $x\in\Z$\\
Dimostrare che se  $[a]\in U_p$\\
allora  $ord_{U_p}([a]_\neq p- 1$\\
\textbf{Esercizio} [classificazione dei gruppi di ordine $pq$]\\
Dimostrare che tutti i gruppi non ciclici di ordine $pq$ con  $p\neq q$ primi, sono fra loro isomorfi e non abeliani\\
\textbf{Soluzione}\\
Dato $G$ tale che $|G| = pq$
Avevamo dimostrato che $\exists ! Q\in Syl_q(G) \ \Rightarrow \ Q\normale G$\\
Inoltre $\exists P\in Syl_p(G) \Rightarrow P\leq G$  \\
Abbiamo verificato che:\\
$\cdot P\cap Q = \{e\}$\\
 $\cdot |PQ| = |G| \Rightarrow PQ = G$ \\
 $ \Rightarrow G\cong Q\semi P \cong C_q\semi C_p$\\
 dove $\phi : C_p \rightarrow Aut C_q\cong C_{q-1}$\\
 $\cot$ se $p\not | q-1 \Rightarrow \phi$ è banale $ \Rightarrow G\cong C_q\times C_p\cong C_{pq}$\\
 $\cdot$ se $p | q-1 \Rightarrow \phi$ potrebbe essere non banale $ \Rightarrow ord_{Aut(C_q)}(\phi_P) = p \Rightarrow Im(\phi)\subseteq Aut(C_q)\cong C_{q-1}$ con $|Im(\phi)| = p$\\
 Sappiamo che  $C_{q-1}$ contiene un unico sottogruppo di ordine $p$ $ \Rightarrow Im(\phi)$ non dipende da $\phi$ (a meno che $\phi$ non banale)\\
 $ \Rightarrow $ A meno di "precomporre" $\phi$ con un automorfismo di $C_q$ la mappa $C_p \rightarrow Aut(C_q)\cong C_{q-1}$ è univocamente determinata\\
 \textbf{Concretamente:}\\
 Dati $\phi,\phi': C_p \rightarrow Aut(C_q)$ non banali $ \Rightarrow $ esiste $B\in Aut(C_p)$ tale che $\phi' = \phi\cdot B \Rightarrow C_q\semi C_p\cong C_q\semi C_p$ quindi esiste un'unica classe d'isomorfismo non ciclica\\
 \newpage
 \subsection{Successioni esatte corte}
 \textbf{Esercizi [Scheda 9]}\\
 \begin{defi}
	 Una successione esatta corta di gruppi è una coppia di omomorfismi\\ $H \xrightarrow r G \xrightarrow \pi K$ dove $r$ iniettivo $\pi$ suriettivo e $Im(r) = ker(\pi)$
	 \begin{itemize}
		 \item $G$ si dice estensione di $K$ tramite $H $
		 \item la successione spezza se $\exists s:K \xrightarrow{} G$ omomorfismo tale che  $\pi\cdot s = Id$
		 \item $S$, se esiste, si chiama sezione
	 \end{itemize}
 \end{defi}
	 \textbf{Esempi}\\
	 Costruire una successione esatta corta (SEC) di $Q_8$ che estende $K_4$ tramite $C_2$\\
	 \textbf{Soluzione}\\
 \[\{Id,\rho\} = C_2 \xrightarrow r Q_8 \xrightarrow \pi K_4\]\\
 $r$ per essere iniettiva deve mandare $\rho$ che è di ordine 2 in un elemento di ordine 2.\\
 $ord(r(\rho)) = 2 \Rightarrow r(\rho) = -1$ 
 \begin{aligned}
 	Id \rightarrow 1\\
	\rho \rightarrow -1
 \end{aligned}\\
 Considero la proiezione al quoziente $Q_8 \rightarrow Q_8/\{\pm 1\} \cong K_4$\\
 $ \Rightarrow $ basta prendere $\pi: Q_8 \rightarrow Q_8/ \{\pm 1\}\cong K_4$\\
 2) Non spezza!:\\
 Se spezzasse dato che una sezione è necessariamente iniettiva (esercizio), ma non esistono omomorfismi iniettivi da $K_4$ in $Q_8$\\
 3) \begin{aligned}
	 $\Z \xrightarrow r&\R \rightarrow S^1\leq C^* $\\
	 $ n \rightarrow &2\pi n\\
			 &\theta \rightarrow e ^{i\theta}$
 \end{aligned} \\è una SEC che non spezza
	\begin{defi}[Spezza]
		Una successione esatta corta $H \rightarrow G \rightarrow K$ spezza se $\exists S: K \rightarrow G$ omomorfismo t.c. $\pi\circ S = Id_K$
	\end{defi}
	\textbf{Osservzione}\\
	Una sezione è iniettiva\\
	\textbf{Esempio:}\\
	$H,K$ gruppi $G:= H\semi K$\\
	per qualche  $\phi : K \rightarrow Aut(H) \Rightarrow $ \\
	\begin{aligned}
		$ H \xrightarrow r &H\semi K \xrightarrow \pi K \\
		h \rightarrow &(h,e_K)\\
			      & (h,k) \rightarrow k$
	\end{aligned} è una SEC che spezza\\
	$\cdot r$ è iniettiva\\
	$\cdot \pi$ è suriettiva\\
	$\cdot Im(r) = \{(h,e_K) | h\in H\} = ker(\pi)$\\
	 $\cdot$ spezza perchè \begin{aligned}
		 $S: &K \rightarrow H\semi K$
		     &k \rightarrow (e_K, k)
	 \end{aligned}\\
	 è una sezione:\\
	 \[
		 (\pi\cdot S)(k) = \pi(e_H,k) = k \ \ \ \forall k\in K
	 .\] 
	\textbf{Esercizio scheda 9}\\
	Data una SEC $H \xrightarrow r G \xrightarrow K$ con $S:K \rightarrow G$ che  spezza $ \Rightarrow G\cng H\semi K$ \\
	\textbf{Soluzione:}\\
	Osservo che:\\
	$\cdot r(H)\leq G \leadsto \cdot r(H) = ker(\pi)\normal G$\\
	 $\cdot S(K)\leq G \leadsto \cdot r(H)\cap S(K) = \{e_G\}$\\
	 $  \Rightarrow $ Sia $x\in r(H)\cap S(K) \Rightarrow \exists h\in H, \exists k\in K$\\
	 $t.c. x = r(h) = S(k)$\\
	 Applicando $\pi:$\\
	 $e_K = \pi(r(h)) = \pi(S(k)) = k \ \Rightarrow \ x = S(k) = S(e_K) = e_G$ \\
	 $\cdot r(H)\cdot S(K) = G$\\
	 $g\in G \leadsto \pi(g)\in K \leadsto f = S(\pi(g))\in S(K)\leq G$ \\
	 Vorremmo ora scrivere $g$ come un'elemento in $r(H)$ per $f$\\
	 Basta quindi mostrare che  $gf^{-1}\in Im(r)$ ma  $Im(r) = ker(\pi)$\\
	 Applicando  $\pi: \pi(gf^{-1}) = \pi(g)\pi(f^{-1}) = \pi(g)\pi(S(\pi(g^{-1}))) = \pi(g)\cdot(\pi\circ S)(\pi(g^{-1}))= \pi(gg^{-1}) = e_K$\\
	 Sapendo che  $f^{-1} = (S(\pi(g)))^{-1}$ e che $(\pi\circ S) = Id_K$\\
	 Quindi  $gf^{-1}\in ker(\pi) = Im(r) \Rightarrow \exists h\in H$ t.c. $gf^{-1} = r(h) \Rightarrow g = r(h)g = r(h)S(\pi(g))$ \\
	 \text{ }\storto \ni \ \ \ \ \ \storto\ni\\
	 $Im(r) \  Im(S)$ \\
	 $\cdot$ Deduciamo che $G\cong r(H)\semi S(K)\cong H\semi K$\\
	 poiché  $r$ e $S$  iniettive $ \Rightarrow H\cong r(H)$ e $ K\cong S(K)$
	 \subsection{Quaternioni}\\
	 $i^2 = j^2 = k^2 = ijk = -1$\\
	 $\mathbb H = \{a + bi + cj + dk | i^2 = j^2 = k^2 = -1, ijk = -1, a,b,c,d\in \R\}$\\
	 è uno spazio vettoriale di dimensione 4. \\
	 Dalla scheda 9 segue che  $\mathbb H^* = \mathbb H\setminus\{0\}$ è un gruppo moltiplicativo.\\
	 \begin{defi}
	 	$n\geq 2 \ \ Dic_n:= <a,j> \leq \mathbb H^*$ dove  $a = \cos(\frac \pi n) + i\sin(\frac \pi n)\in \mathbb H^*$
	 \end{defi}
	 \textbf{Osservazione}\\
	 $(a)$ è un gruppo ciclico di ordine $2n$\\
	  \textbf{Osservazione}\\
	  $n = 2 \leadsto a = \cos(\frac \pi 2 ) + i \sin (\frac \pi 2) = i \ \ \Rightarrow  \ \ Dic_2 = <i,j> = \{\pm 1,\pm i, \pm j,\pm k\} = Q_8$ \\
	  \subsection{Gruppi diciclici}\\
	  $Dic_n = <a,j>\leq\mathbb H^*$\\
	  1)  $ord(a) = 2n \ \ \ ord(j) = 4$\\
	  2) Mostrare  $j^2a^m = a^m + n = a^m j^2$\\
	   \textbf{Soluzione}\\
	   $j^2 = -1$ e $a^n = -1$ tutti i membri delle uguaglianze sono quindi  $-a^m$\\
	   3) Mostrare  $j^{\pm 1} a^m = a^{-m} j^{\pm 1}$\\
	   \textbf{Soluzione }\\
	   $j^-1 = -j$\\
	   $ ja^m = j(cos(\frac{m\pi}{n}) + i\sin(\frac{m\pi}n) = \cos(\frac{m \pi}{n}))= \cos (\frac{-m\pi}{n}) + i\sin(-\frac{m\pi}{n}) = a^{-m}j\\
	   \Rightarrow j a^m - a^{-m} j \Rightarrow -ja^m = a^{-m}(-j) \Rightarrow j^{-1} a^m = a^{-m}j^{-1}$ \\
	   6) Mostrare che ogni elemento in $Dic_n$ può scriversi come $a^m j^k$ con  $0\leq m < 2n$  $0\leq j\leq 1$ segue dalle relazioni precedenti  $ \Rightarrow Dic_n = \{a^m | 0\leq m < 2n\}\cup \{a^m_j | 0\leq m < 2n\}$ \\$ \Rightarrow (6): |Dic_n| = 4n$ \\
	   $8) $ Mostrare che esiste una SEC\\
	   $C_{2n} \xrightarrow{r} Dic_n \rightarrow {\pi} C_2$\\
	   $\rho \rightarrow a$\\
	   $\cdot r(\rho) L= a \Rightarrow r$ iniettiva\\
	   $\cdot \pi : Dic_n \rightarrow C_2$\\
	   Vorrei verificare proiezione al quoziente.\\
	   In effetti  $r(C_{2n}) = <a>\normale Dic_n$ perchè\\
	   $[Dic_n : <a>] = 2$ \ \ \begin{aligned}
		   $\pi : &Dic_m \rightarrowC_2 = <\sigma>\\
			  &a^m \rightarrow e\\
			  a^m j \rightarrow \sigma$

	   \end{aligned} 
	   9) Mostrare che \underline {non} si spezza 
	   \textbf{Soluzione}\\
	   Mi chiedo se esiste una sezione $S: C_2 \rightarrow Dic_n$\\
	   Se $S$ esiste allora $S(\sigma) =a^m j$ per qualche $0\leq m < 2n$\\
	   $ord(a^mj)= 4 \leadsto (a^mj)(a^mj) = a^{m-m}j\cdotj = j^2 = -1\\
	    \Rightarrow ord(S(\sigma))\neq ord(\sigma) \Rightarrow $ assurdo\\
	    10) Mostrare che esiste unna SEC\\
	    $ C_n \xrightarrow r Dic_n \xrightarrow \pi C_4$ ds n dispari:\\

	    \begin{tikzpicture}

% Disegna la circonferenza
\draw[thick] (0,0) circle(1);
\draw[thick, ->] (-1.5,0) -- (1.5,0) node[below right] {\(x\)};
\draw[thick, ->] (0,-1.5) -- (0,1.5) node[above right] {\(y\)};
% Definizione degli angoli per i tre raggi
\foreach \angle/\label in {15/a, 45/a^2, 75/a^3} {
    % Disegna il raggio
    \draw[thick] (0,0) -- (\angle:1);
    % Aggiungi l'etichetta
    \node at (\angle:1.2) {\(\label\)};
}
\draw[thick] (0,0) -- (180:1)\\
	\node at (-1.8,0.3) {\(-1=a^m\)}

\end{tikzpicture}\\
$C_n = <\rho> \xrightarrow r Dic_n\\
\text{    }\ \ \ \ \ \ \ \ \ \ \ \rho \rightarrow a^2$\\
$\pi : Dic_n \rightarrow C_4 = <r> \ \ \pi(a^m) =$ \begin{cases}
	Id \ \ \text{se }m\equiv_2 0\\
	r^2 \ \ \text {se }m\equiv_2 1
\end{cases}\\
 $\pi(a^mj) =$ \begin{cases}
	rm \ \ \text{se }m\equiv_2 0\\
	r^3m \ \ \text {se }m\equiv_2 1
\end{cases}\\
\textbf{Osservazione}\\
$r^2 = \pi(j^2) = \pi(a^n) = \begin{cases}
	Id \ \ \text{se } n \text{ pari}\\
	r^2 \ \ \text{se } n \text{ pari}\\
\end{cases}$\\
2) $n \geq 3$ dispari\\
Dimosrtare che $Dic_n\cong C_n\semi C_4$ per qualche $\phi:C_4 \rightarrow Aut(C_n)$\\
\textbf{Soluzione:}\\
Costruiamo $S: C_4 \rightarrow Dic_n$\\
$\cdot$ dobbiamo solo definire $S(r) = j$\\
$\cdot \ S$  omomorfismo\\
$\cdot \pi\circ S(r) = \pi (j) = r$\\
 \begin{defi}
	 Un gruppo $G$ si dice semplice se i suoi unici sottogruppi normali sono $\{e\}$ e $G$\\
\end{defi}
\textbf{Esempio:}\\
$\cdot Q_8$ non è semplice\\
$\cdot A_3\cong C_3$ è semplice\\
$\cdot A_4$ non è semplice;\\
Ricordo:\\
per $A_4$ sia ha $n_3 = 4$ e $n_2 = 1 \Rightarrow A_4$  contiene un unico $2$-Sylow ("sottogruppo di oridne 4") che quindi è normlae
\[
	V = \{ Id, (12)(34), (13)(24), (14)(23)\} \leadsto V\normale A_4
.\] 
\newpage
\begin{prop}
	$A_n$ è semplice $\forall n\geq 5$\\
\end{prop}
	 $\cdot$ Strategia: Vogliamo procedere per passi dimostrando che:\\
	 1) $\{e\}\neq H\normale A_n \Rightarrow H$ contiene un $3$-ciclo\\
	 2) Se $H$ contiene un $3$-ciclo $ \Rightarrow $ li contiene tutti\\
	 3)  $A_n$ con  $n\geq 5$ è generato dai  $3$-cicli\\
	 
\begin{lemm}
	$\{e\}\neq H\normale A_n$ Allora\\
	 $H$ contiene almeno un $3$-ciclo oppure (almeno un prodtotto di trasposizioni disgiunte)
\end{lemm}
\begin{dimo}
	Sia $\sigma\in H, \sigma\neq Id \Rightarrow \sigma = \sigma_1\circ\sigma_2\circ\ldots\circ\sigma_k$ \\
	con $\sigma_i$ cicli disgiunti.\\
	Caso I: $\sigma_1$ è $m$ ciclo con $m\geq 4$  $\sigma_1 = (a_1a_2a_3\ldots)$\\
	$\tay:= (a_1a_2a_3)\sigma(a_1a_2a_3)^{-1}\in H \Rightarrow \sigma\tau^{-1}$\in H\\
	$ \Rightarrow \sigma\tau^{-1} = \sigma (a_1a_2a_3a)\sigma^{-1}(a_1a_2a_3)^{-1} = (\sigma(a_1)\sigma(a_2)\sigma(a_3))$ \\
	$= (a_2a_3a_4)(a_1a_3a_2)=(a_1a_4a_2)(a_3)\in H$\\
	Caso II $m = 3$ per casa\\
	Caso I : $m = 2$ per casa 
\end{dimo}
%lezione 19
\subsection{Gruppi semplici}
\begin{defi}[Gruppo Semplice]
		Un gruppo di dice semplice se gli unici sottogruppi normali sono banali
	\end{defi}
	\textbf{Obiettivo}\\
	Dimostrare $A_n$ è semplice per  $n \geq 5$\\
	\textbf{Osservazione:}\\
	$A_4$ non è semplice\\
	$A_2$ e $A_3$ sono semplici\\
	%TODO
	\textbf{Strategia}\\
	$n \geq 5$\\
	1)  $\{Id\} \neq H\normale A_n$ allora  $H$ contiene almeno un $3$-ciclo\\
	2) $\{Id\}\neq H\normale A_n$ se $H$ contiene un $3$-ciclo allora li contiene tutti\\
	3) $A_n$ è generato dai suoi $3$-cicli\\
	Ricordo:
	\newpage
	\begin{lemm}
		$n\geq 3 \ \ \{Id\}\neq H\normale A_n$\\
		Allora $H$ contiene almeno un $3$-ciclo oppure un prodotto di trasposizioni disgiunte
	\end{lemm}
	\begin{prop}
		$n\geq 5, \ \ \{Id\}\neq H\normale A_n$ allora $H$ contiene almeno un $3$-ciclo
	\end{prop}
	\begin{dimo}
		Basta verificare che se $\sigma = (a_1a_2)(a_3a_4)\in H, $ allora esiste un $e$-ciclo in $H$.\\
		Dato che  $H\normale A_n$ abbiamo 
		\[
			gHg^{-1}\subseteq H \ \ \forall g\in A_n
		.\] 
		Definiamo $a_5 \not\in \{a_1,a_2,a_3,a_5\}$\\
		$\tau := (a_3a_4a_5)\sigma(a_3a_4a_5)^{-1}\in H$\\
		$ \Rightarrow \sigma \tau^{-1}\in H$ Studiamo $\sigma\tau^{-1}$\\
		$ \Rightarrow \sigma\tau^{-1} = \sigma(a_3a_4a_5)\sigma^{-1}(a_3a_4a_5)^{-1}$ \\
		Dove  $\sigma(a_3a_4a_5) = (\sigma(a_3)\sigma(a_4)\sigma(a_5))$\\
		$\sigma\tau^{-1} = (a_4a_3a_5)(a_3a_5a_4) = (a_3a_4a_5)\in H$
	\end{dimo}
	\begin{teo}
		$n\geq 5$ $\{Id\}\neq H\normale A_n$\\
		Allora  $H$ contiene tutti i $3$-cicli\\
	\end{teo}
	\begin{dimo}
		Basta verificare che dato \\
		$\sigma = (a_1a_2a_3)\in H$\\
		Allora $H$ contiene tutti i $3$-cicli\\
		Sfruttiamo $H\normale A_n$\\
		$ \Rightarrow \tau = (a_3a_4a_5)\sigma(a_3a_4a_5)^{-1}$ \\
		$\tau\in H$\\
		dove  $a_4,a_5\not\in\{a_1,a_2,a_3\}$\\
		Studiamo $\tau:$\\
		$\tau = (a_3a_4a_5)(a_1a_2a_3)(a_3a_4a_5)^{-1}$ 
		$= (a_1a_2a_4)\in H$\\
		Abbiamo dimostrato che se $(a_1a_2a_3)\in H$ allora $(a_1a_2a_4)\in H \ \ \forall a_4\not\in\{a_1,a_2\}$\\
		Dunque mostriamo che il $3$-ciclo arbitrato $(b1,b_2,b_3)\in H$ per qualunqe $b_1,b2,b_3$\\
		$(a_1a_2a_3)\in H$\\
		$ \Rightarrow (a_1a_2a_3)\in H$\\
		$ \Rightarrow  (b_1b_2b_3)\in H$ \\

	\end{dimo}
	\begin{coro}
		$n\geq 5 \ A_n$ è semplice
	\end{coro}
	\begin{dimo}
		Sia $\{e\}\neq H\normale A_n$, dimostriamo che $H=A_n$\\
		Per il teorema  $H$ contiene tutti i $3$-cicli, quindi basta verificare che $A_n$ è generato dai  $3$-cicli, Sia $\sigma\neq Id, \sigma\in A_n\subseteq S_n$\\
Ricordando che  $S_n$ è generato da trasposizioni\\
$ \Rightarrow \sigma = \tau_1\tau_2\ldots\tau_{2i-1}\tau_{2i}\ldots\tau_{2k-1}$ \\
L'idea è verificare che $\tau_{2i-1}\tau_{2i}$ si ottiene come prodotto di  $3$-cicli $\forall i\in \{1,\ldots,k\}$\\
Caso 1  $\tau_{2i-1} = \tau_{2i}$\\
$\tau_{2i-1}\tau_{2i} = Id = (123)(132)$\\
Caso 2  $\tau_{2i-1}=\tau_{2i}$\\
hanno un indice in comune\\
Allora:\\
$\tau_{2i-1} = (ab)\\
\tau_{2i} = (bc)$\\
$ \Rightarrow \tau_{2i-1}\tau_{2i} = (ab)(bc) = (abc)$ \\
Caso 3:\\ $\tau_{2i-1},\tau_{2i}$ non hanno indici in comune.\\
$ \Rightarrow \tau_{2i-1} = (ab), \ \tau_{2i} = (cd)$ \\
$\tau_{2i-1}\tau_{2i} = (ab)(cd)$\\
Ma \\
$(abc)(bcd) = (ab)(cd)$\\
Quindi:
$\tau_{2i-1}\tau_{2i} = (abc)(bcd)$\\
Allora  $\sigma$ è prodotto di $3$-cicli $ \Rightarrow \sigma \in H \Rightarrow H = A_n$
	\end{dimo}
	\textbf{Esercizio}\\
	$n\geq 5$ dimostrare che gli unici sottogruppi normali di $S_n$ sono $\{e\}, A_n, S_n\{e\}, A_n, S_n$
	\textbf{Soluzione}\\
	Osserviamo che se $ H\normale S_n$ allora  $H\cap A_n\normale A_n$ poichè  $H\normale S_n$ significa \\
	$gHg^{-1}\subseteq H \ \ \ \forall g\in S_n$\\
	Quindi  $\{Id\}\neq H\normale S_n$\\
	Studio  $H\cap A_n$\\
	1)  $H\subseteq A_n$\\
	 $ \Rightarrow H = H\cap A_n\trianglelefteq A_n$ \\
	 $ \xrightarrow {A_n \text{semplice}} H = \{Id\}$ oppure $H = A_n$ \\
	 2) $H\not\subseteq A_n$\\
	 $ \Rightarrow [H:H\cap A_n] = 2$ e $H\cap A_n\normale A_n$\\
	 $\xRightarrow{A_n\text{ semplice}} H\cap A_n = \{Id\}$ oppure $H\cap A_n = A_n$\\
	 Se  $H\cap A_n = \{Id\}$\\
	 \begin{aligned}
		 &\Rightarrow [H:H\cap A_n] = 2\\
		 & \Rightarrow |H| = 2\\
		 & \Rightarrow  H = \{Id,\sigma\}\\

	 \end{aligned}
	 con $ord(\sigma) = 2$\\
	 Se tale  $H$ fosse normale allora avremmo\\
	 $g\sigmag^{-1} = \sigma \ \ \forall g\in S_n$\\
	  $ \Rightarrow $ Assurdo perchè $\sigma $ è coniugato a tutti gli elementi con la sua stessa struttura ciclica.
	   $\cdot $ Allora $H\cap A_n = A_n.$\\
	   $ \Rightarrow [H:H\cap A_n] = 2$ \\
	   $ \Rightarrow  |H| = n! \Rightarrow  H = S$ \\
	   ricordando che $H\cap A_n = A_n$\\
	   \subsection{Classi di coniugio in A_n}
	   Obiettivo:\\
	   Studiare le azioni
	   \[
	    \begin{aligned}
		    S_n &\times A_n \rightarrow A_n\\
		    (\tau &,\sigma) \rightarrow\tau\sigma\tau^{-1}
	   	
	   \end{aligned} \ \ 
	   \vline \ \ \ 
	   \begin{aligned}
		   A_n&\times A_n \rightarrow A_n\\
		   (\tau&,\sigma) \rightarrow\tau\sigma \tau^{-1}
	   \end{aligned}
   \]
   \textbf{Ricordo:}\\
   Data $\sigma\in A_n$\\
   $O_\sigma^{S_n} = \{$ permutazioni con la stessa struttura ciclica di $\sigma\}\\$
   Domanda:  $O_\sigma^{A_n} = ?$\\
   A priori abbiamo  $O_\sigma^{A_n}\subseteq O_\sigma^{S_n}$\\
    \textbf{Esempio:} n = 3\\
    $O^{S_3}_{(123)} = \{(123),(132)\}$\\
    infatti $(23)(123)(23)^{-1} = (132)$\\
    $A_3 = \{Id,(123),(132)\}$\\
    $O^{A_3}_{123} = \{(123)\}$\\
    \textbf{Ricordo:}\\
    Data $\sigma\in A_n$\\
    $\cdot C_{A_n}(\sigma) = \{\tau\in A_n|\tau\sigma\tau^{-1}\} = Stab^{A_n}_\sigma$\\
    $\cdot C_{S_n}(\sigma) = \{\tau\in S_n | \tau\sigma\tau^{-1} = \sigma\} = Stab_\sigma^{S_n}$\\
     \textbf{Osservazione}\\
     $C_A = (\sigma) = C_{S_n}(\sigma)\cap A_n$\\
      \begin{teo}
     	$n\geq 2 \ \ \sigma \in A_n$ \\
	1) Se $C_{S_n}(\sigma )\not\subseteq A_n$ allora $O_\sigma^{A_n} = O_\sigma^{S_n}$\\
	2) $C_{S_n}(\sigma)\subseteq A_n$ allora $|O_\sigma^A_n| = \frac 12 |O_\sigma^{S_n}$
\end{teo}
	 \begin{dimo}
		 Supponiamo che $C_{S_n}(\sigma)\not\subseteq A_n$\\
		 Allora  $C_{S_n}(\sigma)\leq S_n$\\
		 $|C_{S_n}(\sigma):C_{S_n}(\sigma)\cap A_n] = 2$\\
		 notando che $C_{S_n}(\sigma)\cap A_n = C_{A_n}(\sigma)$\\
		 $ \Rightarrow |C_{A_n}(\sigma)| = \frac 12 |O_{S_n}(\sigma)|$ \\
		 $ \Rightarrow \begin{cases}
			 n! = |S_n| = |C_{S_n}(\sigma)|\cdot |O_\sigma^{S_n}|\\
			 \frac{n!}2 = |A_n| = |C_{A_n}(\sigma)|\cdot|O_\sigma^{A_n}
		 \end{cases}$ \\
	 $ \Rightarrow |O_\sigma^{S_n}| = |O_\sigma^{A_n}$\\
	 $ \Rightarrow O_\sigma ^{S_n}= O_\sigma^{A_n}$ \\
	 2) Se $C_{S_n}(\sigma)\subseteq A_n$\\
	 $ \Rightarrow C_{S_n}(\sigma) = C_{A_n}(\sigma)$\\
	 $ \Rightarrow \begin{cases}
		 n! = |S_n| = |C_{S_n}(\sigma)|\cdot|O_\sigma^{S_n}|\\
		 \frac{n!}2 = |A_n| = |C_{A_n}(\sigma)|\cdot|O_\sigma^{A_n}|

	 \end{cases}$ \\
	 $ \Rightarrow |O_\sigma ^{A_n}| = \frac 12 |O_\sigma^{S_n}|$
		
	\end{dimo}
	\textbf{Esempio:}\\
	$\sigma = (123) \ \ n = 5$\\
	$ \Rightarrow O^{S_n}_{(123)} = O^{A_n}_{(123)}$\\
	perché $(45)\in C_{S_n}(\sigma)$ MA $(45)\not\in A_5$\\
	\textbf{Esercizio}\\
	$\sigma = \sigma_1,\ldots, \sigma_k\in S_n$ disgiunti.\\
	$\sigma_i$ è $m_i$-ciclio\\
	1) se $\sum^k_{i=1}m_i\leq n-2$\\
	allora $O_\sigma^{S_n} = O_\sigma^{A_n}$ \\
\textbf{IDEA:}\\
dall'ipotesi segue che $\exists a,b\in\{1,\ldots,n\}$ tali che $\sigma(a) = a, \sigma (b) = b$\\
$ \Rightarrow (ab)\in C_{S_n}(\sigma)$ e $sgn(ab) = -1$
%Lezione 20
	\section{Gli anelli}
	\begin{defi}
		Un anello $(R$, $+$, $\cdot)$ è un insieme $R$ dotato di due operazioni, $+,\cdot$ che soddisfano le seguenti:
		\begin{enumerate}
			\item $(R, + )$ è un gruppo abeliano
			\item L'operazione $\cdot$ è associativa $(a\cdot b\cdot c = (a\cdot b)\cdot c = a\cdot (b\cdot c) \ \ \ \forall a,b,c\in R) $ 
			\item $\exists 1\in R $ tale che $1\cdot a  = a \cdot 1  = a \ \ \forall a\in R$ $(R$ è unitario$)$
			\item Vale la legge distributiva \\$ (a + b)\cdot c= a\cdot c + b\cdot c$ \\
				$c\cdot (a + b) = c\cdot a + c\cdot b \ \ \forall a,b,c\in R$
		\end{enumerate}
\textbf{Nota}\\
Artin richiede  anche la commutatività

	\end{defi}
	\begin{defi}
		Un anello $(\R,+,\cdot)$ si dice commutativo se 
		\[
		a\cdot b = b\cdot a \ \ \forall a,b\in R
		.\] 
	\end{defi}
	\newpage \ \\
	\textbf{Esempi}\\
	$1)(\Z, + , \cdot ) $ è un anello commutativo\\
	$2) Mat_{2\times 2}(\mathbb Q)$ è un anello non commutativo\\
	\begin{defi}[Dominiio d'integrità]
		Un dominio d'integrità è un anello commutativo tale che
		\begin{enumerate}
			\item $0\neq 1$
			\item  $\forall a,b\in R$ tale che $a\cdot b = 0$ si ha  $a = 0$ oppure $b = 0$
		\end{enumerate}\\
		$0$ denota l'elemento neutro del gruppo $(R,+)$\\
		Si dice che $R$ non ha divisori dello $0$
	\end{defi}
	\textbf{Esempio:}\\
	$R = \{e\}$\\
	 $e + e = e$\\
	  $e\cdot e = e$\\
	   $(R,+,\cdot)$ è un anello che soddisfa $0 =1$\\
	   Si chiama Anello Banale (Zero Ring) \\
	   \textbf{Esercizio}\\
	   $(R,+,\cdot)$ anello\\
	   $1)$ dimostrare che  \[a \cdot 0 =0\cdot a  = 0 \ \ \forall a\in R\]
   $2)$ dimostrare che  \[(-a)\cdot b = a\cdot (-b) = - (a \cdot b)\]\\
   $3)$ se $0 = 1$ allora  $R$ è l'anello banale (ovvero $|R| = 1$)\\
    \begin{defi}
   	$(R, + ,\cdot)$ anello.\\
	Un sottoanello di $R$ è un sottoinsieme $A\subseteq R$ tale che:\\
	\begin{enumerate}
		\item $(A,+) \leq (R,+)$
		\item $1\in A$
		\item $A$ è chiuso rispetto all'operazione  $\cdot$
	\end{enumerate}
   \end{defi}
\textbf{Esempi}\\
$M\geq 2$ intero\\
 $(\Z/(m), + )$ gruppo abeliano\\
$(\Z/(m), + , \cdot)$ è un anello commutativo\\
IN generale non è un dominio d'integrità.\\
Ad esempio se  $m=6 $\\
$[2][3]=[6]=[0]$\\
quindi  $[2]$ e $[3]$ sono divisori di $[0]$ in $\Z/(6)$
\begin{prop}
	$m\geq 2 $ è intero allora $\Z/(m)$ è un dominio d'integrità se e solo se $m$ è primo
\end{prop}
\begin{dimo}
	Se $m$ non è primo allora esistono $1<a,b < m$ tali che $m = a\codt b$\\
	Allora  $[a]\cdot[b] = [m] = [0]$ e  $[a]$ è un divisore dello zero\\
	Viceversa se $m$ è primo dobbiamo dimostrare che non esistono zero divisori \\
	Considero $[a]\in\Z/(m)$ con $[a]\neq [0]$\\
	Assumo che  $0< a < m$ \\
	Allora $MCD(a,m) = 1$\\
	$ \Rightarrow (a) + (m) = (1) =\Z$ \\
	$ \Rightarrow \exists k,h\in \Z$ tali che $ka + hm = 1$\\
	$ \Rightarrow [k]\cdot [a] = [1]\in \Z/(m)$\\
	Ora se esiste $[b]\in \Z/(m)$ tale che\\
	$[a]\cdot[b] = [0]$\\
	$ \Rightarrow [k]\cdot[a]\cdot[b] = [k]\cdot [0]\\
	\Rightarrow [b] = [0]\\
	\Rightarrow  [a]$  non è zero divisore
\end{dimo}
\textbf{Osservazione}\\
Abbiamo dimostrato che se $a\in R$ ammette un inverso moltiplicativo allora  $R$ è un dominio d'integrità (assumendo "solo" che $R$ sia anello commutativo)\\
\begin{defi}
	Un anello $(R,+, \cdot)$ si dice corpo se\\
	 $0\neq 1$\\
	 $\forall a\in R, \ \exists a^{-1}\in R$ t.c.\\
	 $a^{-1} \cdot a = a\cdot a^{-1} = 1$ 
	 $a^{-1}$ si dice inverso moltiplicativo
\end{defi}
\begin{defi}
	Un campo è un corpo commutativo
\end{defi}
\textbf{Osservazione}\\
Se $(R, +,\cdot)$ anello $a\in R$ che ammette inverso moltiplicativo  $a^{-1}\in R $ Allora $a$ non è zero divisore\\
Infatti se $\exists b \in R$ t.c.  $a\cdot b = 0$ \\
$0 = a^{-1}\cdot (a\cdot b) = a^{-1}\cdot a\cdot b = (a^{-1}\cdot a) \cdot b = b$\\
$1\cdot b = 0 \Rightarrow  b = 0$ \\
 $ \Rightarrow a$ non è divisore di $0$\\
  \begin{coro}
 	Ogni campo è un dominio d'integrità 
 \end{coro}
 \begin{dimo}
	 $\forall a\in R$ esiste  $a^{-1} \Rightarrow R$ dominio d'integrità
 \end{dimo}
 \textbf{Osservazione}\\
 \begin{document}

\begin{center}
	\begin{tikzpicture}

\node (field) at (0, 2) {CAMPO};
\node (ring) at (-2, -2) {ANELLO};
\node (integralDomain) at (2, 0) {DOMINIO D'INTEGRITÀ};
\node (commutativeRing) at (2, -2) {ANELLO COMMUTATIVO};
\node (corpo) at (-2, 0) {CORPO};

\draw[line width=1pt, double distance=3pt,
	arrows = {-Implies}] (field) -- (corpo);
 \draw[line width=1pt, double distance=3pt,
	arrows = {-Implies}](corpo) -- (ring) node[midway, above left] {};
 \draw[line width=1pt, double distance=3pt,
	arrows = {-Implies}](field) -- (integralDomain);
\draw[line width=1pt, double distance=3pt,
	arrows = {-Implies}](integralDomain) -- (commutativeRing);
 \draw[line width=1pt, double distance=3pt,
	arrows = {-Implies}](commutativeRing) -- (ring) node[midway, below left] {};

\end{tikzpicture}
\end{center}
\textbf{Esempio:}
1) $\mathbb H$ quaternioni è un corpo\\
infatti $i^2 = j^2 = k^2 = ijk = -1$  $q\in \mathbb H$ \\
$\leadsto q = x + yi + zj + wk\in \mathbb H, \ \ x,y,z,w\in \R$\\
 $\leadsto \overline q := x-yi - zj - wk$ (coniugato)\\
 $\leadsto |q|^2 = q\overline q = x^2 + y^2 + z^2 + w^2 $\\
 $\leadsto q\cdot \frac {\overline q}{|q|^2} = 1$
 quindi tutti invertibili (tranne 0)  $ \Rightarrow \mathbb H$ è un corpo
 \begin{prop}
 	Ogni dominio d'integrità finito è un campo
 \end{prop}
 \begin{dimo}
	 $(R,+,\cdot)$ dominio finito. Dato $a\in R\setminus\{0\}$ vogliamo dimostrare che esiste  $a^{-1}$\\
	  \textbf{Idea:}\\
	  considero la funzione 
	  \begin{aligned}
		  \varphi_a : &R \rightarrow R\\
		      &b \rightarrow a\cdot b
	  \end{aligned}
	  $ \varphi_a$ è iniettiva. Infatti $\varphi_a$ è un omomorfismo di gruppi\\
	  $(R,+) \rightarrow (R,+)$ per la distributività\\
	  Inoltre\\
	  $ker( \varphi_a) = \{b\in R | \varphi_a (b) = 0\} = \{b\in R | a\cdot b = 0\} = \{0\}$ (dato che $R$ è dominio)\\
	  $ \Rightarrow  \varphi_a$ è iniettiva\\
	  Ora dato che $|R| < +\infty$  $ \varphi_a$ è biunivoca\\ Quindi nell'immagine di $\phi_a$ abbiamo $1$\\
	   $ \Rightarrow \exosts b \in R$ tale che $ \varphi_a(b) = 1$ ovvero $a\cdot b = 1$\\
	    $ \Rightarrow b$ è l'inverso moltiplicativo di $a$
 \end{dimo}
\begin{defi}
	Dati $(R_1, +,\cdot)$ e $(R_2,\oplus,\odot)$ anelli, un omomorfismo di anelli è una funzione $f: R_1 \rightarrow R_2$ tale che
	\begin{enumerate}
	\item $f(a + b) = f(a) \oplus f(b)$
	\item  $f(a \cdot b) = f(a)\odot f(b)$
	\item  $f(1_{R_1}) = 1_{R_2} \ \ \ \ \forall a,b\in R_1$
	\end{enumerate}
\end{defi}
\subsection{Idee per gli esercizi}\\
$1) (R,+,\codt)$ anello  $a\in R$ \\
$0\cdot a = (0 + 0)\cdot a = 0\cdot a  + 0\cdot a$\\
$ \Rightarrow -(0\cdot a) + (0\cdot a) = -(0\cdot a ) + (0\cdot a ) + (0\cdot a)$ \\
$ \Rightarrow  0 = 0 \cdot a$ \\
2) $a,b\in R$ \\
$0 = 0\cdot b = (a + (-a)) \cdot b = a\cdot b + (-a)\cdot b$\\
Sommando $-(a\cdot b)$ ad entrambi i membri ottengo:\\
 $-(a\codt b) = -(a)\codt b$
\end{document}
