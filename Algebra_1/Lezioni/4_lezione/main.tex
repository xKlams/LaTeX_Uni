\documentclass[12px]{article}

\title{Lezione 4 Algebra I}
\date{2024-10-10}
\author{Federico De Sisti}

\usepackage{amsmath}
\usepackage{amsthm}
\usepackage{mdframed}
\usepackage{amssymb}
\usepackage{nicematrix}
\usepackage{amsfonts}
\usepackage{tcolorbox}
\tcbuselibrary{theorems}
\usepackage{xcolor}
\usepackage{cancel}

\newtheoremstyle{break}
  {1px}{1px}%
  {\itshape}{}%
  {\bfseries}{}%
  {\newline}{}%
\theoremstyle{break}
\newtheorem{theo}{Teorema}
\theoremstyle{break}
\newtheorem{lemma}{Lemma}
\theoremstyle{break}
\newtheorem{defin}{Definizione}
\theoremstyle{break}
\newtheorem{propo}{Proposizione}
\theoremstyle{break}
\newtheorem*{dimo}{Dimostrazione}
\theoremstyle{break}
\newtheorem*{es}{Esempio}

\newenvironment{dimo}
  {\begin{dimostrazione}}
  {\hfill\square\end{dimostrazione}}

\newenvironment{teo}
{\begin{mdframed}[linecolor=red, backgroundcolor=red!10]\begin{theo}}
  {\end{theo}\end{mdframed}}

\newenvironment{nome}
{\begin{mdframed}[linecolor=green, backgroundcolor=green!10]\begin{nomen}}
  {\end{nomen}\end{mdframed}}

\newenvironment{prop}
{\begin{mdframed}[linecolor=red, backgroundcolor=red!10]\begin{propo}}
  {\end{propo}\end{mdframed}}

\newenvironment{defi}
{\begin{mdframed}[linecolor=orange, backgroundcolor=orange!10]\begin{defin}}
  {\end{defin}\end{mdframed}}

\newenvironment{lemm}
{\begin{mdframed}[linecolor=red, backgroundcolor=red!10]\begin{lemma}}
  {\end{lemma}\end{mdframed}}

\newcommand{\icol}[1]{% inline column vector
  \left(\begin{smallmatrix}#1\end{smallmatrix}\right)%
}

\newcommand{\irow}[1]{% inline row vector
  \begin{smallmatrix}(#1)\end{smallmatrix}%
}

\newcommand{\matrice}[1]{% inline column vector
  \begin{pmatrix}#1\end{pmatrix}%
}

\newcommand{\C}{\mathbb{C}}
\newcommand{\K}{\mathbb{K}}
\newcommand{\R}{\mathbb{R}}


\begin{document}
	\maketitle
	\newpage
	\section{Altre informazioni sugli omomorfismi}
	\textbf{Esercizio}\\
	Sia $ \varphi: G_1 \rightarrow G_2$ omomorfismo dei gruppi\\
	$ker \varphi = \lbrace g\in G_1 | \varphi(g) = e_2\rbrace$\\
	Dimostrare che\\
	$ \varphi$ è iniettivo $ \Leftrightarrow ker( \varphi) = \lbrace e_1 \rbrace$ \\
	soluzione:\\
	supponiamo che $ker ( \varphi) = \lbrace e_1\rbrace$\\
	Allora dati $g,h \in G_1$ t.c $ \varphi (g) = \varphi(h)$\\
	dobbiamo mostrare che $g=h$\\
	moltiplico per  $ \varphi(h)^{-1}$\\
	\begin{gather*}
		\Rightarrow \varphi(h)^{-1} * \varphi(g) = e_2\\
		\Rightarrow \varphi(h^{-1})* \varphi(g) = e_2\\
		\Rightarrow \varphi(h^{-1}\cdot g) = e_2\\
	 \Rightarrow h^{-1}\cdot g\in ker \varphi\\
	 \Rightarrow h^{-1}\cdot g = e_1\\
	 \Rightarrow g = h\\
	\end{gather*}
	Il viceversa è lasciato  al lettore come esercizio\\
	\textbf{Soluzione di un esercizio passato}\\
	1)Se $H_1\subseteq G_1$ dimostriamo che $ \varphi(H_1)\trianglelefteq \varphi(G_1)$\\
	Verifichiamo che\\
	\[
		f \varphi(H_1) f^{-1}\subseteq \varphi(H_1) \ \ \forall f\in (G_1)
	.\] 
	Quindi basta dimostrare che\\
	$\forall h\in H_1 \ \ \forall g\in G_1$ abbiamo\\
	$ \varphi(g) \varphi(h) \varphi(g)^{-1}\in \varphi(H_1)$\\
	Questo è equivalente a richiedere che\\
	\[
		\varphi(g\cdot h\cdot g^{-1}) \varphi(H_1)
	.\] 
	Ma $ghg^{-1}\in gH_1g^{-1}=H_1$ dato che $H_1\trianglelefteq G_1$\\
	\begin{gather*}
		\exists \tilde h\in H_1 \text { t.c } g\cdot h\cdot g^{-1} = \tilde h\\
		\varphi(ghg^{-1}) = \varphi(\tilde h)\in \varphi(H_1)
	\end{gather*}
	2) Se $H_2\trianglelefteq G_2$ dimostriamo che $ \varphi^{-1}(H_2)\trianglelefteq G_1$\\
	Ho due omomorfismi,\\
	li compongo:
	\[
		\psi: G_1 \xrightarrow[ \varphi]{} G_2 \xrightarrow[\pi]{} G_2/H_2
	.\] 
	Studia il $\ker (\psi)$\\
	 $\ker(\psi):= \lbrace g\in G_1 | \psi (g) = H_2\rbrace = \lbrace g\in G_1 | \varphi(g)H_2 = H_2\rbrace$\\
	 $ker (\psi) = \lbrace g\in G | \varphi(g)\in H_2 \rbrace = \varphi^{-1}(H_2)$\\
	 Quindi $ \varphi^{-1}(H_2)$ è il nucleo di un omomorfismo $\psi : G_1 \rightarrow G_2/H_2$ e dunque $ \varphi^{-1}(H_2)\trianglelefteq G_1$\\
	 \textbf{Osservazione:}\\
	 Se $ \varphi:G_1 \rightarrow G_2$\\
	 omomorfismo di gruppi\\
	 $H_2 = \lbrace e_2 \rbrace \trianglelefteq G_2$\\
	 l'esercizio $(2)$ ci dice che  $\ker( \varphi) = \varphi^{-1}(\lbrace e_2\rbrace)\trianglelefteq G_1$\\
	 \textbf{Osservazione}\\
	 Dalla parte (1) segue che\\
	 $H_1\leq G_1 \Rightarrow \varphi(H_1)\leq G_2$ \\
	 Quindi  se scelgo $H_1=G_1\leq G_1$\\
	 $ \Rightarrow Im( \varphi) = \varphi(G_1)\leq G_2$ 
	 \section{Parte figa della lezione}
	 \begin{lemm}
	 	
	 $(G,\cdot)$ gruppo\\
	 $N\trianglelefteq G, H\trianglelefteq G$ sottogruppi normali\\
	  $\pi: G \rightarrow G/N$\\
	  Allora $\pi(H) = \pi(HN)$\\
	 \end{lemm}
	  \begin{dimo}
	  	$H\subseteq HN$ poiché $e\in N$ ogni elemento di  $H$ lo scrivo come lui stesso e $ \Rightarrow \pi(H)\subseteq \pi(HN)$ \\
		Viceversa dimostriamo che $\pi(HN) \subseteq \pi(H)$\\
		infatti:\\
		 $\forall h\in H \ \ \ \forall n\in N$\\
		  $\pi(hn)=\pi(h)\pi(n)$ (omomorfismo)\\
		   $n\in N$\\
		    $ \Rightarrow \pi(n) = N \rightarrow \pi(h)\pi(e) = \pi(ne)$ \\
		    \pi(e) = N \ \ \ \ =\pi(n)\in \pi H
	  \end{dimo}
	  \newpage
	  \begin{lemm}
	  	$(G,\cdot)$ gruppo\\
		$\cdot H\trianglelefteq G$\\
		$\cdot N\trianglelefteq G$\\
		$\cdot \pi \rightarrow G/N$\\
		Allora:\\
		$1)\pi^{-1}(\pi(H))=HN$\\
		$2)$ se $N\subseteq H \rightarrow\pi^{-1}(\pi(H)) = N$\\
		$3) \bar H\leq G/N \rightarrow \pi(\pi^{-1}(\bar H)) = \bar H$

	  \end{lemm}
	  \begin{dimo}[1]
		  $\pi^{-1}(\pi(H)) = ?$\\
		  osserviamo che dal lemma 1\\
		   $\pi(H) = \pi(HN) = HN$\\
		   dato che  $\pi(hn) = \pi(h)\pi(n) = hn$\\
		   $ \Rightarrow \pi^{-1}(\pi(H)) = \pi^{-1}(\pi(HN)) = \pi^{-1}(HN)\supseteq HN$ \\
		   Resta da verificare che $\pi^{-1}(\pi(H))\subseteq HN$\\
		    \begin{gather*}
			    \pi^{-1}(\pi(H)):=\lbrace g\in G|\pi(g)\in \pi(H)\rbrace\\
			    =\lbrace g\in G|\exists g\in H:\pi(g) = \pi(h)\rbrace\\
			    =\lbrace g\in G| \exists h\in H: \pi(h)^{-1}\pi(g)=N\rbrace \text{ N= elemento neutro in G}\\
			    =\lbrace g\in G|\exists h\in H: \pi(hg) = N\rbrace\\
			    =\lbrace g\in G | \exists h\in H: h^{-1}g\in N\rbrace\\
			    =\lbrace g\in G | \exists h\in H: g\in hNŕbrace \subseteq HN
		   \end{gather*}
		   segue (1)
	  \end{dimo}
	  \begin{dimo}[2]
	  	È un caso particolare del punto 1, infatti se
		\[
		N\subset H \Rightarrow HN = H
		.\] 
	  \end{dimo}
	  \begin{dimo}[3]
	  	Segue dal fatto che $\pi è un omomorfismo suriettivo $
		 \[
			 \pi(\pi^{-1}(\bar H))=\pi (G)\cap \bar H = \bar H
		.\] 
	  \end{dimo}
	  \begin{teo}
		$(G,\cdot)$, $n\trianglelefteq G$\\
		Allora esistono due corrispondenze biunivoche
		 \begin{center}
		 	
			 $\lbrace \text{sottogruppi } H\leq G \ \ t.c. \ \ N\supseteq H\rbrace \rightarrow \{\text{sottogruppi di } G/N\rbrace\\
			 H \rightarrow \pi (H)\\
			 \pi^{-1} \leftarrow \bar H\\
			 \lbrace \text{ sottogruppi normali } H\trianglelefteq G \ t.c \ N\subseteq H\rbrace \rightarrow \lbrace \text{ sottogruppi normali } G/N \rbrace\\
			 H \rightarrow \pi(H)\\
			 \pi^{-1}(\bar H) \rightarrow \bar H$
		 \end{center}
	  \end{teo}
	  \begin{dimo}
		  Il lemma 2 (punti 2 e 3) garantisce che le due applicazioni $H \rightarrow \pi(H)$ $\pi^{-1}(H) \rightarrow\bar H$\\
		  sono una l'inversa dell'altra
	  \end{dimo}
	  \textbf{Osservazione:}\\
	  Per la seconda corrispondenza osserviamo che per la suriettività di $\pi$ e l'esercizio di oggi \\
	  \[
	  H\trianglelefteq G \rightarrow \pi (H)\trianglelefteq G/N
	  .\] 
	  \begin{teo}[Teorema di omomorfismo]
	  	$ \varphi:G_1 \rightarrow G_2$ omomorfismo\\
		$\cdot N\trianglelefteq G_1$\\
		$\pi:G_1 \rightarrow G/N$\\
		Allora:\\
		1) esiste unico omomorfismo\\
		$\bar\varphiL G/N \rightarrow G_2$\\
		t.c. $ \bar  \varphi\circ \pi = \varphi$
		\begin{tikzcd}
G_1 \arrow[r, "\varphi"] \arrow[d, "\pi"] & G_2  \\
G_1 / N \arrow[ur, dashed, "\exists ! \bar\varphi"]
\end{tikzcd} \\
2) $Im (\bar \varphi) = Im ( \varphi)$\\
3) $ \bar \varphi$ è iniettivo \Leftrightarrow $ker\varphi = N$
	  \end{teo}
	  \begin{dimo}
	  	La condizione $\bar \varphi \cdot \pi = \varphi$\\
		Significa\\
		$\forall g\in G_1$ si ha \\
		$\bar \varphi\cdot \pi(g) = \varphi(g)$\\
		ovvero\\
		$ \bar \varphi(gN) = \varphi(g)$\\
		Dobbiamo verificare:\\
		$\cdot $ Unicità (segue da $\bar \varphi\cdot \pi = \varphi)$\\
		$\cdot \bar \varphi$  è ben definita\\
		$\cdot \bar \varphi$ è un omomorfismo\\
		significa che se $gN=fN$ per qualche $g,f\in G_1$, allora $ \varphi(g) = \varphi(f)$\\
		Verifichiamo:\\
		$gN = fN \rightarrow g\equiv f mod N$\\
		$ \Rightarrow \exists n\in N$ t.c. $g^{-1}f = n$\\
		 $ \Rightarrow f=gn \Rightarrow \varphi(f) = \varphi(gn)$ \\
		 $ \Rightarrow \varphi(f) = \varphi(g) \varphi(n) = \varphi(g)$ \\
		 dato che $ \varphi(n) = e_2$ ovvero $N\subseteq\ker \varphi$\\
		 \textbf{Mostriamo adesso che $\bar \varphi$ è un omomorfismo}\\
		 Significa che $\forall f,g\in G$\\
		  \[
		 \bar \varphi((fN)\cdot (gN)) = \bar \varphi(fN)\cdot \bar\varphi(gN)
		 .\] 
		 Per definizione\\
		 \[
		 \bar\varphi ((fN)(gN)) = \bar \varphi(fgN) = \varphi(fg) = \varphi(f) \varphi(g)
		 .\] 
		 $2) \bar \varphi\circ \pi = \varphi$\\
		 dalla suriettività del $\pi$ segue che  $Im(\bar  \varphi) = Im ( \varphi)$\\
		 $3) \bar \varphi$ è iniettivo $ \Leftrightarrow ker \bar \varphi = \lbrace N\rbrace$\\
		 $ker\bar \varphi = \lbrace gN\in G_1/N | \bar \varphi(gN) = e_2\rbrace$\\
		 $=\lbrace gN\in G_1/N | \varphi(g) = e_2\rbrace$\\
		 $=\lbrace gN\in G_1/N | g\in ker( \varphi)\rbrace$
	  \end{dimo}
	  \begin{coro}
	  	$(G,\cdot)$, $N\trianglelefteq G$\\
		Allora esiste una corrispondenza biunivoca\\
		\begin{center}
			$
			 \lbrace \text{omomorfismi } \varphi:G \rightarrow G' \ t.c. \ N\subseteq ker( \varphi)\rbrace \rightarrow \lbrace \text{omomorfismi }G/N \rightarrow G'\rbrace\\
			 \varphi \rightarrow \bar \varphi\\
			 \bar\circ \pi \leftarrow \bar \varphi
			 $
		\end{center}
	  \end{coro}
	  \begin{dimo}
		  basta osservare che\\
		  dato $\bar \varphi: G/N \rightarrow G'$ la composizione\\
		  $\bar \varphi\circ \pi: G \rightarrow G'$ è un omomorfismo\\
		  tale che $ker(\bar \varphi\circ \pi)\supseteq N$\\
		  segue $\pi(N) = N$ che è l'elemento neutro di  $G/N$\\
		   $ \Rightarrow \bar \varphi\circ \pi (N) = e'$ che è l'elemento neutro  di $G'$\\
		   \end{dimo}
		   \newpage
		    \begin{defi}
		   	$ \varphi: G_1 \rightarrow G_2$\\
			omomorfismo si dice isomorfismo se è invertibile\\

		   \end{defi}
		   \begin{teo}[Primo teorema di isomorfismo]
		   	$ \varphi:G_1 \rightarrow G_2$\\
			Allora:\\
			$Im( \varphi) \cong G_1/ker( \varphi)$\\
			Dove $\cong$ (isomorfo) significa che esiste un isomorfismo tra i due gruppi
		   \end{teo}\\
		   \begin{dimo}
		   \[
\begin{tikzcd}
G_1 \arrow[r, "\varphi"] \arrow[d, "\pi"] & G_2 \\
G_1 / N \arrow[ur, "\exists ! \overline{\varphi}"]
\end{tikzcd}
\]	
scelgo $N-ker \varphi$\\
il teorema di isomorfismo fornisce un omomorfismo iniettivo
\[
\bar\varphi: G_1/\ker \varphi \rightarrow G_2
.\] 
Allora mi restringo all'immagine di $\bar \varphi$ così diventa suriettiva\\
\[
G/ker \varphi \cong  Im ( \bar\varphi) \cong Im( \varphi)
.\] 
la prima tramite $\bar \varphi$ la seconda per il teorema di isomorfismo\\
Applicazione:\\
det: $GL_n(\K) \rightarrow (\K^*,\cdot) = (\K\setminus\{0\},\cdot)$ \\
$\ker(det) = SL_n(\K)$ matrici con det 1\\
$ \Rightarrow GL_n(\K)/SL_n(\K) \cong (\K^*,\cdot)$
		   \end{dimo}

	  	

\end{document}
