\documentclass[12px]{article}

\title{Lezione 5 Algebra I}
\date{2024-10-15}
\author{Federico De Sisti}

\usepackage{amsmath}
\usepackage{amsthm}
\usepackage{mdframed}
\usepackage{amssymb}
\usepackage{nicematrix}
\usepackage{amsfonts}
\usepackage{tcolorbox}
\tcbuselibrary{theorems}
\usepackage{xcolor}
\usepackage{cancel}

\newtheoremstyle{break}
  {1px}{1px}%
  {\itshape}{}%
  {\bfseries}{}%
  {\newline}{}%
\theoremstyle{break}
\newtheorem{theo}{Teorema}
\theoremstyle{break}
\newtheorem{lemma}{Lemma}
\theoremstyle{break}
\newtheorem{defin}{Definizione}
\theoremstyle{break}
\newtheorem{propo}{Proposizione}
\theoremstyle{break}
\newtheorem*{dimo}{Dimostrazione}
\theoremstyle{break}
\newtheorem*{es}{Esempio}

\newenvironment{dimo}
  {\begin{dimostrazione}}
  {\hfill\square\end{dimostrazione}}

\newenvironment{teo}
{\begin{mdframed}[linecolor=red, backgroundcolor=red!10]\begin{theo}}
  {\end{theo}\end{mdframed}}

\newenvironment{nome}
{\begin{mdframed}[linecolor=green, backgroundcolor=green!10]\begin{nomen}}
  {\end{nomen}\end{mdframed}}

\newenvironment{prop}
{\begin{mdframed}[linecolor=red, backgroundcolor=red!10]\begin{propo}}
  {\end{propo}\end{mdframed}}

\newenvironment{defi}
{\begin{mdframed}[linecolor=orange, backgroundcolor=orange!10]\begin{defin}}
  {\end{defin}\end{mdframed}}

\newenvironment{lemm}
{\begin{mdframed}[linecolor=red, backgroundcolor=red!10]\begin{lemma}}
  {\end{lemma}\end{mdframed}}

\newcommand{\icol}[1]{% inline column vector
  \left(\begin{smallmatrix}#1\end{smallmatrix}\right)%
}

\newcommand{\irow}[1]{% inline row vector
  \begin{smallmatrix}(#1)\end{smallmatrix}%
}

\newcommand{\matrice}[1]{% inline column vector
  \begin{pmatrix}#1\end{pmatrix}%
}

\newcommand{\C}{\mathbb{C}}
\newcommand{\K}{\mathbb{K}}
\newcommand{\R}{\mathbb{R}}


\begin{document}
	\maketitle
	\newpage
	\section{Teoremi di isomorfismo}
	\begin{teo}[Secondo teorema di isomorfismo]
		$(G,\cdot)$ gruppo\\
		$H,N\normale G$ tali che $N\subseteq H$ Allora
		 \begin{enumerate}
			 \item$ H/M\normale G/N $
			 \item $G/N/H/N \cong G/H$
		 \end{enumerate}
	\end{teo}
	\begin{dimo}
	\begin{tikzcd}
		G \arrow[r, "\varphi = \pi_H"] \arrow[d,"\pi"]  & G/H \\
G / N \arrow[ur, dashed, "\exists ! \overline{\varphi}"]
\end{tikzcd}
$\pi_H$ proiezione sul quoziente H\\
$N\subseteq H = ker ( \varphi)$ \\
Inoltre $Im( \bar\varphi) = Im( \varphi) = G/H$\\
\textbf{Idea:} applicare il primo teorema di isomorfismo\\
\underline{suriettiva} $\bar \varphi: G/N \rightarrow G/H$\\
basta quindi dimostrare che $ker (\bar \varphi) = H/N$\\
Studiamo
\[
ker(\bar \varphi) = \lbrace gN\in G/N | \bar \varphi(gN) = H\rbrace
.\] 
\[
	\{gN\in G/N | gH = H\}
.\] 
\[
	\{gN\in G/N | g\in H\} = H/N
.\] 
	\end{dimo}
	\begin{coro}
		In $(\mathbb Z, +)$ gruppo abeliano\\$a,n\in \mathbb Z$ interi non nulli\\
		Denotiamo con
		\[
			[a] = a + (n) \in \mathbb Z/(n) = \{ [0],[1],[2],\ldots,[n-1]\}
		.\] 
		Allora $ord_{\mathbb Z/(n)}([a]) = \frac n {MCD(a,n)}$
	\end{coro}
	\textbf{Nota:}\\
	se $MCD(n,a) = 1$ allora a genera il gruppo ciclico  $\mathbb Z/(n)$
	 \begin{dimo}
		Consideriamo $G=\mathbb Z \ \ H = (a) + (n) \ \ N= (n)$\\
		Dal II Teorema di isomorfismo
		 \[
			 \bigslant{\mathbb Z/(n)}{([a])
			 }\cong\bigslant{\mathbb Z/(n)}{(a) + (n)/(n)} \cong \bigslant {G/N}{H/N}\cong G/N \cong \mathbb Z/(MCD(a,n))
		.\] 
	\end{dimo}
	Confrontiamo le cardinalità\\
	\[
	MCD(a,n) = |\mathbb Z/(MCD(a,n))|
	.\] 
	\[
		= |\bigslant{\mathbb Z/ (n)}{([a])}|

	.\] 
	\[
		\frac {|Z/(n)|}{([a])} = \frac n {ord([a])}
	.\] 
	\[
		ord([a]) = \frac n {MCD(a,n)}
	.\] 
	\begin{lemm}
		$a,bg\in \mahbb Z$ non nulli\\
		tali che  $a|b$ (allora  $(b)\subseteq (a)$ \\
		Allora\\
		\[
		|(a)/(b)| = \frac ba
		.\] 
	\end{lemm}
	\begin{dimo}
		Studiamo $(a)/(b)$\\
		Per definizione è l'insieme dei laterali\\
		 \[
			 (a)/(b) = \{ta + (b) | t\in \mathbb Z\}
		.\] 
		dobbiamo capire quanti laterali \underline{distinti} esistono\\
		Dati $t,s\in \mathbb Z$ tali che\\
		\[
		ta + (b) = sa + (b)
		.\] 
		\[
		\Leftrightarrow ta \equiv sa \ mod(b)
		.\] 
		\[
		\Leftrightarrow -ta + sa \in(b)
		.\] 
		Allora\\
		\[
			(a)/(b) = \{ta + (b)|tt\in\{1,\ldots,\frac ba\}\}
		.\] 
	\end{dimo}
	\begin{teo}[III teorema di isomorfismo]
		$(G,\cdot)$ gruppo\\
		\begin{itemize}
			\item $N\trianglelefteq G$\\
			\item  $H\leq G$
		\end{itemize}
		Allora
		\begin{enumerate}
			\item$H\cap N\trianglelefteq H$\\
			\item  $\bigslant H {H\cap N}\cong HN/N$
		\end{enumerate}
	\end{teo}
		\begin{dimo}
			$\pi_N:G \rightarrow G/N$\\
			$g \rightarrow gN$\\
			consideriamo la restrizione\\
			\begin{gather*}
				\pi_N|_H:H \rightarrow G/H\\
				h \rightarrow hN\\
				ker (\pi_N|_H) = \{h\in H| \pi_N|_H(h) = N\}\\
					       =\{h\in H|hN = N\}\\
					       =\{h\in H| h\in N\}\\
					       =H\cap N
			\end{gather*}\\
			Deduciamo che $H\cap N\trianglelefteq N$\\
			Idea: Applicare il I teorema di isomorfismo all'omomorfismo
			 \[
			\varphi=\pi_N|_H:H \rightarrow G/N
			.\] 
			Avremo $Im( \varphi)\cong H/ker( \varphi) = H/H\cap N$\\
			Studiamo $Im( \varphi)$\\
	\[
	Im( \varphi) = Im( \pi_N|_H) = \pi_N(H) = \pi_N(HN) = HN/N
	.\] 
	Il penultimo passaggio deriva da un lemma già visto a lezione
		\end{dimo}
		\begin{coro}
			$a,b\in \mathbb Z$ non nullli\\
			Allora $mcm(a,b) = \frac {ab} {MCD(a,b)}$
		\end{coro}
		\begin{dimo}
			$G = \mathbb Z$ \\
			$H = (a)$\\
			 $N= (b)$\\
			  $H+N = (MCD(a,b))$\\
			   $H\cap N = (mcm(a,b))$\\
		Dal III teorema di isomorfismo 
		 \[
			 \bigslant{(a)}{(mcm(a,b))}\cong \bigslant H {H\cap N}\cong \bigslant {HN} N \cong \bigslant{(MCD(a,b))}{(b)}
		.\] 
		Confrontiamo la cardinalità\\
		Per il lemma\\
		\[
			\frac {mcm(a,b)}{a} = |\biglsant {(a)}{(mcm(a,b))}| = |\bigslant {(MCD(a,b))}{(b)} = \frac {b}{MCD(a,b)}
		.\] 
		Quindi 
		\[
			mcm(a,b) = \frac {ab}{MCD(a,b)}
		.\] 
		\end{dimo}
		\section{Classificazione di gruppi di ordine "piccolo" a meno di isomorfismo}\\
		\textbf{Ordine 1}\\
		Se $|G| = 1 \ \ \Rightarrow  \ \ G = \{e\}$\\
		\textbf{Ordine p primo:}\\
		Abbiamo mostrato che se $|G|=p$ allora  $G$ non ammette sottogruppi non banali\\
		Sia $g\in G$ tale che $g\neq e$
		$ \Rightarrow ord(g) = p \ \Rightarrow G = <g>$ \\
		\begin{gather*}
			\varphi: G \rightarrow G_p = <p>\\
			g \rightarrow p
		\end{gather*}\\
		\textbf{Obiettivo:} classificare a meno di isomorfismo i gruppi di ordine 4 e di ordine 6\\
		\begin{defi}[Klein,1884]
			Il gruppo di Klein, $K_4$ è il gruppo delle isometrie del piano che preservano un rettangolo fissato.
		\end{defi}
		\textbf{Esercizio}\\
		Verificare che $K_4 = \{id, \rho,\sigma, \rho\sigma\}$\\
		dove $\rho$ = rotazione di angolo $\pi$\\
		e dove  $\sigma$ = riflessione rispetto ad un lato
		\textbf{Osservazione}\\ tutti gli elementi in $K_4$ hanno ordine $\leq 2$ Quindi  $K_4\neq C_4$\\
		\begin{nota}
			Dato che $K_4 = <\rho,\sigma>$\\
			denoteremo anche
			\[
				K_4 = D_2 \text{ (gruppo diedrale)}
			.\] 
		\end{nota}
		\textbf{Esercizio}\\
		$(G,\cdot)$ gruppo in cui ogni elemento ha ordine $\leq 2$ (equivalentemente ogni elemento è inverso di se stesso)\\
		1) Dimostrare che $G$ è abeliano\\
		2) Se $|G| = 4$ dimostrare che  $G \cong K_4$
		\textbf{Svolgimento}
		1) Dati $f,g\in G$\\
		$fg = (fg)^{-1} = g^{-1}f^{-1} = gf$\\
		2) Sia $|G| = 4$\\
		Scelgo  $g,f\in G$ distinti tali che \begin{cases}
			g\neq e\\
			f\neq e
		\end{cases}\\
		Considero $H = <g,h>$ \\
		Per Lagrange\\
		$H \geq 3$\\
		$ \Rightarrow H = H\\
		\Rightarrow G = \{e,f,g,fg\}$\\
		abeliano\\
		Costruisco l'isomorfismo esplicito con $K_4$\\
		\begin{gather*}
			\varphi:G \rightarrow K_4 = <\rho,\sigma>\\
			e \rightarrow e\\
			f \rightarrow\rho\\
			g \rightarrow\sigma\\
			fg \rightarrow\rho\sigma
		\end{gather*}
		che è chiaramente biunivoca ed è un omomorfismo $ \Rightarrow \varphi$ è un isomorfismo


\end{document}
