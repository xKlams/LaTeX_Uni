\documentclass[12px]{article}

\title{Lezione 17 Algebra I}
\date{2024-11-26}
\author{Federico De Sisti}

\usepackage{amsmath}
\usepackage{amsthm}
\usepackage{mdframed}
\usepackage{amssymb}
\usepackage{nicematrix}
\usepackage{amsfonts}
\usepackage{tcolorbox}
\tcbuselibrary{theorems}
\usepackage{xcolor}
\usepackage{cancel}

\newtheoremstyle{break}
  {1px}{1px}%
  {\itshape}{}%
  {\bfseries}{}%
  {\newline}{}%
\theoremstyle{break}
\newtheorem{theo}{Teorema}
\theoremstyle{break}
\newtheorem{lemma}{Lemma}
\theoremstyle{break}
\newtheorem{defin}{Definizione}
\theoremstyle{break}
\newtheorem{propo}{Proposizione}
\theoremstyle{break}
\newtheorem*{dimo}{Dimostrazione}
\theoremstyle{break}
\newtheorem*{es}{Esempio}

\newenvironment{dimo}
  {\begin{dimostrazione}}
  {\hfill\square\end{dimostrazione}}

\newenvironment{teo}
{\begin{mdframed}[linecolor=red, backgroundcolor=red!10]\begin{theo}}
  {\end{theo}\end{mdframed}}

\newenvironment{nome}
{\begin{mdframed}[linecolor=green, backgroundcolor=green!10]\begin{nomen}}
  {\end{nomen}\end{mdframed}}

\newenvironment{prop}
{\begin{mdframed}[linecolor=red, backgroundcolor=red!10]\begin{propo}}
  {\end{propo}\end{mdframed}}

\newenvironment{defi}
{\begin{mdframed}[linecolor=orange, backgroundcolor=orange!10]\begin{defin}}
  {\end{defin}\end{mdframed}}

\newenvironment{lemm}
{\begin{mdframed}[linecolor=red, backgroundcolor=red!10]\begin{lemma}}
  {\end{lemma}\end{mdframed}}

\newcommand{\icol}[1]{% inline column vector
  \left(\begin{smallmatrix}#1\end{smallmatrix}\right)%
}

\newcommand{\irow}[1]{% inline row vector
  \begin{smallmatrix}(#1)\end{smallmatrix}%
}

\newcommand{\matrice}[1]{% inline column vector
  \begin{pmatrix}#1\end{pmatrix}%
}

\newcommand{\C}{\mathbb{C}}
\newcommand{\K}{\mathbb{K}}
\newcommand{\R}{\mathbb{R}}


\begin{document}
	\maketitle
	\newpage
	\section{Ricordo (Lagrange)}
	$f(x) = a_nx^n + \ldots + a_1 x + a_0\in \Z[x]$ tale che $a_n \not\equiv_p 0$ con  $p > 1$ primo \\
	Allora  $f(x)\equiv_p 0$ ammette al più  $n$ soluzioni
	\begin{coro}[Esercizio]
	Dimostrare che se $p$ primo e $d | (p-1)$ allora  $x^d -1 \equiv_p 0$ ammette esattamente  $d $ soluzioni
\end{coro}
\begin{dimo}[Soluzione]
	Abbiamo che se $d  (p-1)$ allora $(x^d -1) | (x^{p-1} - 1)$\\
	$  \Rightarrow x^{p-1} = (x^d-1)f(x)$\\
	dove $f$ è di grado $(p-1-d)$ \\
	Ora  $x^{p-1}\equiv_p 1$ ammette  $p-1$ soluzioni distinte per il piccolo teorema di Fermat. Le soluzioni sono  $1,2,\ldots, p-1$\\
	Se una di tali soluzioni non risolve  $f(x)\equiv_p 0$ allora risolve  $x^d-1\equiv_p 0$ (Sto usando il fatto che  $\Z/(p)$ è un dominio d'integrità [prodotto commutativo e se il prodotto tra due numeri è 0 allora o uno o l'altro sono 0])\\
	Dato che $ f(x)\equiv_p 0$ ammette al più $p-1-d$ soluzioni distinte deduciamo che $x^d - 1\equiv_p$ ammette almeno $d = (p-1)-(p-1-d)$ soluzioni distinte in  $\Z/(p)$.\\
	D'altra parte per l'esercizio precedente ne ammette al più $d$, e quindi segue la tesi.
\end{dimo}
\begin{coro}[Esercizio]
	$p> 1$ primo, $d | (p-1)$ Allora, esistono esattamente $\phi(d)$ interi, distinti in $U_p$, di ordine $d$ in $U_p$
\end{coro}
\begin{dimo}[Soluzione]
	Introduco $S_d = \{k\in \Z | ord_{U_p}([k]) = d, \ \ 1\leq k\leq p-1\}$\\
	La tesi è equivalente a dimostrare che  $|S_d| = \phi(d)$\\
	Abbiamo una partizione  $\displaystyle\{1,\ldots, p-1\}=\bigcup_{d|p-1}S_d$\\
	Quindi $\displaystyle p-1 = \sum_{d|(p-1)}|S_d|$\\
	Ricordo:\\
	$n = \sum_{d|n} \phi(d)$ (esercizio delle vecchie schede)\\
	Scegliendo $n = p-1$ deduciamo\\
	$\displaystyle \sum_{d|p-1}|S_d| = \sum_{d|p-1} \phi(d)$\\
	Basta allora dimostrare che $|S_d|\leq \phi(d) \ \ \forall d|p-1$\\
	Se $S_d = \emptyset$ $ \Rightarrow |S_d| = 0\leq \phi(d)$ \\
	Se $S_d\neq \emptyset \ \Rightarrow \exists a\in S_d$\\
	$ \Rightarrow \{a, a^2, a^3, \ldots, a^d\}$ sono tutti distinti  $mod(p)$ infatti \\
	\begin{center}
		
	 \begin{aligned}
		 a^i &\equiv_p a^k \\
		     &\storto \Leftrightarrow\\
		 i &\equiv_d j
	 \end{aligned}\\
	\end{center}
	 Quindi $a,a^2,\ldots, a^n$ sono tutte e sole le soluzioni di  $x^d - 1\equiv_p 0$ Quindi gli elementi di ordine  $d$ in $U_p$ sono della forma $a^j$ per qualche $j\in \{1,\ldots, j\}$\\
	 Ma  $ord([a^j]) = \frac{d}{MCD(j,d)}$ (esercizio di una riga) \\
	 Quindi $|S_d| = \phi(d)$

\end{dimo}
\begin{coro}{Esercizio]
	$p > 1 $ primo:\\
	Allora esistono esattamente $ \phi(p-1)$ radici primitive distinte
\end{coro}
\begin{dimo}[Soluzione]
	Basta applicare l'esercizio precedente, scegliendo $d = p-1$
\end{dimo}
\hline\ \\
\textbf{Esercizio}\\
$p> 1$ primo\\
dimostrare che  $Aut(C_p)\cong C_{p-1}$\\
\textbf{Soluzione:}\\
Sappiamo che $Aut(C_p)\cong U_p\cong C_{p-1}$\\
Dove la prima congruenza la sappiamo da teoremi precedenti, la seconda viene data dal precedente corollario\\
\begin{congettura}[Gauss, 1801]
	Esistono infiniti primi per cui $10$ è una radice primitiva
\end{congettura}
\begin{congettura}[E. Artin, 1927]
	$a\in\Z$,  $a\neq \pm 1$\\
	Assumiamo che  $a$ non sia un quadrato perfetto, Allora esistono infiniti primi per cui $a $ è una radice prima
\end{congettura}
\textbf{Osservazione}\\
Oggi sappiamo che la congettura di Artin è vera per infiniti interi $a$, ma non è noto quali
\textbf{Esercizio:}
$p > 1$ primo\\
Sia  $a = x^2$ con  $x\in\Z$\\
Dimostrare che se  $[a]\in U_p$\\
allora  $ord_{U_p}([a]_\neq p- 1$\\
\textbf{Esercizio} [classificazione dei gruppi di ordine $pq$]\\
Dimostrare che tutti i gruppi non ciclici di ordine $pq$ con  $p\neq q$ primi, sono fra loro isomorfi e non abeliani\\
\textbf{Soluzione}\\
Dato $G$ tale che $|G| = pq$
Avevamo dimostrato che $\exists ! Q\in Syl_q(G) \ \Rightarrow \ Q\normale G$\\
Inoltre $\exists P\in Syl_p(G) \Rightarrow P\leq G$  \\
Abbiamo verificato che:\\
$\cdot P\cap Q = \{e\}$\\
 $\cdot |PQ| = |G| \Rightarrow PQ = G$ \\
 $ \Rightarrow G\cong Q\semi P \cong C_q\semi C_p$


\end{document}
