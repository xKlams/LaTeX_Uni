\documentclass[12px]{article}

\title{Lezione 7 Algebra I}
\date{2025-03-24}
\author{Federico De Sisti}

\input{../../../setup.tex}

\begin{document}
\maketitle
\newpage
\subsection{Varie cose su polinomi e UFD}
\begin{defi}
	$R$ \underline{UFD}, $f\in R[x]$ il contenuto di $f$ è 
	\[
		c(f) = MCD(a_0,\ldots,a_n)\in R
	.\] 
	dove $f = \sum^{n}_{i=0}a_ix^i$
\end{defi}
\textbf{Osservazione}\\
$c(f)$ è ben definito a meno di moltiplicazioni per unità di  $R$.
\begin{defi}
	$R$ UFD, $f\in R[x]$ si dice primitivo se  $c(f) = 1$
\end{defi}
\begin{lemm}[Gauss]
	$R$ è UFD $f,g\in R[x]$\\
	Allora
	\[
		c(f\cdot g) = c(f)\cdot c(g)
	.\] 
\end{lemm}
\begin{dimo}
	$f,g\in R[x]$ possiamo scriverli come
	\[
		\begin{cases}
			f = c(f)\cdot f'\\
			g = c(g)\cdot g'
		\end{cases}
	.\] 
	con $f',g'\in R[x]$ primitivi\\
	Inoltre  $c(r\cdot h) = r\cdot c(h) \ \ \forall r\in R$ e  $\forall h\in R[x]$ \\
	Allora \\
	\[
		c(f\cdot g) = c(c(f)\cdot f'\cdot c(g)\cdot g')= c(f)c(g)\cdot c(f'\cdot g')
	.\] 
	dato che $c(f),c(g)\in R$ \\
	Quindi è sufficiente dimostrare che $c(f'\cdot g') = 1$\\
	Equivalentemente verifichiamo che non esiste alcun primo in $q\in R$ tale che $a | c(f'\cdot g')$\\
	Supponiamo per assurdo che esista  $q\in R$ tale che  $q | c(f'\cdot g')$ primo  in  $R$.\\
	$q\in R$ primo  $ \Rightarrow (q)\subseteq R$ ideale primo\\
	$ \Rightarrow \overline R = R/(q)$ è un dominio d'integrità.\\
	$ \Rightarrow \overline R [x]$ dominio d'integrità.\\
	Considero $\bar f', \bar g'\in \bar R[x]$  i polinomi indotti in  $\bar R[x]$ \\riducendo il coefficiente di $f'$ e $g'$ $mod(q)$ \\
	Allora $q|c (f'\cdot g') \Rightarrow \bar f'\cdot \bar g' = 0$ in $\bar R[x]$\\
	Quindi  (dato che $\bar R[x]$ dominio d'integrità)\\ $\bar f' = 0 $ in  $\bar R[x]$ oppure  $\bar g' = 0$ in  $\bar R[x]$ \\
	$ \Rightarrow q | c(f') $ in $R$ oppure  $q | c(g')$ in $R$\\
	Ma  $f',g'$ sono primitivi in  $R[x] \Rightarrow  c(f'),c(g') $ unità in $R \Rightarrow $ assurdo\\
\end{dimo}
\textbf{Ricordo}\\
$R$ dominio d'integrità\\
$X = \{ (a,b) \ | \ a\in R, b\in R\setminus\{0\}\}$\\
$(a,b)\sim (c,d) \Leftrightarrow ad = cb$\\
$Frac(R) = X/\sim$ \ è il campo delle frazioni di  $R$ \\
\textbf{Osservazione}
\begin{center}
	\begin{aligned}
		$ R$ &$\rightarrow Frac(R)$\\
		$ r$ &$\rightarrow (r,1)$
	\end{aligned}
\end{center}
omomorfismo di anelli.
\begin{nota}
	denoteremo $(a,b)$ in $Frac(R)$ come  $a\cdot b^{-1}$
\end{nota}
\begin{lemm}
	$R$ UFD.\ $f\in R[x]$ primitivo  $g\in R[x]$ \\
	Allora $f | g$ in $R[x]$ se e solo se  $f | g$ in $\K[x]$ dove  $\K = Frac(R)$
\end{lemm}
\begin{dimo}
$f | g$ in  $R[x]$ significa  $f\cdot q = g$ per qualche  $q\in R[x]$\\
$f | g$ in $\K[x]$ significa $f\cdot q = g$ per qualche  $q\in \K[x]$\\
Nota che  $\K[x]$ potrebbe avere molti più elementi di $R$ quindi è un'informazione  più generale.\\
	Basta mostrare che la seconda implica la prima\\
	Se $q \in \K[x] \Rightarrow  q = a\cdot b^{-1}$ dove $a\in R[x]$ e $b\in R\setminus \{0\}$\\
	Allora  \\
	$b\cdot g = b\cdot f\cdot q= \cancel b\cdot f \cdot a \cdot  \cancel {b^{-1}}$ \\
	Per il lemma di Gauss
	\[
	 c(b)\cdot c(g) = c(b\cdot g) = c(f\cdot a ) = c(f)\cdot c(a)
	.\] 
	Notando che $c(f) = 1$ deduciamo che  $c(a) = b\cdot c(g)$\\
Allora:\\
\[
	q = a\cdot b^{-1} = c(a)\cdot a' \cdot b^{-1} = \cancel b\cdot c(g)\cdot a '\cdot \cancel{b^{-1}} \in R[x]
.\] 
\end{dimo}
\newpage
\begin{prop}
	$R$ UFD. Allora $f\in R[x]$ è irriducibile se e solo se una delle seguenti condizioni è verificata
	\begin{enumerate}
		\item $f\in R$ e  $f$ irriducibile in $R$
		\item  $f\in R[x]$ primitivo e  $f$ irriducibile in $\K[x]$
	\end{enumerate}
\end{prop}
\begin{dimo}\text{}\ 
	\begin{itemize}
		\item	Le unità di $R[x]$ sono le stesse di $R$. \\
			Infatti se  $f\cdog g = 1$ allora  $deg(f) + deg(g) = deg(f\cdot g) = deg (1) = 0$ 
		\item Se $f\in R$ allora le uniche fattorizzazioni in  $R[x]$  sono quelle in $R.$\\
			Quindi  $f$ irriducibile in $R$ se e solo se $f$ irriducibile in $R[x]$
		\item Resta da studiare il caso in cui  $f\in R[x]$ con  $deg(f)\geq 1$\\
			Supponiamo che  $f$ sia irriducibile in $R[x]$ e sia  $f = u\cdot v$ una fattorizzazione in  $\K[x]$ con  $u,v$ non invertibili. \\
			Abbiamo 
			\[
			 v = a\cdot b^{-1} \text{ con } a\in R[x], b\in R\setminus\{0\}
			.\] 
			$ \Rightarrow f = u\cdot b^{-1}\cdot b\cdot v$ in $\K[x]$\\
			ma  $b\cdot v\in R[x]$\\
			Allora basta verificare che  $f$ sia primitivo, poiché la fattorizzazione precedente fornirebbe una fattorizzazione di $f$ in $R[x]$ per il lemma.\\
			Dimostriamo che  $f$ è irriducibile in $R[x] \Rightarrow f$ primitiva\\
			Consideriamo $f$\\
			 \[
			f = c(f)\cdot f'
			.\] 
			con $f'$ primitivo in $R[x]$ \\
			$f$ irriducibile in $R[x]$\\
			$ \Rightarrow  c(f)$ invertibile in $R[x]$ oppure $f'$ invertibile in $R[x]$ \\
			Ma $deg(f') > 0$\\
			$ \Rightarrow f'$ non invertibile in $R[x]$\\
			 $ \Rightarrow  c(f)$ è invertibile\\
			 $ \Rightarrow f$ primitivo
	\end{itemize}
\end{dimo}
\begin{coro}
	$R$ è UFD $f\in R[x]$ è primo se e solo se è irriducibile.\\
\end{coro}
\begin{dimo}
	primo $\Rightarrow $ irriducibile (sempre vero per domini d'integrità)\\
	Dobbiamo verificare che irriducibile  $ \Rightarrow $ primo\\
	Abbiamo due casi:
	\begin{enumerate}
		\item $f\in R$ irriducibile in  $R$  $ \Rightarrow f$ primo in $R$ \hfill (usiamo  $R$ UFD)\\
			Se $f | u\cdot v \ \ \ $ con  $u,v\in R[x]$\\
			 $ \Rightarrow c(f) = f \ | \ c(u)\cdot c(v)$ in $R$\hfill (Per il lemma di Gauss)\\
			 $ \Rightarrow f|c(u)$ oppure $f | c(v)$\\
			 $ \Rightarrow f| u$ oppure $f | v$\\
			 $\Rightarrow  $ primo in $R[x]$
		 \item  $f$ primitivo e $f$ è irriducibile in $\K[x]$\\
			 Se  $f | u\cdot v$ in  $R[x]$\\
			 $ \Rightarrow f \ | \ u\cdot v$ in $\K[x]$\\
			 $ \Rightarrow f | u $ in $\K[x]$ oppure $f | v$ in $\K[x]$\\
			 Dato che $u,v\in R[x]$ e  $f$ primitivo, questo significa $f | u$ in $R[x]$ oppure  $f | v$ in  $R[x]$
	\end{enumerate}

\end{dimo}
\begin{teo}
	$R$ UFD, Allora $R[x]$ UFD
\end{teo}
\begin{dimo}
	$f\in R[x]$. Dimostriamo ceh esiste una fattorizzazione in irriducibili\\
	 $f = c(f)\cdot f'$
	 \begin{itemize}
	 	\item $c(f)\in R$ si fattorizza come prodotto di irriducibili in  $R$ (che sono anche irriducibili in  $R[x]$ )
		\item $f'\in R[x]$ è primitivo.\\
			se  $f'$ è irriducibile allora  $f$ si fattorizza in irriducibili in $R[x]$\\
			se invece $f'$   non è irriducibile in $R[x]$ allora  $f' = u\cdot v $ con $c,v\in R[x]$ non invertibili e primitivi (per il lemma di  Gauss)\\
			La tesi segue per induzione su $deg(f')$ fattorizzando  $u$ e  $v$
	 \end{itemize}
	 Verifichiamo che la fattorizzazione è unica\\
	 Supponiamo che $f = \varepsilon \cdot b_1\cdot b_2\cdot\ldots\cdot b_k = \eta\cdot c_1\cdot c_2\cdotldots \cdot c_h$ con $ \e \eta $ invertibili,\\ $b_i,c_j$ irriducibili in  $R[x]$\\
	 $b_1 $ irriducibile in $R[x] \Rightarrow  b_1$ primo in $R[x]$\hfill (Per il corollario)\\
	  $ \Rightarrow b_1 \ | \ c_1$ \hfill (a meno di permutare $c_1,\ldots,c_h$)\\
	  $ \Rightarrow b_1,c_1$ associati poiché $c_1$ irriducibile\\
	  $ \Rightarrow  c_1 = \lambda b_1$ con $\lambda$ invertibile\\
	   $ \Rightarrow \varepsilon \cdot b_1\cdot\ldots\cdot b_k = (\eta\cdot\lambda)\cdot b_1\cdot c_2\cdot\ldots\cdot c_h$\\
	   $ \Rightarrow b_1\cdot (\varepsilon b_2\cdot \ldots\cdot b_k - \eta\kambda\cdot c_2 \cdot\ldots\cdot c_h) = 0$ \\
	   $\e b_2\cdot\ldots\cdot b_k = \eta\lambda c_2\cdot\ldots \cdot c_h$\\
	   Si conclude per induzione su $k$

\end{dimo}
\end{document}
