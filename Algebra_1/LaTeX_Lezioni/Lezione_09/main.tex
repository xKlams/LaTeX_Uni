\documentclass[12px]{article}

\title{Lezione 9 Algebra I}
\date{2025-03-31}
\author{Federico De Sisti}

\input{../../../setup.tex}

\begin{document}
	\maketitle
	\newpage
	\subsection{Esercizio 14 scheda 15}
$R$ anello commutativo. $E\subseteq R$ sottoinsieme.\\
$V(E) := \{P\subseteq R \ | \ P$ ideale primo tale che $E\subseteq P\}$\\
$Spec(R) = \{P\subseteq R \ |\ P$ ideale primo$\}$ (spettro di  $R$)\\
\textbf{Obiettivi}\\
Definire una topologia su $Spec(R)$\\
 \textbf{Idea}\\
 Definire gli aperti come complementari dei $V(E)$ (che saranno quindi i chiusi di $Spec(R)$)\\
 dato un omomorfismo di anelli $f: R \rightarrow S$, definire una funzioen continua
 \[
 f^*:Spec(S) \rightarrow Spec(R)
 .\] 
 \textbf{Osservazione}\\
 $I\subseteq R$ ideale
 \begin{enumerate}
	 \item $\sqrt{I}:=\{a\in R \ | \ a^n\in I$ per qualche $n\in\Z_{>0}\}$ \\
	 $Nil(R) = \sqrt{0} = \{$elementi nilpotenti in $R\}$ (elementi che ad una certa potenza fanno 0)
 \item $\pi : R \rightarrow R/I$
	 \[
		 \sqrt{I} = \pi^{-1}(Nil(R/I))
	 .\] 
 $[a]\in Nil(R/I) \Leftrightarrow [a]^n = [0]$ in $R/I$\\
 $ \Leftrightarrow [a^n]=[0]$ in $R/I \Rightarrow  a^n\in I$
\item $V(E) = V(I) = V(\sqrt{I})$ dove $I = (E)$
 \end{enumerate}
 \subsection{Moduli}
 \begin{defi}
 	 $R$ anello, Un gruppo abeliano $(M, +)$ è un $R$-modulo sinistro tramite \\un'applicazione 
	 \[
	 \begin{aligned}
		 $\cdot : &R\times M \rightarrow M$\\
			  &(r,m) \rightarrow r\cdot m
	 \end{aligned}
	 .\] 
	 se valgono le seguenti proprietà:
	 \begin{enumerate}
		 \item $(r + s)\cdot m = r\cdot m + s\cdot m$ \ \ $\forall r,s\in R \ \ \forall m\in M$
		 \item  $(r\cdot s)\cdot m = r\cdot (s\cdot m) \ \ \ \forall r,s\in R\ \ \ \forall m\in M$
		 \item $r\cdot (m + n ) = r\cdot m + r\cdot n \ \ \ \forall r\in R\ \ \forall m,n\in M$
		 \item  $1_R\cdot m = m \ \ \ \forall m\in M$
	 \end{enumerate}
 \end{defi}
 \textbf{Esempi}\\
 \begin{enumerate}
	 \item Se $R$ è un campo gli $R$-moduli sinistri sono gli spazi vettoriali su $R$
	 \item  $(G,+)$ gruppo abeliano è uno $\Z$-modulo (sinistro), basta definire 
		 \[
		 \begin{aligned}
			&\Z\times G \rightarrow \Z\\
			& (n,g) \rightarrow n\cdot g
		 \end{aligned}
		 .\] 
		 dove 
		 \[
		 $n\cdot g = \begin{cases}
			 0 \ \ \ \text{se } n= 0\\
			 \frac{n}{|n|}(g + \ldots + g) \ \ \  \text{se }n\neq 0
		 \end{cases}$
		 .\] 
	 \item $R$ è un $R$-modulo sinistro tramite \[
	 \begin{aligned}
		&R\times R \rightarrow R\\
		&(r,s) \rightarrow r\cdot s
	 \end{aligned}
	 .\] 
 \item $R^n = R\times\ldots\times R \ \ \ \ n\geq 1$\\
	 è un  $R$-modulo sinistro tramite
	 \[
		 \begin{aligned}
		 &R\times R^n \rightarrow R^n \\
		 &(r,\matrice{r_1\\\vdots\\r_n}) \rightarrow \matrice{r\cdot r_1\\\vdots\\r\cdot r_n}
		 \end{aligned}
	 .\] 
 \end{enumerate}
 \textbf{Esercizio}\\
 $R$ anello, $M$ $R$-modulo (sinistro)\\
 Dimostrare:
 \begin{enumerate}
	 \item $0_R\cdot m = 0_M\ \ \ \forall m\in M$
	 \item  $r\cdot 0_M = 0_M \ \ \forall r\in R$
	 \item  $-1_R)\cdot m = -m \ \ \forall m\in M$
 \end{enumerate}
 \begin{defi}
 	$R$ anello\\
	$M,N$  $R$-moduli sinistri, Un omomorfismo di $R$-moduli è una funzione  $f: M \rightarrow N$ tale che:
	\begin{enumerate}
		\item $f(m + m') = f(m) + f(m')\ \ \ \ \forall m,m'\in M$
		\item  $f(r\cdot m) = r\cdot f(m) \ \ \forall m\in M, \ \forall r\in R$
	\end{enumerate}
 \end{defi}
 \begin{defi}
 	$M,N$ due $R$-moduli sinistri,\\
\[Hom_R(M,N) = \{f: M \rightarrow N\ | \ f \text{ omomorfismo di } R\text{-moduli}\}\]
 \end{defi}
 \textbf{Esercizio:}\\
 Se $R$ commutativo allora $Hom_R (M,N)$ è un  $R$-modulo sinistro tramite:
 \[
 \begin{aligned}
	&R\times Hom_R(M,N) \rightarrow Hom_R(M,N)
	& (r,f) \rightarrow (r\cdot f)(m) = r\cdot f(m)
 \end{aligned}
 .\] 
 \textbf{Attenzione}\\
 $R$ commutativo garantisce che $(r\cdot f)$ sia un omomorfismo $\forall r\in R \\ \forall f\in Hom_R(M,N)$\\
 \textbf{Esempio:}\\
 $R$ anello, $I\subseteq R$ ideale sinistro.\\
 Allora $I$ è un $R$-modulo sinistro tramite 
 \[
 \begin{aligned}
	&R\times I \rightarrow I\\
	& (r,m) \rightarrow (r\cdot m)
 \end{aligned}
 .\] 
 \textbf{Osservazione:}\\
 $R$ anello, $M\  R$-modulo sinistro.\\
 \[
 End_R(M) := Hom_R(M,M)
 .\] 
 è un anello con le operazioni di somma tra funzioni
 \[
	 (f + g)(m) := f(m) + g(m)
 .\] 
 e prodotto dato dalla composizione. (verificare per esercizio)\\
 \textbf{Osservazione}\\
 $R$ anello, $M$ $R$-modulo sinistro\\
 $\leadsto End_R(M)$ anello\\
 \textbf{Definiamo}
 \[
	 \begin{aligned}
		 \mu: &R \rightarrow End_R(M)\\
		      &r \rightarrow \mu_r
	 \end{aligned}
 .\]
 dove $\mu_r(m) = r\cdot m \ \ \ \forall r\in R, \ \ \forall m\in M$\\
  $\mu$ è un omomorfismo di anelli.\\
  Dobbiamo verificare:
  \begin{enumerate}
	  \item $\mu_{r+s} = \mu_r + \mu_s$\\
		  $\mu_{r+s}(m) = (r+s)\cdot m = r\cdot m + s\cdot m = \mu_r(m) + \mu_s(m)\hfill \forall m\in M$ 
	  \item  (prodotto in $R$) $\mu_{r\cdot s} = \mu_r\circ\mu_s$ (composizione di funzioni)\\
		  $\mu_{r\cdot s}(m) = (r\cdot s)\cdot m = r\cdot (s\cdot m) = \mu_r(\mu_s(m)) = \mu_r\circ\mu_s(m) \hfill \forall m\in M$
  \end{enumerate}
  \begin{defi}
  	$R$ anello, $M\ R$-modulo sinistro\\
	Un sottogruppo abeliano  $A\subseteq M$ si dice  $R$-sottomodulo di $M$ se 
	 \[
		 r\cdot n \in N \ \ \forall n\in N \ \ \forall r\in R
	.\] 
  \end{defi}
  \textbf{Osservazione }\\
  Se $R$ campo e $M$ spazio vettoriale su $R$ allora i sottomoduli di $M$ sono i sottospazi vettoriali
  \begin{defi}
  	$R$ anello $f: M \rightarrow N$ omomorfismo di $R$-moduli sinistri
	\begin{itemize}
		\item Il nucleo di  $f$ è 
			\[
				\ker(f) = \{m\in M\ | \ f(m) = 0_N\}
			.\] 
		\item L'immagine di  $f$ è 
			\[
				Im(f) = \{ n\in n\ | \ \exists m\in M \ \ f(m) = n\}
			.\] 
	\end{itemize}
  \end{defi}
  \textbf{Esercizio}\\
  Verificare le seguenti affermazioni:
  \begin{enumerate}
	  \item $\ker(f)\subseteq M$ è un $R$-sottomodulo
	  \item $Im(f)\subseteq N$ è un  $R$-sottomodulo
	  \item $f$ è iniettivo se e solo se $\ker(f) = \{O_M\}$
	  \item  $f$ è suriettiva se e solo se $Im(f) = N$.
  \end{enumerate}
  \textbf{Obiettivo:}\
  Teorema di omomorfismo per moduli\\
  \textbf{Osservazione}\\
  $R$ anello. $M$ $R$-sottomodulo sinistro.\\
  Ogni $R$-sottomodulo $N$ di  $M$ è in particolare un sottogruppo ($N\normale M$ poiché $M$ abeliano)\\
  Abbiamo il gruppo quoziente $M/N$ su cui definiamo la struttura di  $R$-modulo sinistro
  \[
  \begin{alinged}
	&R\times M/N \rightarrow M/N\\
	(r,m + N) \rightarrow r \cdot m + N
  \end{alinged}
  .\]
  \textbf{Osservazione}\\
  è ben definita!
  \[
  m - m'\in N \Rightarrow N\ni r\cdot (m-m') = r\cdot m - r\cdot m' 
  .\]
  \begin{teo}[di omomorfismi]
  	$R$ anello\\
	$f: M \rightarrow M'$ omomorfismo di $R$-moduli sinistri\\
	$N\subseteq M\ \ R$-sottomodulo tale che $N\subseteq \ker(f)$\\
	Allora  esiste un unico omomorfismo 
	\[
	\bar f :M/N \rightarrow M'
	.\] 
	tale che  AGGIUNGI DIAGRAMMA 2:49 sia commutativo (ovvero $\pi\circ \bar f = f$)
  \end{teo}
  \begin{coro}
  	$R$ anello. $M,M'\ R$-moduli sinistri. $N\subseteq M$ $R$-sottomodulo.\\
	Allora esiste una corrispondenza biunivoca:
	 \[
		 \bigg\lbrace \begin{aligned}
			 &\text{ omomorfismi } f: M \rightarrow M' \text { t.c. }\\ &N\subseteq \ker(f)
		 \end{aligned}\bigg\rbrace\leftrightarrow \{\text{omomorfismi} \ \bar f : M/N \rightarrow M'\}
	.\] 
  \end{coro}
  \begin{dimo}[del teorema]
  	dato il teorema di omomorfismo per gruppi, basta verificare che l'omomorfismo $\bar f$ sia $R$-lineare\\
	Ricordo
	\[
	\bar f(m + N) = f(m)
	.\] 
	Abbiamo
	\[
	\bar f(r\cdot (m + N)) = \bar f(r\cdot m + N) = f(r\cdot m) = r\cdot f(m) = r\cdot \bar f(m + N)
	.\] 
  \end{dimo}
\end{document}
