\documentclass[12px]{article}

\title{Lezione 1 Algebra}
\date{2024-10-01}
\author{Federico De Sisti}

\usepackage{amsmath}
\usepackage{amsthm}
\usepackage{mdframed}
\usepackage{amssymb}
\usepackage{nicematrix}
\usepackage{amsfonts}
\usepackage{tcolorbox}
\tcbuselibrary{theorems}
\usepackage{xcolor}
\usepackage{cancel}

\newtheoremstyle{break}
  {1px}{1px}%
  {\itshape}{}%
  {\bfseries}{}%
  {\newline}{}%
\theoremstyle{break}
\newtheorem{theo}{Teorema}
\theoremstyle{break}
\newtheorem{lemma}{Lemma}
\theoremstyle{break}
\newtheorem{defin}{Definizione}
\theoremstyle{break}
\newtheorem{propo}{Proposizione}
\theoremstyle{break}
\newtheorem*{dimo}{Dimostrazione}
\theoremstyle{break}
\newtheorem*{es}{Esempio}

\newenvironment{dimo}
  {\begin{dimostrazione}}
  {\hfill\square\end{dimostrazione}}

\newenvironment{teo}
{\begin{mdframed}[linecolor=red, backgroundcolor=red!10]\begin{theo}}
  {\end{theo}\end{mdframed}}

\newenvironment{nome}
{\begin{mdframed}[linecolor=green, backgroundcolor=green!10]\begin{nomen}}
  {\end{nomen}\end{mdframed}}

\newenvironment{prop}
{\begin{mdframed}[linecolor=red, backgroundcolor=red!10]\begin{propo}}
  {\end{propo}\end{mdframed}}

\newenvironment{defi}
{\begin{mdframed}[linecolor=orange, backgroundcolor=orange!10]\begin{defin}}
  {\end{defin}\end{mdframed}}

\newenvironment{lemm}
{\begin{mdframed}[linecolor=red, backgroundcolor=red!10]\begin{lemma}}
  {\end{lemma}\end{mdframed}}

\newcommand{\icol}[1]{% inline column vector
  \left(\begin{smallmatrix}#1\end{smallmatrix}\right)%
}

\newcommand{\irow}[1]{% inline row vector
  \begin{smallmatrix}(#1)\end{smallmatrix}%
}

\newcommand{\matrice}[1]{% inline column vector
  \begin{pmatrix}#1\end{pmatrix}%
}

\newcommand{\C}{\mathbb{C}}
\newcommand{\K}{\mathbb{K}}
\newcommand{\R}{\mathbb{R}}


\begin{document}
	\maketitle
	\newpage
	\section{Cosa c'è su e-leaning di Francesco Mazzini}
	Date appelli\\
	Esercizi settimanali\\
	All'esame ti chiedono due esercizi delle schede scelti a caso\\
	Ci sono 2 esoneri (primo 17 dicembre) (secondo ?? maggio)\\
\textbf{Libri}\\
M. Artin Algebra\\
IN. Hernstein: Algebra (difficile)\\
\section{Gruppi}
\begin{defi}[Gruppo]
	Un gruppo è un dato di un insieme $G$ con un'operazione $\cdot$ tali che:\\
	1) L'operazione è associativa\\
	\[f \cdot (g\codt h) = (f\cdot g)\cdot h \ \ \ \ \ \forall f, g, h\in G\]
	2) Esistenza elemento neutro
	\[
		\exists e\in G \text{    tale che     } g\cdot e = e\cdot g = g \ \ \ \forall g\in G
	.\] 
	3) esistenza degli inversi
	\[
		\forall g\in G \ \ \ \ \ \exists \ \ \ \ g^{-1}\in G \ \ \text{ tale che    } g^{-1}\cdot g = g\cdot g^{-1} = e
	.\] 
\end{defi}
\begin{nome}[notazione]
	$(G,\cdot)$
	dato  $g\in G$ denotiamo con: \\
	1) g^0 = e\\
	2) g^1 = g\\
	3) g^n = g\cdot\ldots\cdot g
	4) g^{-n} = (g^{-1})^n

\end{nome}
\textbf{Osservazione:}\\
Con questa notazione:\\
\[
	(g^n)^m = g^{nm}
\] 
\[
	g^n \cdot g^m = g^{n + m}
\] 
\textbf{Esempi}\\
1) $(\mathbb Z, + ), (\mathbb Q, + ), (\R, +)$\\
2)  $GL_n(\K) = \lbrace{A\in Mat_{n\times n}(\K) | det(A) \neq 0\rbrace$ con prodotto\\
	3) $SL_n(\K) = \lbrace{A\in Mat_{nn}(\K) | det(A) = 1\rbrace$\\
4) $X$ insieme\\
\text{ } \hspace{90px} $S_X = \lbrace$ funzioni  $X \rightarrow X$ invertibili$\rbrace$\\
\textbf{Speciale}
Se $X = \lbrace 1,\ldots, n\rbrace}$\\
Allora chiamiamo 
 \[
S_n = S_X
.\] 
(è i lgruppo di permutazioni su $n$ elementi)\\
Si chiama gruppo simmetrico
\begin{defi}[Gruppo diedrale]
	$n\geq 3$ Consideriamo l'n-agono regoalre nel piano (3-agono, triangolo)\\
	 $D_n$ è l'insieme delle simmetrie del piano che preservano l' $n$-agono\\
	 Si chiama gruppo diedrale, l'operazione è la composizione
\end{defi}
\textbf{Esempio:}\\
Per $n=3$ abbiamo  $D_3$\\
\textbf{TODO INSERISCI DISEGNO gruppo diedrale}\\
\textbf{Esercizio}\\
Determina gli inversi e tutti i possibili prodotti degli elementi di $D_3$
\begin{defi}[Gruppo Abeliano]
	$(G,\codt)$ gruppo si dice Abeliano se l'operazione è commutativa \[f\cdot g = g\cdot f)\]
\end{defi}
\begin{defi}[Gruppo finito]
	$(G,\cdot)$ gruppo si dice finito se la sua cardinalità è finita \[|G| < +\infty\]
\end{defi}
\begin{defi}[Ordine del gruppo]
	L$(G,\cdot)$ gruppo, l'ordine di $G$ è $|G|$
\end{defi}
\begin{defi}[Ordine di un elemento]
	$ord(g) = \min \lbrace{n\in \mathbb N | g^n = e\rbrace$\\
		se $\cancel\exists n\in \mathbb N$  tale che $ g^n = e$\ \ \  poniamo  \ \ $ord(g) = +\infty$
\end{defi}
\begin{defi}[Gruppo ciclico]
	$n\geq 3$ consideriamo  $C_n$ l'insieme delle isometrie del piano che preservano l' $n$-agono e preservano l'orientazione, questo si chiama gruppo cicliclo
\end{defi}
\textbf{Esempio}\\
Nel caso di $n = 3$ abbiamo solamente 3 elementi: identità, e le due rotazioni (ordine dispari)
\textbf{Esercizi}\\
1) si dimostri che l'elemento neutro in un gruppo è unico\\
2) si dimostri che ogni elemento in un gruppo ammette un unico elemento inverso\\
per casa\\
1) Trvoare un'applicazione biunivoca $S_3 \rightarrow D_3$\\
2) Dimostrare che non esiste un'applicazione biunivoca $S_4 \rightarrow D_4$\\
3) Dimostrare che i seguenti nkn sono gruppi\\
$\cdot Mat_{n\times n} (\K)$ con prodotto riche per colonne\\
 $\codt GL(\K)$ con somma tra matrici\\
 $\codt \mathbb Z \mathbb Q \R con il prodotto$\\
  \begin{prop}
	  $(G, \cdot)$ gruppi finito, Allora ogni elemento ah ordine finito
 \end{prop}
 \begin{dimo}
 	$g\in G$ Considero il sottoinsieme
	 \[
	A = \lbrace g, g^2, g^3, \ldots\rbrace\subseteq G
	.\] 
	quindi $|A|<+\infty \Rightarrow \exists s,t\in \mathbb N, s> t$  tali che \[
	g^s = g^t
	.\] 
	Moltiplico per $g^{-t}$ a destra
	\[
		g^s = g^t \ \ \ \Rightarrow  \ \ \ g^s\cdot g^{-t} = g^t\cdot g^{-t} \ \ \ \Rightarrow \ \ \ g^{s-t} = e
	.\] 
	Quindi $n = s-t\geq 1$ e $g^n = e $ $ \Rightarrow ord(g) \leq n < +\infty $
 \end{dimo}
 \begin{defi}[Sottogruppo]
 	$(G, \cdot)$ gruppo $H\subseteq$ G sottosinsieme, si dice che H è un sottogruppo se $(H,\cdot)$ è un gruppo.\\
	In tal caso scriveremo $H \leq G$
 \end{defi}
 \textbf{Osservazione}\\
 $(G,\cdot)$ gruppo,  $G\subseteq G$ sottoinsieme allora  $H\leq G$se H è chiuso rispettto a $\cdot$ \\e  $H$ è chiuso rispetto agli inversi \\(se  $g, h\in G \Rightarrow g\cdot h\in H$ e se $h\in H \Rightarrow h^{-1}\in H)$

 \begin{prop}
 	$(G, \cdot)$ gruppo $H\subseteq G$ sottoinsieme con  $|H| < +\infty$ Allora:\\
	1)  $H \leq G$ se e solo se  $H$ è chiuso rispetto a $\cdot$
 \end{prop}
 \begin{dimo}
 	$( \Rightarrow )$ ovvia\\
	$ ( \Leftarrow )$ basta dimostrare che $H$ è chiuso rispetto alĺ inverso ovvero\\
	se $|H| < +\infty$\\
	e  H chiuso rispetto a  $\cdot$\\
	Allora  $H$ è chiuso rispetto agli inversi\\
	Sia $h\in H$\\
	$A = \lbrace h, h^2, h^3,\ldots\rbrace\subseteq H$\\
	Allora  $|A|<\infty $\\
	Ragionando come prima deduciamo  $ord(h) < + \infty$
	 \[
		 h\cdot h^{ord(h) -1} = h^{ord(h) -1}\cdot h = e
	.\] 
	Quindi $h^{-1} = h^{ord(h)-1} = h\cdot \ldots\cdot h\in H \Rightarrow h^{-1}\in H$
 \end{dimo}
 \textbf{Esempi}\\
 1)$C_n\leq D_n$\\
 2)  $SL_n(\K)\leq GL_n(\K)$\\
 3) $(G, \cdot)$ gruppo $g\in G$ 
 \[
 <g> = \lbrace g^n \in G| n\in \mathbb Z\rbrace
 .\] 
 Allora  $<g>\leq G$\\
  \textbf{Congruenze}\\
  $(G,\cdot)$ gruppo $H\leq G$\\
   \begin{defi}
  	$f,g\in G$ si dicono congruenti modulo  $H$ se 
	 \[
		 f^{-1}\codt g\in H
	.\] 
	In tal caso scriveremo 
	\[
	f\equiv g \ \ \ mod \ \ \ H
	.\] 
  \end{defi}
  \textbf{Esercizio}\\
  Dimostrare che al congruenza modulo $H$ definisce una relazione di equivalenza su  $G$\\
  \textbf{Suggerimento}\\
  $(f^{-1}\cdot g)^{-1} = g^{-1}\cdot (f^{-1})^{-1} = g^{-1}\cdot f\\$
  e $H$ è chiuso rispetto agli inversi\\
  \textbf{Esercizi:}\\
  $(G,\cdot)$ è un gruppo $H\leq G$ Allora la classe di equivalenza di $g\in G$ modulo  $H$ è il sottoinsieme
   \[
   gH = \lbrace g\cdot h | h\in H\rbrace
  .\] 
  C'è una classe di equivalenza speciale in $G$ data da 
  \[
  e\cdot H = H
  .\] 
  l'unica ad essere un sottogruppo\\
\hline \ \\
Dimostrare che esiste un'applicazione biunivoca tra $H \rightarrow gH\ \ \ \forall g\in G$

\end{document}
