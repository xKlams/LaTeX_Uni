\documentclass[12px]{article}

\title{Lezione 1 Algebra I (Secondo semestre)}
\date{2025-03-09}
\author{Federico De Sisti}

\usepackage{amsmath}
\usepackage{amsthm}
\usepackage{mdframed}
\usepackage{amssymb}
\usepackage{nicematrix}
\usepackage{amsfonts}
\usepackage{tcolorbox}
\tcbuselibrary{theorems}
\usepackage{xcolor}
\usepackage{cancel}

\newtheoremstyle{break}
  {1px}{1px}%
  {\itshape}{}%
  {\bfseries}{}%
  {\newline}{}%
\theoremstyle{break}
\newtheorem{theo}{Teorema}
\theoremstyle{break}
\newtheorem{lemma}{Lemma}
\theoremstyle{break}
\newtheorem{defin}{Definizione}
\theoremstyle{break}
\newtheorem{propo}{Proposizione}
\theoremstyle{break}
\newtheorem*{dimo}{Dimostrazione}
\theoremstyle{break}
\newtheorem*{es}{Esempio}

\newenvironment{dimo}
  {\begin{dimostrazione}}
  {\hfill\square\end{dimostrazione}}

\newenvironment{teo}
{\begin{mdframed}[linecolor=red, backgroundcolor=red!10]\begin{theo}}
  {\end{theo}\end{mdframed}}

\newenvironment{nome}
{\begin{mdframed}[linecolor=green, backgroundcolor=green!10]\begin{nomen}}
  {\end{nomen}\end{mdframed}}

\newenvironment{prop}
{\begin{mdframed}[linecolor=red, backgroundcolor=red!10]\begin{propo}}
  {\end{propo}\end{mdframed}}

\newenvironment{defi}
{\begin{mdframed}[linecolor=orange, backgroundcolor=orange!10]\begin{defin}}
  {\end{defin}\end{mdframed}}

\newenvironment{lemm}
{\begin{mdframed}[linecolor=red, backgroundcolor=red!10]\begin{lemma}}
  {\end{lemma}\end{mdframed}}

\newcommand{\icol}[1]{% inline column vector
  \left(\begin{smallmatrix}#1\end{smallmatrix}\right)%
}

\newcommand{\irow}[1]{% inline row vector
  \begin{smallmatrix}(#1)\end{smallmatrix}%
}

\newcommand{\matrice}[1]{% inline column vector
  \begin{pmatrix}#1\end{pmatrix}%
}

\newcommand{\C}{\mathbb{C}}
\newcommand{\K}{\mathbb{K}}
\newcommand{\R}{\mathbb{R}}


\begin{document}
	\maketitle
	\newpage
	\subsection{Ideali primi e massimali}
	\begin{defi}
		Sia $(R, +, \cdot)$ un anello. Un ideale $I\subseteq R$ si dice primo se
		 \begin{itemize}
			 \item $I\neq R$
			 \item $\forall a,b\in R$ se $a\cdot b\in I \Rightarrow a\in I \vee b\in I$
		\end{itemize}
	\end{defi}
	\begin{teo}
		$(R,+,\codt)$ anello $I\subseteq R$ ideale (bilatero). Allora l'anello quoziente $R/I$ è dominio d'integrità $ \Leftrightarrow I$ è ideale primo
	\end{teo}
	\begin{dimo}
		Per ogni $a,b\in I$, la proprietà :
		\[
			[a]\cdot[b] = [a\cdot b] = [0] \text { in } R/I \Rightarrow [a] = [0] \vee [b] = [0] \text{ in } R/I
		.\] 
		è equivalente a richiedere $a\cdot b \in I  \Rightarrow a\in B \vee b\in I$
	\end{dimo}
	\textbf{Esempio}\\
	$R = \C[x]/(x^2)$\\
	Osserviamo che spazio vettoriale  $\C[x]/(x^2) = \C\oplus \C[x]$ \\
	\textbf{Ricorda}\\
	Gli ideali di $\C[x]/(x^2)$ sono in corrispondenza biunivoca con gli ideali di  $\C[x]$ che contengono $(x^2)$\\
	L'ideale  $(x)\in\C[x]$ contiene $(x^2)$ e $(x)/(x^2)$ in $\C[x]/(x^2)$ è un ideale primo\\
	Infatti:\\
	$C[x]/(x^2)/x/(x^2)\con \C \Rightarrow $ è un corpo $ \Rightarrow  $ è un dominio d'integrità\\
	Osserviamo che l'ideale banalae in $C[x]/(x^2)$ è  $(x^2)/(x^2)$\\
	il quale non è primo infatti  $x\cdot x = x^2$\\
	$ \Rightarrow [x]\cdot[x] = [x^2] = [0] $ in $C[x]/(x^2)$\\
	 \textbf{Osservazione}\\
	 $\C[x]/(x^2)$ si chiama
	 \begin{itemize}
		 \item "algebra dei numeri duali"
		 \item "fat point" (Geometricamente è un punto)
	 \end{itemize}
	 \begin{defi}
		 $(R,+,\cdot)$ anello, $I\subseteq R$ si dice ideale massimale se:\\
		 \begin{itemize}
			 \item $I\neq R$
			 \item Dato un ideale $J\subseteq R$ tale che $I\subseteq J$, si ha\\
				 $I = J \ \ \vee \ \ J = R$

		 \end{itemize}
	 \end{defi}
	 \begin{teo}
	 	$(R,+,\cdot)$ anello commutativo,  $I\subseteq R$ ideale\\
		$I$ è massimale se e solo se $R/I$ è un campo
	 \end{teo}
	 \begin{dimo}
		 Ricordo che esiste una corrispondenza biunivoca tra $\{$ Ideali di $R$ che  contengono $I\} \leftrightarrow\{$  ideali di  $R/I\}$\\
		  $J -> J/I$\\
		   $ \Rightarrow I$ massimale se e solo se $R/I$ contiene solo ideali banali\\
		   $ \Rightarrow $ Sappiamo inoltre che (data la commutatività per ipotesi), $R/I$ contiene solo ideali banali  $ \Leftrightarrow R/I$ è banale
	 \end{dimo}
	 \textbf{Esercizio}\\
	 $n\geq 1 $ intero, $(n)\subseteq \Z$ ideale in  $(\Z,+,\cdot)$\\
	 dimostra che sono equivalenti
	  \begin{itemize}
		  \item $(n) $ è ideale primo\\
		  \item $n$ è numero primo
	 \end{itemize}
	 \subsection{Polinomi}
	 In questa sezione lavoriamo con anelli commutativi.\\
	 Problema $S$ anello commutativo, $R\subseteq S$ sottoanello, $t\in S$\\
	 Vogliamo costruire il più piccolo sottoanello $B$ di $S$ che contenga $R$ e $t$\\
	 \textbf{Osservazione}\\
	 Ogni sottoanello è chiuso rispetto alle operazioni.\\
	 \begin{itemize}
		 \item $t\in B \Rightarrow t^n = t\cdot\ldots\cdot t\in B$ ($n$ volte) \ \ $\forall n\geq 1$ intero
		 \item  $r\in R \Rightarrow r\cdot t^n \in B$ 
		 \item $r_1,\ldots, r_k\in R\subseteq B \Rightarrow r_0 + r_1t + \ldots r_kt^k\in B$ \\
	 \end{itemize}
	 Deduciamo che $R[t]\subseteq B$ dove $R[i] = \{r_0, + r_1 t + \ldots + r_k t^k \ | \ k\in\N, r_0,\ldots, r_l\in R\}$
	 \begin{prop}
		 $R[t]=B$
	 \end{prop}
	 \begin{dimo}
		 La dimostrazione è lasciata al lettore (basta verificare che $R[t] $ è sottoanello di $S$
	 \end{dimo}
	 \textbf{Esempi}\\
	 1) $R = \R, S=\C, t=i$
	  \[
		  R[t] = R[i] = \{r_0 + r_1i + r_2i^2 + \ldots + r_ki^k \ | r_1,\ldots,r_k\in \R\}
	 \] 
	 \[
	  = \{c_0 + c_1i \ | \ c_0,c_1\in\R\} = \C
	 .\] 
	 Qual'è il problema? La scrittura $r_0 + r_1t + \ldots + r_kt^k$ non è unica.
	 \begin{defi}
	 	$R\subseteq S$ sottoanello (commutativo), $t\in S$, allora $t$ è trascendente su $R$ se la scrittura $r_0 + \ldots + r_kt^k$ è unica
	 \end{defi}
	 \begin{prop}
	 	$t$ è trascendente su $R$ se e solo se $r_0 + r_1t + \ldots + r_kt^k = 0 \Leftrightarrow r_0= r_1= \ldots = r_k = 0$
	 \end{prop}
	 \begin{dimo}
	 	$ (\Rightarrow )$ Se $t$ è trascendente $ \Rightarrow 0
		\in R$ ammette scrittura unica $ \Rightarrow $ vale la proprietà\\
		$ ( \Leftarrow)$ Se vale tale proprietà \\
		P.A.\\
		 $a_0 + a_1t + \ldots + a_kt^k = b_0 + b_1t + \ldots + b_ht^h$\\
		 Assumo $k\geq h$ senza perdita di generalità\\
		 Porto tutto a sinistra\\
		  $(a_0 - b_0) + (a_1 - b_1)t + \ldot + (a_h - b_h)t^h + \ldots a_kt^k = 0$\\
		  Dove tutti i termini sono gli  $r_i$ nella struttura precedente\\
		  Per ipotesi $ \Rightarrow a_i =b_i \ \ \forall i\leq h, a_j = 0\ \forall h < j \leq k\\
		   \Rightarrow $ la scrittura è unica 
	 \end{dimo}
	
\end{document}
