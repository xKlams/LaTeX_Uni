\documentclass[12px]{article}

\title{Lezione 8 Algebra I}
\date{2024-10-26}
\author{Federico De Sisti}

\usepackage{amsmath}
\usepackage{amsthm}
\usepackage{mdframed}
\usepackage{amssymb}
\usepackage{nicematrix}
\usepackage{amsfonts}
\usepackage{tcolorbox}
\tcbuselibrary{theorems}
\usepackage{xcolor}
\usepackage{cancel}

\newtheoremstyle{break}
  {1px}{1px}%
  {\itshape}{}%
  {\bfseries}{}%
  {\newline}{}%
\theoremstyle{break}
\newtheorem{theo}{Teorema}
\theoremstyle{break}
\newtheorem{lemma}{Lemma}
\theoremstyle{break}
\newtheorem{defin}{Definizione}
\theoremstyle{break}
\newtheorem{propo}{Proposizione}
\theoremstyle{break}
\newtheorem*{dimo}{Dimostrazione}
\theoremstyle{break}
\newtheorem*{es}{Esempio}

\newenvironment{dimo}
  {\begin{dimostrazione}}
  {\hfill\square\end{dimostrazione}}

\newenvironment{teo}
{\begin{mdframed}[linecolor=red, backgroundcolor=red!10]\begin{theo}}
  {\end{theo}\end{mdframed}}

\newenvironment{nome}
{\begin{mdframed}[linecolor=green, backgroundcolor=green!10]\begin{nomen}}
  {\end{nomen}\end{mdframed}}

\newenvironment{prop}
{\begin{mdframed}[linecolor=red, backgroundcolor=red!10]\begin{propo}}
  {\end{propo}\end{mdframed}}

\newenvironment{defi}
{\begin{mdframed}[linecolor=orange, backgroundcolor=orange!10]\begin{defin}}
  {\end{defin}\end{mdframed}}

\newenvironment{lemm}
{\begin{mdframed}[linecolor=red, backgroundcolor=red!10]\begin{lemma}}
  {\end{lemma}\end{mdframed}}

\newcommand{\icol}[1]{% inline column vector
  \left(\begin{smallmatrix}#1\end{smallmatrix}\right)%
}

\newcommand{\irow}[1]{% inline row vector
  \begin{smallmatrix}(#1)\end{smallmatrix}%
}

\newcommand{\matrice}[1]{% inline column vector
  \begin{pmatrix}#1\end{pmatrix}%
}

\newcommand{\C}{\mathbb{C}}
\newcommand{\K}{\mathbb{K}}
\newcommand{\R}{\mathbb{R}}


\begin{document}
	\maketitle
	\newpage
	\section{Prodotti tra gruppi}
	\subsection{Prodotto diretto di gruppi}
	\begin{defi}
		Siano $(G_1,\cdot)$, $(G_2, *)$ gruppi il loro prodotto diretto risulta l'insieme $(G_1\times G_2)$ dotato dell'operazione:
		\[
			(g_1,g_2)\cdot(f_1,f_2) = (g_1\cdot f_1, g_2 * f_2) \ \ \forall g_1,f_1\in G_1, \ \ \forall g_2,f_2\in G_2
		.\]
		e lo indichiamo con $(G_1\times G_2)$
	\end{defi}
	\begin{prop}bold
		$(G_1\times G_2, \cdot)$ è un gruppo
	\end{prop}
	\begin{dimo}
		L'associatività segue da quella di $\cdot$ e  $*$ l'elemento neutro è  $(e_1,e_2)$\\
		l'inverso di $(g,f)$ con  $g\in G_1$ e $f\in G_2$ risulta $(g^{-1},f^{-1})$
	\end{dimo}
	\textbf{Esercizio}\\
	$(G_1, \cdot)$ e  $(G_2,*)$ gruppi\\
	Dimostrare:
	1) $|G_1\times G_2| = |G_1||G_2|$\\
	2) $G_1\times G_2$ è abeliano se e solo se $G_1$ e $G_2$ sono entrambi abeliani\\
	3) Dati due sottogruppi $H\leq G_1$ e $K\leq G_2 \Rightarrow H\times K\leq G_1\times G_2$ \\
	4) Dati $H\trianglelefteq G_1$ e $K\trianglelefteq G_2 \Rightarrow H\times K \trianglelefteq G_1\times G_2$ \\
	5) Dati  $H\trianglelefteq G_1$ e $K\trianglelefteq G_2$\\
	\[
		G_1/H \times G_2/H \cong \bigslant{G_1\times G_2}{H\times K}
	.\] 
	\begin{dimo}[4,5]
	\[
\begin{tikzcd}
G_1 \times G_2 \arrow[r, " \varphi"] \arrow[d] & \frac{G_1}{H} \times \frac{G_2}{K} \\
\frac{(G_1 \times G_2)}{ker \varphi} \arrow[ur, dotted, "\exists ! \bar \varphi"]
\end{tikzcd}
\]

dove 
\[
\varphi(g_1, g_2) = \left( g_1 H, g_2 K \right)
\]
	Dal primo teorema di isomorfismo
	\[
		Im \varphi \cong \frac {G_1\times G_2}{ker \varphi}
	.\] 
	$\cdot \varphi$ suriettiva poichè $\pi_H e \pi_K$ sono suriettive\\
	\begin{aligned*}
	$\cdot$ $\ ker \varphi = \{(g_1,g_2)\in G_1\times G_2| \varphi(g_1,g_2) = (H,K)\} \\= \{(g_1,g_2) | g_1H = H \ e \ g_2K = K\}$ \\
	$\{(g_1,g_2) | g_1\in H,g_2\in K\} = H\times K$
	\end{aligned*}\\
	quindi $H\times K\trianglelefteq G_1\times G_2$
	\[
		\frac{G_1\times G_2}{H\times K}\cong G_1/H\times G_2/K
	.\] 
	\end{dimo}
	\textbf{Esercizio (importante)}\\
	$(G_1,\cdot)$ e $(G_2,*)$ gruppi\\
	$H,K\normale G_1\times G_2$ tali che $H\cap K = \{\tilde e\}$ dove  $\tilde e  = (e_1,e_2)$\\
	Dimostrare che ogni elemento di $H$ commuta con ogni elemento di K.
	\textbf{dimo}Consideriamo $h\in H, k\in K$ e verifichiamo che  $hk = kh$\\
	 \textbf{Idea:}\\
	 Dimostrare che $hkh^{-1}k^{-1} = e$\\
	 Data l'ipotesi  $H\cap K = \{e\}$ è sufficiente dimostrare che  $hkh^{-1}k^{-1}\in H\cap K$
	  \textbf{Sfruttare la normalità di H e K}\\
	  Per l'esercizio sotto chiedi a Marco\\
	  \textbf{Esercizio}\\
	  $(G_1,\cdot)$, $(G_2,*)$ gruppi
	  \[
		  H:= G_1\times \{e_2\} = \{(g,e_2)|g\in G_1\}\leq G_1\times G_2\\
	  .\] 
	  \[
		  H:= e_1\times G_2\} = \{(e_1,g)|g\in G_2\}\leq G_1\times G_2\\
	  .\] 
	  Verificare che $H$ e $K$ soddisfano le ipotesi dell'esercizio precedente
	  \begin{defi}
	  	$(G,\cdot)$ gruppo $H,K\leq G$\\
		Diremo che  $G$ è \\
		Prodotto diretto interno di $H$ e $K$ se:\\
		1) $H,K\normale G$\\
		2)  $H\cap K = \{e\}$\\
		3)  $HK = G$
	  \end{defi}
	  \begin{teo}
	  	$(G,\cdot)$ gruppo\\
		1) Se $G$ è un prodotto diretto interno di $H,K\leq G$ allora  $G\cong H\times K$\\
		2) Se  $G\cong G_1\times G_2$ allora esistono $H,K\leq G$ tali che $G$ sia prodotto diretto interno di $H$ e $K$ e inoltre $H\cong G_1, K\cong G_2$
	  \end{teo}
	  \begin{dimo}[1]
	  	$\psi: H\times K \rightarrow G$\\
		$\ \ \ \ (h,k) \rightarrow hk$\\
		Dobbiamo verificare che $\psi$ sia isomorfismo\\
		1)$\psi$ è suriettiva perchè ogni elemento di $G$ si scrive come $hk$ quindi $Im(\psi) = G$\\
		2)È anche iniettiva infatti se  $\psi(g_1,k_1) = \psi(h_2,k_1)$
		\begin{gather*}
			\Rightarrow h_1k_1 = h_2k_2\\
			\Rightarrow h_2^{-1}h_1k_1 = k_2\\
			\Rightarrow h_2^{-1}h_1 = k_2k_1^{-1}\in H\cap K = \{e\}\\
			\Rightarrow \begin{cases}
				h_2^{-1}h_1 = e\\
				k_2k_1^{-1} = e
			\end{cases} \Rightarrow (h_1,k_1) = (h_2,k_2)\\
			\Rightarrow \psi \text{ iniettiva}
		\end{gather*}\\
		Bisogna in fine dimostrare che $\psi$ è un omomorfismo, ovvero che\\
		\[
		\psi(h_1h_2,k_1k_2) = \psi(h_1,k_1)\psi(h_2,k_2)
		.\] 
		dunque
		\[
		\psi(h_1h_2,k_1k_2) = h_1h_2k_1k_2 = h_1(h_2k_1)k_2 = h_1(k_1h_2)k_2 = \psi(h_1,k_1)\psi(h_2,k_2)
		.\] 
		Ricordando che tutti gli elementi di $H$ commutano con quelli di $K$
	  \end{dimo}
	  \begin{dimo}[2]
	  	Per ipotesi esiste un isomorfismo
		$ \varphi: G_1\times G_2 \rightarrow G$\\
		$\ \ \ (g_1,g_2) \rightarrow \varphi(g_1,g_2)$\\
		considero\\
		$H:= \varphi(G_1,\{e_2\})$\\
		$K:= \varphi(\{e_1\}\times G_2)$ \\
		Abbiamo visto che\\
		$\cdot G_1\times \{e_2\}\trianglelefteq G_1\times G_2 \rightarrow H\normale G$\\
		$\cdot \{e_1\}\times G_2\normale G_1\times G_2 \rightarrow K\normale G$ \\
		\[
			H\cap K = \varphi((G_1\times\{e_2\})\cap(\{e_1\}\times G_2)) = \{e\}
		.\] \\
		\[
			HK = \varphi((G_1\times \{e_2\})(\{e_1\}\times G_2)) = G
	.\]
	Le opportune restrizioni di $ \varphi$ forniscono gli isomorfismi
	\[
		H\cong G_1\times \{e_2\}\cong G_1
	.\] 
	\[
		K\cong \{e_1\}\times G_2\cong G_2
	.\] 

	  \end{dimo}
\textbf{Esempio:}\\
Siano $n,m\in\mathbb Z_{>0}$ t.c.\\
$MCD(n,m) = 1$\\
Consideriamo  $C_{nm} = <p>$\\
dove  $ord(p) = nm$\\
Considero
 \[
H = <\rho^m> \ \ \ K = <\rho^n>
.\] 
$|H| = ord(\rho^m) = n$\\
$|K| = ord(\rho^n) = m$\\
Verifichiamo che
 \[
	 C_{nm} \cong H\times K
.\] 
Dobbiamo mostrare:
\begin{enumerate}
	\item $H,K\normlae C_{nm}$\\
	\item $H\cap K = \{Id\}$\\
	\item $HK = C_{nm}$
	
\end{enumerate}\\
1) $C_{nm}$ abeliano, quindi $H,K\nomrale C_{nm}$\\
2) $H\cap K = ?$\\
sia $\rho^h\in H\cap K$\\
Allora
 \[
\begin{cases}
	\rho^h = (\rho^m)^{t_1}\\
	\rho^h = (\rho^h)^{t_2}
\end{cases} \ \ \ \ \ \begin{cases}
	m|h\\
	n|h
\end{cases}
.\] 
Ma $h\geq mcm(m,n) = mn \Rightarrow  h = mn \Rightarrow \rho^h = Id \Rightarrow H\cap K = \{Id\}$
\[
	|HK| = \frac{|H||K|}{|H\cap K|} = \frac {nm}1
.\] 
$ \Rightarrow HK$ è tutto chiuso quindi è $C_{nm}$\\
\begin{defi}[Automorfismo]
	$(G,\cdot)$ gruppo\\
	Un automorfismo di $G$ è un isomorfismo  $ \varphi:G \rightarrow G$
\end{defi}
\textbf{Osservazione}\\
$(G,\cdot)$ gruppo\\
$ \Rightarrow Aut(G) = \{\text{automorfismi di } G\}\\$
è un gruppo (rispetto alla composizione)\\
\textbf{Esempio:}\\
$(G,\cdot)$ gruppo\\
Fissato $g\in G$ definiamo\\
 \begin{aligned}
	 I_g: &G \rightarrow G\\
	      &f \rightarrow gfg^{-1}
\end{aligned}\\
$I_g$ si dice automorfismo interno\\
$Int(G) = \{\text{automorfismi interni di } G\}$ \\
\begin{prop}
	$Int(G)\normale Aut(G)$
\end{prop}
\begin{dimo}
	$If_G = I_e\in Int(G)$\\
	dato  $g\in G$ allora\\
	\[
		I_{g^{-1}} = I_g^{-1} \rightarrow \begin{cases}
			I_g\in Aut(G)\\
			Int(G) \text{ è chiuso rispetto agli inversi}
		\end{cases}
	.\] 
	$I_{g_2}\cdot I_{g_1}(f) = g_3g_2fg_2^{-1}g_2^{-1}
	= (g_2 g_1)f(g_2g_1)^{-1} = I_{g_2g_1}(f)$\\
	$I_{g_2}\cdot I_{g_1} = I_{g_2g_1}$\\
	quindi $Int(G)$ è chiuso rispetto alla composizione\\
	Quindi $Int(G)\leq Aut(G)$\\
	Basta verificare che:\\
	$ \varphi\circ Int(G)\circ \varphi^{-1}\subseteq Int(G)  \ \ \forall \varphi\in Aut(G)$\\
	ovvero dato $g\in G$\\
	 \[
		 \varphi\circ I_g\circ \varphi^{-1}\in Int(G)
	.\] 
	$\forall f\in G$\\
	\begin{aligend}
		 &\varphi\circ I_g \circ \varphi^{-1}(f) = \varphi(g \varphi^{-1}(f) g^{-1}) = \\
		 &\varphi(g) \varphi( \varphi^{-1}(f)) \varphi(g^{-1}) = \\
		 &= \varphi(g) f \varphi(g) =\\
		 & = I_{ \varphi(g)}(f)\\
		 &\Rightarrow \varphi\circ I_g \circ \varphi^{-1} = I_{ \varphi(g)}\in Int(G)
	\end{aligend}
\end{dimo}
\begin{defi}[Centro di un gruppo]
	$(G,\cdot)$ gruppo\\
	Il centro di $G$ è 
	\[
		Z(G):=\{g\in G| gf = fg \ \ \forall f\in G\}
	.\] 
\end{defi}
	\textbf{Osservazione}\\
	$Z(G)\normale G$\\
	 \textbf{Osservazione:}\\
	 $(G,\cdot)$ gruppo\\
	 Definiamo un omomorfismo\\
	 \begin{aligned}
		 \varphi: \ &G \rightarrow Int(G)\\
			& g \rightarrow I_g
	 \end{aligned}\\
		 \cdot $ \varphi$ è suriettiva\\
		 \cdot $ \varphi$ è omomorfismo\\
	 $ \varphi(g_2g_1)= \varphi(g_2) \varphi(g_1)$\\
	 $I_{g_2g_1} = I_{g_2}\circ I_{g_1}$\\
	 Chi è il $ker( \varphi)$\\
	 \begin{aligned}
		 ker( \varphi) &=  \{ g\in G | \varphi(g) = Id\} = \\
				 &= \{ g\in G | I_g = Id\} = \\
				 & = \{g\in G | \forall f\in G : I_g(f) = Id(f)\} = \\
				 & = \{g\in G | \forall f\in G : gfg^{-1} = f \} = Z (G)
	 \end{aligned}\\
	 Dal I teorema di isomorfismo si ha che
	 \[
	 Int(G) \cong G/Z(G)
	 .\] 
	 \subsection{Prodotto semidiretto}\\
	 Consideriamo due gruppi\\
	 $(N,\cdot)$ e $(H,*)$\\
	 Fissiamo un omomorfismo\\
	  \begin{aligned}
		  \phi: &H \rightarrow Aut(N)\\
			& h \rightarrow \o_n
	 \end{aligned}
	 \begin{defi}[Prodotto semidiretto]
	il prodotto semidiretto di $N$ e $H$ tramite $\o$ è l'insieme $N\times H$ dotato dell'operazione
	 \[
		 (n_1,h_1)\cdot (n_2,h_{2}) = (n_1\cdot \o_{h_1}(n_2), h_1*h_2)
	.\] 
	$\forall n_1,n_2\in N \ \ \ \ \forall h_1,h_2\in H$
\end{defi}
\begin{nota}
Indichiamo il prodotto semidiretto tra $N$ e $H$ con il simbolo $N\rtimes_{\o}$$H$
\end{nota}
\begin{prop}
	$N\rtimes_{\o} H$ è un gruppo
\end{prop}
\begin{dimo}
	Dato $(n,h)\in N\rtimes_{\o} H$\\
	l'inverso è dato da  $(\o_{h^{-1}}(n^{-1}),h^{-1})$
\end{dimo}
\newpage
\begin{defi}
	$(G,\cdot)$ gruppo\\
	$N,H\leq G$ Diremo che\\
	 $G$ è prodotto semidiretto interno di $N$ e $H$ se\\
	 \begin{itemize}[noitemsep]
		 \item $N\normale G$\\
		 \item $N\cap H = \{e\}$\\
		 \item $NH = G$
	 \end{itemize}
\end{defi}
\textbf{Esempio}\\
$D_n = <\rho, \sigma> \ \ N = <\rho> \normale D_n$\\
 $ H = <\sigma > \leq D_n$. Allora $D_n$ è prodotto semidiretto interno di $N$ e $H$
\end{document}

