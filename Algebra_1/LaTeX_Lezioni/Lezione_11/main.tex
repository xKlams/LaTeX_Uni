\documentclass[12px]{article}

\title{Lezione 11 Algebra 1}
\date{2025-04-14}
\author{Federico De Sisti}

\usepackage{amsmath}
\usepackage{amsthm}
\usepackage{mdframed}
\usepackage{amssymb}
\usepackage{nicematrix}
\usepackage{amsfonts}
\usepackage{tcolorbox}
\tcbuselibrary{theorems}
\usepackage{xcolor}
\usepackage{cancel}

\newtheoremstyle{break}
  {1px}{1px}%
  {\itshape}{}%
  {\bfseries}{}%
  {\newline}{}%
\theoremstyle{break}
\newtheorem{theo}{Teorema}
\theoremstyle{break}
\newtheorem{lemma}{Lemma}
\theoremstyle{break}
\newtheorem{defin}{Definizione}
\theoremstyle{break}
\newtheorem{propo}{Proposizione}
\theoremstyle{break}
\newtheorem*{dimo}{Dimostrazione}
\theoremstyle{break}
\newtheorem*{es}{Esempio}

\newenvironment{dimo}
  {\begin{dimostrazione}}
  {\hfill\square\end{dimostrazione}}

\newenvironment{teo}
{\begin{mdframed}[linecolor=red, backgroundcolor=red!10]\begin{theo}}
  {\end{theo}\end{mdframed}}

\newenvironment{nome}
{\begin{mdframed}[linecolor=green, backgroundcolor=green!10]\begin{nomen}}
  {\end{nomen}\end{mdframed}}

\newenvironment{prop}
{\begin{mdframed}[linecolor=red, backgroundcolor=red!10]\begin{propo}}
  {\end{propo}\end{mdframed}}

\newenvironment{defi}
{\begin{mdframed}[linecolor=orange, backgroundcolor=orange!10]\begin{defin}}
  {\end{defin}\end{mdframed}}

\newenvironment{lemm}
{\begin{mdframed}[linecolor=red, backgroundcolor=red!10]\begin{lemma}}
  {\end{lemma}\end{mdframed}}

\newcommand{\icol}[1]{% inline column vector
  \left(\begin{smallmatrix}#1\end{smallmatrix}\right)%
}

\newcommand{\irow}[1]{% inline row vector
  \begin{smallmatrix}(#1)\end{smallmatrix}%
}

\newcommand{\matrice}[1]{% inline column vector
  \begin{pmatrix}#1\end{pmatrix}%
}

\newcommand{\C}{\mathbb{C}}
\newcommand{\K}{\mathbb{K}}
\newcommand{\R}{\mathbb{R}}


\begin{document}
	\maketitle
	\newpage
	\subsection{Boh}
	\textbf{Obiettivo}\\
	Dare un teorema di struttura per i moduli finitamente generati su $R \ PID$ \\
	\begin{lemm}[Esercizio]
		$R$ $PID$, $N\subseteq R^n$ sottomodulo. Supponiamo che esista un elemento  $m\in N$ tale che $m = \icol{d\\0\\\vdots\\ 0}$ di lunghezza minimale in  $N$\\ allora $l(m)$ è di lunghezza \ubderline{minima} in  $N$
	\end{lemm}
	Ricordiamo la definizione di lunghezza minimale:\\
$l(m)\in R/\approx, l(m) = [d]$  $l(m)$ è un elemento minimale nell'insieme \\
$\{l(m')\in R/\approx\ | \ m'\in N\}$\\
\begin{dimo}
	Sia $a  = \icol{a_1\\\vdots\\a_n}\in N\subseteq R$\\
	Dimostriamo che $d \ |\  a_j \ \forall j\in \{1,\ldots,n\}$\\ (se vero, allora  $d \ | \ MCD(a_1,\ldots, a_n)$ quindi $[d]\preceq l(a))$ \\
	Procediamo in due passi:
	\begin{enumerate}
		\item[I. Step] $d \ | \ a_1$ a priori abbiamo $d_1 = MCD(d,a_1)$ la tesi diventa $d_1=d$ \\
			Per l'identità di Bezout, $\exists h,k\in R\ : \ d_1 = hd + ka_1$\\
			Allora \\
		$b:= hm + ka \in N$ (combinazione lineare di elementi in $N$)\\
		$b = \icol{hd + ka_1\\ka_2\\\vdots\\ka_n} = \icol{d_1\\ka_2\\\vdots\\ka_n}\in N$\\
		Quindi: $b\in N$ soddisfa\\
		\[
			l(b)  \preceq [d_1]\preceq [d] = l(m)
		.\] 
		Per la minimalità di $m$ in $N \Rightarrow  l(b) = l(m) \Rightarrow  [d_1] = [d] \Rightarrow  d \ | \ a_1$ 
	\item[II. step]  $d \ | \ a_j \ \ \forall j\in \{1,\ldots,n\}$\\
		Dato che  $d\ | \ a_1\ \ \ \exists h\in R$ tale che $a_1 = hd $\\
	\[c: = (1-h)m + a\in N\]
		Osserviamo che 
		\[
			c = \matrice{(1-h)d + a_1\\a_2\\\vdots\\ a_n} = \matrice{d\\a_2\\\vdots\\a_n}
		.\] 
		Quindi $c\in N$ soddisfa\\
		$l(c)\preceq [d] = l(m)$\\
		Per minimalità di $m$ in  $N$\\
		$l(c) = [d]$\\
		Allora  $d \ | \ a_j\ \ \ \forall j\in\{2,\ldots,n\}$
	\end{enumerate}
\end{dimo}
\begin{lemm}[Esercizio]
	Sia $R\ PID$  $N\subseteq R\oplus R^{n+1} = R\oplus R^n$ sottomodulo di  $R^n$, Supponiamo che esista un elemento di lunghezza minimale  $m = \icol{d\\0\\\vdots\\0}$\\
	Allora esiste  un $R$-sottomodulo $N'\subseteq R^n$  tale che $N = (d)\oplus N'$
\end{lemm}
\begin{dimo}
	Consideriamo $N' = \lbrace\icol{a_1\\\vdots\\a_n}\in R^n\ | \ \icol{0\\a_1\\\vdots\\a_n}\in N\rbrace$ \\
	Consideriamo la doppia inclusione:
	\begin{itemize}
		\item \Lbrac $(d)\oplus N'\subseteq N$\\
			infatti
			 \[
				 r_1\icol{d_1\\ 0 \\ \vdots \\ 0 } + r_2\icol{0\\a_1\\\vdots\\a_n} \in N
			.\] 
			poiché combinazione di elementi in $N$\\
		\item Viceversa, verifichiamo $N\subseteq (d)\oplus N'$ \\
			Sia $\icol{a_0\\a_1\\\vdots\\a_n}\in N$\\
			abbiamo dimostrato nel lemma precedente che $d\ | \ a_0 \Rightarrow  a_0 = hf$ \\
			$ \Rightarrow  \icol{a_0\\a_1\\\vdots\\a_n} = h\icol{d\\0\\\vdots\\0} + \icol{0\\a_1\\\vdots\\a_n}\in (d)\oplus N$
	\end{itemize}
\end{dimo}
\newpage
\begin{teo}[Struttura dei sottomoduli di moduli liberi di rango finito]
	$R\ \ PID$,  $N\subseteq R^n$ sottomodulo \\
	Allora esistono $d_1,\ldots, d_n\in R:$ 
	\begin{enumerate}
		\item $d_1 = \min\{l(m')\ | \ m'\in N\}$ 
		\item $d_j\ | d_{j + 1} \ \ \ \forall j\in \{1,\ldots, n-1\}$
		\item esiste un isomorfismo di  $R$-moduli, $\phi :R \rightarrow R^n$ tale che \\$\phi(N) = (d_1)\oplus\ldots\oplus (d_n)$
	\end{enumerate}

\end{teo}
\begin{dimo}
	Per induzione su $n$
	 \begin{itemize}
		 \item $n = 1$ allora $N\subseteq R$ è un ideale, quindi $R \ PID \Rightarrow  N = (d_1)$
		 \item $n > 1$ assumiamo l'enunciato per sottomoduli di  $R^n$ e dimostriamolo per sottomoduli di  $R^{n+1}$\\
			 Sia  $N\subseteq R^{n+1}$  $R$-sottomodulo\\
			 e sia $m\in N$ un elemento di lunghezza minimale in  $N$ (sappiamo che esiste!)\\
			 Esiste anche un isomorfismo di $R$-moduli $\phi_1: R^{n+1} \rightarrow R^{n+1}$\\
			 tale che $\phi(m) = \icol{d_1\\0\\\vdots\\0}$\\
			 Ora $\phi(m)$ è di lunghezza minima in  $\phi(N)$\\
			 Quindi esiste un complementare  $N'\subseteq R^n$ tale che  \\
			 $\phi_1(N)\cong (d_1)\oplus N'$\\
			 Per ipotesi induttiva a meno di un isomorfismo $\phi_2 : R^n \rightarrow R^n$ abbiamo $\phi_2(N) = (d_2)\oplus\ldots\oplus (d_{n+1})$\\
			 Abbiamo:\\
			 $N \xrightarrow{\phi_1} (d_1)\oplus N' \xrightarrow{id \oplus \phi_2} (d)\oplus \phi_2(N') = (d_1)\oplus\ldots\oplus (d_{n+1})$
	\end{itemize}
\end{dimo}
\begin{teo}[Struttura dei moduli finitamente generati su PID]
	$R\ PID$  $M$ $R$-modulo finitamente generato \\
	Allora esistono $d_1,\ldots d_n\in R$ tali che
	\begin{itemize}
		\item $d_j\ | \ d_{j+1} \ \ j\in\{1,\ldots,n-1\}$
		\item $M\cong R/(d_1)\oplus \ldots\oplus R/(d_n)$
	\end{itemize}
\end{teo}
\begin{dimo}
	Siano $\{m_1,\ldots,m_n\}$ generatori di $M.$\\
	Allora consideriamo l'omomorfismo suriettivo di $R$-moduli
	 \[
	\begin{aligned}
		R^n &\rightarrow M\\
		\icol{a_1\\\vdots\\a_n} &\rightarrow \sum^{j=1}_{n}a_jm_j
	\end{aligned}
	.\] 
	Dal primo teorema di isomorfismo segue
	\[
	M\cong R^n/ker(\phi)
	.\] 
	Dato che $ker(\phi)$ è un $R$-sottomodulo di $R^n$ esistono  $d_1,\ldots,d_n\in R$ tali che
	\begin{enumerate}
		\item $d_j \ | \ d_{j+1} \ \ \forall j\in\{1,\ldots, n-1\}$
		\item  $ker(\phi)\cong (d_1)\oplus\ldots\oplus (d_n)$
	\end{enumerate}\\
	$ \Rightarrow M\cong R^n/ker(\phi)\cong R/((d)\oplus\ldots\oplus (d_n))\cong R/(d_1)\oplus\ldots\oplus R/(d_n)$
\end{dimo}
\textbf{Osservazione:}
\begin{enumerate}
	\item Alcuni $d_j$ possono essere nulli o anche ripetersi.
	\item La scelta dei $d_j$ è "unica" (esercizio)
	\item Un gruppo abeliano $G$ ha un'unica possibile struttura di $\Z$-modulo.\\
		Quindi i concetti di gruppo abeliano e di $\Z$-modulo sono equivalenti.
\end{enumerate}
\begin{coro}
	$G$ gruppo abeliano, Allora esistono $d_1,\ldots,d_n\in \Z$ tali che 
	\begin{enumerate}
		\item $d_j\ | \ d_{j+1} \ \ \forall j\in \{1,\ldots, n-1\}$
		\item $G\cong \Z/(d_1)\oplus\ldots\oplus \Z/(d_n)$
	\end{enumerate}
\end{coro}
\begin{dimo}
	Segue dal teorema con $R = \Z$
\end{dimo}
\textbf{Osservazione}\\
$G$ gruppo abeliano
\[
\begin{aligned}
	\Z\times G &\rightarrow G\\
	(n,g) &\rightarrow g+ \ldots + g (\text {n volte})
\end{aligned}
.\] 
\newpage
\section{Successioni esatte corte}
Su $R$ anello
	\begin{defi}
		Una successione esatta corta di $R$-moduli è una coppia di omeomorfismi 
		\[
			M' \xrightarrow{i} M \xrightarrow{\pi} M''
		.\] 
	tali che
	\begin{enumerate}
		\item $i$ iniettiva
		\item $\pi$ suriettiva
		\item $\ker(\pi) = \text{im}(i)$
	\end{enumerate}
	\end{defi}
	\textbf{Esercizio}\\
	Dimostrare che 
	\begin{enumerate}
		\item $M$ finitamente generato $ \Rightarrow M''$ finitamente generato.
		\item $M', M''$ finitamente generati  $ \Rightarrow  M$ finitamente generato.
	\end{enumerate}
	\textbf{Soluzione}\\
	1)$\{m_1,\ldots, m_n\}$ generatori di $M$\\
	Allora dato che  $\pi$ è suriettiva\\
	$\{\pi(m_1),\ldots,\pi(m_n)\}$ sono generatori di $M''$.\\
	$\{m_1',\ldots, m_h'\}$ generatori di $M'$\\
	$\{m_1'',\ldots,m_k''\}$ generatori di $M''$\\
	Considero $\{m_1,\ldots, m_k\}\subseteq M$ tali che\\
	$\pi(m_j) = m_j''\ \ \ \forall j\in \{1,\ldots,k\} $\\
	Dimostriamo che
	 \[
	 \{i(m_1'),\ldots,i(m_h'), m_1,\ldots,m_k\}
	.\] 
	sono generatori di $M$\\
	Sia  $m\in M$\\
	 $\displaystyle \Rightarrow  \pi(m) = \sum^{k}_{j=1}r_jm_j''\in M'' = \pi( \sum^{k}_{j=1}r_jm_j)$\\
	 $ \displaystyle\Rightarrow m - \sum^{k}_{j = 1}r_j m_j\in \ker (\pi) = im(i)$, che è generata da $\{ i(m_1'),\ldots,i(m_n')\}$\\

	
\end{document}
