\documentclass[12px]{article}

\title{Lezione 4 Algebra 1}
\date{2025-03-13}
\author{Federico De Sisti}

\input{../../../setup.tex}

\begin{document}
	\maketitle
	\newpage
	\subsection{Divisibilità}
	\begin{nota}[Divisibilità]
		$R$ dominio d'integrità $a,b\in R$\\
		Diremo che:
		\begin{itemize}
			\item $a$ divide $b$ se esiste $q\in R$ tale che  $a\cdot q = b$ (scriviamo  $a|b$)
		\item se  $a|b$ allora  $a$ è \underline{divisore} di $b$ e $b$ è \underline{multiplo} di $a$
		\end{itemize}
	\end{nota}
	\textbf{Esercizio}\\
	$R$ dominio d'integrità $a,b,c\in R$
	 \begin{enumerate}
		 \item se $a|b$ e $b|c \ \ \Rightarrow  \ \ a | c$ 
		 \item se $a |b$ e $a|c \ \   \Rightarrow \ \ a|(b\pm c)$ 
		 \item se $a|b$ allora $a|b$
		 \item se  $a|b$ e $b|a$ 
		 \item se $a|b$ e $b|a$  $ \ \Rightarrow  \ \exists u,v\in R\ : \ \begin{cases}
		 	a = ab\\b = va \\ uv = 1
		 \end{cases}$
	\end{enumerate}
	\textbf{Soluzione 4}\\
	per definizione esistono $u,v\in R$ tali che  $ \begin{cases}
		a = ub \\ b = va
	\end{cases}$\\
	se $b = 0 $\\
	allora  $a = 0 \ \Rightarrow \ $ avremmo potuto scegliere $u=v=1$\\
	 se $b\neq 0 $\\
	 Allora $b = va = vub \Rightarrow b(1 0 vu) = 0$ \\
	 $ \Rightarrow 1 - vu = 0$ perché $R$ dominio\\
	  $ \Rightarrow u\cdot v = 1$ \\
	  \begin{defi}
	  	$R$ dominio d'integrità:
		\begin{enumerate}
			\item $a\in R$ si dice unità se  $a | 1$.
			\item  $a,b\in R$ si dicono associati se  $a|b$ e $b|a$
		\end{enumerate}
	  \end{defi}
	  \textbf{Osservazione}
	  \begin{enumerate}
		  \item $a\in R$ è unità se e solo se ammette inverso moltiplicativo.
		  \item dall'esercizio segue che $a,b$ sono associati se si ottengono l'uno dall'altro moltiplicando per un invertibile.
	  \end{enumerate}
	  \textbf{Esercizio}\\
	  $R$ dominio d'integrità, dimostrare che:
	  \begin{enumerate}
		  \item $a,b\in\Z$ sono associati se e solo se  $a = \pm b$
		  \item  $\K$ campo. $f,g\in \K[x]$ sono associati se e solo se  $f = \lambda g$ con $\lambda\in\K\setminus\{0\}$
	  \end{enumerate}
	  \begin{defi}
		  $R$ dominio d'integrità. $a\in\R\setminus\{0\}$\\
		   \begin{enumerate}
			   \item diremo che $a$ è \unerline{irriducibile} se $a = b\cdot c$  $ \Rightarrow \ a,b$ associati oppure $a,c$ associati
			   \item diremo che $a$ è \underline{primo} $a | b\cdot x \Rightarrow a | b$ oppure $a|c$
		  \end{enumerate}
	  \end{defi}
	  \textbf{Esercizio}\\
	  determinare tutti gli irriducibile e i primi in $\Z$ \\
	 \textbf{Osservazione}\\
	 $R$ dominio d'integrità  $a\in R$ primo\\
	  $ \Rightarrow (a)\subseteq R$ è un ideale primo\\
	  Se $b,c\in R$ tali che  $b\cdot c\in (a)$ allora  $b\in (a)$ oppure  $c\in (a)$\\
	  Ma  $b\cdot c \in (a)$ se e solo se $a | b\cdot c$\\
	  L'ipotesi chi primalità implica che  $a | b$ oppure  $a | c$  $ \Rightarrow b\in (a)$ oppure $c\in (a)$ \\
	  \textbf{Esercizio:}\\
	  $R$ dominio ad ideali principali. Allora $a\in R$ irriducibile $ \Rightarrow (a)\subseteq R$ massimale.\\
	  \textbf{Soluzione}\\
	  Sia $J\subseteq R$ ideale tale che $(a)\in J$. Per ipotesi  $J = (b)$ pewr qualche $b\in R$\\
	   $(a)\subseteq(b) \Rightarrow  b | a \Rightarrow  \exists p\in R: a=b \cdot q$ \\
	   Abbiamo due casi:\\
	   primo caso: $a,b$ associati\\
	   Allora $\exists q\in R$ invertibile tale che $b = ua \ \Rightarrow  \ (b)\subseteq (a)$ \\
	   $J = (b) = (a)$\\
	   secondo caso:  $a,q$ associati\\
	   $  \Rightarrow $ esiste $v\in R$ invertibile tale che $q = v\cdot a$\\
	    $ \Rightarrow a = bq = bva$ \\
	    $ \Rightarrow a\codt ( 1 - bv) = 0$ \\
	    $ \Rightarrow 1 - bv = 0$ \\
	    $ \Rightarrow b$ invertibile\\
	    $ \Rightarrow J = (b) = R$ \\
	    Quindi $(a)\subseteq R$ è massimale\\
	    \textbf{Esercizio:}\\
$R$ dominio a ideali principali, allora se $a$ è primo, $a$ è irriducibile\\
\textbf{Soluzione}\\
$a\in \R\setminus \{0\}$ $a$ primo verifichiamo che  $a$ irriducibile se $a = b\codt c$ allora :
\begin{cases}
	b|a\\ c |a \\ a| b\cdot c
\end{cases}
Deduciamo che $a,b$ associati oppure $a,c$ associati $ => a$ irriducibile\\
\begin{coro}
	In $\Z$ $a$ è primo se e solo se $a$ è irriducibile
\end{coro}
\begin{dimo}
	$ ( \Rightarrow)$ $(\Z,+,\cdot)$ è dominio a ideali principali quindi per l'esercizio  $a $ primo $ \Rightarrow \ a$  irriducibile\\
	$ ( \Leftarrow)$  $a$ irriducibile $ \Rightarrow $ $(a)\subseteq \Z$  massimale  $ \Rightarrow  (a)\subseteq \Z$ è ideale primo $ \Rightarrow  a$ è primo in $\Z$
\end{dimo}
\textbf{esercizio:}\\
$R = \Z[\sqrt{-5}] = \{a + b\sqrt{-5} | a,b\in\Z\}$\\
$1)$ dimostrare che  $R$ è un dominio d'integrità\\
$2)$  $3\in R$ non è primo\\
$ 3)$  $3\in R$ è irriducibile
\textbf{Soluzione}\\
$ 1)$  $\Z[\sqrt{-5}]$ è un sottoanello di $\C$ ma  $\C$ è un campo  $ \Rightarrow \C$ dominio d'integrità,\\
Quindi anche $\Z[\sqrt{-5}]$ è un dominio d'integrità.\\
$ 2)$  $3\cdot 7 = (4 + \sqrt{-5})(4 - \sqrt{-5})$.\\
Allora\\
$3$ divide $(4 + \sqrt{-5})(4 - \sqrt{-5})$ \\
D'altra parte $\nmid (4 \pm\sqrt{-5})$\\
Infatti.\\
$3(a + b\sqrt{-5}) = 3a + 3b\sqrt{-5}$ Ma  $3\nmid 4$ in  $\Z$ \\
$3)$ Verifichiamo che $3\in R$ è irriducibile\\
Supponiamo che  $3 = (a+b\sqrt{-5})(c + d\sqrt{-5})$\\
Vogliamo verificare che  $3$ e il primo termine oppure $3$ e il secondo termine sono associati.\\
Considero $||\cdot || $ norma in $\C$\\
\begin{gather*}
	 \Rightarrow  9 = ||a + b\sqrt{-5}||^2 + ||c + d\sqrt{-5}||^2\\
	 = (a^2 + 5b^2)\cdot (c^2 + 5a^2)
\end{gather*}
Quindi $a^2 + 5b^2 = \begin{cases}
	1\\3\\9
\end{cases}$\\
$ a^2 + 5b^2 = 3 \leadsto $ impossibile.\\
se $a^2 +  5b^2 = 9 \\\Rightarrow c^2 + 5d^2 = 1 \Rightarrow \begin{cases}
	c = \pm 1 \\ d = 0
\end{cases}$ \\
Quindi\\
$3 = \alpha +\beta$ allora  $\alpha = \pm 1 \leadsto 3,\beta$ associati\\
oppure  $\beta = \pm 1 \leadsto 3,\alpha$  associati
\subsection{UFD}
\begin{defi}
	$ R$ dominio d'integrità, $R$ si dice \underline{dominio a fattorizzazione unica} se:
	\begin{enumerate}
		\item per ogni $\a\in R\setminus\{0\}$ esiste una "fattorizzazione"  $a = u\cdot b_1\cdot\ldots\cdot b_h$\\
			tale che 
			\begin{itemize}
				\item $u$ unità in $R$
				\item  $b_i$ irriducibile per ogni $i\in\{1,\ldots,h\}$
			\end{itemize}
		\item Se
			$a\cdot b_1 \cdot \ldots\cdot b_h = v\cdot c_1\cdot\ldots\cdot c_k$\\
			con 
			\begin{itemize}
				\item $h = k$
				\item $\exists \omega\in S_h$ tale che $b_i,c_{\sigma(i)}$ associati per ogni  $i\in\{1,\ldots,h\}$
			\end{itemize}
	\end{enumerate}
\end{defi}
\begin{teo}
	$R$ dominio d'integrità,\\
	Allora  $R$ è UFD se e solo se valgono le seguenti condizioni:
	 \begin{enumerate}
		 \item Ogni elemento irriducibile è primo
		 \item Data una successione in $R$ di elementi\\
			 $a_1,\ldots,a_2,\ldots,a_r,\ldots$\\
			 tali che $a_{i+1}\  | \ a_i \ \ \ \ \forall i$ \\
			 si ha che esiste $i\in \Z_{>1}$ tale che $a_j,a_h$ siano associati $\forall h,k >  1$
	\end{enumerate}
\end{teo}
\begin{dimo}
	Supponiamo che $R$ sia UFD\\
	Verifichiamo (1):\\
	Sia $a\in R\setminus\{0\}$ irriducibile.\\
	Considero  $b,c\in R$ tali che  $a | bc$\\
	Allora  $\exists q\in R$ tale che.\\
	 $a\cdot q = b\cdot c$\\
	 Sfrutto l'ipotesi UFD
	  \[
	  q = \varepsilon \cdot t_1,\ldots,t_m
	 .\] 
	  \[
	  b = \eta \cdot r_1,\ldots,r_n
	 .\] 
	  \[
	  c = \delta \cdot s_1,\ldots,s_h
	 .\] 
	 dove $\varepsilon,\eta,\delta$ unità in  $R$\\
	  $t_i,s_i,r_i$ irriducibili in  $R$\\
	    $ \Rightarrow \varepsilon\cdot a \cdot t_1\ldots t_m = (\delta\eta)r_1,\ldots,r_n\cdot s_1,\ldots,s_h$\\
	    Per unicità della fattorizzazione $a$ è associato a un qualche $r_i$ (se $a | b$) oppure $s_i$ (se  $a | c$ ) \\
	    quindi $a$ è primo\\[10px]
	    Verifichiamo che  $UFD \Rightarrow 2 $ \\
	    Sia $a_1,\ldots, a_i,\ldots$\\
	    una successione in $R$ tale che $a_{i+1} \ | \ a_i \ \ \ \forall i$\\
	    Denotiamo:
	     $n_i = $ numero di irriducibili in una (qualsiasi) fattorizzazione di $a_i$\\
	     $ \Rightarrow  n_{i+1}\leq n_i$ \\
	     Ho una successione $n_1,n_2,\ldots,n_i,\ldots$ monotona decrescente\\
	     $ \Rightarrow $ definitivamente costante\\
	     $ \Rightarrow $ $\exists \underline i\in\Z_{\geq 1}$ tale che  $n_j = n_{\underline i} \ \ \forall j\geq \undelrine i$\\
	     Allora l'ipotesi\\
	     $a_k \ | \ a_{\underline i} \ \ $ per $k\geq \underline i$\\
	     $ \Rightarrow a_k\cdot q_k = a_{\underline i}$ \\
	     e UFD $ \Rightarrow  q_k$ invertibile.\\
	     Quindi $a_k,a_{\underline i}$ associati\\
	     $\forall k\geq \uderline{i} \Rightarrow (2)$ \\
	     Supponiamo ora che esistano valgano  $(1)$ e $(2)$ verifichiamo che  $R$ è $UFD$\\
	     Esistenza: sia $a_1\in R$ non invertibile e non irriducibile\\
	     $ \Rightarrow a_1 = a_2\cdot b_2$ tale che \\
	     $a_1,a_2$ non associati\\
	     $a_1,b_2$ non associati\\
	     Se per assurdo $a_1$ non ammette fattorizzazione lo stesso vale per $a_2$ oppure $b_2$\\
abbiamo costruito $a_2$ che 
\begin{itemize}
	\item $a_2 \ |\ a_1$
	\item $a_2$ non ammette fattorizzazione e non è invertibile e non è associato ad $a_1$
\end{itemize}

\end{dimo}

	  
	
\end{document}
