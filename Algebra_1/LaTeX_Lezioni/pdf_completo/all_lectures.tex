\documentclass{article}
\usepackage[utf8]{inputenc}
\usepackage{amsmath, amssymb}
\title{Compendio Lezioni del Corso: Algebra_1/}
\date{\today}
\author{Federico De Sisti}
\include{../../../setup.tex}
\begin{document}
\maketitle
\maketitle
	\newpage
	\subsection{Ideali primi e massimali}
	\begin{defi}
		Sia $(R, +, \cdot)$ un anello. Un ideale $I\subseteq R$ si dice primo se
		 \begin{itemize}
			 \item $I\neq R$
			 \item $\forall a,b\in R$ se $a\cdot b\in I \Rightarrow a\in I \vee b\in I$
		\end{itemize}
	\end{defi}
	\begin{teo}
		$(R,+,\codt)$ anello $I\subseteq R$ ideale (bilatero). Allora l'anello quoziente $R/I$ è dominio d'integrità $ \Leftrightarrow I$ è ideale primo
	\end{teo}
	\begin{dimo}
		Per ogni $a,b\in I$, la proprietà :
		\[
			[a]\cdot[b] = [a\cdot b] = [0] \text { in } R/I \Rightarrow [a] = [0] \vee [b] = [0] \text{ in } R/I
		.\] 
		è equivalente a richiedere $a\cdot b \in I  \Rightarrow a\in B \vee b\in I$
	\end{dimo}
	\textbf{Esempio}\\
	$R = \C[x]/(x^2)$\\
	Osserviamo che spazio vettoriale  $\C[x]/(x^2) = \C\oplus \C[x]$ \\
	\textbf{Ricorda}\\
	Gli ideali di $\C[x]/(x^2)$ sono in corrispondenza biunivoca con gli ideali di  $\C[x]$ che contengono $(x^2)$\\
	L'ideale  $(x)\in\C[x]$ contiene $(x^2)$ e $(x)/(x^2)$ in $\C[x]/(x^2)$ è un ideale primo\\
	Infatti:\\
	$C[x]/(x^2)/x/(x^2)\con \C \Rightarrow $ è un corpo $ \Rightarrow  $ è un dominio d'integrità\\
	Osserviamo che l'ideale banalae in $C[x]/(x^2)$ è  $(x^2)/(x^2)$\\
	il quale non è primo infatti  $x\cdot x = x^2$\\
	$ \Rightarrow [x]\cdot[x] = [x^2] = [0] $ in $C[x]/(x^2)$\\
	 \textbf{Osservazione}\\
	 $\C[x]/(x^2)$ si chiama
	 \begin{itemize}
		 \item "algebra dei numeri duali"
		 \item "fat point" (Geometricamente è un punto)
	 \end{itemize}
	 \begin{defi}
		 $(R,+,\cdot)$ anello, $I\subseteq R$ si dice ideale massimale se:\\
		 \begin{itemize}
			 \item $I\neq R$
			 \item Dato un ideale $J\subseteq R$ tale che $I\subseteq J$, si ha\\
				 $I = J \ \ \vee \ \ J = R$

		 \end{itemize}
	 \end{defi}
	 \begin{teo}
	 	$(R,+,\cdot)$ anello commutativo,  $I\subseteq R$ ideale\\
		$I$ è massimale se e solo se $R/I$ è un campo
	 \end{teo}
	 \begin{dimo}
		 Ricordo che esiste una corrispondenza biunivoca tra $\{$ Ideali di $R$ che  contengono $I\} \leftrightarrow\{$  ideali di  $R/I\}$\\
		  $J -> J/I$\\
		   $ \Rightarrow I$ massimale se e solo se $R/I$ contiene solo ideali banali\\
		   $ \Rightarrow $ Sappiamo inoltre che (data la commutatività per ipotesi), $R/I$ contiene solo ideali banali  $ \Leftrightarrow R/I$ è banale
	 \end{dimo}
	 \textbf{Esercizio}\\
	 $n\geq 1 $ intero, $(n)\subseteq \Z$ ideale in  $(\Z,+,\cdot)$\\
	 dimostra che sono equivalenti
	  \begin{itemize}
		  \item $(n) $ è ideale primo\\
		  \item $n$ è numero primo
	 \end{itemize}
	 \subsection{Polinomi}
	 In questa sezione lavoriamo con anelli commutativi.\\
	 Problema $S$ anello commutativo, $R\subseteq S$ sottoanello, $t\in S$\\
	 Vogliamo costruire il più piccolo sottoanello $B$ di $S$ che contenga $R$ e $t$\\
	 \textbf{Osservazione}\\
	 Ogni sottoanello è chiuso rispetto alle operazioni.\\
	 \begin{itemize}
		 \item $t\in B \Rightarrow t^n = t\cdot\ldots\cdot t\in B$ ($n$ volte) \ \ $\forall n\geq 1$ intero
		 \item  $r\in R \Rightarrow r\cdot t^n \in B$ 
		 \item $r_1,\ldots, r_k\in R\subseteq B \Rightarrow r_0 + r_1t + \ldots r_kt^k\in B$ \\
	 \end{itemize}
	 Deduciamo che $R[t]\subseteq B$ dove $R[i] = \{r_0, + r_1 t + \ldots + r_k t^k \ | \ k\in\N, r_0,\ldots, r_l\in R\}$
	 \begin{prop}
		 $R[t]=B$
	 \end{prop}
	 \begin{dimo}
		 La dimostrazione è lasciata al lettore (basta verificare che $R[t] $ è sottoanello di $S$
	 \end{dimo}
	 \textbf{Esempi}\\
	 1) $R = \R, S=\C, t=i$
	  \[
		  R[t] = R[i] = \{r_0 + r_1i + r_2i^2 + \ldots + r_ki^k \ | r_1,\ldots,r_k\in \R\}
	 \] 
	 \[
	  = \{c_0 + c_1i \ | \ c_0,c_1\in\R\} = \C
	 .\] 
	 Qual'è il problema? La scrittura $r_0 + r_1t + \ldots + r_kt^k$ non è unica.
	 \begin{defi}
	 	$R\subseteq S$ sottoanello (commutativo), $t\in S$, allora $t$ è trascendente su $R$ se la scrittura $r_0 + \ldots + r_kt^k$ è unica
	 \end{defi}
	 \begin{prop}
	 	$t$ è trascendente su $R$ se e solo se $r_0 + r_1t + \ldots + r_kt^k = 0 \Leftrightarrow r_0= r_1= \ldots = r_k = 0$
	 \end{prop}
	 \begin{dimo}
	 	$ (\Rightarrow )$ Se $t$ è trascendente $ \Rightarrow 0
		\in R$ ammette scrittura unica $ \Rightarrow $ vale la proprietà\\
		$ ( \Leftarrow)$ Se vale tale proprietà \\
		P.A.\\
		 $a_0 + a_1t + \ldots + a_kt^k = b_0 + b_1t + \ldots + b_ht^h$\\
		 Assumo $k\geq h$ senza perdita di generalità\\
		 Porto tutto a sinistra\\
		  $(a_0 - b_0) + (a_1 - b_1)t + \ldot + (a_h - b_h)t^h + \ldots a_kt^k = 0$\\
		  Dove tutti i termini sono gli  $r_i$ nella struttura precedente\\
		  Per ipotesi $ \Rightarrow a_i =b_i \ \ \forall i\leq h, a_j = 0\ \forall h < j \leq k\\
		   \Rightarrow $ la scrittura è unica 
	 \end{dimo}

% -------------------- Fine Lezione 1 --------------------

\maketitle
	\newpage
	\subsection{Ricordo}
$S$ anello commutativo $R\subseteq S$ sottoanello  $t\in S$\\
Abbiamo dimostrato che  $R[t] = \{ \sum^{k}_{i=0}r_it_i \ | \ k\in \Z_{>0}, \ \ t_i\in\R\}$\\
è il più piccolo sottoanello si $S$ contenente $\R$ e $t$ \\
\begin{defi}
	$t\in S$ si dice trascendente su $\R$ se per ogni $a\in\R[t]$ la scrittura
	\[
	s = \sum^{k}_{i=0}r_it^i
	.\] 
	è unica
\end{defi}
\textbf{Esercizio:}\\
Dimostrare che $i\in S$ è trascendente su  $\R $ se e solo se vale la seguente condizione
\[
	(*) \ \ \ r_0 + r_1t + \ldots + r_k t^k = 0 \Rightarrow r_0=r_1=\ldots=r_k = 0
.\] 
\textbf{Soluzione}\\
Se $t$ è trascendente allora $0\in\R$ ammette struttura polinomiale unica   $ \Rightarrow $ vale la proprietà.\\
Viceversa suppongo che valga $(*)$. Se \\
\[
a_0 + a_1t + \ldots + a_k t^k = b_0 + b_1t + \ldots + b_ht^h
.\] 
Assumo $k \geq k$
 \[
	 (a_0 - b_0) + (a_1-b_1)t + \ldots + (a_h - b_h)t^h + \ldots a_kt^k = 0
.\] 
$(*) \Rightarrow a_0 = b_0, \ldots, a_i = b_i \ \ \forall i \leq h, \ a_j = 0 \ \ \forall h < j \leq k$ \\
\textbf{Idea}\\
$R$ anello commutativo\\
$x$ simbolo\\
\[
	R[x] = \{ \sum^{k}_{i=0}r_ia^i \ | \ k\in\Z_{>0}, \ r_i\in\R\}
.\]
\textbf{Operazioni:}\\
$\displaystyle\left( \sum^{k}_{i=1}a_ix^i \right) + \left( \sum^{h}_{i=0}b_ix^i \right) = \sum^{max(h,k)}_{i=0}(a_i + b_i)x^i\\
\left( \sum^{h}_{i=0}a_ix^i \right) \cdot \left( \sum^{k}_{i=0 }b_ix^i \right) = \sum^{h + k}_{j = 0} \left( \sum^{}_{p + q = j}(a_p\cdot b_q) \right)$\\
\textbf{Osservazion}\\
Su $\R[x]$ è definita la funzione grado
\begin{center}
	\begin{aligned}
		deg : &$\R[x] \rightarrow \Z_{\geq0}\\
		      & p \rightarrow deg(p)$
	\end{aligned}
\end{center}
Se $p = \sum^{k}_{i=0}a_ix^i \ \ \ a_k\neq 0$\\
allora $deg(p) = k$ e  $p$ si dice  \textbf{monico} se $a_k = 1$ dove $k = deg(p)$\\
\begin{teo}[Divisione Euclidea]
	$R$ anello commutativo\\
	$f,g\in R[x], g$ monico\\
	Allora esistono $q,r\in\R[x]$ tali che 
	 \[
	f = q\cdot g + r
	.\] 
	con $deg(r) < def(g)$\\
	Tali $q$ e $r$ sono unici
\end{teo}
\begin{dimo}
	Procediamo per induzione su $deg(f)$\\
	Se  $deg(f) < deg (g)$\\
	scelgo  $q = 0$ e $f = r$\\
	Altrimenti  \\
	$deg(f) \geq deg(g)$\\
	scriviamo $f = \sum^{h}_{i=0  }a_ix^i$\\
	$g =  \left( \sum^{k-1}_{i=0}b_ix^i \right) + x^k$\\
	Considero\\
	\[
		\hat f : = f - a_kx^{h + k}\cdot g
	.\] 
	$ \Rightarrow deg(\hat f) < deg (f)$\\
	Per ipotesi induttiva\\
	$\exists \hat q, \hat r\in R[x]$ tali che\\
	$\hat f = \hat q \cdot g + \hat r$ \ \ \ con  $deg(\hat r) < deg(g)$\\
	Allora\\
	\[
		f  - a_kx^{h-k}\cdot g = \hat q\cdot g + \hat r \Rightarrow g = (a_hx^{h-k} + \hat q)\cdot g + \hat r
	.\] 
	con $deg(r) = deg(\hat r) < deg(g)$\\
	Supponiamo\\
	 \[
	f = q_1\cdot g + r_1 = q_2\cdot g + r_2
	.\] 
	\[
	\Rightarrow  (q_1-q_2)\cdot g = (r_2-r_1)
	.\] 
	$deg(q_1-q_2)\cdot g) \geq deg(g) > deg(r_2r_1)$\\
	$ \Rightarrow $ Assurdo\\
	$ \Rightarrow q_1=q_2 \ \ \Rightarrow r_2= r_1$
\end{dimo}
\begin{teo}
	$R$ anello commutativo\\
	$\phi: R \rightarrow S$ omomorfismo di anelli $r\in S$\\
	Allora esiste un unico omomorfismo di anelli  $\bar \phi: R[x] \rightarrow S$ tale che
\begin{enumerate}
	\item $\bar\phi(x) = t$\\
	\item $\bar\phi|_R = 0$
\end{enumerate}
\end{teo}
\begin{dimo}
	Le richieste danno $\var\phi:$
	 \[
		 \bar\phi \left( \sum^{k}_{i=0}r_ix^i \right) = \sum^{k}_{i=0}\phi(r^i)t^i
	.\] 
\end{dimo}
\textbf{Osservazione}\\
Stiamo dicendo che esiste l'omomorfismo $R \rightarrow R[x]$ dato dall'inclusione
\[
\begin{tikzcd}
R \arrow[r, "\phi"] \arrow[d, " \text{i} "] & S \\
R[x] \arrow[ru, "\exists \bar{\phi}", dashed]
\end{tikzcd}
\]
\textbf{Esercizio}\\
$R$ anello commutativo\\
$R[x]$ anello commutativo\\
$R[x][y]$ anello commutativo\\
\[
	\sum^{k}_{j=0} \left( \sum^{m_i}_{i =0}a_{ij}x^i \right)y^j
.\] 
E se procediamo al contrario?\\
$R[y][x]$ è uguale a quello precedente?\\
\[
	\sum^{k}_{j=0} \left( \sum^{m_i}_{i =0}a_{ij}y^i \right)x^j
.\] 
Dimostrare che esiste un isomorfismo di anelli 
\[
	\psi :R[x][y] \rightarrow R[y][x]
.\] 
che soddisfa
\begin{enumerate}
	\item $\psi(r) = r_1$
	\item $\psi(x) = x$
	\item  $\psi(y) = y$
\end{enumerate}
\textbf{Soluzione}\\
%TODO aggiugni grafico
%TODO Aggiungi grafico
\begin{center}
\begin{tikzpicture}
  % Nodes
  \node (R) at (0,4) {$R$};
  \node (Ryx) at (2,4) {$R[y][x]$};
  \node (Rx) at (0,2) {$R[x]$};
  \node (Rxy) at (0,0) {$R[x][y]$};

  % Arrows
  \draw[->] (R) -- (Rx) node[midway, left] {\tilde \nu};                    % Vertical arrow from R to R[x]
  \draw[->] (R) -- (Ryx);                   % Horizontal arrow from R to R[y][x]
  \draw[down hook, ->] (Rx) -- (Rxy);                  % Vertical arrow from R[x] to R[x][y]
  \draw[dotted,thick, <-] (Ryx) to  [hook] (Rxy); % Curved arrow from R[y][x] to R[x][y]
  \draw[red, dotted,thick,  ->] (Rx) to  (Ryx);     % Two-way arrow between R[x] and R[y][x]
\end{tikzpicture}
\end{center}

esiste un omomorfismo $\psi$ con le proprietà cercate.\\
Per dimostrare che $\psi$ è un \underline{iso}morfismo basta costruire l'inverso in modo analogo.
\begin{prop}
	$R$ anello commutativo $R$ dominio d'integrità se e solo se $R[x]$ dominio d'integrità
\end{prop}
\begin{dimo}
	Chiaramente se $R[x]$ è dominio d'integrità allora lo è anche  $R$\\
	Viceversa siano $f,g\in R[x] \setminus\{0\}$ allora il coefficiente di grado massimo di $f\codt g$ è il prodotto dei coefficienti di grado massimo di $f$ e di $g$. Quindi se $R$ dominio $ \Rightarrow f\cdot g\neq 0$

\end{dimo}
\newpage
\section{Domini Euclidei}
\begin{defi}
	$R$ anello commutativo\\
	\[
		\nu : R \rightarrow\Z_{>0} \text{ funzione tale che}
	.\] 
	\begin{enumerate}
		\item $P(r) =0 \Leftrightarrow r= 0$\\
		\item dati $a,b,c\in\R$ tali che $b\neq 0$ e  $c = a\cdot b$ allora
			 \[
			\nu (c)\geq \nu(a)
			.\] 
		\item $\forall f,g\in R$ con  $g\neq 0$ esistono $q,r\in R$ tali che 
			 \[
			g = q\cdot g + r
			.\] 
			dove $\nu(r) < \nu (q)$
	\end{enumerate} 
	Tale $\nu$ si chiama si valutazione e $(R,\nu)$ si chiama dominio Euclideo
\end{defi}
\textbf{Esempio}\\
$\mathbb K$ campo  $(\mathbb K[x],\nu)$ è un dominio euclideo dove $\nu(p) = deg(p) + 1$ e  $\nu(0) = 0$ \\
$(\Z,\nu)$ è un domino euclideo dove $\nu(n) = |n|$\\
$\mathbb K$ campo  $(\mathbb K,\nu)$ dominio euclideo dove $\nu(0) = 0$ e  $\nu(r) = 1 \  \forall r\in\mathbb K\setminus\{0\}$ \\
\textbf{Esercizio}\\
Dimostrare che $(\Z[i], \nu)$ è domino euclideo dove $\nu[a+ib] = a^2 + b^2$\\
 \textbf{Esempio}\\
 $f = 4 + 3i, \  \ g = 3 + 2i\neq 0$ Cerco  $q,r\in\Z[i] $ tale che $f = q\cdot g + r$ e  $\nu(e)<\nu(g) = 13$\\
 \textbf{Idea generale}
 \[
	 \frac{a + ib}{c = id} = \alpha + i\beta \ \ \ \alpha,\beta\in\Q
 .\] 
 \begin{defi}
 	$R$ anello commutativo.\\
	Definiamo gli insiemi $U_i$ iterativamente\\
	 \begin{gather*}
		 U_0 = \{0\}\subseteq \R\\
		 U_{i+1} = \{p\in\R | U_i  \rightarrow \R/(p) \ \text{ è suriettivo}\}\cup \{0\}
	\end{gather*}
 \end{defi}
 \textbf{Osservazione 1}\\
 L'omomorfismo $U_i \rightarrow R/(p)$ è la composizione\\
 \[
	 U_i \xrightarrow{inc} R \xrightarrow{\pi} R/(p)
 .\] 
 \textbf{Osservazione 2}\\
 La suriettività di $U_i \rightarrow R/(p)$ significa 
 \[
	 \forall f\in R \ \exists q\in R, \ r \in U_i\ \ \text{ tali che } f - q\cdot p = r
 .\] 
 ovvero $f = q \cdot p + r$ \\
 \textbf{Osservazione 3/esercizio}\\
 $U_i\subseteq U_{i+1} \ \ \forall i\geq 0$ \\
 \textbf{Osservazione 4}\\
 Chi è $U_1$?\\
 $U_1 = \{p\in R | \{0\} \rightarrow R/(p)$ è suriettiva$\}$\\
 $\{q\in R \  | \ (p) = R\}$ \\
 $\{p\in R \ | \ p$ invertibile$\}$\\
 \begin{teo}
	 $R$ dominio d'integrità, Allora $R$ è un dominio euclideo se e solo se \[R = \bigcup^{+\infty}_{i=0}U_i.\]
 \end{teo}
 \begin{dimo}
 	Supponiamo che $(R,\nu)$ sia un dominio Euclideo.\\
	$Im(\nu) = \{0,a_0,a_1,\ldots,a_n,\ldots\}\subseteq\Z_{\geq0}$\\
con $\{a_k\}$ successione strettamente crescente.\\
Definiamo
\[
	V_i = \{p\in R  \ | \ \nu(p) \leq a \}
.\] 
In particolare $V_0 = \{0\}$\\
\[
R = \bigcup^{+\infty}_{i =0 }V_i
.\] 
La tesi segue verificando che $V_i = U_i \ \ \forall i\geq 0$ (esercizio)\\
Viceversa: Se $R = \bigcup^{+\infty}_{i =0}U_i$\\
vogliamo definire $\nu: R \rightarrow\Z_{>0}$\\
tale che $(R,\nu)$ dominio Euclideo, Dato $r\in \R$  $\exists i\geq 0$ tale che $r\in U_{i+1}\setminus U_o$ \\
Definiamo $\nu(t) = i+1$ \\
Si possono verificare le 3 proprietà di $\nu$.\\
Vediamo $(2)$: dati  $a,b,c\in R$ con  $b\neq 0$ tali che  $c = a + b$\\
vogliamo misurare  $\nu(c) \geq \nu(a)$\\
 $(c)\subseteq(a)$\\
  $ \Rightarrow R/(a) \Rightarrow R/(a)/(a)/(c)\cong R/(a)$ \\
  Se $U_i \rightarrow R/(c)$ è suriettiva\\
  allora $U_i \rightarrow R/(c) \rightarrow R/(a)$ è suriettiva\\
  Ovvero
  \[
	  c\in U_{i+1} \Rightarrow a\in U_{i+1}
  .\] 
  quindi 
   \[
	   \nu(c) = i + 1 \Rightarrow  c \in U_{i+1} \Rightarrow a\in U_{i+1} \Rightarrow \nu(a)\leq i+1 = \nu(c)
  .\] 
 \end{dimo}

% -------------------- Fine Lezione 2 --------------------

\maketitle
	\newpage
	\subsection{Seconda parte della lezione}
	\textbf{Domanda:}\\
	Cosa cambia in $\K[2]$ quando  $\K$ è un campo?\\
	$u_1 = \K$ \\
	Chi è $u_2 ?$\\
	$p\in u_2$ se e solo se
	\[
		\K \rightarrow \K[x]/(p) \ \text{ è suriettiva}
	.\] 
	se e solo se $deg(p) = 1 \vee deg(p) = 0$\\
	In generale\\
	$\forall i\geq 1$ \ \  $u_{i+1}\setminus u_i$ è l'insieme dei polinomi di grado $i$
	\textbf{Attenzione} $\K[x,y]$ non è domino euclideo.\\
	$u_1 = \K$\\
	 $u_2 = ?$\\
	 \begin{defi}
	 	$R$ anello commutativo, Dati $r_1,\ldots, r_k\in R$ chiamiamo
		\[
			(r_1,\ldots,r_k) = \{ \sum^{k}_{i=1}a_ir_i \ | \ k\in Z_{\geq 1}\ \ a_i\in R\}
		.\] 
Ideale generato da $r_1,\ldots, r_k$ in $R$\\
	 \end{defi}
\textbf{Osservazione}\\
$(r_1,\ldots, r_k)$ è il più piccolo ideale di $R$ contenente $r_1,\ldots,r_k$
\begin{defi}[Ideale principale]
	$R$ anello commutativo $I\subseteq R$ ideale, si dice principale se $\exists \ r\in R$ tale che $I = (r)$
\end{defi}
\begin{defi}
	$R$ anello commutativo.
	\begin{itemize}
		\item $R$ si dice Anello a ideali principali se tutti i suoi ideali sono principali.
		\item  $R$ si dice dominio a ideali principali se è un dominio d'integrità e un anello a ideali principali.
	\end{itemize}
\end{defi}
\textbf{Esempio}\\
$R = (\Z,+,\cdot)$ è un dominio a ideali principali.\\
\textbf{Esercizio}\\
Trovare un anello a ideali principali che non sia un dominio\\
$n\in\Z$,  $n$ composto\\
$ \Rightarrow \Z/(n)$ è un anello a ideali principali che non è un dominio
\begin{prop}
	$\K$ campo. $R = \K[x]$ è un dominio a ideali principali
\end{prop}
\begin{dimo}
	$\K[x]$ è dominio d'integrità poiché $\K$ lo è.\\
	Sia $I\subseteq R[x]$ ideale ,  $I\neq \{0\}$\\
	Sia $f\in I\setminus\{0\}$ di grado minimo in  $I$\\
	Vogliamo dimostrare che  $I = (f)$
	 \begin{itemize}
		 \item $(f) \subseteq I$, infatti se $f\in I$ allora $q\cdot f\in I \ \ \forall q\in \K[x]$
		 \item  $I\subseteq (f),$ infatti $g\in I$ usiamo la divisione per  $f$\\
			  $ \Rightarrow g = q\cdot f + r$ con $deg(r) < deg(f) \Rightarrow r = g - q\cdot f \in I$ \\
			  $ \Rightarrow r =0 \Rightarrow g = q\cdot f\in (f)$
	\end{itemize}

\end{dimo}
\textbf{Esercizio}\\
Dimostrare che se
\begin{itemize}
	\item $R$ dominio d'integrità
	\item $R[x]$ dominio a ideali principali
\end{itemize}
Allora $R$ è un campo\\
\textbf{Soluzione}\\
Dobbiamo verificare che dato $a\in R\setminus\{0\}$ esiste l'inverso moltiplicativo.\\
Consideriamo l'ideale $(a,x)\subseteq R[x]$\\
$R[x]$ a ideali principali $ \Rightarrow \exists p\in R[x]$ tale che $(p) = (a,x)$\\
Quindi:
\begin{gather*}
	\Rightarrow a = q_1\cdot p\\
	 \Rightarrow x = q_2\cdot p \ \rightarrow \ ax = \tilde q_2\cdot p
\end{gather*}
Deduciamo che $q_1$ e $p$ sono entrambi costanti.\\
Infatti il termine di grado più alto del prodotto $q_1\cdot p$ è il prodotto dei termini direttivi di $p$ e di $q_1$ (Stiamo usando il fatto che $R$ sia dominio d'integrità)\\
Se $p$ costante\\
\[
	\Rightarrow q_2 = hx \text{ con } h\cdot p = 1
\] 
$p$ invertibile $ \Rightarrow  (p) = R[x]$ \\
$ 1\in (a,x) \Rightarrow $ esistono $s,t\in R[x]:$
 \[
1 = a\cdot s + t\cdot x \Rightarrow s = \sum^{}_{i\geq 0}s_ix^i \Rightarrow a s_0 = 1
.\] 
\textbf{Esercizio/Proposizione}\\
$R$ dominio a ideali principali. $I$ ideale, Se $I$ è primo, allora $I$ è massimale.\\
\textbf{Soluzione}\\
$ I = (p)\subseteq R$\\
 $I$ primo. Supponiamo che esista un ideale $J = (q)\subseteq R$ tale che  $I\subseteq J$\\
  $I\subseteq J \Rightarrow (p)\subseteq (q) \Rightarrow  p = a\cdot q$ per qualche $a\in R$\\
   $I$ primo $ \Rightarrow a\in I$ oppure $q\in I$
   \begin{center}
   	
    \begin{aligned}
	    $q\in I &\Rightarrow q\in (p) \\
		    & \Rightarrow  (q)\subseteq(p)\\
		    & \Rightarrow 	J = I\\$
   \end{aligned}
   \end{center}
   \begin{center}
   	\begin{aligend}
		$a\in I &\Rightarrow a\in (p)\\
			& \Rightarrow a = k\cdot p \text { per qualche } k\in R\\
			& \Rightarrow p = a\cdot q = p\cdot k\cdot q\\
			& \Rightarrow p\cdot (1-k\cdot q)= 0\\
			& \Rightarrow 1 + k\cdot q = 0 \Rightarrow q$ invertibile\\
			&$ J = R$
   	\end{aligend}
   \end{center}
   \begin{coro}
   	$R$ dominio a ideali principali (PID) allora un ideale è primo se e solo se è massimale
   \end{coro}
   \begin{dimo}
   	Resta da verificare che $I$ massimale $ \Rightarrow I$ primo\\
	$I$ massimale $ \Rightarrow R/I$ campo $ \Rightarrow R/I$ dominio integrità $ \Rightarrow I$ primo 
   \end{dimo}

% -------------------- Fine Lezione 3 --------------------

\end{document}
