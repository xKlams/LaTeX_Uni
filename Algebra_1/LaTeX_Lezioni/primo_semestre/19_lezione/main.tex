\documentclass[12px]{article}

\title{Lezione 19 Algebra I}
\date{2024-12-03}
\author{Federico De Sisti}

\usepackage{amsmath}
\usepackage{amsthm}
\usepackage{mdframed}
\usepackage{amssymb}
\usepackage{nicematrix}
\usepackage{amsfonts}
\usepackage{tcolorbox}
\tcbuselibrary{theorems}
\usepackage{xcolor}
\usepackage{cancel}

\newtheoremstyle{break}
  {1px}{1px}%
  {\itshape}{}%
  {\bfseries}{}%
  {\newline}{}%
\theoremstyle{break}
\newtheorem{theo}{Teorema}
\theoremstyle{break}
\newtheorem{lemma}{Lemma}
\theoremstyle{break}
\newtheorem{defin}{Definizione}
\theoremstyle{break}
\newtheorem{propo}{Proposizione}
\theoremstyle{break}
\newtheorem*{dimo}{Dimostrazione}
\theoremstyle{break}
\newtheorem*{es}{Esempio}

\newenvironment{dimo}
  {\begin{dimostrazione}}
  {\hfill\square\end{dimostrazione}}

\newenvironment{teo}
{\begin{mdframed}[linecolor=red, backgroundcolor=red!10]\begin{theo}}
  {\end{theo}\end{mdframed}}

\newenvironment{nome}
{\begin{mdframed}[linecolor=green, backgroundcolor=green!10]\begin{nomen}}
  {\end{nomen}\end{mdframed}}

\newenvironment{prop}
{\begin{mdframed}[linecolor=red, backgroundcolor=red!10]\begin{propo}}
  {\end{propo}\end{mdframed}}

\newenvironment{defi}
{\begin{mdframed}[linecolor=orange, backgroundcolor=orange!10]\begin{defin}}
  {\end{defin}\end{mdframed}}

\newenvironment{lemm}
{\begin{mdframed}[linecolor=red, backgroundcolor=red!10]\begin{lemma}}
  {\end{lemma}\end{mdframed}}

\newcommand{\icol}[1]{% inline column vector
  \left(\begin{smallmatrix}#1\end{smallmatrix}\right)%
}

\newcommand{\irow}[1]{% inline row vector
  \begin{smallmatrix}(#1)\end{smallmatrix}%
}

\newcommand{\matrice}[1]{% inline column vector
  \begin{pmatrix}#1\end{pmatrix}%
}

\newcommand{\C}{\mathbb{C}}
\newcommand{\K}{\mathbb{K}}
\newcommand{\R}{\mathbb{R}}


\begin{document}
	\maketitle
	\newpage
	\section{Gruppi}
	\textbf{Obiettivo}\\
	Dimostrare $A_n$ è semplice per  $n \geq 5$\\
	\textbf{Osservazione:}\\
	$A_4$ non è semplice\\
	$A_2$ e $A_3$ sono semplici\\
	%TODO
	\textbf{Strategia}\\
	$n \geq 5$\\
	1)  $\{Id\} \neq H\normale A_n$ allora  $H$ contiene almeno un $3$-ciclo\\
	2) $\{Id\}\neq H\normale A_n$ se $H$ contiene un $3$-ciclo allora li contiene tutti\\
	3) $A_n$ è generato dai suoi $3$-cicli\\
	Ricordo:
	\begin{lemm}
		$n\geq 3 \ \ \{Id\}\neq H\normale A_n$\\
		Allora $H$ contiene almeno un $3$-ciclo oppure un prodotto di trasposizioni disgiunte
	\end{lemm}
	\begin{prop}
		$n\geq 5, \ \ \{Id\}\neq H\normale A_n$ allora $H$ contiene almeno un $3$-ciclo
	\end{prop}
	\begin{dimo}
		Basta verificare che se $\sigma = (a_1a_2)(a_3a_4)\in H, $ allora esiste un $e$-ciclo in $H$.\\
		Dato che  $H\normale A_n$ abbiamo 
		\[
			gHg^{-1}\subseteq H \ \ \forall g\in A_n
		.\] 
		Definiamo $a_5 \not\in \{a_1,a_2,a_3,a_5\}$\\
		$\tau := (a_3a_4a_5)\sigma(a_3a_4a_5)^{-1}\in H$\\
		$ \Rightarrow \sigma \tau^{-1}\in H$ Studiamo $\sigma\tau^{-1}$\\
		$ \Rightarrow \sigma\tau^{-1} = \sigma(a_3a_4a_5)\sigma^{-1}(a_3a_4a_5)^{-1}$ \\
		Dove  $\sigma(a_3a_4a_5) = (\sigma(a_3)\sigma(a_4)\sigma(a_5))$\\
		$\sigma\tau^{-1} = (a_4a_3a_5)(a_3a_5a_4) = (a_3a_4a_5)\in H$
	\end{dimo}
	\begin{teo}
		$n\geq 5$ $\{Id\}\neq H\normale A_n$\\
		Allora  $H$ contiene tutti i $3$-cicli\\
	\end{teo}
	\begin{dimo}
		Basta verificare che dato \\
		$\sigma = (a_1a_2a_3)\in H$\\
		Allora $H$ contiene tutti i $3$-cicli\\
		Sfruttiamo $H\normale A_n$\\
		$ \Rightarrow \tau = (a_3a_4a_5)\sigma(a_3a_4a_5)^{-1}$ \\
		$\tau\in H$\\
		dove  $a_4,a_5\not\in\{a_1,a_2,a_3\}$\\
		Studiamo $\tau:$\\
		$\tau = (a_3a_4a_5)(a_1a_2a_3)(a_3a_4a_5)^{-1}$ 
		$= (a_1a_2a_4)\in H$\\
		Abbiamo dimostrato che se $(a_1a_2a_3)\in H$ allora $(a_1a_2a_4)\in H \ \ \forall a_4\not\in\{a_1,a_2\}$\\
		Dunque mostriamo che il $3$-ciclo arbitrato $(b1,b_2,b_3)\in H$ per qualunqe $b_1,b2,b_3$\\
		$(a_1a_2a_3)\in H$\\
		$ \Rightarrow (a_1a_2a_3)\in H$\\
		$ \Rightarrow  (b_1b_2b_3)\in H$ \\

	\end{dimo}
	\begin{defi}
		Un gruppo di dice semplice se gli unici sottogruppi normali sono banali
	\end{defi}
	\begin{coro}
		$n\geq 5 \ A_n$ è semplice
	\end{coro}
	\begin{dimo}
		Sia $\{e\}\neq H\normale A_n$, dimostriamo che $H=A_n$\\
		Per il teorema  $H$ contiene tutti i $3$-cicli, quindi basta verificare che $A_n$ è generato dai  $3$-cicli, Sia $\sigma\neq Id, \sigma\in A_n\subseteq S_n$\\
Ricordando che  $S_n$ è generato da trasposizioni\\
$ \Rightarrow \sigma = \tau_1\tau_2\ldots\tau_{2i-1}\tau_{2i}\ldots\tau_{2k-1}$ \\
L'idea è verificare che $\tau_{2i-1}\tau_{2i}$ si ottiene come prodotto di  $3$-cicli $\forall i\in \{1,\ldots,k\}$\\
Caso 1  $\tau_{2i-1} = \tau_{2i}$\\
$\tau_{2i-1}\tau_{2i} = Id = (123)(132)$\\
Caso 2  $\tau_{2i-1}=\tau_{2i}$\\
hanno un indice in comune\\
Allora:\\
$\tau_{2i-1} = (ab)\\
\tau_{2i} = (bc)$\\
$ \Rightarrow \tau_{2i-1}\tau_{2i} = (ab)(bc) = (abc)$ \\
Caso 3:\\ $\tau_{2i-1},\tau_{2i}$ non hanno indici in comune.\\
$ \Rightarrow \tau_{2i-1} = (ab), \ \tau_{2i} = (cd)$ \\
$\tau_{2i-1}\tau_{2i} = (ab)(cd)$\\
Ma \\
$(abc)(bcd) = (ab)(cd)$\\
Quindi:
$\tau_{2i-1}\tau_{2i} = (abc)(bcd)$\\
Allora  $\sigma$ è prodotto di $3$-cicli $ \Rightarrow \sigma \in H \Rightarrow H = A_n$
	\end{dimo}
	\textbf{Esercizio}\\
	$n\geq 5$ dimostrare che gli unici sottogruppi normali di $S_n$ sono $\{e\}, A_n, S_n\{e\}, A_n, S_n$
	\textbf{Soluzione}\\
	Osserviamo che se $ H\normale S_n$ allora  $H\cap A_n\normale A_n$ poichè  $H\normale S_n$ significa \\
	$gHg^{-1}\subseteq H \ \ \ \forall g\in S_n$\\
	Quindi  $\{Id\}\neq H\normale S_n$\\
	Studio  $H\cap A_n$\\
	1)  $H\subseteq A_n$\\
	 $ \Rightarrow H = H\cap A_n\trianglelefteq A_n$ \\
	 $ \xrightarrow {A_n \text{semplice}} H = \{Id\}$ oppure $H = A_n$ \\
	 2) $H\not\subseteq A_n$\\
	 $ \Rightarrow [H:H\cap A_n] = 2$ e $H\cap A_n\normale A_n$\\
	 $\xRightarrow{A_n\text{ semplice}} H\cap A_n = \{Id\}$ oppure $H\cap A_n = A_n$\\
	 Se  $H\cap A_n = \{Id\}$\\
	 \begin{aligned}
		 &\Rightarrow [H:H\cap A_n] = 2\\
		 & \Rightarrow |H| = 2\\
		 & \Rightarrow  H = \{Id,\sigma\}\\

	 \end{aligned}
	 con $ord(\sigma) = 2$\\
	 Se tale  $H$ fosse normale allora avremmo\\
	 $g\sigmag^{-1} = \sigma \ \ \forall g\in S_n$\\
	  $ \Rightarrow $ Assurdo perchè $\sigma $ è coniugato a tutti gli elementi con la sua stessa struttura ciclica.
	   $\cdot $ Allora $H\cap A_n = A_n.$\\
	   $ \Rightarrow [H:H\cap A_n] = 2$ \\
	   $ \Rightarrow  |H| = n! \Rightarrow  H = S$ \\
	   ricordando che $H\cap A_n = A_n$\\
	   \section{Classi di coniugio in A_n}
	   Obiettivo:\\
	   Studiare le azioni
	   \[
	    \begin{aligned}
		    S_n &\times A_n \rightarrow A_n\\
		    (\tau &,\sigma) \rightarrow\tau\sigma\tau^{-1}
	   	
	   \end{aligned} \ \ 
	   \vline \ \ \ 
	   \begin{aligned}
		   A_n&\times A_n \rightarrow A_n\\
		   (\tau&,\sigma) \rightarrow\tau\sigma \tau^{-1}
	   \end{aligned}
   \]
   \textbf{Ricordo:}\\
   Data $\sigma\in A_n$\\
   $O_\sigma^{S_n} = \{$ permutazioni con la stessa struttura ciclica di $\sigma\}\\$
   Domanda:  $O_\sigma^{A_n} = ?$\\
   A priori abbiamo  $O_\sigma^{A_n}\subseteq O_\sigma^{S_n}$\\
    \textbf{Esempio:} n = 3\\
    $O^{S_3}_{(123)} = \{(123),(132)\}$\\
    infatti $(23)(123)(23)^{-1} = (132)$\\
    $A_3 = \{Id,(123),(132)\}$\\
    $O^{A_3}_{123} = \{(123)\}$\\
    \textbf{Ricordo:}\\
    Data $\sigma\in A_n$\\
    $\cdot C_{A_n}(\sigma) = \{\tau\in A_n|\tau\sigma\tau^{-1}\} = Stab^{A_n}_\sigma$\\
    $\cdot C_{S_n}(\sigma) = \{\tau\in S_n | \tau\sigma\tau^{-1} = \sigma\} = Stab_\sigma^{S_n}$\\
     \textbf{Osservazione}\\
     $C_A = (\sigma) = C_{S_n}(\sigma)\cap A_n$\\
      \begin{teo}
     	$n\geq 2 \ \ \sigma \in A_n$ \\
	1) Se $C_{S_n}(\sigma )\not\subseteq A_n$ allora $O_\sigma^{A_n} = O_\sigma^{S_n}$\\
	2) $C_{S_n}(\sigma)\subseteq A_n$ allora $|O_\sigma^A_n| = \frac 12 |O_\sigma^{S_n}$
\end{teo}
	 \begin{dimo}
		 Supponiamo che $C_{S_n}(\sigma)\not\subseteq A_n$\\
		 Allora  $C_{S_n}(\sigma)\leq S_n$\\
		 $|C_{S_n}(\sigma):C_{S_n}(\sigma)\cap A_n] = 2$\\
		 notando che $C_{S_n}(\sigma)\cap A_n = C_{A_n}(\sigma)$\\
		 $ \Rightarrow |C_{A_n}(\sigma)| = \frac 12 |O_{S_n}(\sigma)|$ \\
		 $ \Rightarrow \begin{cases}
			 n! = |S_n| = |C_{S_n}(\sigma)|\cdot |O_\sigma^{S_n}|\\
			 \frac{n!}2 = |A_n| = |C_{A_n}(\sigma)|\cdot|O_\sigma^{A_n}
		 \end{cases}$ \\
	 $ \Rightarrow |O_\sigma^{S_n}| = |O_\sigma^{A_n}$\\
	 $ \Rightarrow O_\sigma ^{S_n}= O_\sigma^{A_n}$ \\
	 2) Se $C_{S_n}(\sigma)\subseteq A_n$\\
	 $ \Rightarrow C_{S_n}(\sigma) = C_{A_n}(\sigma)$\\
	 $ \Rightarrow \begin{cases}
		 n! = |S_n| = |C_{S_n}(\sigma)|\cdot|O_\sigma^{S_n}|\\
		 \frac{n!}2 = |A_n| = |C_{A_n}(\sigma)|\cdot|O_\sigma^{A_n}|

	 \end{cases}$ \\
	 $ \Rightarrow |O_\sigma ^{A_n}| = \frac 12 |O_\sigma^{S_n}|$
		
	\end{dimo}
	\textbf{Esempio:}\\
	$\sigma = (123) \ \ n = 5$\\
	$ \Rightarrow O^{S_n}_{(123)} = O^{A_n}_{(123)}$\\
	perché $(45)\in C_{S_n}(\sigma)$ MA $(45)\not\in A_5$\\
	\textbf{Esercizio}\\
	$\sigma = \sigma_1,\ldots, \sigma_k\in S_n$ disgiunti.\\
	$\sigma_i$ è $m_i$-ciclio\\
	1) se $\sum^k_{i=1}m_i\leq n-2$\\
	allora $O_\sigma^{S_n} = O_\sigma^{A_n}$ \\
\textbf{IDEA:}\\
dall'ipotesi segue che $\exists a,b\in\{1,\ldots,n\}$ tali che $\sigma(a) = a, \sigma (b) = b$\\
$ \Rightarrow (ab)\in C_{S_n}(\sigma)$ e $sgn(ab) = -1$
	\end{document}
