\documentclass[12px]{article}

\title{Lezione 3 Algebra I}
\date{2024-10-08}
\author{Federico De Sisti}

\usepackage{amsmath}
\usepackage{amsthm}
\usepackage{mdframed}
\usepackage{amssymb}
\usepackage{nicematrix}
\usepackage{amsfonts}
\usepackage{tcolorbox}
\tcbuselibrary{theorems}
\usepackage{xcolor}
\usepackage{cancel}

\newtheoremstyle{break}
  {1px}{1px}%
  {\itshape}{}%
  {\bfseries}{}%
  {\newline}{}%
\theoremstyle{break}
\newtheorem{theo}{Teorema}
\theoremstyle{break}
\newtheorem{lemma}{Lemma}
\theoremstyle{break}
\newtheorem{defin}{Definizione}
\theoremstyle{break}
\newtheorem{propo}{Proposizione}
\theoremstyle{break}
\newtheorem*{dimo}{Dimostrazione}
\theoremstyle{break}
\newtheorem*{es}{Esempio}

\newenvironment{dimo}
  {\begin{dimostrazione}}
  {\hfill\square\end{dimostrazione}}

\newenvironment{teo}
{\begin{mdframed}[linecolor=red, backgroundcolor=red!10]\begin{theo}}
  {\end{theo}\end{mdframed}}

\newenvironment{nome}
{\begin{mdframed}[linecolor=green, backgroundcolor=green!10]\begin{nomen}}
  {\end{nomen}\end{mdframed}}

\newenvironment{prop}
{\begin{mdframed}[linecolor=red, backgroundcolor=red!10]\begin{propo}}
  {\end{propo}\end{mdframed}}

\newenvironment{defi}
{\begin{mdframed}[linecolor=orange, backgroundcolor=orange!10]\begin{defin}}
  {\end{defin}\end{mdframed}}

\newenvironment{lemm}
{\begin{mdframed}[linecolor=red, backgroundcolor=red!10]\begin{lemma}}
  {\end{lemma}\end{mdframed}}

\newcommand{\icol}[1]{% inline column vector
  \left(\begin{smallmatrix}#1\end{smallmatrix}\right)%
}

\newcommand{\irow}[1]{% inline row vector
  \begin{smallmatrix}(#1)\end{smallmatrix}%
}

\newcommand{\matrice}[1]{% inline column vector
  \begin{pmatrix}#1\end{pmatrix}%
}

\newcommand{\C}{\mathbb{C}}
\newcommand{\K}{\mathbb{K}}
\newcommand{\R}{\mathbb{R}}


\begin{document}
	\maketitle
	\newpage
	\section{Altra roba sui gruppi}
	\begin{prop}[Caratterizazzione dei sottogruppi normali]
		$(G,\cdot)$ gruppo, $N\leq G$\\ 
		Le seguenti sono equivalenti:\\
		$1) gNg^{-1}\subseteq N \ \ \ \forall g\in G$\\
		$2) gNg^{-1} = N \ \ \ \forall g\in G$\\
		$3) N\normale G$\\ 
		 $4)$ L'operazione $G/N\times G/N \rightarrow G/N$ \\
		 è ben posta \ \ \ \ \ \ \ $(fN,gN) \rightarrow fgN$ \\
		 o equivalentemente $N\backslash G\times n\backslash G \rightarrow n\backslash G$\\
		 \text{} \hspace{90px} $(Nf,Ng) \rightarrow Nfg$

	\end{prop}
	\begin{dimo}
		$1 \rightarrow 2$\\
		Verifichiamo che $N\subseteq gNg^{-1}$\\
		Dato che  $n\in N \Rightarrow n = g(g^{-1}ng)g^{-1}$ basta dimostrare che $g^{-1}ng\in N$\\
		D'altra parte  $g^{-1}ng\in g^{-1}Ng\subseteq N$ (per ipotesi 1)\\
		$2 \rightarrow 3$\\
		$\forall g\in G \ \ \forall n\in N$\\
		$gng^{-1}\in N$ (per ipotesi 2)
		\[
		\begin{cases}
			gn\in Ng\\
			ng^{-1}\in g^{-1}N
		\end{cases} \Rightarrow \begin{cases}
			gN\subseteq Ng (1)\\
			Ng^{-1} \subseteq g^{-1}N (2)
		\end{cases}
		.\] 
		Il che è equivalente a dire che $gN = Ng$ la prima condizione mi dice $G/N\subseteq G/N$ e la seconda dell'arbitrarietà di  $g$\\
		 $G/N\subseteq G/N$\\
		 3  \rightarrow 4\\
		 Dati $f$ e $g\in G$ abbiamo
		 \[
			 (Nf)(Ng) = (fN)(Ng) = fNg = (fN)g = (Nf)g = Nfg
		 .\] 
		 $4 \rightarrow 1$\\
		 Per ipotesi 4 $(Nf)(Ng)=Nfg \ \ \forall f,g\in G$ quindi 
		  \[
		 nfn'g\in Nfg \ \ \ \forall n,n'\in N
		 .\] 
		 dall'arbitrarietà di g, scelgo $g=f^{-1}$, quindi 
		 \[
			 nfn'f^{-1}\in N \ \ \forall f\in G .\] 
			 Moltiplico (a sinistra) per $n^{-1}$ e ottengo\\
			 \[
				 fn'f^{-1}\in N \ \ \forall f\in G
			 .\] 
			 Dall'arbitrarietà di $n'$ otteniamo  $fNf^{-1}\subseteq N\ \ \forall f\in G$ che è la condizione (1) \\
	\end{dimo}
	\textbf{Osservazione}\\
	 $(G,\cdot)$ gruppo, la proposizione ci dice che un sottogruppo $H\leqG$ è normale se e solo se l'operazione indotta su $G/H$ è ben definita 
	 \begin{teo}
	 	$(G,\cdot)$ gruppo $N\trianglelefteq G$\\
		Allora $(G/N,\cdot)$ è un gruppo (detto gruppo quoziente)
	 \end{teo}
	 \begin{dimo}
	 	Associatività, ovvia\\
		elemento neutro : $N=Ne$\\
		elemento inverso di  $Ng$ è $Ng^{-1}$ \ \  $\forall g\in G$
	 \end{dimo}
	 \textbf{Osservazione}\\
	 $(G,\cdot)$ gruppo e $H\leq G$ t.c.  $[G:H] = 2$ Allora  $H\trianglelefteq G$\\
	 Infatti esistono solo due laterali sinistri o destri: H, G/H\\
	 \textbf{Osservazione}\\
	 $(G,\cdot)$ gruppo abeliano $\Rightarrow$ ogni sottogruppo è normale\\
	 \textbf{\underline{Non} vale sempre il viceversa}\\
	 \textbf{Esempio}\\
	 Dimostrare che $Q = \lbrace \pm 1, \pm i,\pm j,\pm k\rbrace$\\ 
	 è un gruppo (rispetto al prodotto)
	 non abeliano in cui però tutti i sottogruppi sono normali\\
	  \textbf{Prodotti:}\\
	  $i^2 = k^2 = j^2 = -1$\\
	   $ij = k \ \ jk = i \ \ ki = j$\\
	    $ji = -k \ \ kh = -i \ \ ik = -j$
	 \begin{defi}
		Siano $(G_1,\cdot)$ e $(G_2,*)$ gruppi\\
		Sia $ \varphi$ un'applicazione\\
		$
		\varphi: G_1  \rightarrow G_2$ si dice omomorfismo se:
		\[
		\varphi(g\cdot f) = \varphi(g)* \varphi(f) \ \ \ \forall g,f\in G_1
		.\] 
		
	\end{defi}
	\textbf{Osservazione}\\
	Graficamente $ \varphi$ è un omomorfismo se
	
\begin{tikzcd}
	(g, f) \arrow[d, ] & G_1 \times G_1 \arrow[r, "\cdot"] \arrow[d, " \varphi \times \varphi"'] & G_1 \arrow[d, " \varphi"] &  \color{red}(g,f) \arrow[r,red] & \color{red}g\cdot f\arrow[d, red]\\
	(\varphi(g), \varphi(f)) & G_2 \times G_2 \arrow[r, "*"] & G_2 & {} & \color{red}\varphi(g\cdot f)
\end{tikzcd}\\
\newpage
\textbf{Esempi:}\\
$(\R, +)$ gruppo additivo reali\\
$(\R_{> 0}, \cdot)$ gruppo moltiplicativo reali positivi\\
\textbf{Allora}\\
$exp: \R \rightarrow \R_{>0}$\\
\text{}\ \ \ \ $x \rightarrow e^x$\\
è un omomorfismo infatti: $\forall x,y\in \R$\\
 \[
	 e^{x+y} = e^x\cdot e^y
.\] 
\textbf{Esempio}\\
$ln:\R_{>0} \rightarrow \R$\\
\text{}\ \ \ \ \ $x \rightarrow ln(x)$\\
è un omomorfismo, infatti $\ln (x\cdot y) = \ln(x) + \ln(y) \ \ \ \forall x,y\in\R_{>0}$\\
 \textbf{Osservazione:}\\
 $l^0 = 1 \ \ ln(1) = 0$\\
  $0$ è l'elemento neutro in $(\R, +)$\\
  $1$ è l'elemento neutro in  $(\R_{>0},\cdot)$\\
  \textbf{Osservazione:}\\
  $e^{-x} = \frac{1}{e^x}$\\
  Inverso di  $x$ in $(\R, +)$ \\
  è invero di $e^x$ in  $(\R_{>0},\cdot)$\\
   $\ln(\frac 1 x) = -\ln(x)$\\
   \textbf{Esercizio}\\
   $ \varphi : G_1 \rightarrow G_2$ omomorfismo. Dimostrare\\
   $1) \varphi(e_1) = e_2$\\
   $2) \varphi(g^{-1}) = \varphi(g)^{-1} \ \ \forall g\in G_1$\\
   \textbf{Soluzione:}\\
   $ \varphi (e_1) = \varphi(e_1\cdot e_2) = \varphi(e_1)* \varphi(e_1)$\\
   moltiplico per $ \varphi(e_1)^{-1}$\\
   $ \Rightarrow e_2 = \varphi(e_1)^{-1}* \varphi(e_1) = \varphi(e_1)^{-1}*( \varphi(e_1)* \varphi(e_1)) = \varphi(e_1)$ \\
   \textbf{Esempio:}
   $(G,\cdot)$ gruppo, $N\trianglelefteq G$\\
   Allora\\
   $\pi:G \rightarrow G/N$\\
   \text{} \ \ \ $g \rightarrow gN$\\
   è un omomorfismo\\
   \textbf{Esempio}\\
   $det: GL_n (\K) \rightarrow \K^*$ \\
   dove $\K$ campo\\
    $\K^* = \K \setminus \lbrace 0 \rbrace$ è un gruppo rispetto l prodotto\\
    allora  $det$ è un omomorfismo\\
    infatti:
     \[
    \forall A,B\in GL_n(\K) \ \ \ \ det(AB) = det(A)det(B)
    .\] 
    in particoalre:\\
    $det(Id) = 1$\\
    $det(A^{-1}) = \frac 1 {det(A)} \ \ \ \forall A\in GL_n(\K)$
     \begin{defi}
    	$ \varphi: G_1 \rightarrow G_2$ omomorfismo\\
	il nucleo di $ \varphi$ è $ker( \varphi) := \lbrace g\in G_1| \varphi(g) = e\rbrace$\\
	L'immagine di $\phi$ è \\
	$Im ( \varphi) = \lbrace{h\in H_2 | \exists g\in G_1: \varphi (g) = h \rbrace$
    \end{defi}
    \textbf{Esercizio:}\\
    $\varphi: G_1 \rightarrow G_2 $ omomorfismo\\
    Allora $ker( \varphi)\trianglelefteq G_1)$\\
    \textbf{Soluzione}\\
    Chiamo $H:ker ( \varphi )$\\
    vorrei verificare che $g Hg^{-1}\subseteq H \ \ \forall g\in G_1$\\
    scegliamo $h\in H$ (ovvero $ \varphi(g) = e_2)$\\
    $ \Rightarrow  \varphi(ghg^{-1}) = \varphi(g) \varphi(h) \varphi(g^{-1})$ = per esercizio = $ \varphi(g) \varphi(h) \varphi(g)^{-1} = e_2$\\
    $ \Rightarrow ghg^{-1}\in H\forall h\in H, \forall g\in G \Rightarrow gHg^{-1}\subseteq H$\\
    \textbf{Osservazione}\\
    $(G,\cdot)$ gruppo, $H\leq G$. Allora  $H\trinaglelefteq G$ se e solo se esiste  $ \varphi: G_1 \rightarrow G_2$ omomorfismo tale che $H=ker( \varphi)$\\
    \begin{dimo}
    	Resta solo l'implicazione $ \Rightarrow $\\
	Sia $H\trianglelefteq G$. considero l'omomorfismo\\
$	\pi: G \rightarrow G/H\\
\text{}\ \ \ g \rightarrow gH$\\
chi è $ker(\pi)$\\
 ker(\pi) = \lbrace g\in G | gH = H\rbrace = \lbrace g\in G | g\in H\rbrace = H$
    \end{dimo}\\
     \textbf{Esempio}\\
     $det: GL_n(\K) \rightarrow K^*$\\
     $ker(det) := \lbrace A\in GL_n(\K) | det(A) = 1\rbrace = SL_n(\K)\\
     quindi\\
     $SL_n(\K)\trianglelefteq GL_n (\K)$\\
     \textbf{Esercizio}\\
     $(G,\cdot)$ gruppo  $g\in G$ fissato\\
      $ \varphi: \mathbb Z \rightarrow G$ \\
     $ \text{} \ \ \ n \rightarrow g^n$\\
     è un omomorfismo\\
     determinare $ker \varphi$ e $Im \varphi$\\
     \textbf{Esercizio}\\
     Sia $  \varphi: G_1 \rightarrow G_2$ omomorfismo\\
     $1)$ Se $H_1 \leq G_1 \Rightarrow \varphi(H_1)\leq G_2$ \\
     se $H_1 \trianglelefteq G_1 \Rightarrow \varphi(H_1)\trianglelefteq \varphi (G_1)$\\
     $2)$ Se $H_3 \leq G_2 \Rightarrow \varphi^{-1}(H_2)\leq G_1$ \\
     se $H_1 \trianglelefteq G_2 \Rightarrow \varphi^{-1}(H_2)\trianglelefteq \varphi (G_1)$\\


\end{document}
