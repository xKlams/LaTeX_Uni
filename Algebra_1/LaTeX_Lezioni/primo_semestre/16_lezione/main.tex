\documentclass[12px]{article}

\title{Lezione 16 Algebra I}
\date{2024-11-21}
\author{Federico De Sisti}

\usepackage{amsmath}
\usepackage{amsthm}
\usepackage{mdframed}
\usepackage{amssymb}
\usepackage{nicematrix}
\usepackage{amsfonts}
\usepackage{tcolorbox}
\tcbuselibrary{theorems}
\usepackage{xcolor}
\usepackage{cancel}

\newtheoremstyle{break}
  {1px}{1px}%
  {\itshape}{}%
  {\bfseries}{}%
  {\newline}{}%
\theoremstyle{break}
\newtheorem{theo}{Teorema}
\theoremstyle{break}
\newtheorem{lemma}{Lemma}
\theoremstyle{break}
\newtheorem{defin}{Definizione}
\theoremstyle{break}
\newtheorem{propo}{Proposizione}
\theoremstyle{break}
\newtheorem*{dimo}{Dimostrazione}
\theoremstyle{break}
\newtheorem*{es}{Esempio}

\newenvironment{dimo}
  {\begin{dimostrazione}}
  {\hfill\square\end{dimostrazione}}

\newenvironment{teo}
{\begin{mdframed}[linecolor=red, backgroundcolor=red!10]\begin{theo}}
  {\end{theo}\end{mdframed}}

\newenvironment{nome}
{\begin{mdframed}[linecolor=green, backgroundcolor=green!10]\begin{nomen}}
  {\end{nomen}\end{mdframed}}

\newenvironment{prop}
{\begin{mdframed}[linecolor=red, backgroundcolor=red!10]\begin{propo}}
  {\end{propo}\end{mdframed}}

\newenvironment{defi}
{\begin{mdframed}[linecolor=orange, backgroundcolor=orange!10]\begin{defin}}
  {\end{defin}\end{mdframed}}

\newenvironment{lemm}
{\begin{mdframed}[linecolor=red, backgroundcolor=red!10]\begin{lemma}}
  {\end{lemma}\end{mdframed}}

\newcommand{\icol}[1]{% inline column vector
  \left(\begin{smallmatrix}#1\end{smallmatrix}\right)%
}

\newcommand{\irow}[1]{% inline row vector
  \begin{smallmatrix}(#1)\end{smallmatrix}%
}

\newcommand{\matrice}[1]{% inline column vector
  \begin{pmatrix}#1\end{pmatrix}%
}

\newcommand{\C}{\mathbb{C}}
\newcommand{\K}{\mathbb{K}}
\newcommand{\R}{\mathbb{R}}


\begin{document}
	\maketitle
	\newpage
	\section{OPIS}
	il codice opis del corso è 
	\[
	7K817KGS
	.\] 
	\section{Cazzi e mazzi}
	\subsection{Ricordo:}
	\begin{teo}
		$p < q$ primi $G$ gruppo finito di ordine $pq$\\
		Allora:\\
		$\cdot$ se $p\not | q + 1$ allora $G\cong C_{pq}$\\
		 $\cdot$ se $p | q + 1$ allora $ G\cong C_q\semi C_p$
	\end{teo}
	\textbf{Inserisci tabella fino ad ordine 9}
	\begin{coro}
		$q > 2$ primo, $G$ gruppo di ordine $2q$\\
		Allora $G\cong C_{2q}$ oppure $G\cong D_q$
	\end{coro}
	\begin{dimo}
		Dal teorema basta studiare gli omomorfisimi
		\begin{center}
		\begin{aligned}
			\phi:&C_2 \rightarrow Aut(G)\\
			     &s \rightarrow (\phi_s : r \rightarrow s)
			
		\end{aligned}
		\end{center}
		Affinchè $\phi$ sia un omomorfismo, dato che $ord_{C_2}(s) = 2$\\
		dobbiamo imporre che $ord_{Aut(G)}(\phi_s) \in \{1,2\}$\\
		Se è uguale a 1 $\phi_s = Id \Rightarrow \phi$ omomorfismo banale\\
		$ \Rightarrow $ il prodotto è diretto\\
		$ \Rightarrow G\cong C_q\times C_2\cong C_{2q}$ \\
		Nell'altro caso $ord_{Aut(G)}(\phi_s) = 2$\\
		$ \Rightarrow \phi_s\circ\phi_s = Id_{C_q} \Rightarrow \phi_s(\phi_s(r))=r$ \\
		$\phi_s(r^k) = r$\\
		$ \Rightarrow k^2\equiv_{ord_{C_1}(r)} 1 \Rightarrow k^2\equiv_q 1$ \\
		$ \Rightarrow (k-1)(k+1)\equiv_q 0$ \\
		$ \Rightarrow k\equiv_q \pm 1$ \\
		Se $k \equiv_q 1$\\
		$ \Rightarrow \phi_s = Id_{C_q} \Rightarrow G\equiv C_{2q}$ \\
		Se $k\equiv_q -1$\\
		$ \Rightarrow \phi_s(r) = r^{-1}$\\
		$ \Rightarrow G\cong C_q\semi C_2\cong D_q$ (già visto)

	\end{dimo}
	\newpage
	\section{Gruppi di ordine 12}
	Studiamo $G$ tramite i teoremi di Sylow\\
	$\cdot Syl_2(G)\neq \emptyset$\\
	 $\cdot Syl_3(G)\neq\emptyset$ \\
	 \hline\ \\
	 Dal Sylow III abbiamo
	 \[
	 \begin{cases}
	 	n_2 \equiv_2 1\\
		3\equiv_{n_2} 0
	 \end{cases}
	 .\] 
	 $ \Rightarrow n_2= 1$ oppure $n_2 = 3$\\
	 Dal Sylow II
	  \[
		  \begin{cases}
		  	
	 n_3\equiv_3 1\\
	 4\equiv_{n_3} 0
		  \end{cases}
	 .\] 
	 $n_3 = 1$ oppure $n_3 = 4$\\
	  \textbf{Osservazione:}\\
	  Esiste un sottogruppo normale in $G$
	  \begin{dimo}
	  	se $n_3 = 4$\\
		Allora in  $G$ esistono 4 sottogruppi di ordine 3\\
		Ognuno dei quali contenente due elementi di ordine 3.\\
		Quindi $G$ contiene 8 elementi di ordine 3.\\
		Quindi i restanti 3 elementi di ordine diverso da 3 formano necessariamente l'unico $2$-Sylow
	  \end{dimo}
	  \textbf{Esercizio:}\\
	  Se $|G| = 12$ e  $n_3 = 4$ allora esiste un omomorfismo iniettivo  $G \rightarrow S_4$\\
	  \textbf{Nota}\\
	  Da questo segue che $G\cong A_4$ perchè $A_4$ è l'unico sottogruppo di ordine $12$ in $S_4$
	  \begin{dimo}
	  	$G\times Syl_3(G) \rightarrow Syl_3(G)$\\
		$(g,H) \rightarrow g H g^{-1}$\\\
		$n_3 = 4$\\
		$ \Rightarrow Syl_3(G) = \{H_1,H_2,H_3,H_4\}$ \\
		Definiamo\\
		\begin{aligend}
			\psi : &G \rightarrow S_4\\
			       &g \rightarrow \tau_g
		\end{aligend}\\
		$\tau_g(i) = j \Leftrightarrow gHg^{-1} = H_j$ con $i\in\{1,2,3,4\}$ (Questa è l'idea da utilizzare negli esercizi delle schede\\
Verifiche:\\
$1) \psi$ è ben definita, Infatti $\tau_g$ è invertibile con inversa  $\tau_{G^{-1}}$\\
 $2) \ \psi $ è un omomorfismo, ovvero
 \[
 \psi(gf) = \psi(g)\psi(f)
 .\] \newpage \ \\
 $\tau_{gf}(i) = j$ \\\begin{aligned}
	\Leftrightarrow&(gf)H(gf)^{-1} = H_j\\
	\Leftrightarrow& g(fHf^{-1})g^{-1} = H_j\\
	\Leftrightarrow& \tau_g(\tau_f(i)) = j
 \end{aligned}\\
 3) $\psi$ iniettiva\\
 supponiamo che $\tau_g = \tau_f$\\
 $gHg^{-1} = fHf^{-1} \ \ \forall H\in Syl_3(G)$\\
 $ \Rightarrow (f^{-1}g)H(f^{-1}g)^{-1} = H \ \ \forall H\in Syl_3(G)$ \\
 $ \Rightarrow f^{-1}g\in N_G(H) \ \ \forall H\in Syl_3(G)$ \\
 $ \displaystyle\Rightarrow f^{-1}g\in \bigcap_{H\in Syl_3(G)} N_G(H)$ \\
 $ \displaystyle\Rightarrow f^{-1}g\in \bigcap_{H\in Syl_3(G)} H = \{e\} \Rightarrow f^{-1}g = 3 \ \Rightarrow  \ f = g$\\
 Resta da verificare che $H = N_G(H)$ \\
 $4 = n_3 = [G:N_G(H)] \displaystyle = \frac { |G|}{|N_G(H)|} = \frac {12}{N_G(H)} \Rightarrow |N_G(H)| = 3$ \\
 Ma $H\leq N_G(H) \Rightarrow H = N_G(H)$
	  \end{dimo}
	  \subsection{Studiare gruppi di ordine 12 in cui $n_3 = 1$}
Da Sylow III Segue che $\exists ! Q\in Syl_3(G) \Rightarrow Q\normale G$ \\
Esiste in $G$ almeno un $2$-Sylow $P\leq G$\\
Ora:\\
 $\cdot G\normale G, \ \ P\leq G$\\
 $\cdot Q\cap P = \{e\}$ (perchè l' $MCD(|Q|,|P|) = 1$\\
 $\displaystyle\cdot |QP| = \frac{|Q||P|}{|Q\cap P|} = \frac{3\cdot 4}1 = 12$\\
  $ \Rightarrow QP = G$ \\
  Allora $G\cong Q\semi P$ per qualche\\
  $\phi: P \rightarrow Aut(Q)\cong C_2$\\
Quindi studiamo i possibili omomorfismi\\
$\phi : P \rightarrow Aut(C_3)\hfill$ se $P\cong C_4$\\
$C_4 = <\gamma> \ \ \ \ C_3 = <r>$\\
\begin{aligned}
	\phi: <&\gamma> \rightarrow Aut(C_3)\\
	      &\gamma \rightarrow (\phi_\gamma : r \rightarrow r^k \text{ con }$k \pm$ 1)
\end{aligned}
nel csao $k = 1$ abbiamo  $\phi$ banale\\
$ \Rightarrow$ prodotto diretto\\
$ \Rightarrow G\cong C_3\times C_4\cong C_{12}$ \\
nel caso $k = -1$\\
abbiamo  $G\cong C_3\semi C_4\cong Dic_3$\\
dove\\
\begin{aligned}
	\phi: &C_4 \rightarrow Aut(C_3)\\
	      &\gamma \rightarrow (\phi_\gamma:r \rightarrow r^{-1})
\end{aligned}\\
$P\cong K_4$ \\
\begin{aligned}
	$\phi : &K_4 \rightarrow Aut (C_3)$\\
	$&\{Id,a,b,ab\}$\\
	 &a \rightarrow (\phi_a : r \rightarrow r^{\pm 1})\\
	 &b \rightarrow (\phi_b : r \rightarrow r^{\pm 1})\\
	 &ab \rightarrow (\phi_{ab} : r \rightarrow r^{\pm 1})\\
\end{aligned}\\
Se $\phi$ è banale\\
$ \Rightarrow$ prodotto diretto\\
\begin{aligned}
	$\Rightarrow G&\cong C_3\times K_4$ \\
	$ &\cong C_3\times C_2\times C_2$ \\
	$ & \cong C_6\times C_2$
	
\end{aligned}\\
Se $\phi$ è non banale, a meno di rinominare gli elementi $\{a,b,ab\}$ avremo che \\ \begin{aligned}
	\text{    \hspace{20px}} &  $\phi_a r\rightarrow r$\\
	&$\phi_b r\rightarrow r^{-1}$\\
	&$\phi_{ab} r\rightarrow r^{-1}$
\end{aligned}
Grazie (!) a Esercizio 1 di scheda 7 tutti i restanti prodotti semidiretti sono isomorfi\\
$G\cong C_3\semi K_4\cong D_6$\\
Infatti $|D_6| = 12$\\
$D_6$ non è isomorfo ad alcuno dei precedenti casi\\
$1) C_2$ è ciclico\\
$2) C_6\times C_2$ è abeliano, ma non ciclico\\
$3) A_4$ unico caso in cui $n_3 = 4$\\
$4) Dic_3$ non è abeliano e contiene elementi di ordine 4\\
 $5) D_6$ non è abeliano e non contiene elementi di ordine 4 ($C_4)$ \\
 \begin{defi}[Radice primitiva modulo (n)]
	 Un intero $a$ si definisce radice primitiva modulo (n) se $ord_{U_n}([a]) = \phi(n)$
 \end{defi}
 \textbf{Osservzaione:}\\
 Per teorema di Eulero\\
 \[
	 a^{\phi(n)}\equiv_n 1
 .\] 
 $ \Rightarrow ord_{U_n}([a]) = \phi(n)$ \\
 \textbf{Osservazione}\\ $a$ radice primitiva mod (n) significa che $U_n = <[a]>$ \\
 \textbf{Obiettivo} (Scheda 7)\\
 Dimostrare che se $ p > 1$ primo allora $\exists $ radice primitiva modulo $(p)$\\
  \textbf{Esempi}\\
  Non esistono radici primitive mod(8)\\
  Studio $U_8 = \{[1],[3],[5],[7]\}$
   \[
  \phi(8) = 2^3 - 2 ^2 = 4
  .\] 
  \begin{aligned}
	1^2\equiv_8  1\\
  	3^2\equiv_8  1\\
  	5^2\equiv_8  1\\
  	7^2\equiv_8  1
  \end{aligned}\\
  \textbf{Es(ercizio esempio)}\\
  $3$ è radice primitiva mod(7)\\
  \textbf{Svolgimento:}\\
  $3^1 \equiv_7 3$\\
  $3^2 \equiv_7 2$\\
  $3^3 \equiv_7 1$\\
  $3^4 \equiv_7 3$\\
  $3^5 \equiv_7 2$\\
  $3^6 \equiv_7 1$\\
  \hline \ \\
  $2$ è radice primitiva mod(9)\\
  \textbf{Da fare}\\
  \textbf{Esercizio}(Scheda 7)\\
  Dimostrare che \\
  $Aut(C_p)\cong C_{p-1}$\\
   \textbf{Soluzione}\\
   Sappiamo che\\
   $Aut(C_p)\cong U_p\cong C_{\phi(p)}\cong C_{p-1}$\\
   \textbf{Esercizio}\\
   $p$ primo\\
   $f(x) = a_nx^n + \ldots + a_1x + a_0$\\
   $f(x)\equiv_p 0$ ammette al più $p$ soluzioni distinte in  $\Z / (p)$\\
   \begin{dimo}
   	per induzione su $n$ \\
	se $n = 1$  $ \Rightarrow a_1x\equiv_p -a_0$ \\
	$ \Rightarrow x \equiv_p = -a\cdot a_1^{-1}$\hfill $a_1$ invertibile in $\Z/(p)$ per ipotesi \\
	$n > 1$ \\
	Se $f(x) \equiv_p 0$\\
	non ammette soluzioni ok \\
	Se invece a è soluzione dividiamo\\
	$f(x) = (x - a ) q(x) + r$\\
	 $ \Rightarrow f(x)\equiv_p (x-a)q(x) + r$ \\
	 Valuto in $a$:\\
	 $ \Rightarrow 0\equiv_p f(a) \equiv_p (a-a)q(a) + r$ \\
	 $ \Rightarrow f(x) \equiv_p (x - a)q(x)$\\
	 Sia  $b\not \equiv_p a$ tale che  $f(b)\equiv_p 0$\\
	  $0\equiv_p f(b) \equiv_p (b-a)q(b)$\\
	  $\Z/(p)$ dominio d'integrità\\
	  $q(b)\quiv_p 0$\\
	  Ma per induzione  $q(x)\equiv_p 0$\\
	  ammette al più  $n-1$ soluzioni distinte\\
	  $ \Rightarrow  f(x)\equiv_p 0$ ammette al più n soluzioni 

   \end{dimo}

\end{document}
