\documentclass[12px]{article}

\title{Lezione 9 Algebra I}
\date{2024-10-30}
\author{Federico De Sisti}

\usepackage{amsmath}
\usepackage{amsthm}
\usepackage{mdframed}
\usepackage{amssymb}
\usepackage{nicematrix}
\usepackage{amsfonts}
\usepackage{tcolorbox}
\tcbuselibrary{theorems}
\usepackage{xcolor}
\usepackage{cancel}

\newtheoremstyle{break}
  {1px}{1px}%
  {\itshape}{}%
  {\bfseries}{}%
  {\newline}{}%
\theoremstyle{break}
\newtheorem{theo}{Teorema}
\theoremstyle{break}
\newtheorem{lemma}{Lemma}
\theoremstyle{break}
\newtheorem{defin}{Definizione}
\theoremstyle{break}
\newtheorem{propo}{Proposizione}
\theoremstyle{break}
\newtheorem*{dimo}{Dimostrazione}
\theoremstyle{break}
\newtheorem*{es}{Esempio}

\newenvironment{dimo}
  {\begin{dimostrazione}}
  {\hfill\square\end{dimostrazione}}

\newenvironment{teo}
{\begin{mdframed}[linecolor=red, backgroundcolor=red!10]\begin{theo}}
  {\end{theo}\end{mdframed}}

\newenvironment{nome}
{\begin{mdframed}[linecolor=green, backgroundcolor=green!10]\begin{nomen}}
  {\end{nomen}\end{mdframed}}

\newenvironment{prop}
{\begin{mdframed}[linecolor=red, backgroundcolor=red!10]\begin{propo}}
  {\end{propo}\end{mdframed}}

\newenvironment{defi}
{\begin{mdframed}[linecolor=orange, backgroundcolor=orange!10]\begin{defin}}
  {\end{defin}\end{mdframed}}

\newenvironment{lemm}
{\begin{mdframed}[linecolor=red, backgroundcolor=red!10]\begin{lemma}}
  {\end{lemma}\end{mdframed}}

\newcommand{\icol}[1]{% inline column vector
  \left(\begin{smallmatrix}#1\end{smallmatrix}\right)%
}

\newcommand{\irow}[1]{% inline row vector
  \begin{smallmatrix}(#1)\end{smallmatrix}%
}

\newcommand{\matrice}[1]{% inline column vector
  \begin{pmatrix}#1\end{pmatrix}%
}

\newcommand{\C}{\mathbb{C}}
\newcommand{\K}{\mathbb{K}}
\newcommand{\R}{\mathbb{R}}


\begin{document}
	\maketitle
	\newpage
	\section{Ricapitolando}

Siano \( (N, \cdot), (H, *) \) gruppi.

\begin{defi}
Il \textit{prodotto semidiretto} di \( N \) e \( H \) tramite un omomorfismo \( \theta : H \to \text{Aut}(N) \) è l'insieme \( N \times H \) dotato dell'operazione
\[
	(n_1, h_1) \cdot (n_2, h_2) = (n_1 \cdot \o_{h_1}(n_2), h_1 * h_2).
\]
\end{defi}
\textbf{Osservazione:}\\
$h_1\in H, \ \ \o_{h_1}\in Aut(N) \ \ \o_{h_1}(n_2)\in N$\\
\textbf{Esempio}\\
Scegliendo\\
\begin{aligned}
	\o : &H \rightarrow Aut(N)\\
	     & h \rightarrow \o_h
\end{aligned}\\
con $\o_n := Id_N \ \ \forall h\in H$\\
Abbiamo:
 \[
	 (n_1,h_1)\cdot (n_2,h_2) = (n_1\cdot n_2, h_1 * h_2)
.\] 
Quindi il prodotto diretto è un caso particolare del prodotto semidiretto

\section{Prodotto semidiretto interno:}
Un gruppo \( G \) si dice \textit{prodotto semidiretto interno} di \( N \) e \( H \leq G \) se:
\begin{enumerate}
    \item \( N \trianglelefteq G \),
    \item \( N \cap H = \{ e \} \),
    \item \( N H = G \).
\end{enumerate}
\textbf{Esercizio}\\
Sia $\o : H \rightarrow Aut(N)$ un omomorfismo\\
Dimostrare:\\
$1) |N\rtimes_{\o} H | = |N||H|$\\
$2) N\rtimes_{\o} H$ è abeliano $ \Leftrightarrow N,H$ abeliani \\
$3) \tilde H \leq H, \tilde N \leq N$ (sottogruppo caratteristico)\\
 \[
\tilde N \semi \tilde H := \{ (n,h)\in N\semi H | 
	n\in \tilde N,
	n\in \tilde H
 \}
.\] 
è un sottogruppo di $N \semi H$\\
\begin{defi}[Sottogruppo caratteristico]
	$\tilde N\leq N$ sottogruppo caratteristico se \\$ \varphi(n)\in \tilde N\ \ \forall n\in \rilde N\ \ \ \forall \varphi\in Aut(N)$
\end{defi}
\begin{teo}
Sia \( G \) un gruppo. \\1) Se \( G \) è prodotto semidiretto di \( N \) e \( H \leq G \), allora esiste un omomorfismo \( \o : H \to \text{Aut}(N) \) tale che $G\cong N\semi H$\\
2) Se  $G\cong \tilde N\semi \tilde H$ allora esistono $N,h\leq G$ t.c.
\begin{itemize}
	\setlength\itemsep{-1em}
	\item $G$ sia prodotto semidiretto interno di $N$ e $H$ \\
	\item $N\cong \tilde N, h\cong \tilde H$
\end{itemize}
\end{teo}
\begin{dimo}[1]
	Definiamo l'applicazione\\
	\begin{aligned}
		\o :& H \rightarrow Aut(N)\\
		    &	h \rightarrow \o_n
	\end{aligned}\\
	dove $\o_{h}(n) := (hnh^{-1})\in hNh^{-1} = N \ \ \forall n\in N$\\
	Dato che abbiamo assunto  $N$ normale\\
	Abbiamo verificato la volta scorsa che è un omomorfismo.\\
	Definiamo l'applicazione\\
	\begin{aligned}
		\psi: &N\semi H \rightarrow G\\
		      &(n,h) \rightarrow nh
	\end{aligned}\\
	$\psi$ è suriettiva poiché $N\cdot H = G$ \\
	$\psi$ è iniettiva poichè\\
	\begin{gather*}
		n_1h_1 = n_2h_2 \rightarrow n_2^{-1}h_1 = h_2h_1^{-1}\in H\cap N = \{e\}\\ 
		\Rightarrow \begin{cases}
			n_2^{-1} n_1 = e\\
			h_2h_1^{-1} = e
		\end{cases} \rightarrow (n_1,h_1) = (n_2,h_2)
	\end{gather*}
	\textbf{$\psi$ è omomorfismo:}\\
	\begin{aligned}
	&\psi((n_1,h_1)\cdot(n_2,h_2)) =\\
	&=\psi((n_1\o_{h_1}(n_2),h_1h_2))\\
	& =n_1\o_{h_1}(n_2)h_1h_2\\ 
	&= n_1h_1n_2h^{-1}_1h_1h_2 = \psi(n_1,h_1)\cdot\psi(n_2,h_2)
		
	\end{aligned}\\
	Omomorfismo biunivoco
\end{dimo}
\newpage
	\begin{dimo}[2]
		Dato un isomorfismo\\
		$\psi :\tilde N\semi\tilde H \rightarrow G $\\
		definiamo:\\
		$N := \psi (\tilde N\semi \{e_{\tilde H}\})\normale G$\\
		$H:=\psi(\{e_n\}\semi \tilde H)$\\
		Osserviamo che:\\
		$\cdot \tilde N \cong \tilde N\semi \{e_{\tilde H}\}\cong N$\\ $\cdot \tilde H\cong \{e_{\tilde H\}\semi \tilde H\cong H$\\
			$\cdot N\cap H = \{e\}$\\
			 $\cdot NH = e$ \\(analogo alla dimostrazione per prodtto diretto)
	\end{dimo}


\end{document}
