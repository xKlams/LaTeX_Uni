\documentclass[12px]{article}

\title{Lezione 2 Algebra 1}
\date{2024-10-03}
\author{Federico De Sisti}

\usepackage{amsmath}
\usepackage{amsthm}
\usepackage{mdframed}
\usepackage{amssymb}
\usepackage{nicematrix}
\usepackage{amsfonts}
\usepackage{tcolorbox}
\tcbuselibrary{theorems}
\usepackage{xcolor}
\usepackage{cancel}

\newtheoremstyle{break}
  {1px}{1px}%
  {\itshape}{}%
  {\bfseries}{}%
  {\newline}{}%
\theoremstyle{break}
\newtheorem{theo}{Teorema}
\theoremstyle{break}
\newtheorem{lemma}{Lemma}
\theoremstyle{break}
\newtheorem{defin}{Definizione}
\theoremstyle{break}
\newtheorem{propo}{Proposizione}
\theoremstyle{break}
\newtheorem*{dimo}{Dimostrazione}
\theoremstyle{break}
\newtheorem*{es}{Esempio}

\newenvironment{dimo}
  {\begin{dimostrazione}}
  {\hfill\square\end{dimostrazione}}

\newenvironment{teo}
{\begin{mdframed}[linecolor=red, backgroundcolor=red!10]\begin{theo}}
  {\end{theo}\end{mdframed}}

\newenvironment{nome}
{\begin{mdframed}[linecolor=green, backgroundcolor=green!10]\begin{nomen}}
  {\end{nomen}\end{mdframed}}

\newenvironment{prop}
{\begin{mdframed}[linecolor=red, backgroundcolor=red!10]\begin{propo}}
  {\end{propo}\end{mdframed}}

\newenvironment{defi}
{\begin{mdframed}[linecolor=orange, backgroundcolor=orange!10]\begin{defin}}
  {\end{defin}\end{mdframed}}

\newenvironment{lemm}
{\begin{mdframed}[linecolor=red, backgroundcolor=red!10]\begin{lemma}}
  {\end{lemma}\end{mdframed}}

\newcommand{\icol}[1]{% inline column vector
  \left(\begin{smallmatrix}#1\end{smallmatrix}\right)%
}

\newcommand{\irow}[1]{% inline row vector
  \begin{smallmatrix}(#1)\end{smallmatrix}%
}

\newcommand{\matrice}[1]{% inline column vector
  \begin{pmatrix}#1\end{pmatrix}%
}

\newcommand{\C}{\mathbb{C}}
\newcommand{\K}{\mathbb{K}}
\newcommand{\R}{\mathbb{R}}


\begin{document}
	\maketitle
	\newpage
	\section{Nelle lezioni precedenti...}
	\begin{defi}
		$(G,\cdot)$ gruppo $H\leq G$
	$f,g\in G$ si dicono congruenti modulo $H$ se\\ $f^{-1}\cdot g\in H$
	\end{defi}
	\section{Classi di equivalenza}
	\begin{nota}
		classi di equivalenza:\\
		\[
		G/H
		.\] 
		%TODO INSERISCI IMMAGINE
	\end{nota}
	\textbf{Esempi importanti}\\
	$(G,\cdot) = (\mathbb Z, +)$
	$H = (m) = \lbrace a m | a\in \mathbb Z\rbrace$ con m fissato\\
	%TODO AGGIUNGI IMMAGINE
	 $G/H = \mathbb Z/(m)$ \\\
	 \textbf{Attenzione}\\
	 potete definire $f = g$ mod $H$ tramite la condizione  $f\cdot g^{-1}$ \\
	 Le due definizioni non sono equivalenti [La chiameremo congruenza destra]\\
	 \begin{nota}
	L'insieme delle classi di equivalenza destra si indica con
	\[
		H\backslash G
	.\] 
	 \end{nota}
	 \begin{defi}
	 	Gli elementi di $G/H$ si chiamano laterali sinistri, quelli di  $H\backslash G$ si chiamano laterali destri
	 \end{defi}
	 \textbf{Esercizio:}\\
	 $(G,\cdot)$ gruppo\\
	 $H\leq G \ \ g\in G$ fissato\\
	 Allora il laterale sinistro a cui appartiene  $g$ è\\ \[
	 gH = \lbrace g\cdot h | h\in H\rbrace
	 .\] 
	 \textbf{Soluzione}\\
	 fisso $f\in G$ e osserviamo che  \[
		 g\equiv f \text{ mod } H
	 .\] 
	 Se e solo se $g^{-1}\cdot f\in H$.\\
	 Questo è equivalente a 
	  \[
		  \exists h\in H \text{ tale che } g^{-1}\cdot f=h
	 .\] 
	 ovvero
	 \[
		 \exists h\in H \text{ tale che } f = g\cdot h
	 .\] 
	 \ \\ \hline \ \\ 
 \textbf{Esercizio}\\
 $H\leq G$\\
 Allora  $|G/H| = |H\backslash G|$ \\
 \textbf{Soluzione}\\
 Basta eseguire un'applicazione biunivoca tra i due insiemi
 %TODO AGGUGNI IMMAGINE\\
 \begin{defi}
 	$(G,\cdot)$ gruppo $H\leq G$ si dice sottogruppo normale se  $gH = Hg \ \ \ \forall g\in G$
 \end{defi}
 \textbf{Esempio}\\
 $G=S_3$ ricordo che  $S_3$ è il gruppo di permutazioni dell'insimee $\lbrace 1,2,3\rbrace$\\
 Quali sono gli elementi di  $S_3$?\\
 \[
	 \matrice{1&2&3\\1&3&2} = (2,3)
 \] 
 \[
	 \matrice{1&2&3\\2&3&1} = (3,2,1)
 \] 
scambio il 3 con l'uno , il 2 con il 2\\
\begin{aligend}
	&(2,3,1)\\
	&(1,3)\\
	&(1,2)\\
	&Id
\end{aligend}
\[
H_1 = <(1,2)> = \lbrace id, (1,2)\rbrace
.\] 
\[
H_2=<(3,2,1)> = \lbrace id, (3,2,1),(2,3,1)\rbrace
.\] 
\textbf{Esercizio}|\
Dimostrare che $H_1\leq S_3$ \underline{non} è normale, mentre $H_2\leq S_3$ è normale
\begin{nota}
	Se $H\leq G$ è normale scriveremo
	 \[
	H \trianglelefteq G
	.\] 
\end{nota}
\textbf{Esercizio}\\
$H\leq G$ sottogruppo dimostrare che l'applicazione  \begin{*aligend}&\phi:H \rightarrow gH\\
	&g \rightarrow g\cdot h\end{*aligend}\\
	\textbf{Soluzione}\\
	$ \phi$ è suriettiva per definizione di $gH$\\
	è anche iniettiva infatti se  $h_1,h_1\in H$ soddisfano 
	\[
		gh_1 = gh_2 \ \ 
	.\] 
	allora $h_1 = h_2$ (per la legge di cancellazione)\\
	\textbf{Ossercazione}\\
	$(G,\cdot)$ gruppo\\
	$H\leq G$ Allora
	 \[
	|gH| = |Hg| \ \ \forall g\in G
	.\] 
	anche se $gH\neq Hg$ poiché hanno entrambi la stessa cardinalità di  $H$\\
	Inoltre tutti i laterali sinistri (e destri) hanno la stessa cardinalità\\
	 \begin{defi}
		 $(G,\cdot)$ gruppo, $H\leq G$ l'indice di  $H$ in $G$ è 
		  \[
			  [G:H] = |G/H|
		 .\] 
		 dove $|G/H|$ è il numero di classi laterali sinistre
	\end{defi}
	\textbf{Osservazione}\\
	$H\leq G$ sottogruppo\\
	Se  $G$ è abeliano allora  $H\leq G$\\
	Il viceversa è falso! Possono esistere sottogruppi normali in gruppi non abeliani\\
	\begin{prop}
		$(G,\cdot)$ gruppo $H\leq G$ allora
		 \[
			 |G| = [G:H]|H|
		.\] 
	\end{prop}
	\begin{dimo}
		Basta ricordare che la cardinalità di ciascun laterale sinistro è pari a $|H|$
	\end{dimo}
	\textbf{Osservazione}\\
	$\displaystyle H\subseteq G => [G:H] = \frac {|G|}{|H|}$\\
	\begin{teo}[Lagrange]
		$(G,\cdot)$ gruppo $H\leq G$ Allora l'ordine di  $H$ divide l'ordine di  $G$
	\end{teo}
	\begin{dimo}
		Dall'osservazione segue $\displaystyle\frac{|G|}{|H|} = [G:H]\in \mathbb N$
	\end{dimo}
	\begin{coro}
		$(G,\cdot)$ gruppo di ordine primo (ovvero $|G| = p$ con $p$ primo)\\
		Allora $G$ non contiene sottogruppi non banali (tutto il gruppo o il gruppo minimale)
	\end{coro}
	\begin{dimo}
		Sia $H\leq G$ allora per Lagrange abbiamo
		 \[
			 |H| \text{ divide } p
		.\] 
		$ \Rightarrow |H| = 1$ quindi  $H = \lbrace e \rbrace$\\
		oppure 
		$ \Rightarrow |H| = p$ quindi $H=H$

	\end{dimo}
	\begin{coro}
		$(G,\cdot)$ gruppo (finito)\\
		Dato $g\in G $ si ha  $ord(g)$ divide l'ordine di $G$
		
	\end{coro}
	\begin{dimo}
		Dato $g\in G$ considero\\
		\[
			<g> = \lbrace e, g, g^2, \ldots, g^{n-1}\rbrace
		\]
		\[
		|<g>| = ord(g)
		.\] 
		La tesi segue ora da Lagrange
	\end{dimo}
	\section{Operazioni fra sottogruppi}
	\begin{prop}
		$(G,\cdot)$ gruppo $H,K \leq G$\\
		Allora  $H\cap K\leq G$
	\end{prop}
	\begin{dimo}
		$H\cap K$ è chiuso rispetto all'operazione e agli inversi poiché sia $H$ che $K$ che lo sono 
	\end{dimo}
	\textbf{Esercizio}\\
	Esibire due sottogruppi $H,J\leq G$ tali che  $H\cup K$ non è un gruppo\\
	\begin{defi}
		Dati $H,K\leq G$ definiamo il \underline {sottoinsieme}\\
		 \[
		HK = \lbrace h\cdot k | h\in H, k\in K\rbrace
		.\] 
		\textbf{Attenzione} non è necessariamente un sottogruppo
	\end{defi}
	\textbf{Esercizio}\\
	Dimostrare che $HK$ è un sottogruppo, di $G$ se e solo se 
	\[
	HK = KH
	.\] 
	\textbf{Soluzione}\\
	Supponiamo che $HK$ sia un sottogruppo
	\[
		HK = (HK)^{-1} = \lbrace (h\cdot k)^{-1} | h\in H, k\in K\rbrace = K^{-1}H^{-1} = KH
	.\] 
	Viceversa supponiamo che $HK = KH$\\
	1) Dimostro che  $KH$ è chiuso rispetto all'operazione.\\
	$h_1\codt k_1\in HK$ e $h_2\cdot k_2\in HK$\\
\[
	(h_1\cdot k_1)\cdot (h_2\cdot k_2) = h_1\cdot (k_1\cdot h_2)\cdot k_2 = h_1\cdot h_3\cdot k_3\cdot k_2 = (h_1\cdot h_3)\cdot(k_3\cdot  k_1)
.\] 
2) $HK$ è chiuso rispetto agli inversi
\[
	h\cdot k\in HK \leadsto (h\cdot k)^{-1} = k^{-1}\cdot h^{-1} = h_4\cdot k_4\in HK
.\] 
\begin{defi}[Sottogruppo generato da un sottoinsieme]
	$(G,\cdot)$ gruppo $X\subseteq G$ sottoinsieme\\
	Il sottogruppo generato da $X$ è
	\[
		<X> = \bigcap_{H\leq G, X\subseteq H} H
	.\] 
\end{defi}
\begin{nota}
	$\cdot H,K\leq G$\\
	 \[
	<H,K> := <H\cup K>
	.\] \\
	$\cdot g_1,\ldtos,g_n\in G$\\
	\[
	<g_1,\ldots, g_n> := <\lbrace{g_1,\ldots, g_n\rbrace >
	.\] 
\end{nota}
\textbf{Caso Speciale}\\
$(G,\cdot) = (\mathbb Z, +)\ \ \  m\in \mathbb Z$\\
$(m) := <m>$
\section{Sottogruppi di \mathbb Z}
\textbf{Ricordo}\\
dato $a\in \mathbb Z$ si ha  $(a)\leq \mathbb Z$\\
 \textbf{Obbiettivo}\\
 non esisotno altri sottogruppi
 \begin{teo}
 	$H\leq \mathbb Z$ allora esiste $m\in \mathbb Z$ tale che  $H = (m)$
 \end{teo}
 \begin{dimo}
 	Distinguiamo due casi:\\
	1) $H= (0)$ finito\\
	2) $H\neq(0)$ allora H contiene (almeno) un intero positivo, Definiamo
	\[
		m := \min\lbrace n\in\mathbb Z | n\geq 1, n\in H\rbrace
	.\] 
	Vogliamo verificare che $H = (m)$ Sicureamente $(m)\subseteq H$ poich\e H\leq \mathbb Z\\
	Viceversa supponiamo che  $\exists n\in Hx(n)$.\\
	Allora
	 \[
		 n = qm - r \text{ per qualche } q\in \mathbb Z \ \ 0 < r < m
	.\] 
	$ \rightarrow r = n - qm \in H$\\
	Ma $r>0, r<m$ quindi otteniamo l'assurdo per minimalità di m

 \end{dimo}
 \begin{prop}
 	$a,b\in \mathbb Z$, Allora:\\
	$1) (a)\cap (b) = (n)$ dove $m := mcm\lbrace a,b\rbrace$\\
	$2) (a) + (b) = (d)$ dove  $d := MCD\lbrace a, b \rbrace$

 \end{prop}
 \textbf{Osservazione}\\ 
 $(a)+(b)$ è della forma $HK$ con $ H = (a)$ e $K = (b)$\\
 inoltre $(a) + (b)\leq \mathbb Z$ poich\e $(\mathbb Z, +)$ è abeliano
 \begin{dimo}
	 $1)(a)\cap (b)$ è il sottogruppo dei multipli di $a$ e di $b$\\
	 Dunque $(a)\cap(b) = (m)$ \\
	 $2)a+b\leq \mathbb Z \Rightarrow (a) + (b) = (d')$ per teorema\\
	 Dobbiamo verificare che $d' = d$ \\
	 \[
		 (d) = (a)+(b)\supseteq (a) \Rightarrow d'|a ( d'\text{ divide }a)
	 .\] 
	 \[ \Rightarrow  \begin{cases}
	 	d'|a\\
		d'|b
	\end{cases} \Rightarrow d'\leq d\]\\
	 $d'\in (a) + (b) \Rightarrow \exists h,k\in \mathbb Z$ tale che $d' = ha + kb$\\
	 Dunque:\\
	 \[
	  \begin{cases}
	  	d|a\\
		d|b
	  \end{cases} \Rightarrow d|d' => d \leq d'\]
	  Allora $d=d'$
 \end{dimo}
 \section{Gruppi $D_n$ e $C_n$}
  \textbf{Ricordo}\\
  $n\geq 3$\\
  Fissiamo un  $n-agono$ \\
  $D_n = \lbrace$isometrie che preservano l'n-agono$\rbrace$\\
   $C_n  = \lbrace$isometrie che preserrvano l'n-agono e l'orientazione$\rbrace$\\
 \begin{teo}
 	$n\geq 3$ Allora\\
	\[|D_n| = 2n\]
\[|C_n| = n\]
 \end{teo}
 \begin{dimo}
	 Fissiamo un lato l dell'$n$-agono. Un'isometria $ \varphi\in D_n$ è univocamente determinata dall'immagine di $ \varphi(l)$\\
	 Ho $n$ scelte per il lato e per ogniuna di queste ho 2 scelte per le orientazione (mando il lato in se stesso? in quello dopo? in quello dopo ancora?, posso anche invertire la sua orientazione, i successivi lati vengono definiti da dove viene mandato il primo)\\
	 se non scegliamo l'orientazione, ci rimane il gruppo ciclico, e ciò conclude la dimostrazione
 \end{dimo}
 \textbf{Osservazione}\\
 La dimostrazione prova che
 \[
  C_n = <\rho>
 .\]
 dove $\rho$ è la rotazione di angolo $\frac {2\pi}{n}$ attorno al centro dell'$n$-agono\\
 Infatti $\rho\in C_n \Rightarrow <\rho>\subseteq C_n$ ma l'ordine di questa rotazione è $n$
  \[
 |<\rho>| = ord(\rho) = n = |C_n| => C_n = <\rho>
 .\] 
 \textbf{Osservazione}\\
 Dalla dimostrazione segue che $D_n$ è costituito da $n$ rotazioni \\(della forma $\rho^i \ \ i\in\lbrace 1,\ldots, n\rbrace$ \\
 e $n$ riflessioni\\
 \begin{prop}
 	$n\geq 3$ Allora:\\
	 $1)D_n = <\rho,\sigma >$\\
	 Dove  $\sigma$ è una rotazione qualsiasi  $(\sigma \in D_n\setminus C_n)$ \\
	 $2)\rho^i \sigma = \sigma \rho^{n-i}$ 
 \end{prop}
 \newpage
 \begin{dimo}
 	1)Sicuramente $<\rho,\sigma>\subseteq D_n$\\
	$H = <\rho> = \lbrace{Id, \rho, \rho^2, \ldots,\rho^{n-1}\rbrace $\\
	  $K = <\sigma> = \lbrace Id, \sigma \rbrace$ \\
	  $H\cap K = \lbrace Id\rbrace$
	   \[
		   |KH| = \frac{|H||K|}{|H\cap K|} = 2n
	   .\] 
	   $ \Rightarrow HK \subseteq D_n$ (In particolare HK è sottogruppo)
	   $ \Rightarrow D_n = HK = <\rho, \sigma>$\\
	   $\rho\sigma$ \underline{non} preserva l'orientazione\\
	   $ \Rightarrow \rho^i\sigma$ è riflessione\\
	   $ \Rightarrow ord(\rho^i\sigma) = 2$\\
	   $ \Rightarrow \rho^i\sigma\rho^i\sigma = Id$ \\
	   $ \Rightarrow \rho^i\sigma\rho^i=\sigma$ \\
	   $ \Rightarrow \sigma\rho^i = \rho^{n-1}\sigma$
 \end{dimo}

\end{document}
