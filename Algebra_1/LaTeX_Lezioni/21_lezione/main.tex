\documentclass[12px]{article}

\title{Lezione 21 Algebra I}
\date{2024-12-10}
\author{Federico De Sisti}

\usepackage{amsmath}
\usepackage{amsthm}
\usepackage{mdframed}
\usepackage{amssymb}
\usepackage{nicematrix}
\usepackage{amsfonts}
\usepackage{tcolorbox}
\tcbuselibrary{theorems}
\usepackage{xcolor}
\usepackage{cancel}

\newtheoremstyle{break}
  {1px}{1px}%
  {\itshape}{}%
  {\bfseries}{}%
  {\newline}{}%
\theoremstyle{break}
\newtheorem{theo}{Teorema}
\theoremstyle{break}
\newtheorem{lemma}{Lemma}
\theoremstyle{break}
\newtheorem{defin}{Definizione}
\theoremstyle{break}
\newtheorem{propo}{Proposizione}
\theoremstyle{break}
\newtheorem*{dimo}{Dimostrazione}
\theoremstyle{break}
\newtheorem*{es}{Esempio}

\newenvironment{dimo}
  {\begin{dimostrazione}}
  {\hfill\square\end{dimostrazione}}

\newenvironment{teo}
{\begin{mdframed}[linecolor=red, backgroundcolor=red!10]\begin{theo}}
  {\end{theo}\end{mdframed}}

\newenvironment{nome}
{\begin{mdframed}[linecolor=green, backgroundcolor=green!10]\begin{nomen}}
  {\end{nomen}\end{mdframed}}

\newenvironment{prop}
{\begin{mdframed}[linecolor=red, backgroundcolor=red!10]\begin{propo}}
  {\end{propo}\end{mdframed}}

\newenvironment{defi}
{\begin{mdframed}[linecolor=orange, backgroundcolor=orange!10]\begin{defin}}
  {\end{defin}\end{mdframed}}

\newenvironment{lemm}
{\begin{mdframed}[linecolor=red, backgroundcolor=red!10]\begin{lemma}}
  {\end{lemma}\end{mdframed}}

\newcommand{\icol}[1]{% inline column vector
  \left(\begin{smallmatrix}#1\end{smallmatrix}\right)%
}

\newcommand{\irow}[1]{% inline row vector
  \begin{smallmatrix}(#1)\end{smallmatrix}%
}

\newcommand{\matrice}[1]{% inline column vector
  \begin{pmatrix}#1\end{pmatrix}%
}

\newcommand{\C}{\mathbb{C}}
\newcommand{\K}{\mathbb{K}}
\newcommand{\R}{\mathbb{R}}


\begin{document}
	\maketitle
	\newpage
	\section{Esercizi Schede}
$G = GL_2(\C)$ \\
$X = Mat_{2\times 2}(\C)$\\
	\begin{aligend}
		&G\times X \rightarrow X\\
		&(A,B) \rightarrow A\cdot B
	\end{aligend}
	è un'azione tra gruppi\\
	Studiare le orbite\\
	\textbf{Soluzione}\\
	$O_{\icol{0&0\\0&0}} = \lbrace\matrice{0 & 0 \\ 0& 0}\rbrace$\\
	$O_{Id} = \{$ matrici invertibili$\}$\\
	Restano da studiare solo i casi di matrici non invertibili e non nulle
\\
	Se  $det(B) = 0$ allora  $B =\matrice{x & y\\ \lambda x & \lambda y}$ \ \ \ \  $x,y,\lambda\in \C$\\
	 $O_B = ?$\\
	 \textbf{Caso 1}\\Se  $x = 0$  $ \Rightarrow y\neq 0$ \\
	$ \Rightarrow  B = \matrice{0 &y\\0 & \lambda y}$\\
	Allora scelgo \\
	$A = \matrice{\frac 1 y & ?\\ ? & 1}$\\
	$AB = \matrice{\frac 1 y & 0\\ -\lambda & 1}\cdot \matrice{0 &y\\0 & \lambda y} = \matrice{0 & 1\\ 0 & 0}$\\
	Dove ho messo al posto dei punti interrogativi numeri appositi per arrivare alla matrice $e_{12}$\\
	Quindi se $x\neq 0 \Rightarrow O_B= O_\icol{0&1\\0&0}$ \\
	\textbf{Caso II}\\
	Se $x\neq 0$\\
	Scelgo  $A = \matrice{\frac 1x & 0 \\ ? & 1}$\\
	$AB = \matrice{\frac 1x & 0 \\ - \lambda & 1} \matrice{x &y\\ \lambda x& \lambda y} = \matrice{1 & \frac yx\\ 0 & 0}\\$
	$O_B = O_\icol{ 1 &\frac yx \\ 0 & 0 }$\\
	La matrice $A = \matrice{0&1\\1&0}$\\
	Scambia le righe di $B$ \\
	$ B = \matrice { a & b \\c & d}$\\
	$AB = \matrice{ c&b\\a&b}$\\
	\subsection{Ideali}
	\begin{defi}[Ideali]
		$(R,+,\cdot)$ anello\\ 
		Un ideale è un sottogruppo $(I,+)\leq (R,+)$ tale che
		\begin{enumerate}
		 \item $\forall a\in I$ \ \ $\forall x\in R $\\
			 $ \Rightarrow x\cdot a\in I$ \hfill [Ideale Sinistro]
		 \item $\forall a \in I \ \ \forall x\in R$\\
			 $ \Rightarrow a\cdot x\in I$ \hfill [Ideale Destro]
		 \item $\forall a\in I, \forall x\in R\\
			 \Rightarrow \begin{cases}
			 	x\cdot a \in I\\
				a\cdot x\in I
			\end{cases}$\hfill [Ideale bilatero]
			
		\end{enumerate}
	\end{defi}
	\textbf{Osservazione}\\
	Se $R$ è commutativo allora un sottogruppo (additivo) $I\leq R$\\ è ideale sinistro $ \Leftrightarrow$ è un ideale destro $ \Leftrightarrow$ è un ideale bilatero.
	\begin{nota}
		$R$ anello $I$ ideale bilatero lo chiameremo semplicemente \textbf{ideale}
	\end{nota}
	\textbf{Osservazione}\\
	$R$ anello $ \Rightarrow (R,+)$ è un gruppo abeliano\\
	$ \Rightarrow I\subseteq R$  ideale è un sottogruppo additivo normale\\
	\textbf{Esercizio:}\\
	$R$ anello $I\subseteq R$ ideale $ \Rightarrow (R/I, +)$ gruppo abeliano.\\
	Dimostrare che l'operazione  \\
	\begin{aligned}
		$\cdot \ : \ R&/I \times R/I \rightarrow R/I$\\
		&$(aI,bI) \rightarrow (ab)I$ \\
	\end{aligned}\\
	è ben definita e dedurre che $(R/I,+,\cdot)$ è un anello.\\
	\textbf{Esempio}\\
	$(\R,+,\cdot)$ è un anello\\
	 $(\Z,+) \leq (\R,+)$\\
	  $(\Z,+,\cdot)$ è un sottoanello\\
	  $(\Z, +,\cdot)$ non è un ideale in  $(\R,+,\cdot)$\\
	  Infatti  $\sqrt 2 \cdot 1\not\in \Z$\\
	   \textbf{Esempi}\\
	   $R$ anello\\
	   $ \Rightarrow  I=\{0\}$ è un ideale\\
	   $ \Rightarrow I=R$ è un ideale\\
	   \begin{defi}
	   	$R$ anello commutativo. $I\subseteq R$ ideale\\
		$I$ si dice primo se $I\neq R$ e $ab\in I \Rightarrow a\in I$ oppure $b\in I$
	   \end{defi}
	   \textbf{Esercizio}\\
	   $R = (\Z, + , \cdot)$\\
	   Determinare tutti gli ideali primi di $R$\\
	   \textbf{Esercizio:}\\
	   $R$ anello $I\subseteq R$ ideale\\
	   Dimostrare che le seguenti sono equivalenti
	    \begin{enumerate}
		    \item $R/I$ è un dominio d'integrità
		    \item  Se $a\cdot b\in I \Rightarrow a\in I$ oppure $b\in I$
	   \end{enumerate}
	   \begin{teo}[Omomorfismo per anelli]
	   	Dato $\varphi: R \rightarrow S$ un omomorfismo di anelli abbiamo
		\begin{enumerate}
			\item $ker (\varphi)\subseteq R$ è un ideale
			\item esiste un unico omomorfismo di anelli $\bar \varphi : R/ker(\varphi) \rightarrow S$\\
				tale che  \textbf{TODO inserisci immagine omomorfismo schema (foto sul telefono)}
			\item Esiste un isomorfismo di anelli\\
				$R/ker(\varphi) \cong Im(\varphi)$
		\end{enumerate}
	   \end{teo}
	   \begin{dimo}[Esercizio]
	   	1) Basta verificare che se $x\in ker( \varphi)$ e $y\in R$
		allora  \begin{cases}
			$x\cdot y\in ker( \varphi)\\
			y\cdot x\in ker( \varphi)$
		\end{cases}\\
		$x\in ker( \varphi) \Rightarrow \varphi(x) = 0$ \\
		Quindi\\
		\begin{aligned}
			$ \cdot \ \varphi(x\cdot y) &= \varphi(x)\cdot \varphi(y) \\&= 0\cdot \varphi(y) \\&= 0$
		\end{aligned}\\
		$ \Rightarrow x\cdot y\in Ker( \varphi)$ \\
		\begin{aligned}
			$ \cdot \ \varphi(y\cdot x) &= \varphi(y)\cdot \varphi(x) \\&= \varphi(y)\cdot 0 \\&= 0$
		\end{aligned}\\
		$ \Rightarrow y\cdot x\in ker( \varphi)$
	   \end{dimo}
	   \newpage
	   \subsection{Caratteristica}
	   Voglio associare ad ogni anello un numero intero che ci possa dare qualche informazione su di esso.
	   \begin{defi}
		   $(R,+,\cdot)$ anello.\\
		   Considero l'omomorfismo di anelli\\
		   \begin{aligned}
			   $\psi:& \Z \rightarrow R\\
			       &1 \rightarrow 1_R\\
			       &n \rightarrow (1_R +\ldots +1_R)\\$
		   \end{aligned}\\
	   Osserviamo che $\psi$ è un omomorfismo di anelli\\
	   \[
	   \psi(n\codt m) = \psi(n)\cdot \psi(m)
	   .\] 
	   Infatti\\
	   $\psi(n)\cdot\psi(m)=$\\
	    $= (1_R+\ldots+1_R)(1_R+\ldots+1_R)$\\
	    \text{} \ \ \ \ \ \ \ n volte \ \ \ \ \ \ \ \  \ \ m volte\\
	     $ = 1_R(1_R+\ldots+1_R) + \ldots + 1_R(1_R+\ldots+1_R)$\\
	     \text{} \ \ \ \ \ \ \ \ \ \  m volte \hspace{50px} \ \ \ \ \ \ \ m volte\\
	     $ = \psi(n\cdot m)$ \\
	     Allora $ker(\psi) = (m)\subseteq \Z$ per qualche  $m\geq 0$\\
	      $m$ si dice caratteristica di $R$.
	   \end{defi}
	   \textbf{Osservazione}\\
	   Supponiamo che $R$ abbia caratteristica positiva  $m > 0$ allora  $m = ord_{(R,+)}(1_R)$  \\
	   \textbf{Esercizio:}\\
	   $(R,+,\cdot)$ campo\\ 
	   Dimostrare che la caratteristica $R$ è $0$ oppure un numero primo\\
	   \textbf{Esempi:}\\
	   $\R,\mathbb Q,\C$\\
	   sono campi di caratteristica 0\\
Mentre $\Z/(p)$ è un campo di caratteristica $p$ (con  p primo)\\
\textbf{Esercizio}\\
Un anello commutativo è un campo se e solo se non possiede ideali non banali\\
\textbf{Soluzione}\\
Supponiamo che $R$ sia un campo e sia $I\subseteq R$ un ideale  $I\neq \{0\}$\\
Allora dobbiamo mostrare che  $I = R$.\\
Se  $a\in I\neq\{0\}$ considero  $a^{-1}\in R$\\
$a^{-1}\cdot a = 1\in I$ \hfill [Dato che è un ideale]\\
Dato  $b\in R:$\\
$b = b\cdot 1\in I$ \hfill [Dato che $1$ è nell'ideale ]\\
Viceversa:\\
dato $a\in R\setminus\{0\}$ dobbiamo verificare che esiste  $b\in R$ tale che  $a\cdot b = 1$ \\
Definiamo $I := \{ a\cdot r | r\in R\}\subseteq R$\\
$I$ è un ideale.
Inoltre $I \neq \{0\}$ poiché $a\in I$\\
$ \Rightarrow  I = R \Rightarrow 1\in I$ \hfill [Per ipotesi]\\
Quindi esiste $b\in R$  tale che $a\cdot b = 1$\\
 \textbf{Osservazione}\\
 Se $R$ campo e $\psi: R \rightarrow S$  è un omomorfismo di anelli, allora $\psi$ è iniettivo oppure $\psi $ è l'omomorfismo nullo.\\
 Abbiamo verificato che $ker(\psi)$ è un ideale in $R$.\\
 Quindi:
 \begin{itemize}
	 \item $ker(\psi) = \{0\}$\\
		  $ \Rightarrow $ iniettivo
	  \item $ker (\psi) = R$\\
		   $ \Rightarrow \psi(r) = 0 \ \ \ \forall r\in R$
 \end{itemize}



\end{document}
