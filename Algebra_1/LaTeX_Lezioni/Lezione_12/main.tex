\documentclass[12px]{article}

\title{Lezione 12 Algebra 1}
\date{2025-04-24}
\author{Federico De Sisti}

\usepackage{amsmath}
\usepackage{amsthm}
\usepackage{mdframed}
\usepackage{amssymb}
\usepackage{nicematrix}
\usepackage{amsfonts}
\usepackage{tcolorbox}
\tcbuselibrary{theorems}
\usepackage{xcolor}
\usepackage{cancel}

\newtheoremstyle{break}
  {1px}{1px}%
  {\itshape}{}%
  {\bfseries}{}%
  {\newline}{}%
\theoremstyle{break}
\newtheorem{theo}{Teorema}
\theoremstyle{break}
\newtheorem{lemma}{Lemma}
\theoremstyle{break}
\newtheorem{defin}{Definizione}
\theoremstyle{break}
\newtheorem{propo}{Proposizione}
\theoremstyle{break}
\newtheorem*{dimo}{Dimostrazione}
\theoremstyle{break}
\newtheorem*{es}{Esempio}

\newenvironment{dimo}
  {\begin{dimostrazione}}
  {\hfill\square\end{dimostrazione}}

\newenvironment{teo}
{\begin{mdframed}[linecolor=red, backgroundcolor=red!10]\begin{theo}}
  {\end{theo}\end{mdframed}}

\newenvironment{nome}
{\begin{mdframed}[linecolor=green, backgroundcolor=green!10]\begin{nomen}}
  {\end{nomen}\end{mdframed}}

\newenvironment{prop}
{\begin{mdframed}[linecolor=red, backgroundcolor=red!10]\begin{propo}}
  {\end{propo}\end{mdframed}}

\newenvironment{defi}
{\begin{mdframed}[linecolor=orange, backgroundcolor=orange!10]\begin{defin}}
  {\end{defin}\end{mdframed}}

\newenvironment{lemm}
{\begin{mdframed}[linecolor=red, backgroundcolor=red!10]\begin{lemma}}
  {\end{lemma}\end{mdframed}}

\newcommand{\icol}[1]{% inline column vector
  \left(\begin{smallmatrix}#1\end{smallmatrix}\right)%
}

\newcommand{\irow}[1]{% inline row vector
  \begin{smallmatrix}(#1)\end{smallmatrix}%
}

\newcommand{\matrice}[1]{% inline column vector
  \begin{pmatrix}#1\end{pmatrix}%
}

\newcommand{\C}{\mathbb{C}}
\newcommand{\K}{\mathbb{K}}
\newcommand{\R}{\mathbb{R}}


\begin{document}
	\maketitle
	\newpage
	\subsection{Successioni esatte corte di moduli}
	\textbf{Ricordo}\\
	\begin{defi}
		Una SEC è una coppia di omeomorfismi $R$-moduli
		\[
			M' \xrightarrow{i} M \xrightarrow{\pi} M''
		.\] 
		tali che 
		\begin{enumerate}
			
			\item $i$ iniettiva
			\item $\pi$ suriettiva
			\item $Ker(\pi) = Im(i)$
		\end{enumerate}
	\end{defi}
	Abbiamo dimostrato
	\begin{enumerate}
		\item Se $M$ è finitamente generato allora  $M''$ è finitamente generato.
		\item Se M' e M'' sono finitamente generati allora M è finitamente generato.
	\end{enumerate}
	\textbf{Esercizio}\\
	$R = \R[x_1,x_2,\ldots]$
	\begin{itemize}
		\item $M = R$ è un $R$-modulo (libero di ragno 1) su se stesso (è generato da $1_R$).
		\item $I  = (x_1,x_2,\ldots)$\\
			Verificate che $I$ non è finitamente generato come $R$-modulo.

	\end{itemize}
	\textbf{Obiettivo}\\
	Studiare  il caso di (sotto) moduli su anelli Noetheriani.
	\begin{lemm}
		$R$ anello commutativo.\\
		Allora $R$ è Noetheriano se e solo se ogni ideale di $R$ è finitamente generato.
	\end{lemm}
	\begin{dimo}
		\begin{enuemrate}
		\item Se $R$ è Noetheriano, assumiamo per assurdo che esista $I\subseteq R$ ideale  $\underline{non}$ finitamente generato. Sia  $r_1\in I\setminus\{0\}$\\
			definiamo $I_1 = (r_1)\subsetneq I$\\
			Sia $r_2\in I\setminus I_1$\\
			definiamo $I_2 = (r_1,r_2)\subsetneq I$ \\
			Iteriamo:\\
			Sia $r_k \in I\setminus I_{k-1}$\\
			definiamo  $I_k = (r_1,\ldots,r_k)\subsetneq I$\\
			Abbiamo una catena infinita di ideali\\
			$I_1\subsetneq I_2\subsetneq I_3\subsetneq\ldots\subsetneq I_k\subsetneq \ldots$\\
			che contraddice l'ipotesi su $R$ 
		\item Viceversa, supponiamo che ogni ideale di $R$ sia finitamente generato\\
			Consideriamo una catena di ideali
			\[
			I_1\subseteq I_2\subseteq \ldots\subseteq I_k\subseteq\ldots
			.\] 
			Definiamo $I = \bigcup^{}_{k\in\Z_{\geq 1}}I_k$ 
			$I\subseteq R$ è un ideale.\\
			Allora $I = (r_1,\ldots,r_h)$ per qualche scelta di $r_1,\ldots,r_h\in R$\\
			Allora $\forall j\in\{1,\ldots,h\}$ \\
			$\exists k_j\in\Z_{\geq 1}$ tale che  $r_j\in I_{k_j}$\\
			Definiamo  $\underline k=\max\{k_1,\ldots k_h\}$ 
			\[
				\Rightarrow r_j\in I_{k_j}\subseteq I_{\undelrline k} \ \ \ \forall j\in\{1,\ldots h\}
			.\] 
			Quindi $I = (r_1,\ldots, r_h)\subseteq I_{\underline k}\subseteq I$\\
			$ \Rightarrow  I = I_{\underline k}$ \\
			$ \Rightarrow  I_k  = I_{\underline k}\ \ \ \ \ \forall k\geq \underline k$ \\
			$ \Rightarrow R$ Noetheriano
		\end{enuemrate}
	\end{dimo}
	\begin{teo}
		$R$ anello Noetheriano $M \ R$-modulo finitamente generato, Allora ogni sottomodulo di  $M$ è finitamente generato
	\end{teo}
	\begin{dimo}
		\textbf{Passo I:} Studiamo il caso $M = R^n$ \\
		Per induzione su $n$.\\
		$n = 1$  $M = R$ e i suoi sottomoduli sono gli ideali di  $R$, che sono finitamente generati per il lemma. \\
		$n > 1 \ \ $ Sia  $N\subseteq R^n$ sottomodulo\\
		Consideriamo
		\[
		\begin{aligned}
			R \xrightarrow{i} &R^n \xrightarrow{\pi} R^{n-1}\\
			r \rightarrow \icol { 0\\ \vdots\\0 \\ 1} \rightarrow& \icol{a_1\\\vdots\\a_n} \rightarrow\icol{a_1\\\vdots\\a_{n-1}}
		\end{aligned}
		.\] 
		\[
		\begin{aligned}
			R \xrightarrow i &R^n \xrightarrow \pi R^{n-1}\\
			K = \ker(\pi|_N) \xrightarrow {i|_N} &N \xrightarrow {\pi|_N} \pi(N)

		\end{aligned}
		.\] 
		Quindi $N$ finitamente generato\\
		\textbf{Passo II:} $M\ R$-modulo finitamente generato\\
		Esiste un omomorfismo suriettivo
		\[
		\begin{aligned}
			R^n &\xrightarrow \phi M\\
			 \icol{a_1\\\vdots\\a_n} &\rightarrow \sum^{n}_{j = 1}a_jm_j
		\end{aligned}
		.\] 
		Sia $N\subseteq M$ sottomodulo, Definiamo
		 \[
			 \tilde N = \phi^{-1}(N)\subseteq R^n
		.\] 
		Abbiamo una SEC
		\[
			\ker(\phi|_{\tilde N}) \rightarrow\tilde N \xrightarrow{\phi|_{\tilde N}} N
		.\] 
		Quindi $\tilde N$ finitamente generato per il primo passo\\
		 $ \Rightarrow N$ finitamente generato
	\end{dimo}
	\textbf{Ricordo:}
	\begin{teo}[di struttura]
		$R \ PID$  $M\ R$-modulo finitamente generato. Allora esistono $d_1,\ldots, d_n\in R$
		\begin{enumerate}
			\item $d_j\ | d_{j+1}\ \ \ \ \forall j\in \{1,\ldots,n-1\}$
			\item  $M\cong R/(d_1)\oplus R/(d_2)\oplus \ldots\oplus R/(d_n)$
		\end{enumerate}
	\end{teo}
	\textbf{Osservazione}\\
	1) possiamo assumere  $d_j$ non invertibile $\forall j\in \{1,\ldots,n\}$\\
	2) i  $d_j$ sono univocamente determinati a meno di associati.\\
	3) Alcuni dei $d_j$ potrebbero essere 0.\\
	\textbf{Esempi:}\\
	1) $R = \Z$.\\
	2)  $R = \K[t]$ con  $\K$ campo qualsiasi.
	\subsection{Decomposizione primaria}
	\begin{coro}
		$R\ PID$,  $M\ R$-modulo finitamente generato.\\
		Allora esistono\\
		$h,k\in \Z_{\geq 0}, n_1,\ldots, n_k\in \Z_{\geq 1}$\\
		e desistono primi $p_1,\ldots,p_k\in R$\\
		tali che 
		\[
			M\cong R/(p_1^{n_1})\oplus\ldots\oplus R/(p_k^{n_k})\oplus R^h
		.\] 
	\end{coro}
	\begin{dimo}
		Dal teorema di struttura:\\
		$M\cong R/(d_1)\oplus \ldots\oplus R/(d_k)\oplus R^h$ \\
		Basta studiare il caso $R/(d)$ con  $d\neq 0$ e  $d$ non invertibile \\
		Ricordiamo che $PID \Rightarrow  UFD$ \\
		Quindi 
		\[
			d = p_1^{n_1}\cdot\ldots\cdot p_s^{n_s}$ 
		.\] 
		con $p_1,\ldots,p_s$ distinti\\
		Dal teorema cinese del resto \[
			 R/(d)\cong R/(p_1^{n_1})\oplus\ldots\oplus R/(p_s^{n_s})
		.\] 
	\end{dimo}
	\section{Forma canonica razionale}
	\textbf{Osservazione}\\
	il dato di un $\K[t]$-modulo è equivalete al dato di una coppia $(V,T)$ dove  $V$ è un $\K$-spazio vettoriale e $T:V \rightarrow V$ applicazione $\K$-lineare\\
	dato $M\ \ \K[t]$-modulo poniamo  $V =M$ \\
	dove la moltiplicazione per scalari $a\in \L$ è data dalla moltiplicazione per polinomi costanti  $a\in \K[t]$\\
Inoltre
 \[
\begin{aligned}
	T: V & \rightarrow V\\
	v & \rightarrow t\cdot v
\end{aligned}
.\] 
Viceversa, data una coppia $(V,T)$ vogliamo costruire un  $\K[t]$-modulo $M$.\\
Poniamo  $M = V$\\
e\\
\[
	\begin{aligned}
		\K[t]\times V &\rightarrow V\\
		(p(t),v) & \rightarrow p(T)\cdot v
	\end{aligned}
\]
\textbf{Osservazione}\\
se $p(t) = t$\\
$ \Rightarrow  (p(t),v) \rightarrow T(v)$ 
\begin{defi}
	Dato un polinomio monico $p(t) = t^n + a_{n-1} t^{n-1} + \ldots + a_0\in \K[t]$\\
	La matrice compagna di  $p$ è
	 \[
		 C_p = \matrice{0 & 0 & \ldots & 0 & -a_0\\
			 1 & 0 & \ldots  & 0 & -a_1\\
			 0 & 1 & \ldots & 0& -a_2\\
			 \vdots & \ldots & \ddots & 0 & \vdots\\
0 & \ldots & 0 &1 & -a_{n-1}}
	.\] 
\end{defi}
\textbf{Esercizio}\\
Dimostrare che il polinomio caratteristico di $C_p$ è $p$ \\
$ det(t iD - C_p)= p(t)$ \\
\begin{teo}[forma canonica razionale]
	$\K$ campo $V\ \ \K$-spazio vettoriale di dimensione finita $T\in End_\K (V,V)$\\
	Allora esistono polinomi monici  $f_1,\ldots, f_k\in\K[t]$ ed esiste una base $B$ di $V$ tali che
	 \begin{enumerate}
		 \item $f_k\ | \ f_{j+1} \ \ \ \forall j\in\{1,\ldots,k-1\}$
		 \item la matrice che rappresenta  $T$ nella base $B$ è 
			 \[
				 T = \matrice{C_{f_1} & 0 & 0\\ 0 &\ddots & 0 \\
				 0 & 0 & C_{f_k}}
			 .\] 
		 \item il polinomio caratteristico di $T$ è $f_1,\ldots,f_k$
	\end{enumerate}
\end{teo}
\begin{dimo}
	Per il teorema di struttura \\ $V\cong \K[t]/(f_1)\oplus\ldots\oplus \K[t]/(f_k)$ \hfill isomorfismo di $\K[t]$-moduli\\
	\textbf{Osserviamo}\\
	\begin{enumerate}
		\item $dim_\K(V) < +\infty \Rightarrow  f_j\neq 0 \ \forall j\leq k$ 
		\item possiamo scegliere $f_j$ monico $\forall j$
		\item $f_j \ | \ f_{j+1}\ \ \ \forll j\in\{1,\ldots , k-1\}$\\
	\end{enumerate}
			Consideriamo il sottospazio 
			 \[
				 U_j = \K[t]/(f_j)
			.\] 
			scegliamo base 
			\[
				B_j = \{v_1^j,\ldots,v_{g_j}^j\}
			.\] 
			dove $g_j = deg(f_j)$ e  $v_i^j = t^{i-1}$\\
			 $T(v_i^j) = t\cdot v^j_i= \begin{cases}
				 v_{i+1}^j\ \ \ \forall i\in\{1,\ldots,g_j-1\}\\ 
				 - \sum^{g_j -1}_{i = 0}a_i\cdot v_{i+1}^j \ \ i = g_j\\

			 \end{cases}$
			 dove $f_j(t) = t^{g_j} + \sum^{g_j-1}_{i=1}a_i\cdot t_i$\\
			 Quindi:
			 \[
				 T = \matrice{T|_{U_1}&0&0\\0&\ddots & 0\\
				 0 & 0 & T|_{U_k}}
			 .\] 
				 e $T|_{U_j} = \matrice{0 & 0 &\ldots&- a_0\\
					 1 & 0 & \ldots & \vdots\\
					 \vdots & \vdots & \ddots & \vdots \\
				 0& \ldots & \ldots & -a_{n-1}} = C_{f_j}$
\end{dimo}
\subsection{Cayley-Hamilton}
\begin{defi}
	$V\ \K$-spazio vettoriale $T\in End_K(V)$\\
	$f(t) = t^n + a_{n-1}t^{n-1} + \ldots + a_0\in\K[t]$\\
	diremo che $T$ annulla $f$ se\\
	\[
		f(T) = T^n + a_{n-1}T^{n-1}+\ldots+a_1T + a_0 Id
	.\]
	è il polinomio nullo
\end{defi}
\begin{teo}[Hamilton (1853), Cayley (1858), Frobenius (1878)]
	$\K$ campo $V$ spazio vettoriale, $T\in End_\K(V)$,  $dim_\K(V)< +\infty$\\
	Allora  $T$ annulla il suo polinomio caratteristico
\end{teo}
\begin{dimo}
	Dalla forma canonica razionale possiamo studiare il caso in cui $T$ sia rappresentato dalla matrice compagna
	 \[
		 C_p = \matrice{0 & 0 & \ldots & 0 & -a_0\\
			 1 & 0 & \ldots  & 0 & -a_1\\
			 0 & 1 & \ldots & 0& -a_2\\
			 \vdots & \ldots & \ddots & 0 & \vdots\\
0 & \ldots & 0 &1 & -a_{n-1}}
	.\] 
	Sia $\{v_1,\ldots,v_n\}$ la base in cui $T$ ha questa forma\\
	$T^n(v_k) = T^{k-1}(v_n)$\\
	$T^n (v_k) = T^{k-1}(v_n) = T^{k-1}(- \sum^{n-1}_{j=0}a_jv_{j+1})$\\
$ = - \sum^{n-1}_{j=0}a_j T^{k-1}(v_{j+1}) = -\sum^{n-1}_{j=0}a_j T^{k -1 + j}(v_1)$\\
$ = - \sum^{n-1}_{j=0}a_jT^j(T^{k-1}(v_1)$\\
$ = - \sum^{n-1}_{j-0}a_jT^j(v_k)$\\
Quindi:\\
$(T^n + \sum^{n-1}_{j=0}a_jT^j)(v_k) = 0\ \ \forall k$ \\
$ \Rightarrow  T$ annulla il polinomio
\[
	f(t) = t^n + a_{n-1}t^{n-1}+ \ldots a_0
.\]  che è il polinomio caratteristico di $C_f = T$
\end{dimo}
\begin{defi}
	$V\ \ \K$-spazio vettoriale\\
	 $T \in End_\K(V)$ il polinomio minimo di  $T$ è il polinomio monico di grado minimo che è annullato da $T$
\end{defi}
\begin{teo}[Forma canonica di Jordan]
	$\K$ campo algebricamente chiuso, $V\ \ \K$-spazio vettoriale  $T\in End_\K(V)$ \  $dim_\K(V) < +\infty$\\
	Allora esiste una base di  $V$ in cui $T$ si rappresenta come una matrice
	\[
		T =\matrice{J_{\lambda_1} & 0 & 0\\
			0 & \ddots & 0\\
			0 & 0 & T_{\lambda_k}}
	.\] 
			dove $J_{\lambda_i} = \matrice{\lambda_i & 1 & 0 &\ldots & 0 \\
				0 & \ddots & \ddots & \ddots &\vdots\\
				\vdots & \ddots & \ddots & \vdots & 1\\
			0 & \ldots & \ldots & 0 & \lambda_i}
				$
\end{teo}
\begin{dimo}
	dalla decomposizione primaria segue
	\[
		V\cong \K[t]/((t-\lambda_1)^{s_1})\oplus\ldots\oplus\K[t]/((t-\lambda_k)^{s_k})
	.\] 
	poiché primo $ \Leftrightarrow$ irriducibile $ \Leftrightarrow\ deg =1 \hfill $ di $\K$ algebricamente chiuso \\
	Si tratta  di trovare una base per $U = \K[t]/((t-\lambda)^{s})$ in cui $T$ si rappresenta come $J_\lambda$ \\
	Scelgo la base $\{v_1,\ldots,v_n\}$ per $i\in\{1,\ldots $ con  $v_i = (t-\lambda)^{s-i}$\\
		Ora \\
		$T(v_i) = t\cdot v_i = [(t-\lambda) + \lambda] \cdot v_ = (t- \lambda)v_i + \lambda v_i\\
		= \begin{cases}
			v_{i - 1} + \lambda v_i \text{ se } i\neq 1\\
			\lambda v_1 \ \ \ \text{ se } i = 1
		\end{cases}$
		Allora:\\
		$T|_{U_j} = J_{\lambda_j} = \matrice{\lambda_j & 1 & 0 &\ldots & 0 \\
				0 & \ddots & \ddots & \ddots &\vdots\\
				\vdots & \ddots & \ddots & \vdots & 1\\
			0 & \ldots & \ldots & 0 & \lambda_j}
				$
\end{dimo}


\end{document}
