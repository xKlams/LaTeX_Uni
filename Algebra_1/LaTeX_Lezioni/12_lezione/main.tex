\documentclass[12px]{article}

\title{Lezione 12 Algebra I}
\date{2024-11-07}
\author{Federico De Sisti}

\usepackage{amsmath}
\usepackage{amsthm}
\usepackage{mdframed}
\usepackage{amssymb}
\usepackage{nicematrix}
\usepackage{amsfonts}
\usepackage{tcolorbox}
\tcbuselibrary{theorems}
\usepackage{xcolor}
\usepackage{cancel}

\newtheoremstyle{break}
  {1px}{1px}%
  {\itshape}{}%
  {\bfseries}{}%
  {\newline}{}%
\theoremstyle{break}
\newtheorem{theo}{Teorema}
\theoremstyle{break}
\newtheorem{lemma}{Lemma}
\theoremstyle{break}
\newtheorem{defin}{Definizione}
\theoremstyle{break}
\newtheorem{propo}{Proposizione}
\theoremstyle{break}
\newtheorem*{dimo}{Dimostrazione}
\theoremstyle{break}
\newtheorem*{es}{Esempio}

\newenvironment{dimo}
  {\begin{dimostrazione}}
  {\hfill\square\end{dimostrazione}}

\newenvironment{teo}
{\begin{mdframed}[linecolor=red, backgroundcolor=red!10]\begin{theo}}
  {\end{theo}\end{mdframed}}

\newenvironment{nome}
{\begin{mdframed}[linecolor=green, backgroundcolor=green!10]\begin{nomen}}
  {\end{nomen}\end{mdframed}}

\newenvironment{prop}
{\begin{mdframed}[linecolor=red, backgroundcolor=red!10]\begin{propo}}
  {\end{propo}\end{mdframed}}

\newenvironment{defi}
{\begin{mdframed}[linecolor=orange, backgroundcolor=orange!10]\begin{defin}}
  {\end{defin}\end{mdframed}}

\newenvironment{lemm}
{\begin{mdframed}[linecolor=red, backgroundcolor=red!10]\begin{lemma}}
  {\end{lemma}\end{mdframed}}

\newcommand{\icol}[1]{% inline column vector
  \left(\begin{smallmatrix}#1\end{smallmatrix}\right)%
}

\newcommand{\irow}[1]{% inline row vector
  \begin{smallmatrix}(#1)\end{smallmatrix}%
}

\newcommand{\matrice}[1]{% inline column vector
  \begin{pmatrix}#1\end{pmatrix}%
}

\newcommand{\C}{\mathbb{C}}
\newcommand{\K}{\mathbb{K}}
\newcommand{\R}{\mathbb{R}}


\begin{document}
	\maketitle
	\newpage
	\section{Divisione Euclidea}
	\begin{teo}
		$a,b\in\Z$ con  $b\neq 0$ allora $\exists q,r\in \Z$ tale che \\
		$	\cdot a = qb+r\\
			\cdot 0\leq r< |b|$
	\end{teo}
	\begin{dimo}
		Procediamo per passi\\
		$1) a,b\in \Z_{>0}$
		 \[
			 A = \{k\in\Z | kb>a\}
		.\] 
		Osserviamo che $A\neq \emptyset$\\
		Infatti  $(a+1)b=ab+b>ab\geq a \Rightarrow a + \in A$ \\
		Per il principio del buon ordinamento di $\mathbb N$
		 \[
		 \Rightarrow \exists m := min\{k\inA\}\in\Z^+
		.\] 
		Definiamo 
		\[
		q:=m-1\in \Z^+
		.\] 
		$q\not\in A$ e $q+1\in A$\\
		 $qb\leq a < (q+1)b=qb + b$ \\
		 $ \Rightarrow 0\leq a - qb < b$ \\
		 Definiamo $r = a - qb$ e otteniamo:\\
		  $0\leq r < b\\
		  a = qb + r$\\
		  2)  $a\in\Z\ b > 0$\\
		  Se  $a\geq 0 \ ($ok per 1)\\
		  Se  $a < 0 \Rightarrow - a > 0$ \\
		  $ \Rightarrow -a = qb + r$ con $0\leq r< b$\\
		   $ \Rightarrow a = (-q)b - r$ \\
		   Se $r = 0$ abbiamo finito \\
		   Se invece $0 < r < b$\\
		   definiamo  $r' = b-r \Rightarrow 0 < r' < b$ \\
		   $a = (-q)b - b + \frac{b-r}{r'}$\\
		    $ \Rightarrow a = (-q-1)b + r' = q'b + r'$\\
		    3) $a\in\Z, b < 0$\\
		     $ \Rightarrow -b > 0\\
		     a = q(-b) + r$ con $0\leq r<-b$\\
		      $ \Rightarrow a = (-q)b + r \ \ 0\leq r < |b|$
	\end{dimo}
	\newpage
	\section{Esercizi delle schede}
	\begin{cases}
		
	x\equiv 50\ mod(110)\\
	x \equiv 47 mod(73)
	\end{cases}\\
	Dal teorema cinese del resto sappiamo che esiste un'unica soluzione modulo il prodotto $mod(110 * 73) = mod(8030)$\\
	Come lo costruisco?\\
	$\bar x = 50 \cdot 73 \cdot m_1 + 47 \cdot 110 \cdot m_2$\\
	L'idea è di infilare al posto di $m_1$ l'inverso di $73\ mod(110)$
	\[
	\begin{cases}
		73\cdot m_1\equiv 1 \ (110)\\
110 \cdot m_2 \equiv 1 \ (73)
	\end{cases}
	.\] 
	Bisogna determinare $m_1, m_2$\\
	\textbf{Idea:} Sfruttare l´identità di Bezoit: $(n_1) + (n_2) = (MCD(n_1,n_2)) = (1)$\\
	obiettivo: $n_1\cdot e + n_2\cdot s = 1$\\
	Nel nostro caso cerco $110 \cdot r + 73\cdot s = 1$  \ \ $r,s\in\Z$\\
 Perché è importante
 $110\cdot r \equiv 1\ mod(73)$\\
  $73\cdot s\equiv 1 \ mod(110)$\\
Il nuovo obiettivo è determinare  $r,s$ \\
Procedo con la divisione euclidea tra $110$ e $73$\\
\begin{gather*}
	110 = 73 + 37\\
	73 = 2 \cdot 37 - 1\\
	\Rightarrow 1 = 2\cdot 37 - 73\\
	\Rightarrow 2\cdot(110-73)-73 = 1\\
	\Rightarrow 2\cdot 110 - 3\cdot 73
\end{gather*}
Quindi:\\
$1 = 2 \cdot 110 - 3\cdot 73$\\
da cui \\
 \begin{gather*}
	m_1 = -3\\
	m_2 = 2
\end{gather*}
 \[
\bar x \equiv 5-\cdot 73\cdot(-3) + 47\cdot 110 \cdot (2) \equiv -620 \ mod(8030)
.\] 
\pene\\
\textbf{Nuovo Esercizio}\\
\begin{cases}
	x\equiv_6 2\\
	x\equiv_{10} 3
\end{cases}
Non possiamo sfruttare il teorema cinese del resto \\
\begin{gather*}
	x\equiv_6 2\\
	\storto{ \Leftrightarrow}\\
	x = 2 + 6k \ \ k\in\Z\\
	\storto{ \Leftrightarrow}\\
	x = 2(1 + 3k)
\end{gather*}
\begin{gather*}
	x\equiv_{10} 3\\
	\storto \Leftrightarrow\\
	x = 3 + 10h \ \ h\in \Z\\
	\storto \Leftrightarrow\\
	x = 2(5h + 1) + 1
\end{gather*}
Dunque dalla prima congruenza segue
\[
x\equiv_2 0 
.\] 
dalla seconda
\[
x\equiv_2 1
.\] 
\pene \\
\textbf{Nuovo Esercizio}\\
\begin{cases}
	3x\equiv_{15} 6\\
	7x\equiv_9 2
\end{cases}\\
Non posso usare $TCB$ studio $3x\equiv_{15} 6$ \\
\begin{gather*}
	3x\equiv 6 + 15k\\
	\storto \Leftrightarrow\\
	3x = 3(2 + 5k)\\
	\storto \Leftrightarrow\\
	x = 2 + 5k
\end{gather*}
\begin{cases}
	x\equiv_5 2\\
	7x\equiv_9 2
\end{cases}\\
Ora $MCD(3,9) = 1$ Vorrei sfruttare TCR, per farlo dobbiamo eliminare i coefficienti\\
Noto che $7$ e $9$ sono coprimi $ \Rightarrow [7]\in U_9$ (invertibili modulo 9)\\
Cerchiamo l'inverso moltiplicativo di $[7]\in U_9$\\
ovvero cerco $s\in\Z$ tale che  $7s \equiv_9 1$\\
Utilizzo la divisione euclidea\\
\begin{gather*}
	9 = 7 + 2\\
	7 = 3\cdot 2 + 1\\
	\Rightarrow 1 = 7 - 3\cdot 2\\
	\Rightarrow 1 = 7 - 3\cdot (9 - 7)\\
	\Rightarrow 1 = 4\cdot 7 - 3\cdot 9
\end{gather*}
Quindi $s = 4$\\
 \begin{gather*}
 	7x\equiv_9 2\\
	\storto \Leftrightarrow\\
	4\cdot 7 \equiv_9 4\cdot 2\\
	\storto \Leftrightarrow\\
	x\equiv_9 8\\
 \end{gather*}
Il sistema è quindi equivalente a \\
\[\begin{cases}
	x\equiv_5 2\\
	x\equiv_9 8
\end{cases}\]
Applico TCR\\
La soluzione esiste ed è unica modulo $(45)$\\
Soluzione:
 \[
\bar x \equiv_{45} 2\cdot 9 \cdot m_1 - 1\cdot 5\cdot m_2
.\] 
Dove : \ \ 
\begin{cases}
	5m_2\equiv_9 1\\
	9m_1\equiv_5 1
\end{cases}
Divisione euclidea
\begin{gather*}
	9 = 5 + 4\\
	5 = 4 + 1\\
	1 = 5 - 4\\
	1 = 5 - (9 - 5)\\
	1 = 2\cdot 5 - 9
\end{gather*}
$ \Rightarrow m_2 = 2 \ \ m_1 = -1$ 
\[
\bar x \equiv_{45} -18 -10 \equiv_{45} -28
.\] 
\section{Azioni di gruppi}
\begin{defi}
	Un'azione di un gruppo $(G,\cdot)$ su un insieme $X$ è un'applicazione 
	\\
	\begin{center}
		
	\begin{aligned}
		G&\times X \rightarrow X\\
		 &(g,x) \rightarrow g.x
	\end{aligned}
	\end{center}
	tale che\\
	1) $e.x = x$\\
	2)  $(f \cdot g).x = f(g.x) \ \ \ \forall f,g\in G \ \ \ \forall x\in X$
\end{defi}\\
\textbf{Esempi:}\\
1)$(G,*)$ gruppo scelgo $X = G$ agisce per moltiplicazione sinistra\\
\begin{center}
	\begin{aligned}
		&G\times X \rightarrow X\\
		&(g,x) = g\c* x
	\end{aligned}
\end{center}
2) $ G = S_n \ \ \ X = \{1,\ldots, n\}$\\
 \begin{center}
	\begin{aligend}
		&S_n\times X \rightarrow X\\
		(\sigma, x) \rightarrow \sigma (x)
	\end{aligend}
\end{center}
3) $n,m\in \Z^+$\\
$G := GL_n(\R)\times GL_n(\R)$\\
$X = Mat_{n,m}(\R)$\\
 \begin{center}
	\begin{aligned}
		&G\times X \rightarrow X\\
		&(AB, C) \rightarrow BCA^{-1}
	\end{aligned}
\end{center}\\
4) $G = GL_n(\R) \ \ X = \R^n$\\
\begin{center}
	
 \begin{aligend}
	&G\times X \rightarrow X\\
	&(A,v) \rightarrow Av
\end{aligend}
\end{center}
5) $G = GL_n(\R) \ \ \ X = Mat_{n,m}(\R)$\\
\begin{center}
	\begin{aligned}
		&G\times X \rightarrow X\\
		&(A,C) \rightarrow ACA^{-1}
	\end{aligned}
\end{center}
6) $(G,\cdot)$ gruppo $X = G$\\
 \begin{center}
	\begin{aligned}
		&G\times X \rightarrow X\\
		& (g,x) \rightarrow g*x*g^{-1}
	\end{aligned}
\end{center}
\begin{defi}
	Data un'azione di un gruppo $G$ su un insieme $X$ si dice transitiva se
	 \[
		 \forall x,y\in X \ \exist g\in G \text{ tale che } g.x = y
	.\] 
\end{defi}
\begin{defi}
	Un'azione si dice semplicemente transitiva se 
	\[
		\forall x,y\in X \ \ \exist g\in G\text{ tale che } g.x = y
	.\] 
\end{defi}
\textbf{Esercizio:}\\
1) Dimostrare che gli esempi dati sono azioni\\
2) stabilire quali degli esempi sono semplicemente transitivi, transitivi o nessuna delle due
\begin{nota}
	Scriveremo $G\curvearrowright X$ per indicare che il gruppo  $G$ agisce sull'inseme $X$
\end{nota}
\begin{defi}
	$G\agisce X$, Dato  $x\in X$ definiamo:\\
	$\cdot$ l'orbita di x come il sottoinsieme
	\[
		O_x = \{g.x|g\in G\}\subseteq X
	.\] 
	$\codt$ lo stabilizzatore di $x$ il sottogruppo:
	\[
		Stab_x = \{g\in G| g.x = x\}\subseteq G
	.\] 
\end{defi}
\textbf{Esercizio:}\\
Dimostra che lo stabilizzatore di ogni elemento è sempre un sottogruppo (non necessariamente normale\\
\textbf{Esercizio:}\\
Sia $G$ gruppo finito ($|G| < +\infty)$ con $G\agisce X$, per ogni  $x\in X$ si ha:\\
1) $|Stab_x|<+\infty$ \ \ (banale)\\
2)  $|O_x|<+\infty$\\
3)  $|G| = |O_x||Stab_x|$\\
\textbf{Suggerimento:}\\
2) Abbiamo un'applicazione suriettiva \\
\begin{aligned}
	&G \rightarrow O_x\\
	&g \rightarrow g.x
\end{aligned}\\
3) L'idea è di dimostrare che esiste una corrispondenza biunivoca fra gli elementi dell'orbita e i laterali sinistri dello stabilizzatore, poi concludete ricordando che $[G:Stab_x] = \frac {|G|}{|Stab_x|}$ (numero di laterali sinistri)\\
 \textbf{Idea}(per la corrispondenza biunivoca)\\
 Verificare che $\forall g,f\in G$
  \begin{gather*}
  g\equiv f mod(Stab_x)\\
  \storto \Leftrightarrow\\
  g.x = f.x
 \end{gather*}
 \begin{teo}[Cauchy]
 	Sia $G$ un gruppo finito, Sia $p$ primo tale che $p \ |\ |G|$\\
	Allora esistono (almeno)  $p - 1$ elementi di ordine  $p$ in G
 \end{teo}
 \begin{dimo}
 	1) In generale se $G\agisce X$ allora  $X$ è unione disgiunta di orbite\\
	Definiamo la relazione di equivalenza $\tilde$ su $X$ come
	 \[
		 x\tilde y \Leftrightarrow \exist g\in G \text{ tale che }g.x = y
	.\]  
	Basta dimostrare che è una relazione d'equivalenza\\
	2) $X = \{(g_1,\ldots, g_n)\in G\times\ldots\times G | g\cdot\ldots\cdot g_p = e\}$\\
Vogliamo definire un'azione del gruppo ciclico $C_p =<p>$ su $ X$
 \begin{center}
	\begin{aligned}
		&C_p\times X \rightarrow X\\
		&\rho.(g_1,\ldots,g_p) \rightarrow (g_2,g_3,\ldots, g_p,g_1)
	\end{aligned}
\end{center}
Verifichiamo che l'azione sia ben definita ovvero che\\ $\rho.(g_1,\ldots,g_p)\in X \ \ \ \forall (g_1,\ldots,g_p)\in X$
\[
	g_2\cdot\ldots\cdot g_pg_1 = (g_1^{-1}g_1)(g_2\cdot\ldots\cdot g_p)g_1 = g_1^{-1}(g_1\cdot\ldots\cdot g_p)g_1 = g_1^{-1}g_1 = e
.\] 
3) Studio $| X|$ abbiamo   $|X| = |G|^{p-1}$ infatti:\\
$\forall (g_1,\ldots, g_{p-1},g_p)\in X$ dove $_p = (g_1,\ldots,g_{p-1})^{-1} \Rightarrow$ in particolare $p | |X|$ \\
4)Studiamo le orbite dell'azione $C_p\agisce X$, Sappiamo che  $|C_p| = |O_x||Stab_x| \ \forall x\in X$\\
Quindi  $|O_x| = 1 \ \ \vee\ \  |O_x| = p$\\
5) Dato che $X$ è unione disgiunta di orbite e $p | |X|$\\
Allora il numero di orbite formate da  $(x)$ unico elemento è un multiplo di  $p$\\
6) Studio tali orbite\\
L'orbita  $O_{(g_1,\ldots,g_p)}$ è formata da un singolo elemento se e solo se\\ $g_1 = g_2=\ldots=g_p$
 \end{dimo}
 Dunque abbiamo una corrispondenza biunivoca 
 \[
	 \{O_x : \ |O_x| = 1\} \ \leftrightarrow \  \{g\in G | g^p = e\}
 .\] 
 Quindi $p$ divide $|\{g\in G|g^p = e\}| $\\
 d'ora in poi  $ A = \{g\in G | g^p = e\}$\\
  $7) A \neq\emptyset$ poiché  $e\in A$\\
 \[
	 A = \{e\}\cup \{g\in G | ord(g) = p\}
.\] 
Quindi modulo $(p)$ abbiamo
\[
	0\equiv_p 1 + |\{g\in G | ord(g) = 1\}|
.\] 
Quindi l'insieme di elementi di ordine $p$ in  $G$ è non uvoto e 
\[
	|\{g\inG|ord(g)=p\} \equiv_p p - 1
.\] 
Deduciamo 
\[
	|\{g\in G | ord(g) = p\} = kp-1\geq p-1\\
.\] 
con $k\in Z^+$
\section{Torniamo alle schede}
\begin{cases}
	3x\equiv_{15} 6\\
	21x\equiv_{49} 13
\end{cases}
La prima congruenza è equivalente a $x\equiv_5 2$\\
$MCD(21,49) = 7$\\
La seconda congruenza significa\\
 \[
 21x = 13 + 49k \ \ k\in \Z
 .\] 
 \begin{gather*}
 	21x - 49k = 13\\
	7(3x-7k) = 13\ \
 \end{gather*}
 \textbf{Osservazione:}\\
 Se $MCD(a,n) \not | b$\\
 allora 
  $ax\equiv_n b$ non ammette soluzioni\\
  Infatti:  $d = MCD(a,n)$ \\ con  $d\not | b$ allora\\
  con $d$ divide il membro di sinistra ma non quello di destra\\
  \textbf{Esercizio}\\
  $G$ gruppo $g\in G$ \ \  $ord(g) = n$\\
  Allora,  $g^h = g ^k$ se e solo se $h\equiv_n k$\\
   \textbf{Soluzione}\\
   Assumiamo che $g^h=g^k$ Divisione Euclidea\\
   $h-k=qn + r$ con $0\leq r< n$
    \begin{gather*}
	    g^h = g^k \Rightarrow g^{h-k} = e\\
	    \Rightarrow g^{qn+r}=e\\
	    \Rightarrow (g^n)^qg^r = e\\
	    \Rightarrow g^r = e
   	
   \end{gather*}
   Assurdo se $0<r<n$ \  $r = 0$\\
   $h-k=qn \Rightarrow h\equiv_n k$\\
   \textbf{Esercizio}\\
   per quali $n,m\in\Z$ si ha  $2^n + 2^m$ divisibile per $9$
    \textbf{Soluzione}\\
    Studio
    \begin{gather*}
    	2^n + 2^m\equiv_9 0\\
	\storto \Leftrightarrow
	2^n\equiv_9 -2^m\\
	\storto \Leftrightarrow\\
	2^{n-m} \equiv_9 -1\\
	\storto \Leftrightarrow\\
	2^{n-m} \equiv_9 8\\
	\storto \Leftrightarrow\\
	2^{n-m}\equiv_{9} 2^3
    \end{gather*}
    Sfruttiamo l'esercizio precedente con $G = U_9$ \\
    La congruenza è verificata se e solo se
    \[
	    n - m \equiv 3 \ mod(ord_{U_9}([2]))
    .\] 
    \begin{gather*}
    	2\\
	2^2 = 4\\
	2^3 = -1\\
	2^4 = -2\\
	2^5 = -4\\
	2^6 = 1
    \end{gather*}
    quindi $ord([2]) = 6$\\
    Soluzione: $n-m \equiv_6 3$
\end{document}
