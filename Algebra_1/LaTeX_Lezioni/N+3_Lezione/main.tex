\documentclass[12px]{article}

\title{Lezione N+3 Algebra I}
\date{2025-05-22}
\author{Federico De Sisti}

\usepackage{amsmath}
\usepackage{amsthm}
\usepackage{mdframed}
\usepackage{amssymb}
\usepackage{nicematrix}
\usepackage{amsfonts}
\usepackage{tcolorbox}
\tcbuselibrary{theorems}
\usepackage{xcolor}
\usepackage{cancel}

\newtheoremstyle{break}
  {1px}{1px}%
  {\itshape}{}%
  {\bfseries}{}%
  {\newline}{}%
\theoremstyle{break}
\newtheorem{theo}{Teorema}
\theoremstyle{break}
\newtheorem{lemma}{Lemma}
\theoremstyle{break}
\newtheorem{defin}{Definizione}
\theoremstyle{break}
\newtheorem{propo}{Proposizione}
\theoremstyle{break}
\newtheorem*{dimo}{Dimostrazione}
\theoremstyle{break}
\newtheorem*{es}{Esempio}

\newenvironment{dimo}
  {\begin{dimostrazione}}
  {\hfill\square\end{dimostrazione}}

\newenvironment{teo}
{\begin{mdframed}[linecolor=red, backgroundcolor=red!10]\begin{theo}}
  {\end{theo}\end{mdframed}}

\newenvironment{nome}
{\begin{mdframed}[linecolor=green, backgroundcolor=green!10]\begin{nomen}}
  {\end{nomen}\end{mdframed}}

\newenvironment{prop}
{\begin{mdframed}[linecolor=red, backgroundcolor=red!10]\begin{propo}}
  {\end{propo}\end{mdframed}}

\newenvironment{defi}
{\begin{mdframed}[linecolor=orange, backgroundcolor=orange!10]\begin{defin}}
  {\end{defin}\end{mdframed}}

\newenvironment{lemm}
{\begin{mdframed}[linecolor=red, backgroundcolor=red!10]\begin{lemma}}
  {\end{lemma}\end{mdframed}}

\newcommand{\icol}[1]{% inline column vector
  \left(\begin{smallmatrix}#1\end{smallmatrix}\right)%
}

\newcommand{\irow}[1]{% inline row vector
  \begin{smallmatrix}(#1)\end{smallmatrix}%
}

\newcommand{\matrice}[1]{% inline column vector
  \begin{pmatrix}#1\end{pmatrix}%
}

\newcommand{\C}{\mathbb{C}}
\newcommand{\K}{\mathbb{K}}
\newcommand{\R}{\mathbb{R}}


\begin{document}
	\maketitle
	\newpage
	\subsection{Ordine lessicografico sui monomi}
	$\F$ campo\\
	Definiamo l'ordinamento lessicografico sull'insieme dei monomi monici in $\F[x_1,\ldots,x_n]$\\
	Diremo che \\
	$x^{k_1}_1\cdot\ldots \cdot x_n^{k_n}\ \ \ k_1,\ldots,k_n\in\Z_{\geq 0}$\\
	è maggiore di $x_1^{h_1}\cdot\ldots\cdot x_n^{h_n}$ \ $h_1,\ldots,h_n\in\Z_{\geq 0}$\\
	se esiste $s\in \{1,\ldots, n-1\}$ tale che:
	 \begin{enumerate}
		 \item $k_j = h_j\ \ \forall j\in \{1,\ldots,s\}$
		 \item  $k_{s+1} > h_{s + 1}$
	\end{enumerate}\\
	\textbf{Esempio}\\
$n = 2$
 \begin{itemize}
	 \item $x_1^2 > x_1$
	 \item $x_1 > x_2$
	 \item $x_1 > x_2^{69}$
\end{itemize}
\subsection{Polinomi simmetrici}
\begin{defi}
	$\F$ campo\\
	\begin{enumerate}
		\item $p\in \F[x_1,\ldots, x_n]$ si dice polinomio simmetrico\\
			$s\in p(x_1,\ldots,x_n) = p(x_{\tau(1)},\ldots,x_{\tau(n)})$ \\
			per ogni $\tau\in S_n$
		\item I polinomi simmetrici elementari sono 
			 \[
				 \varepsilon_s(x_1,\ldots,x_n) = \sum_{1 \leq j_1 < \ldots< j_s \leq n} x_{j_1}\cdot\ldots\cdot x_{j_s}
			 .\]  con $s\in \{1,\ldots,n\}$ 
		 \item L'insieme dei polinomi simmetrici si denota $\F[x_1,\ldot,x_n]^{S_n}$
	\end{enumerate}
\end{defi}
\textbf{Obiettivo}\\
Ogni polinomio simmetrico si scrive in modo unico come polinomio nei polinomi simmetrici elementari
\begin{teo}
	$\F[x_1,\ldots,x_n]^{S_n}\cong F[\varepsilon_1, \ldots, \varepsilon_n]$
\end{teo}
\begin{dimo}
	Sia $p\in \F[x_1,\ldots, x_n]^{S_n}$ \\
	Sia $\alpha x_1^{k_1}\cdot\ldots\cdot x_n^{k_n}$ il monomio massimo fra quelli che appaiono in $p$.
	 \begin{itemize}
		 \item dimostriamo che $k_1\geq k_2\geq \ldots\geq k_n$\\
			 Se per assurdo $k_1 < k_2$ allora potremmo applicare $\tau  = ( 12)\in S_n$per ottenere il monomio
			  \[
				  \alpha x_2^{k_1}\cdot x_1^{k_2}\cdot x_3^{k_3}\cdot\ldots\cdot x_n^{k_n} \ \ \text{ in } p
			 .\] 
			 Ma questo monomio è 
			 \[
				 \alpha\cdot x_1^{k_2}\cdot x_2^{k_1}\cdot x_3^{k_3}\cdot \ldots\cdot x_n^{k_n} > \alpha x_1^{k_1}\cdot x_2^{k_2}\cdot\ldots\cdot x_n^{k_n}
			 .\] 
			 da cui l'assurdo per l'ipotesi di massimalità
		 \item $\phi_1(x_1,\ldots,x_n): = \e_1^{k_1-k_2}\cdot\ldots\cdot\e_s^{k_s - k_{s+1}}\cdot\ldots\cdot \e^{k_n}_n$ \\
			 $ \Rightarrow \phi_1\in\F[x_1,\ldots, x_n]^{S_n}$ \\
			 Il monomio massimo di $\phi_1$ è 
			 \[
				 (x_1)^{k_1-k_2}\cdot (x_1x_2)^{k_2-k_3}\cdot\ldots\cdot(x_1\ldots x_n)^{k_n}
			 .\] 
			 ovvero
			 \[
				 x^{k_1}_1\cdot x_2^{k_2}\cdot\ldots\cdot x_n^{k_n}
			 .\] 
		 \item $p - \alpha\cdot \phi_1\in \F[x_1,\ldots,x_n]^{S_n}$\\
			 ha monomio massimo minore rispetto a $p$
		 \item Iterando il procedimento un numero finito di volte avremo
			  \[
				  p = \alpha_1\phi_1 + \ldots + \alpha_r\phi_r\in \F[\e_1,\ldots,\e_n]
			 .\] 
	\end{itemize}
	\textbf{Unicità}\\
	"basta" verificare che se $\exists g\in \F[z_1,\ldots, z_n]$ tale che
	\[
	g(\e_1,\ldots,\e_n) = 0
	.\] 
	allora $g(z_1,\ldots,z_n)= 0$\\
Dimostriamo che se 
\[
g(z_1,\ldots,z_n)\neq 0 
.\] 
allora
 \[
g(\e_1,\ldots, \e_n)\neq 0
.\] 
Dato $g\in \F[z_1,\ldots,x_n]\setminus\{0\}$\\
consideriamo il monomio massimo in $g$
 \[
	 \beta \cdot z_1^{h_1}\cdot\ldots\cdot x_n^{h_n}
.\] 
Allora il monomio massimo in $g(\e_1,\ldots,\e_n)$ rispetto\\
alle variabili $x_1,\ldots,x_n$\\
\[
	\beta\cdot x_1^{h_1}\cdot (x_1x_2)^{h_2}\cdot\ldots\cdot(x_1\ldots x_n)^{h_n}
.\] 
ovvero 
\[
	\beta \cdot x^{h_1 +\ldots +  h_n}\cdot x_2^{h_2 + \ldots + h_n}\cdot\ldots\cdot x_n^{h_n}
.\]  
In particolare
\[
g(\e_1,\ldots,\e_n) \neq 0
.\] 
\end{dimo}
\subsection{Estensioni radicali e gruppi risolubili}
\begin{defi}[Estesione radicale]
	$\F$ campo con $char(F) = 0$, Un'estensione  $\F\subseteq\K$ si dice radicale se\\
	$\exists m,n_1,\ldots,n_m\in\Z_{\geq 1}$ e $\exists \alpha_1,\ldots,\alpha_m\in\K$ tali che
	\begin{enumerate}
		\item $\F\subseteq\F(\alpha_1,\ldots,\alpha_m) = \K$
		\item $\alpha_1^{n_1}\in \F$ 
		\item $\alpha_j^{n_j}\in\F(\alpha_1,\ldots,\alpha_{j-1}) \ \ \forell j\in\{2,\ldots,m\}$
	\end{enumerate}
\end{defi}
\begin{defi}
	$f\in\F[x]$ risolubile per radicali se un suo campo di spezzamento  $\F\subseteq\K$ esiste un'estensione  $\K\subseteq\Le$ tale che $\F\subseteq\Le$ è un'estensione radicale.
\end{defi}
\begin{prop}
	Sia $\F$ campo, $char(\F) = 0$ $\F\subseteq\K$ estensione radicale, allora esiste un'estensione  $\K\subseteq\Le$ tale che la composizione $\F\subseteq \Le$ sia Galoisiana radicale (di grado finito)
\end{prop}
\begin{dimo}
	Sia $\alpha_1,\ldots,\alpha_m$ successione radicale per $\F\subseteq\K$ Siano  $p_j\in\F[x]$ polinomi minimi di  $d_j$ per  $j\in\{1,\ldots,m\}$\\
	$f = p_1\cdot\ldots\cdot p_m\in \F[x]$\\
	Sia $\F\subseteq \Le$ campo di spezzamento di  $f \Rightarrow  \F\subseteq\Le$ Galoisiano di grado finito
	\begin{itemize}
		\item Dimostriamo che $\F\subseteq\Le$  è la composizione dell'estensione  $\F\subseteq\K$ con un'estensione  $\K\subseteq\Le$\\
			Siano  $\beta_1,\ldots\beta_m\in\Le$ radici (qualsiasi) di $p_1,\ldots p_m$ rispettivamente\\
			INSERISCI IMMAGINE 5 31 22 maggio\\
			Quindi abbiamo l'estensione $\K \hookrightarrow \Le$ che rende commutativo il diagramma\\
			INSERISCI IMMAGINE 5 33 22 maggio
		\item Resta da verificare che $\F\subseteq\Le$ sia radicale.\\
			Abbiamo dimostrato che $\beta_1,\ldots,\beta_m$ è una successione radicale per $\F\subseteq\F(\beta_1,\ldots,\beta_n)$\\
			Siano $\{\beta_{j,r}\}_{k\in\{1,\ldots,deg(p_j)\}}$ le radici di $p_j$  in  $\Le$\\
			$ \Rightarrow  F\subseteq F(\beta_{1,1},\ldots,\beta_{1,\deg(p_1)},\beta_{2,1},\ldots,\beta_{2,\deg(p_1)},\ldots ,\dlots,  \beta_{n,1},\ldots,\beta_{n,\deg(p_1)} $
	\end{itemize}
\end{dimo}
\begin{lemm}
	$\F$ campo, $char(\F) = 0 , a\in\F\setminus\{0\}$ supponiam oche  $\F$ contega tutte le radiii $n-$ e
\end{lemm}
PARTE CHE MANCA RECUPERALA DA LEONARDO
\begin{prop}
	$\F$ campo, $char(\F) = 0$   $\F\subseteq\K$ estensione Galoisiana e radicale. Allora  $G(\K,\F)$ è risolubile
\end{prop}
\begin{dimo}
	$\alpha_1, \ldots, \alpha_k$ successione radicale di esponenti $n_1,\ldots n_m$ per $\F\subseteq\K$\\
	$n = mcm(n_1,\ldots,n_m)\in \Z_{\geq 1}$ Abbiamo $\alpha_j^n\in\F(\alpha_1,\ldots,\alpha_{j-1})$ \\
	Sia $\omega\in\A$ radice  $n$-esima primitiva dell'unità .
	\[
		\F = \Le_0\subseteq \Le_1 = \Le_0(\omega)\subseteq \Le_2 = \Le_1(\alpha_1\subseteq\ldots\subseteq\Le_j = \Le_{j-1}(\alpha_j)\subseteq\ldots\subseteq\Le = \K(\omega)
	.\] 
	dimostriamo che $G(\Le, \F)$ è risolubile.\\
	 $\F\subseteq\Le$ è un'estensione Galoisiana perché composizione 
	  \[
	 \F\subseteq\K\subseteq\K(\omega) = \Le
	 .\] 
	 la prima giustificata perché Galoisiana per ipotesi e la seconda per cambio di sp. di $x^n - 1 \in \K[x]$ \\
	 Quindi per il teorema di Galois abbiamo catena di sottogruppi
	 \[
		 G(\Le,\F) =  G(\Le, \Le_0)\geq G(\Le,\Le_1)\geq\ldots\geq G(\Le,\Le) = \{id\}
	 .\] 
Verifichiamo che
\begin{enumerate}
	\item $G(\Le,\Le_j)\trianglerighteq G(\Le, \Le_{j+1})$ infatti l'estensione $\Le_j\subseteq\Le_{j+1}$ è normale, perché è campo di spezzamento del polinomio  $x^n -\alpha_{j+1}^n\in\Le_j[x]$
	\item  $\displaystyle\frac{G(\Le, \Le_j)}{G(\Le,\Le_{j+1})}\cong G(\Le_{j+1},\Le_j)$ che è abeliano (lemma)
\end{enumerate}\\
Quindi $G(\Le,\F)$ è risolubile
\begin{itemize}
	\item Resta da verificare che $G(\Le,\F)$ è risolubile\\
		Abbiamo  $\F\subseteq\K\subseteq\K(\omega) = \Le$ \\
		dove $\F\subseteq\K$ normale per ipotesi Dal teorema di Galois:
		 \[
			 \frac{G(\Le,\F)}{G(\Le,\K)}\cong G(\K,\F)
		.\] 
		quoziente di un gruppo risolubile $ \Rightarrow $ risolubile
\end{itemize}
\end{dimo}
\begin{teo}
	$\F$ campo,  $char(\F) =0 $  $f\in \F[x]$ risolubile per radicali.\\
	Allora il suo campo di spezzamento  $\F\subseteq\K$ ha gruppo  di Galois $G(\K,\F)$ risolubile.
\end{teo}
\begin{dimo}
	Sappiamo che esiste estensione $\K\subseteq\Le$  tale che $\F\subseteq\Le$ sia radicale e Galoisiana\\
	Dalla proposizione segue che $G(\Le,\F)$ risolubile\\
	Ora, $\F\subseteq\K$ normale (poiché campo di spezzamento) quindi per teorema di Galois
	 \[
		 G(\K,\F)\cong \frac{G(\Le,\F)}{G(\Le,\K)}
	.\] 
	che è risolubile poiché quoziente di risolubile.
\end{dimo}
\subsection{Teorema di Abel-Ruffini}
\begin{defi}
	$\F = \Q(a_1,\ldots,a_n)$ dove $a_1,\ldots,a_n$ variabili trascendenti.\\
	\[
		f(x) = x^n + a_1x^{n-1} + \ldots + a_{n-1}a + a_n\in\F[x]
	.\] 
	si dice polinomio generico di grado $n$.
\end{defi}
\begin{teo}[Abel-Ruffini (1799) - Galois (1846)]
	Il polinomio generico di grado $n\geq 5$ non è risolubile per radicali 
\end{teo}
\begin{dimo}
	Dato $\F\subseteq\K$ campo di spezzamento di  $f$, dimostriamo che
	\[
	G(\K,\F)\cong S_n
	.\] 
	e dunque non risolubile per $n\geq 5$ 
	\begin{itemize}
		\item $\alpha_1,\ldots,\alpha_n\in\K$ radici di $f\in \F[x]$\\
			$\displaystyle f(x) = \prod_{j=1}^n(x-\alpha_j) = x^n + \sum^{n}_{s=1}(-1)^n \left(\sum^{n}_{1\leq j_1<\ldots< j_s\leq n}a_{j_1}\cdot\ldots\cdot a_{j_s} \right)x^{n-s}$\\
			quindi $a_s\in\Q(\alpha_1,\ldots,\alpha_n)$ \ \ $\dorall s\in\{1,\ldots,n\}$\\
			Quindi il campo di spezzamento di $f$ è\\
			 \[
			\F = \Q(\alpha_1,\ldots,\alpha_n)\subseteq\K = \Q(\alpha_1,\ldots,\alpha_n)
			.\] 
		\item Esiste un omomorfismo di gruppi (iniettivo!)
			\[
				\begin{aligned}
					i: S_n &\rightarrow G(\K,\F)\\
					\tau & \rightarrow i_\tau
				\end{aligned}
			.\] 
			dove $i_\tau(\alpha_j) = \alpha_{\tau(j)}$\\
			$H := im(i)\leq G(\K,\F)$ con  $H\cong S_n$
		\item Dal teorema di Galois
		\[
		S_n\cong H = G(\K,\K_H) = G(\K,\F)
		.\] 
		dove l'ultima uguaglianza segue dal fatto che $\Q[x_1,\ldots,x_n]^{S_n} = \Q[\e_1,\ldots,\e_n]$ 
	\end{itemize}
\end{dimo}
\end{document}
