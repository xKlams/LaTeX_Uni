\documentclass[12px]{article}

\title{Lezione 4 Algebra I}
\date{2025-03-10}
\author{Federico De Sisti}

\usepackage{amsmath}
\usepackage{amsthm}
\usepackage{mdframed}
\usepackage{amssymb}
\usepackage{nicematrix}
\usepackage{amsfonts}
\usepackage{tcolorbox}
\tcbuselibrary{theorems}
\usepackage{xcolor}
\usepackage{cancel}

\newtheoremstyle{break}
  {1px}{1px}%
  {\itshape}{}%
  {\bfseries}{}%
  {\newline}{}%
\theoremstyle{break}
\newtheorem{theo}{Teorema}
\theoremstyle{break}
\newtheorem{lemma}{Lemma}
\theoremstyle{break}
\newtheorem{defin}{Definizione}
\theoremstyle{break}
\newtheorem{propo}{Proposizione}
\theoremstyle{break}
\newtheorem*{dimo}{Dimostrazione}
\theoremstyle{break}
\newtheorem*{es}{Esempio}

\newenvironment{dimo}
  {\begin{dimostrazione}}
  {\hfill\square\end{dimostrazione}}

\newenvironment{teo}
{\begin{mdframed}[linecolor=red, backgroundcolor=red!10]\begin{theo}}
  {\end{theo}\end{mdframed}}

\newenvironment{nome}
{\begin{mdframed}[linecolor=green, backgroundcolor=green!10]\begin{nomen}}
  {\end{nomen}\end{mdframed}}

\newenvironment{prop}
{\begin{mdframed}[linecolor=red, backgroundcolor=red!10]\begin{propo}}
  {\end{propo}\end{mdframed}}

\newenvironment{defi}
{\begin{mdframed}[linecolor=orange, backgroundcolor=orange!10]\begin{defin}}
  {\end{defin}\end{mdframed}}

\newenvironment{lemm}
{\begin{mdframed}[linecolor=red, backgroundcolor=red!10]\begin{lemma}}
  {\end{lemma}\end{mdframed}}

\newcommand{\icol}[1]{% inline column vector
  \left(\begin{smallmatrix}#1\end{smallmatrix}\right)%
}

\newcommand{\irow}[1]{% inline row vector
  \begin{smallmatrix}(#1)\end{smallmatrix}%
}

\newcommand{\matrice}[1]{% inline column vector
  \begin{pmatrix}#1\end{pmatrix}%
}

\newcommand{\C}{\mathbb{C}}
\newcommand{\K}{\mathbb{K}}
\newcommand{\R}{\mathbb{R}}


\begin{document}
	\maketitle
	\newpage
	\subsection{Seconda parte della lezione}
	\textbf{Domanda:}\\
	Cosa cambia in $\K[2]$ quando  $\K$ è un campo?\\
	$u_1 = \K$ \\
	Chi è $u_2 ?$\\
	$p\in u_2$ se e solo se
	\[
		\K \rightarrow \K[x]/(p) \ \text{ è suriettiva}
	.\] 
	se e solo se $deg(p) = 1 \vee deg(p) = 0$\\
	In generale\\
	$\forall i\geq 1$ \ \  $u_{i+1}\setminus u_i$ è l'insieme dei polinomi di grado $i$
	\textbf{Attenzione} $\K[x,y]$ non è domino euclideo.\\
	$u_1 = \K$\\
	 $u_2 = ?$\\
	 \begin{defi}
	 	$R$ anello commutativo, Dati $r_1,\ldots, r_k\in R$ chiamiamo
		\[
			(r_1,\ldots,r_k) = \{ \sum^{k}_{i=1}a_ir_i \ | \ k\in Z_{\geq 1}\ \ a_i\in R\}
		.\] 
Ideale generato da $r_1,\ldots, r_k$ in $R$\\
	 \end{defi}
\textbf{Osservazione}\\
$(r_1,\ldots, r_k)$ è il più piccolo ideale di $R$ contenente $r_1,\ldots,r_k$
\begin{defi}[Ideale principale]
	$R$ anello commutativo $I\subseteq R$ ideale, si dice principale se $\exists \ r\in R$ tale che $I = (r)$
\end{defi}
\begin{defi}
	$R$ anello commutativo.
	\begin{itemize}
		\item $R$ si dice Anello a ideali principali se tutti i suoi ideali sono principali.
		\item  $R$ si dice dominio a ideali principali se è un dominio d'integrità e un anello a ideali principali.
	\end{itemize}
\end{defi}
\textbf{Esempio}\\
$R = (\Z,+,\cdot)$ è un dominio a ideali principali.\\
\textbf{Esercizio}\\
Trovare un anello a ideali principali che non sia un dominio\\
$n\in\Z$,  $n$ composto\\
$ \Rightarrow \Z/(n)$ è un anello a ideali principali che non è un dominio
\begin{prop}
	$\K$ campo. $R = \K[x]$ è un dominio a ideali principali
\end{prop}
\begin{dimo}
	$\K[x]$ è dominio d'integrità poiché $\K$ lo è.\\
	Sia $I\subseteq R[x]$ ideale ,  $I\neq \{0\}$\\
	Sia $f\in I\setminus\{0\}$ di grado minimo in  $I$\\
	Vogliamo dimostrare che  $I = (f)$
	 \begin{itemize}
		 \item $(f) \subseteq I$, infatti se $f\in I$ allora $q\cdot f\in I \ \ \forall q\in \K[x]$
		 \item  $I\subseteq (f),$ infatti $g\in I$ usiamo la divisione per  $f$\\
			  $ \Rightarrow g = q\cdot f + r$ con $deg(r) < deg(f) \Rightarrow r = g - q\cdot f \in I$ \\
			  $ \Rightarrow r =0 \Rightarrow g = q\cdot f\in (f)$
	\end{itemize}

\end{dimo}
\textbf{Esercizio}\\
Dimostrare che se
\begin{itemize}
	\item $R$ dominio d'integrità
	\item $R[x]$ dominio a ideali principali
\end{itemize}
Allora $R$ è un campo\\
\textbf{Soluzione}\\
Dobbiamo verificare che dato $a\in R\setminus\{0\}$ esiste l'inverso moltiplicativo.\\
Consideriamo l'ideale $(a,x)\subseteq R[x]$\\
$R[x]$ a ideali principali $ \Rightarrow \exists p\in R[x]$ tale che $(p) = (a,x)$\\
Quindi:
\begin{gather*}
	\Rightarrow a = q_1\cdot p\\
	 \Rightarrow x = q_2\cdot p \ \rightarrow \ ax = \tilde q_2\cdot p
\end{gather*}
Deduciamo che $q_1$ e $p$ sono entrambi costanti.\\
Infatti il termine di grado più alto del prodotto $q_1\cdot p$ è il prodotto dei termini direttivi di $p$ e di $q_1$ (Stiamo usando il fatto che $R$ sia dominio d'integrità)\\
Se $p$ costante\\
\[
	\Rightarrow q_2 = hx \text{ con } h\cdot p = 1
\] 
$p$ invertibile $ \Rightarrow  (p) = R[x]$ \\
$ 1\in (a,x) \Rightarrow $ esistono $s,t\in R[x]:$
 \[
1 = a\cdot s + t\cdot x \Rightarrow s = \sum^{}_{i\geq 0}s_ix^i \Rightarrow a s_0 = 1
.\] 
\textbf{Esercizio/Proposizione}\\
$R$ dominio a ideali principali. $I$ ideale, Se $I$ è primo, allora $I$ è massimale.\\
\textbf{Soluzione}\\
$ I = (p)\subseteq R$\\
 $I$ primo. Supponiamo che esista un ideale $J = (q)\subseteq R$ tale che  $I\subseteq J$\\
  $I\subseteq J \Rightarrow (p)\subseteq (q) \Rightarrow  p = a\cdot q$ per qualche $a\in R$\\
   $I$ primo $ \Rightarrow a\in I$ oppure $q\in I$
   \begin{center}
   	
    \begin{aligned}
	    $q\in I &\Rightarrow q\in (p) \\
		    & \Rightarrow  (q)\subseteq(p)\\
		    & \Rightarrow 	J = I\\$
   \end{aligned}
   \end{center}
   \begin{center}
   	\begin{aligend}
		$a\in I &\Rightarrow a\in (p)\\
			& \Rightarrow a = k\cdot p \text { per qualche } k\in R\\
			& \Rightarrow p = a\cdot q = p\cdot k\cdot q\\
			& \Rightarrow p\cdot (1-k\cdot q)= 0\\
			& \Rightarrow 1 + k\cdot q = 0 \Rightarrow q$ invertibile\\
			&$ J = R$
   	\end{aligend}
   \end{center}
   \begin{coro}
   	$R$ dominio a ideali principali (PID) allora un ideale è primo se e solo se è massimale
   \end{coro}
   \begin{dimo}
   	Resta da verificare che $I$ massimale $ \Rightarrow I$ primo\\
	$I$ massimale $ \Rightarrow R/I$ campo $ \Rightarrow R/I$ dominio integrità $ \Rightarrow I$ primo 
   \end{dimo}
\end{document}
