\documentclass[12px]{article}

\title{Lezione N + 2 Algebra I}
\date{2025-05-19}
\author{Federico De Sisti}

\usepackage{amsmath}
\usepackage{amsthm}
\usepackage{mdframed}
\usepackage{amssymb}
\usepackage{nicematrix}
\usepackage{amsfonts}
\usepackage{tcolorbox}
\tcbuselibrary{theorems}
\usepackage{xcolor}
\usepackage{cancel}

\newtheoremstyle{break}
  {1px}{1px}%
  {\itshape}{}%
  {\bfseries}{}%
  {\newline}{}%
\theoremstyle{break}
\newtheorem{theo}{Teorema}
\theoremstyle{break}
\newtheorem{lemma}{Lemma}
\theoremstyle{break}
\newtheorem{defin}{Definizione}
\theoremstyle{break}
\newtheorem{propo}{Proposizione}
\theoremstyle{break}
\newtheorem*{dimo}{Dimostrazione}
\theoremstyle{break}
\newtheorem*{es}{Esempio}

\newenvironment{dimo}
  {\begin{dimostrazione}}
  {\hfill\square\end{dimostrazione}}

\newenvironment{teo}
{\begin{mdframed}[linecolor=red, backgroundcolor=red!10]\begin{theo}}
  {\end{theo}\end{mdframed}}

\newenvironment{nome}
{\begin{mdframed}[linecolor=green, backgroundcolor=green!10]\begin{nomen}}
  {\end{nomen}\end{mdframed}}

\newenvironment{prop}
{\begin{mdframed}[linecolor=red, backgroundcolor=red!10]\begin{propo}}
  {\end{propo}\end{mdframed}}

\newenvironment{defi}
{\begin{mdframed}[linecolor=orange, backgroundcolor=orange!10]\begin{defin}}
  {\end{defin}\end{mdframed}}

\newenvironment{lemm}
{\begin{mdframed}[linecolor=red, backgroundcolor=red!10]\begin{lemma}}
  {\end{lemma}\end{mdframed}}

\newcommand{\icol}[1]{% inline column vector
  \left(\begin{smallmatrix}#1\end{smallmatrix}\right)%
}

\newcommand{\irow}[1]{% inline row vector
  \begin{smallmatrix}(#1)\end{smallmatrix}%
}

\newcommand{\matrice}[1]{% inline column vector
  \begin{pmatrix}#1\end{pmatrix}%
}

\newcommand{\C}{\mathbb{C}}
\newcommand{\K}{\mathbb{K}}
\newcommand{\R}{\mathbb{R}}


\begin{document}
	\maketitle
	\newpage
	\subsection{Applicazioni dell'altra volta}
	\[
	\begin{aligned}
		\psi:\mathcal F_{\K,\F} &\rightarrow \mathcal G_{\K\F}\\
					(\F\subseteq\Le\subseteq\K) &\rightarrow G(\K,\Le)
	\end{aligned}
	.\] 
	\[
	\begin{aligned}
		 \mathcal F_{\K,\F} &\leftarrow \mathcal G_{\K,\F}: \Phi\\
		(\F\subseteq\K_H\subseteq\K) &\leftarrow (\G\leq G(\K,\F))
	\end{aligned}
	.\] 
	\subsection{Il teorema di Galois}
	\textbf{Esercizio:}\\
	$\F\subseteq \K$ un campo di spezzamento di  $f\in\F[x]$.\\
	Siano  $\alpha,\beta\in\K$ radici di un polinomio irriducibile $p\in\F[x]$.\\
	Allora  $\exists\sigma\in G(\K,\F)$, tale che  $\sigma(\alpha)=\beta$
	 \textbf{Soluzione}\\
	 Sappiamo che $\exists \psi: \F(\alpha) \rightarrow \F(\beta)$ isomorfismo di anelli tale che:\\
	 \[
	 \psi|_\F=Id \ \ e \  \ \psi(\alpha) = \beta
	 .\] 
	 AGGIUNGI FOTO 19 MARZO 1 20 \\
	 Ora:
	 \begin{itemize}
		 \item $\F(\alpha)\subseteq\K$ è campo di spezzamento di $f\in\F(\alpha)[x]$
		 \item  $\F(\alpha)\substack{\cong\\\phi}\F(\beta)\subseteq\K$ campo di spezzamento di  $f\in \F(\alpha)[x]$
	 \end{itemize}
	 Quindi dall'unicità (non canonica) dei campi di spezzamento esiste l'isomorfismo $\sigma$ cercato.\\
	 \hline\ \\
	 \textbf{Esercizio:}\\
	 $\F\subseteq\K$ estensione Galoisiana di grado finito. Allora 
	  \[
		  |G(\K,\F)| = [\K:\F]
	 .\] 
	 \textbf{Soluzione}\\
	 $|G(\K,|F)| = [\K:\K_{G(\K,\F)}] = [\K:\F]\ \ \hfill \square$\\
	 \hline\ \\
	 \textbf{Esercizio:}\\
	 $\F\subseteq\K$ estensione normale e di grado finito. \\
	 Data un'estensione $\F\subseteq\Le\subseteq\K$
	  \begin{enumerate}
		  \item dimostrare che da $\Le\subseteq\K$ è estensione normale
		  \item esibire un esempio in cui $\F\subseteq\Le$ \underline{non} è normale
	  \end{enumerate}
	 \textbf{Soluzione:}
 \begin{enumerate}
	 \item $\F\subseteq\K$ è campo di spezzamento per  $f\in F[x]$.\\
		 Allora  $\Le\subseteq\K$ è campo di spezzamento di  $f\in \Le[x]$ 
	 \item $\Q\subseteq\Q(\sqrt[3]{2}, \omega)$ dove  $\omega = \frac {-1 + i\sqrt 3}{2}\in\A$ tale estensione è campo di spezzamento di $x^3-2\in\Q[x]$  \\
		 Scegliamo $\Le = \Q(\sqrt[3]2)$ $\Q\subseteq\Le$ non è normale, infatti, $\sqrt[3]2, \sqrt[3]2\cdot\omega$ sono coniugati su  $\Q$, eppure  $\sqrt[3]2\cdot\omega\not\in\Q(\sqrt[3]2)$
\end{enumerate}
\begin{lemm}
	$\F\subseteq\K$ estensione Galoisiana di grado finito.  $\F\subseteq\Le\subseteq\K$ estensione intermedia.\\
	Allora sono equivalenti due condizioni:
	 \begin{enumerate}
		 \item $\F\subseteq\Le$ estensione normale
		 \item  $\sigma|_\Le\in G(\Le, \F)$ \ \ $\forall \sigma\in G(\K,\F)$\\
			 $($ovvero  $\sigma(l)\subseteq\Le\ \ \forall\sigma \in G(\K,\F))$
	 \end{enumerate}
\end{lemm}
\begin{dimo}
	$1) \Rightarrow 2)$ \\
	Siano $\sigma\in F(\K,\F), \ \  l\in\Le$\\
	Allora  $l,\sigma(l)$ sono coniugati $ \Rightarrow  \sigma(l)\in \Le$ poiché $\F\subseteq\Le$ è normale\\
	 $ 2) \Rightarrow 1)$ siano $k\in\K,l\in\Le$ due elementi coniugati su  $\F$\\
	  $ \Rightarrow  \exists \sigma\in G(\K,\F)$ tale che $k=\sigma(l)\in\Le \hfill$ (esercizio 1)\\
	  Quindi $\F\subseteq\Le$ normale
\end{dimo}
\begin{teo}[Galois, 1846]
	$\F\subseteq\K$ estensione Galoisiana di grado finito. $\F\subseteq\Le\subseteq\K$ estensione intermedia. Allora:
	 \begin{enumerate}
		 \item $\psi$ e  $\Phi$ sono una l'inversa dell'altra
		 \item  $|G(\K,\Le)| = [\K:\Le]$
		 \item  $[G(\K,\F):H] = [\K_H:\F]$
		 \item $G(\K,\Le)\leq G(\K,\F)$ è normale se e solo se  $\F\subseteq\Le$ è normale
		 \item Se  $\F\subseteq\Le$ è normale allora
			  \[
				  \frac{G(\K,\F)}{G(\K,\Le)}\cong G(\Le,\F)
			 .\] 
	\end{enumerate}
\end{teo}
\begin{dimo}
	$1)$ Già visto\\
	 $2)$ Basta applicare uno degli esercizi di oggi all'estensione Galoisiana di grado finito  $\L\subseteq\K$ \\
	 $3)$  $[G(\K,\F):H] = \frac{|G(\K, \F)|}{|H|}= \frac{|G(\K,\F)|}{|G(\K,\K_H)|}  \substack{(2)\\ =}\frac{[\K:\F]}{[\K:\K_H]} = [\K_H:\F]$ 
	 $4)$ dal lemma  $\F\subseteq\Le$ normale se e solo se  $\sigma|_\Le\in G(\Le,\F)\ \ \ \forall\sigma\in G(\K,\F)$\\
	  $ \Leftrightarrow \forall \tau\in G(\K,\Le)$ e $\forall \sigma \in G(\K,\F)$ si ha  $\tau \circ \sigma|_\Le = \sigma|_\Le$\\
	  $ \Leftrightarrow (\sigma^{-1}\circ\tau\circ\sigma)|_\Le = Id_\Le$\\
	  $ \Leftrightarrow \sigma^{-1}G(\K,\Le)\sigma\subseteq G(\K,\Le)\ \ \ \forall \sigma\in G(\K,\F)$\\
	  $5)$ Definiamo l'omomorfismo
	  \[
	  \begin{aligned}
		  g:  G(\K,\F) & \rightarrow G(\Le, \F)\\
		  \sigma & \rightarrow \sigma|)\Le
	  \end{aligned}
	  .\] 
	  $g$ è ben definito per il lemma
	  \begin{itemize}
		  \item $\ker(g) = \{\sigma\in G(\K,\F)\ | \ \sigma|_\Le = id_\Le\} = G(\K,\Le)$
		  \item verifichiamo che  $g$ suriettiva dato $\psi\in G(\Le,\F)$ abbiamo\\
			  INSERISCI IMMAGINE 2 05 19 MAGGIO\\
			  $F\subseteq\K$ campo di spezzamento di  $f\in \F[x]$. Allora 
			  \begin{itemize}
				  \item $\Le\subseteq \K$ campo di spezzamento di  $f\in \Le[x]$
				  \item $\Le \substack{\cong\\\psi}\Le\subseteq\K$ campo di spezzamento per  $f\in \Le[x]$ 
			  \end{itemize}
			  $ \Rightarrow  \exist\omega\in G(\K,\F)$ tale che $g(\sigma) = \psi$
	  \end{itemize}
Dal teorema di isomorfismo tra anelli segue la tesi
\end{dimo}
\textbf{Esercizio}\\
$G$ gruppo, $|G| = 2^n$  $n\in \Z_{\geq 1}$. Allora  $\exists G\leq G$ tale che  $[G:H] =2$\\
 \textbf{Esempio:}\\
 $f(x) = x^3 - 2\in \Q[x]$ \\
 $\Q\subseteq\Q(\sqrt[3]2,\omega) = \K$,  $\omega = \frac{-1 + i\sqrt 3}2\in\A$\\
 è estensione Galoisiana di grado finito,\\
 Elenchiamo tutti gli elementi di  $G(\K,\Q)$
  \[
 \rho: \begin{cases}
 	\omega \rightarrow \omega\\
	$\sqrt[3] 2 \rightarrow \sqrt[3] 2\omega$
 \end{cases}\ \ \ \omega: \begin{cases}
 	\omega \rightarrow \omega^2\\
	\sqrt[3]2 \rightarrow \sqrt[3]2

 \end{cases}
 \] 
 \[
 \rho^2 = \begin{cases}
 	\omega \rightarrow \omega\\
	\sqrt[3] 2 \rightarrow \sqrt[3]2\omega^2
 \end{cases} 
 \ \ \ \ \sigma^2 = \rho^3 = Id
 \] 
 \[
 \sigma\rho^2 = \rho\sigma : \begin{cases}
 	\omega \rightarrow\omega^2\\
	\sqrt[3]2 \rightarrow \sqrt[3]2\omega
 \end{cases}
 \] 
 \[
 \rho^2\sigma = \sigma\rho : \begin{cases}
 	\omega \rightarrow \omega^2\\
	\sqrt[3]2 \rightarrow \sqrt[3]2\omega^2
 \end{cases}
 \] 
	$G(\K:\Q) = <\rho,\sigma>\cong S_3\cong D_3$ \\
	AGGIUNGI IMMAGINE 2 33\\
	Elenchiamo tutte le estensioni intermedie
	\[
		\K_{<\rho,\sigma>} = \Q
	.\] 
	\[
		\K_{<\rho>} = \Q(\omega)
	.\] 
	AGGIUNGI IMMAGINE 2 38
	\subsection{Teorema fondamentale dell'algebra}
	\begin{teo}
		$\C$ è algebricamente chiuso
	\end{teo}}
	\begin{dimo}
		Procediamo per passi.
		\begin{enumerate}
			\item $\R\subseteq\K$ estensione Galoisiana finita allora  $2 \ | \ [\K:\R]$ \\
				Infatti un polinomio $f\in \R[x]$ di grado dispari soddisfa
				 \[
					 \lim_{x \rightarrow \pm\infty}f(x) = \pm\infty
				.\] 
				$\Rightarrow$ esiste una radice di $f$\hfill (Bolzano)\\
				 $ \Rightarrow  f$ non è irriducibile.\\
				 L'estensione $\R \subseteq\K$ è semplice \hfill (teorema elemento primitivo)\\
				 $ \Rightarrow  [\K:\R]$ è il grado di un polinomio irriducibile in $\R[x]$
			 \item  $\R\subseteq\K$ estensione Galoisiana finita.\\
				 Allora  $[\K:\R] = 2^n$ con  $n\in \Z_{\geq 0}$\\
				 Infatti:
				  \[
					  2 \ | \ [\K:\R] = |G(\K,\R)|
				 .\] 
				 $ \Rightarrow  \exists H\in Syl_2(G(\K,\R))\hfill $ (Sylow)\\
				 $ \displaystyle\Rightarrow [\K_H: \R] = \frac{[\K:\R]}{[\K:\K_H]} = \frac{|G(\K,\R)|}{|G(\K,\K_H)|} = \frac{|G(\K,\R)|}{|H|}$ \\
				 \underline{non} è divisibile per $2$!\\
				  $ \Rightarrow  \K_H = \R$.\\
				  $ \Rightarrow  H = G(\K,\K_H)=G(\K,\R)$ \\
				  $ \Rightarrow |G(\K,\R)| = |H| = 2^n$ 
			  \item $\C\subseteq\K$ campo di spezzamento di  $f\in \C[x]$. Allora  $\C = \K$.\\
				  Infatti:\\
				  $2^n = [\K:\R] = [\K:\C]\cdot[\C:\R] = [\K:\C]\cdot 2$\\
				  $ \Rightarrow  [\K:\C] = 2^{n-1}$ \\
				  Dobbiamo dimostrare che $n = 1$\\
Se  $ n > 1$
 \[
	 |G(\K,\C)| = 2^{n-1}
.\] 
$ \Rightarrow  \exists H\leq G(\K,\C)$ di indice 2.\\
$ \Rightarrow  \displaystyle [\K_H: \C] = \frac {[\K:\C]}{[\K:\K_H]} = \frac {|G(\K,\C)|}{|G(\K,\K_H)|} = \frac{|G(\K,\C)|}{|H|} = [G(\K,\C):H] = 2$ \\
$C\subseteq \K_H$ è estensione semplice\\
$ \Rightarrow  \exists\alpha\in\K_H:$
\[
	\C\subseteq\C[\alpha] = \K_H
.\] 
e il polinomio minimo p di $\alpha$ su $\C$ ha grado $[\K_h:\C] = 2$ \\
Quindi $p\in\C[x]$ è irriducibile e di grado 2, assurdo.\\
 $ \Rightarrow  n = 1 \Rightarrow \K = \C$
		\end{enumerate}

	\end{dimo}
\end{document}
