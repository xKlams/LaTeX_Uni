\documentclass[12px]{article}

\title{Lezione 15 Algebra I}
\date{2024-11-19}
\author{Federico De Sisti}

\usepackage{amsmath}
\usepackage{amsthm}
\usepackage{mdframed}
\usepackage{amssymb}
\usepackage{nicematrix}
\usepackage{amsfonts}
\usepackage{tcolorbox}
\tcbuselibrary{theorems}
\usepackage{xcolor}
\usepackage{cancel}

\newtheoremstyle{break}
  {1px}{1px}%
  {\itshape}{}%
  {\bfseries}{}%
  {\newline}{}%
\theoremstyle{break}
\newtheorem{theo}{Teorema}
\theoremstyle{break}
\newtheorem{lemma}{Lemma}
\theoremstyle{break}
\newtheorem{defin}{Definizione}
\theoremstyle{break}
\newtheorem{propo}{Proposizione}
\theoremstyle{break}
\newtheorem*{dimo}{Dimostrazione}
\theoremstyle{break}
\newtheorem*{es}{Esempio}

\newenvironment{dimo}
  {\begin{dimostrazione}}
  {\hfill\square\end{dimostrazione}}

\newenvironment{teo}
{\begin{mdframed}[linecolor=red, backgroundcolor=red!10]\begin{theo}}
  {\end{theo}\end{mdframed}}

\newenvironment{nome}
{\begin{mdframed}[linecolor=green, backgroundcolor=green!10]\begin{nomen}}
  {\end{nomen}\end{mdframed}}

\newenvironment{prop}
{\begin{mdframed}[linecolor=red, backgroundcolor=red!10]\begin{propo}}
  {\end{propo}\end{mdframed}}

\newenvironment{defi}
{\begin{mdframed}[linecolor=orange, backgroundcolor=orange!10]\begin{defin}}
  {\end{defin}\end{mdframed}}

\newenvironment{lemm}
{\begin{mdframed}[linecolor=red, backgroundcolor=red!10]\begin{lemma}}
  {\end{lemma}\end{mdframed}}

\newcommand{\icol}[1]{% inline column vector
  \left(\begin{smallmatrix}#1\end{smallmatrix}\right)%
}

\newcommand{\irow}[1]{% inline row vector
  \begin{smallmatrix}(#1)\end{smallmatrix}%
}

\newcommand{\matrice}[1]{% inline column vector
  \begin{pmatrix}#1\end{pmatrix}%
}

\newcommand{\C}{\mathbb{C}}
\newcommand{\K}{\mathbb{K}}
\newcommand{\R}{\mathbb{R}}


\begin{document}
	\maketitle
	\newpage
	\section{Nella lezione precedente..}
	\begin{teo}[$1^o$ Teorema di Sylow]
		$p$ primo che divide $|G|$ Allora
		 $Syl_p(G)\neq\emptyset$
	\end{teo}
	\begin{teo}[$2^o$ Teorema di Sylow]
		$p$ primo divide $|G|$ allora:
		\[
			\forall H,K\in Syk_p(G) \ \ \exists g\in G \text{ tale che } H=gKg^{-1}
		.\] 

	\end{teo}
	\section{Roba nuova}
		\begin{coro}
			$p$ primo che divide $|G|$ allora  $H\in Syl(G)$ è normale se e solo se\\ $n_p = |Syl_p(G)| = 1$
		\end{coro}
		\textbf{Osservazione}\\
		è importante sapere se $n_p = 1$ perché l'esistenza di sottogruppi normali spesso permette di realizzare un gruppo come prodotto semidiretto
	\\
	\begin{teo}[$3^o$ teorema di Sylow]
		$G$ gruppo finito
		\begin{itemize}
			\item $|G| = p^r m$
			\item $r,p,m\in \Z_{>0}$
			\item $p$ primo
			\item $MCD(p,m) = 1$
		\end{itemize}
		Allora:\\
		1) $n_P = [G:N_G(H)]$ dove  $H\in Syl_p(G)$ \\
		2) $m\equiv_{n_p} 0$\\
		3)  $n_p\equiv_p 1$
	\end{teo}
	Prima della dimostrazione vogliamo estendere la nozione di centralizzatore (o centralizzante)
	\newpage
	\begin{defi}[Normalizzatore]
		$G$ gruppo $S\subseteq G$ sottoinsieme\\
		1)Il centralizzatore di  $S$ in  $G$ è
		 \[
			 C(S) = \{g\in G| gs = sg \ \ \forall s\in S\}
		.\] 
		2) Il normalizzatore di $S$ in $G$ è 
		\[
			N_G(S) = \{g\in G | gS = Sg\}
		.\] 
	\end{defi}
	\textbf{Esercizio:}\\
	Dimostrare che\\
	1) Se $S\subseteq G \Rightarrow C(S) \leq G$ \\
	2) $S\subseteq G \Rightarrow N_G(S)\leq G$ \\
	3) $S\leq G \Rightarrow S\leq N_G(S)$ \\
	\begin{dimo}
		Considero l'azione\\
		$G\times Syl_p(G) \rightarrow Syl_p(G)$\\
		$(g,H) \rightarrow g.H := gHg^{-1}$ \\
		Allora $\forall H\in Syl_p(G)\\
		p^rm = |G| = [G:Stab_H]\cdot |Stab_H|$\\
		 $= [G:N_G(H)]\cdot |N_G(H)|$\hfill
		 (dato che $Stab_h = N_G(H)$)\\
		 $=[G:B_G(H)][N_G(H):H]|H|$\hfill
		 (dato che $H\leq N_G(H)$) \\
		 Deduciamo che $m = [G:N_G(H)]\cdot[N_G(H):H]$\\
		 Ora:\\
	 $n_p = |Syl_p(G)| = |O_H^G|$\hfill (II Teorema di Sylow)\\
	 $ = [G:Stab_H]$\\
	 Quindi abbiamo dimostrato $(1)$ e $(2)$\\
	 Resta da dimostrare $(3)$\\
	 di un fissato  $K\in Syl_p(G)$\\
	 \[
		 K\times Syl_p(G) \rightarrow Syl_p (G)
	 .\] 
	 \[
		 (k,H) \rightarrow k.H := kHk^{-1}
	 .\] 
	 Questa azione avrà $r + 1$ orbite (con $r \geq 0$)\\
	 $O_K^K, O_{H_1}^K,\ldots,O_{H_r}^K$\\
	 Abbiamo una decomposizione in orbite disgiunte
	 \[
		 Syl_p(G) = O_K^K\cup O_{H_1}^K\cup\ldots\cup O_{H_r}^K
	 .\] 
	 \[ \Rightarrow n_p = |Syl_p (G)| = |O_K^K| + \sum^r_{j=1}|O_{H_j}^K|\]
	 \[
		 =|O_K^K| + \sum^r_{j=1}[K:Syl_{H_j}^K]
	 .\] 
	 \[
		 = |O_K^K| + \sum^r_{j=1}[K:N_K(H_j)]
	 .\]
	 \textbf{Idea}\\
	 Basta ora verificare che
	 \begin{itemize}
		 \item $|O_K^K| = 1$
		 \item  $O_{H_j}^K\equiv_p 0 \ \ \ \forall 1\leq j\leq r$
	 \end{itemize}\\
	 Abbiamo:\\
	  \[
		  O_K^K = [K:N_K(K)] = 1
	 .\] 
	 Dato che $H\leq N_G(H)\leq G \Rightarrow N_K(K) = K$ 
	 \[
		 |O_{H_j}^K| = [K:N_K(H_j)] = \frac {|K|}{|N_K(H_{j})|} = \frac {p^r}{|N_K(H_j)|}
	 .\] 
	 dato che $K \in Syl_p(G)$\\
	 Quindi resta da escludere il caso  $N_K(H_j) = K$\\
	 Ma questo è equivalente a  $KH_j = H_j K$ 
	 \[
	 \Rightarrow \begin{cases}
	 	KH_j\leq G\\
		|KH_j| = \frac { |K||H_j|}{|K\cap H_j|} = \frac{p^{2r}}{p^{s_j}}
	 \end{cases}
	 .\] 
	 dove $0\leq s_j < r$ dato che  $H_j\neq K_j$\\
	 Ma  $p^{2r-s_j}\not | \ p^rm$ da cui l'assurdo per Lagrange
	\end{dimo}
	\section{Applicazioni di Sylow}
	Possiamo (ri)-dimostrare un vecchio risultato
	\begin{teo}[Cauchy]
		$G$ gruppo finito, $p$ primo che divide $|G|$ allora\\
		 \[
			 \Exists g\in G\text{ tale che } $ord(g) = p$
		.\] 
		
	\end{teo}
	\begin{dimo}
		Da Sylow I segue che esiste $H\in Syl_p(G)$ \\
		Scegliamo $h\in H$ tale che  $h\neq e$\\
		Ora  $ord(h) = p^s$ per qualche $s>0$\\
		Definiamo $f = h^{p^{s-1}}$\\
		 $f = h^{p^{s-1}}\neq e \Rightarrow ord(h)\neq 1$ \\
		 $f^p = (h^{p^{s-1}})^p = h^{p^s} = e \Rightarrow ord(f) = p$
	\end{dimo}
	\begin{teo}[Wilson]
		$p$ primo allora $(p-1)!\equiv_p p-1$
	\end{teo}
	\begin{dimo}
		Scelgo $G= S_p$ Studio $n_p$\\
		I  $p$-Sylow in $S_p$ hanno ordine $p$\\
		 $ \Rightarrow$ sono tutti i sottogruppi ciclici di ordine $p$ in $S_p$\\
		 $\cdot$ Gli unici elementi di ordine  $p$ in $S_p$ sono i $p$-cicli.\\
		 fissato il primo elemento, abbiamo $p-1$ scelte per il secondo, $p-2$ per il terzo e così via\\
		 quindi i $p$-cicli sono $ (p-1)!$\\
		 Quindi i sottogruppi di $S_p$ di ordine $p$ sono $\frac{(p-1)!}{(p-1)} = (p-2)!$ perché in ogni tale sottogruppo appaiono  $p-1$ $p$-cicli\\
		 $ \Rightarrow (p-2)! = n_p \equiv_p 1 \ \ \Rightarrow \ \ (p-1)! \equiv_p p-1$
	\end{dimo}
	\begin{teo}[Classificazione dei gruppi $pq$]
		$G$ gruppo finito, $p,q > 1$ tali che \\
		 $\cdot p,q$ primi\\
		 $\cdot p < 1$\\
		  $\cdot|G| = pq$ \\
		  Allora\\
		  1) Se $p\not | \ q-1$ allora  $G\cong C_{pq}$\\
		  2) Se  $p | q-1$ allora  $G\cong C_q\semi C_p$
	\end{teo}
	\begin{dimo}
		Studio $n_p$ \\
		\begin{cases}
			p = m \equiv_{nq} 0\\
			n_q \equiv_q 1
		\end{cases}\\
		$ \Rightarrow $ \begin{cases}
			n_q =1 \text { oppure } m_q = p\\
			\text{ seconda esclude } n_q = p \text{ perchè } p < q

		\end{cases}\\
		$ \Rightarrow n_q = 1$ \\
		$ \Rightarrow \exists ! Q\in Syl_p(G)$\\
		$ \Rightarrow Q\trianglelefteq G$ e $|G| = q \Rightarrow Q\cong C_q$\\
		Studio $n_p$ nel caso $p\not | q -1$\\
		 \begin{cases}
			 $q-m \equiv_{n_p} 0$\\
			  $n_p\equiv_p 1$
		\end{cases}
		$ \Rightarrow n_p = 1$ oppure $n_p = q$\\
		 $ \Rightarrow n_p\neq q$ perché\\
		 $q\not\equiv_p 1$ per ipotesi\\
		  $n_p = 1 \Rightarrow \exists ! P\in Syl_p(G)$ \\
		  $ \Rightarrow P\triangleleftwq G$ e $|P| = p \Rightarrow P\cong C_p$ \\
		  Ora abbiamo due sottogruppi normali $P,Q\trianglelefteq G$\\
		  tali che\\
	  $\cdot P\cap Q = \{e\}$  perchè  $|P\cap Q|$ divide sia  $|P| = p$ che $|Q| = q$\\
	  $\cdot PQ| = \frac{|P||Q|}{|P\capQ|} = pq = |G|$\\
$ \Rightarrow G\cong P\times Q\cong C_p\times C_q\cong C_{pq}$

	Resta il caso $p | q-1$\\
	 $\cdot \exists ! Q\in Syl_p(G) \leadsto Q\trianglelefteq G$\\
	  $\cdot \exists P\in Syl_p(G)\leadsto P\leq G$\\
	  Ora\\
	  $\cdot P\cap Q = \{ e\}$ come prima\\
	   $PQ = G$ come prima\\
	   Quindi $G$ prodotto semidiretto interno\\
	   $ \Rightarrow G\cong Q\semi P \Rightarrow C_q \semi C_p$ \\
	   per qualche omomorfismo $\phi :C_p \rightarrow Aut(C_q)$
	\end{dimo}
	\textbf{Esercizio:}\\
	Classificare i gruppi di ordine $2q$ con  $q>2$ primo

\end{document}
