\documentclass[12px]{article}

\title{Lezione N+1 Algebra 1}
\date{2025-05-15}
\author{Federico De Sisti}

\input{../../../setup.tex}

\begin{document}
	\maketitle
	\newpage
	\subsection{Teorema dell'elemento primitivo}
	\textbf{Ricordo:} Abbiamo dimostrato
	\begin{teo}
		$\F$ campo infinito, allora ogni estensione separabile di grado finito è semplice
	\end{teo}
	\begin{teo}[elemento primitivo]
		$\F$ campo. Ogni estensione separabile di grado finito di $\F$ è semplice
	\end{teo}
	\begin{dimo}
		si tratta di studiare il caso $|\F| < +\infty$. Sappiamo che il gruppo moltiplicativo $U_\F = \F\setminus\{0\}$  è ciclico. \\
		Sia ora $\F \in \K$ estensione finita\\
		 $ \Rightarrow \exists \alpha\in \K$  tale che $U_\K = <\alpha> \Rightarrow  \F\subseteq\F(\alpha) = \K$  
	\end{dimo}
	\textbf{Osservazione}\\
	Nel caso finito non abbiamo usato l'ipotesi di estensione separabile\\
	Il prossimo risultato è dimostrato da Galois$(1831)$ di  Steinitz$(1910)$ 
	\section{Teoria di Galois}
\begin{defi}
	$\varphi: \F \rightarrow\K$ estensione di campi\\
	$\overline \varphi_\K :\K \rightarrow \overline \K$ chiusura algebrica\\
	Un $\F$-omomorfismo di $\K$ è un estensione  $\psi : \K \rightarrow \overline\K$ tale che il diagramma (INSERISCI IMMAGINE 4 30), sia commutativo.
\end{defi}
\textbf{Esercizio:}\\
L'insieme $I(\K,\F)$ degli  $\F$-omomorfismi di  $\K$ non dipende dalla scelta di $\ovrline \varphi_\K$\\
 \textbf{Esempio:}\\
 $\Q\hookrightarrow \Q(\sqrt[3]{2})\hookrightarrow \A$\\
 dove la prima freccia è $ \varphi$ e la seconda è $\overline\phi_k$\\
 il polinomio minimo di  $\sqrt[3]{2}$ è  $x^3-2\in\Q[x]$\\
 $x^3-2=(x-\sqrt[3]{2})(x -\sqrt[3]{2}\omgea)(x-\sqrt[3]2\omega^2)$\\
 dove $\omega = \frac{-1+i\sqrt{3}}{2}\in\A$\\
 L'estensione 
  \[
 \begin{aligned}
	 \psi : &\Q(\sqrt[3]{2}) \rightarrow \A\\
		&\sqrt[3]2 \rightarrow \sqrt[3]{2}\omega
 \end{aligned}
 .\] 
 \begin{prop}
 	Sia $\F\subseteq\K$ estensione separabile e di grado finito, allora 
	 \[
		 |I(\K,\F)| = [\K:\F]
	.\] 
 \end{prop}
 \begin{dimo}
 	Dal teorema dell'elemento primitivo $\exists \alpha\in\K$ tale che  $\F\subseteq\F(\alpha) = \K$ \\
	Sia $ \psi\in I(\K,\F)$ \\
	$\psi$ è univocamente determinato da $\psi(\alpha)$
	 \begin{itemize}
		 \item dimostriamo che $\alpha, \psi(\alpha)$ sono coniugati\\
			 sia  $f\in\F[x]$ il polinomio minimo di  $\alpha$, allora $0=\psi(0)=\psi(f(\alpha)) = f(\psi(\alpha))$  \\
			 $ \Rightarrow  f$ è il polinomio minimo di $\psi(\alpha)$
		 \item per ogni radice  $\beta$ di $f$ l'applicazione
			 \[
			 \begin{aligned}
				 \psi_{\beta }&:\K\setminus \F(\alpha) \rightarrow \overline\K\\
					      &\alpha \rightarrow \psi_\neta(\alpha) = \beta
			 \end{aligned}
			 .\] 
			 è un omomorfismo di anelli (con $im(\psi_\beta) = \F(\beta))$
		 \item Dato che  $\F\subseteq\K$ è estensione separabile  $ \Rightarrow f$ ammette $\deg(f)$ radici distinte, quindi
			  \[
				  |I(\K,\F)| = deg(f) = [\K:\F]
			 .\] 
	\end{itemize}
 \end{dimo}
 \begin{defi}[Gruppo di Galois]
 	$\F\subseteq K$ estensione. Il suo gruppo di Galoi s è $(G(\K,\F),\circ)$ dove
	 \[
		 G(\K,\F) = \{\sigma: \K \rightarrow \K \ | \ \sigma \text{ è isomorfismo di anelli e }\sigma|_\F = id_\F\}
	.\] 
	e $\circ$ è la composizione
 \end{defi}
 \textbf{Osservazione:}\\
 Data l'estensione $ \varphi: \F \rightarrow \K$ e dato $\omega: \K  \rightarrow\K$ scriveremo $\omega|_\F = id_\F$intendendo che, inserisci diagramma, è un diagramma commutativo.
  \begin{prop}
 	$\F\subseteq\L$ estensione 
	 \[
	|G(\K,\F)|\leq |I(\K,\F)|
	.\] 
 \end{prop}
 \begin{dimo}
 	Costruiamo un'applicazione iniettiva fra insiemi\\
	\begin{aligned}
		$G(\K,\F) &\rightarrow I(\K,\F)$\\
			  &inserisci immagine
	\end{aligned}\\
	dove $\overline \varphi_\K: \K \rightarrow\overline\K$ è una chiusura algebrica fissata\\
	\begin{itemize}
	\item$X$ è ben definita poiché $\sigma\in G(\K,\F)$ allora\\
	INSERISCI IMMAGINE 4 58\\
è un diagramma commutativo
\item Inoltre $X$ è iniettiva poiché.\\
	$X(\sigma_1) = X(\sigma_2)$\\
	$ \Rightarrow \overline\phi_\K = \sigma_1 = \overline\sigma_\K = \sigma_2 \Rightarrow  \sigma_1 = \sigma_2$  
	\end{itemize}
 \end{dimo}
 \begin{coro}
 	$\F\subseteq\K$ estensione separabile di grado finito, allora
	 \[
		 |G(\K,\F)|\leq [\K:\F]
	.\] 
 \end{coro}
 \begin{dimo}
 	questo segue dalle proposizioni precedenti.
 \end{dimo}
 \begin{defi}
 	$\K$ campo  $H\leq Aut(\K)$\\
	Poniamo
	\[
		\K_H:=\{k\in\K\ |\ \omega(k) = k \forall\sigma\in H\}
	.\] 
	si dice campo fissato da $H$.
 \end{defi}
 \textbf{Esercizio:}\\
 Dimostrare che è un campo.\\
 \begin{defi}[Galois]
 	$\F\subseteq\K$ estensione
	 \[
		 \mathcal F_{\K,\F} = \{ \text{estensioni intermedie } \F\subset\mathbb L \subset\K\}
	.\] 
	\[
		\mathcal G_{\K,\F} = \{\text{sottogruppi di }G(\K,\F)
	.\] 
	\[
	\begin{aligned}
		\psi: &\mathcal F_{\K,\F} \rightarrow \mathcal G_{\K,\F}\\
		      &(\mathcal F\subseteq\mathbb L\subseteq\K ) \rightarrow G(\K,\mathbb L)\leq G(\K,\F)
	\end{aligned}
	.\] 
	\[
		\begin{aligned}
		\Phi: &\mathcal G_{\K,\F} \rightarrow\mathcal F_{\K,\F}\\
		$H\leq &G(\K,\F) \rightarrow \F\subseteq\K_H\subseteq \K$
		\end{aligned}
	.\] 
	$\psi$ e $\phi$ si dicono corrispondenze di Galois
 \end{defi}
 \textbf{Domande:}\\
 \begin{enumerate}
	 \item $\Phi(\psi(\L)) = \L?$ per ogni estensione $\F\subseteq\mathbb L\subseteq\K$
	 \item  $\psi(\Phi(H))=H?$ per ogni  $H\leq G(\K,\F)$
 \end{enumerate}
 \textbf{Esercizi:}\\
 $\F\subseteq\K$ estensione, dimostrare
  \begin{enumerate}
	  \item $G(\K,\K) = \{id\}$
	  \item  $\F\subseteq\Le\subseteq\Le_2\subseteq\K$\\
		   $ \Rightarrow  \psi(\Le_2)=G(\K,\Le_2) \leq G(\K,\Le_1) = \psi(\Le_1)$
	   \item $\K_\{id\} = \K$
	   \item  $H_1\leq H_2\leq G(\K,\F)$ \\
		   $ \Rightarrow F\subseteq\K_{H_2} = \Phi(H_2)\subseteq \K_{H_1} = \Phi(H_1)\subseteq \K$
	   \item $H\leq G(\K,\F) \Rightarrow H\leq G(\K,\K_H)= \psi(\K_H) = \psi(\Phi(H))$ 
	   \item $\F\subseteq\Le\subseteq\K \Rightarrow  \F\subseteq\Le\subseteq\K_{G(\K,\Le)}\subseteq\K $ \\
\[
\begin{aligned}
	\psi: &\mathcal F_{\K,\F} \rightarrow \mathcal G_{\K,\F}\\
	      &(\F\subseteq\Le\subseteq\K) \rightarrow \mathcal G(\K,\Le)
\end{aligned}
.\] 
\[
\begin{aligned}
	\psi: &\mathcal G_{\K,\F} \rightarrow \mathcal F_{\K,\F}
	      &H\leq G(\K,\F) \rightarrow F\subseteq\K_G\subseteq \K
\end{aligned}
.\] 
 \end{enumerate}
 \textbf{Osservazione}
\begin{itemize}
	\item $H\leq \psi(\Phi(H))$
	\item  $\Le\subseteq\Phi(\psi(\Le))$
\end{itemize}
\textbf{Obiettivo}\\
Esibire condizioni su $\F\subseteq\K$ affinché $\Phi,\psi$ siano una l'inversa dell'altra
 \begin{teo}
	$\F\subseteq\K$ separabile di grado finito. Allora dato  $H\leq G(\K,\F)$ abbiamo
	 \begin{itemize}
		 \item $|H| = [\K,\K_H]$
		 \item  $\psi(\Phi(H))= H$
	\end{itemize}
\end{teo}
\begin{dimo}
	Abbiamo $H\leq G(\K,\K_H)$ \\
	\[
		\Rightarrow  |H|\leq |G(\K,\K_H)| \leq [\K,\K_H]
	.\] 
	basta verificare che 
	\[
		|H| \geq [\K,\K_H]
	.\] 
	per dedurre entrambi gli enunciati.\\
Dal teorema dell'elemento primitivo esiste $\alpha\in \K$ tale che  
\[
\F\subseteq\F(\alpha) = \K
.\] 
Vogliamo costruire un polinomio $f\in \K_H$ di cui  $\alpha$ sia radice.\\
\[
	H = \{\sigma_1 = Id,\sigma_2,\ldots,\sigma_h\}
.\] 
Definiamo:\\
$\displaystyle\alpha_s := \sum^{}_{1\leq j_1 < \ldots < j_s\leq h}\sigma_{j_1}(\alpha)\cdot\ldots\cdot\sigma_{j_s}(\alpha)\in\K$\\
Poniamo
\[
	f(x) = \prod_{j=1}^h(x-\sigma_j(\alpha)) = x^h - \alpha_1^{h-1} + \ldots + (-1)^h\alpha_h\in\K[x]
.\] 
Chiaramente $f(\delta) = 0$. Verifichiamo $\alpha_s\in\K_H$ ovvero  $\sigma_t(\alpha_s) = \alpha_s$ $\forall t,s\in\{1,\ldots,h\}$\\
Abbiamo
 \[
\sigma_t(\alpha_s) = \sum^{}_{1\leq j_1 < \ldots < j_s\leq h}\sigma_{j_1}(\alpha)\cdot\ldots\cdot\sigma_{j_s}(\alpha) = \sigma_s
.\] 
Dove l'ultima uguaglianza segue dal fatto che 
\[
\begin{aligned}
	& H \rightarrow H\\
	& \sigma_j \rightarrow \sigma_t\cdot \sigma_j
\end{aligned}
.\] 
è un isomorfismo $\forall t\in \{1,\ldots,h\}$\\
$ \Rightarrow f\in \K_H[x]$ \\
$ \Rightarrow  |H| = h = deg(f)\geq deg($ polinomio minimo di $\alpha$ su $\K_H) =[\K_H(\alpha):\K_H] = [\K:\K_H]$

\end{dimo}
\begin{teo}
	$\F\subseteq\K$ estensione separabile di grado finito.\\
	 Allora sono equivalenti:
	 \begin{enumerate}
		 \item $\F\subseteq\K$ estensione normale
		 \item  $\Phi(\psi(\Le))=\Le$ per ogni  $\F\subseteq\Le\subseteq\K$
	 \end{enumerate}
\end{teo}
\begin{dimo}
	
	$2) \Rightarrow  1)$ per ipotesi abbiamo $\F = \K_{G(\K,\F)}$\\
	Per dimostrare che  $\F\subseteq\K$ è normale, basta verificare che sia un campo di spezzamento di un polinomio $f\in \F[x]$.\\
Per il teorema dell'elemento primitivo \\
$\exists \alpha\in \K\  \ $ tale che  $\F\subseteq\F(\alpha) = \K$ \\
Inoltre $|G(\K,\F)| < +\infty$ \\
$G(\K,\F) = \{\sigma_1,\ldots,\sigma_h\}$\\
$f(x) = \prod^h_{j=1}(x-\sigma_j(\alpha)) = x^h - \alpha_1x^{h-1}+\ldots+(-1)^h\alpha_h$\\
dove 
\[
	\alpha_s = \sum^{}_{i\leq j_1<\ldots<j_s\leq h}\sigma_{j_1}(\alpha)\cdot\ldots\cdot\sigma_{j_s}(\alpha)
.\] 
$f(x)\in\K[x]$\\
Osserviamo che  $\dorall s,t\in\{1,\ldots,h\}$\\
\[
	\sigma_t(\alpha_s) = \sum^{}_{1\leq j_1 < \ldots< j_s\leq h} (\sigma_t\sigma_{j_1})(\alpha)\cdot\ldots\cdot(\sigma_t\sigma_{j_s})(\alpha)
.\] 
$ \Rightarrow \alpha_s\in\K_{G(\K,\F)}=\F \Rightarrow  f(x)\in\F[x]$ \\
\begin{itemize}
	\item $f$ si decompone in fattori lineari in $\K[x]$
	\item  $f(\alpha) = 0$ quindi data un'estensione  $\F\subseteq\Le\subseteq\K$ abbiamo  $\alpha\in\Le$\\
		 $ \Rightarrow \K = \F(\alpha)\subseteq\Le\subseteq\K$\\
		 $ \Rightarrow \Le = \K$ \\
		 $ \Rightarrow  \F\subseteq\K$ è campo di spezzamento di $f$.\\
\end{itemize}
$1) \Rightarrow  2)$  Dobbiamo verificare che se $\F\subseteq\K$ è separabile, di grado finito e normale, allora
 \[
	 \Phi(\psi(\Le))=\Le \  \ \ \ \text{per ogni }\F\subseteq\Le\subseteq\K
.\] 
$\Le \subseteq \Phi(\psi(\Le)) = \K_{G(\K,\Le)}$\\
Verificare che $\K_{G(\K,\Le)}\subseteq \Le$\\
Sappiamo che  $\Le\subseteq\K$\\
è estensione normale di grado finito.\\
Quindi esiste polinomio $f\in\Le[x]$ tale che  $\Le\subseteq\K$ sia campo di spezzamento di  $f$.\\
Procedo per induzione su  $[\Le:\K]$\\
se  $[\Le:\K] = 1 \Rightarrow  \Le = \K$ \\
$ \Rightarrow  \K_{G(\K,\Le)} =\K = \Le$ \\
Se $[\K:\Le] > 1$ allora\\
$f: p_1\cdot\ldots\cdotp_r$ fattorizzazione in irriducibili\\
Possiamo assumere che $deg(p_1) > 1$.\\
Siano $\alpha_1,\alpha_2,\ldots,\alpha_h\in\K$ le radici di $p_1$ $(deg(p_1) = h > 1)$\\
\[
	[\K: \Le] =[\K:\Le(\alpha_1)][\Le(x):\Le]
.\] 
con $[\Le(x):\Le] = h$\\
Quindi 
\[
	[\K:\Le(\alpha_1)] < [\K:\Le]
.\] 
e per ipotesi induttiva\\
\[
	\K_{G(\K,\Le(\alpha_1)} = \Le(\alpha_1)
.\] 
Definiamo
\[
\sigma_1,\ldots,\sigma_h\in G(\K,\Le)
.\] 
\[
\begin{aligned}
	\sigma_s: &\K \rightarrow \K\\
		  &\alpha_1 \rightarrow \alpha_s
\end{aligned}
.\] 
(si può estendere a tutto $\K$)\\
Dobbiamo verificare $\K_{G(\K,\Le)}\subseteq\Le$\\
Sia  $k\in \Ke_{G(\K,\Le)}\subseteq\K_{G(\K,\Le(\alpha))} = \Le(\alpha_1)$\\
$ \Rightarrow k = c_0 + c_1\alpha_1 + \ldots + c_{h-1}\alpha_1^{h-1}\in\Le(\alpha_1)$\\
$ \Rightarrow  \sigma_s(k) = k \ \ \ \forall s\in\{1,\ldots,h\}.$ \\
poiché $\sigma_s\in G(\K,\Le)$\\
Ora  $f(x) = (c_0-k) + c_1x + \ldots, c_hx^{h-1}\in\Le(\alpha_1)[x]$\\
$\alpha_s   = \sigma_s(\alpha_1)$ è radice di $f(x) \ \ \forall s\in\{1,\ldots,h\}$ che sono tutte distinte.\\
Quindi ho $h$ radici ma $deg(f) = h-1 \Rightarrow  f(x) = 0 \Rightarrow  k = c_0\in\Le$

\end{dimo}
\begin{defi}
	$\F\subseteq\K$ si dice estensione Galoisiana se è separabile e normale
\end{defi}
\begin{coro}
	$\F\subseteq\K$ estensione Galoisiana di grado finito, allora  $\Phi, \psi$ sono una l'inversa dell'altra.
\end{coro}
\end{document}
