\documentclass[12px]{article}

\title{Dispense di Geometria I
\\\large{Seconda Parte}}
\author{Federico De Sisti}
\date{}

\input{../../setup.tex}

\begin{document}
	\maketitle
	\newpage
	\section{Prefazione}
	Scrivo queste dispense in vista del secondo esonero dato il mio (potentissimo) 21 al primo. Governo ladro speriamo che il secondo vada meglio\\[10px]
	L'utilizzo di queste dispense è solamente riservato a chi mi sta particolarmente simpatico quindi siate lieti di essere in questa cerchia.\\[10px]
	Ciò che è scritto in queste dispense viene solamente dalle lezioni del Papi del 2024, potrei aggiungere qualcosa del libro di Edoardo Sernesi (mio padre).\\
	Detto ciò cominciamo subito questo magico viaggio!
	\newpage
	\subsection{Prodotti Hermitiani}
	$V$ spazio vettoriale complesso
	\begin{defi}[Funzione sesquilineare]
		Una funzione sesquilineare su $V$ è un'applicazione $h: V\times V \rightarrow \mathbb{C}$\\
		che è lineare nella prima variabile e antilineare nella seconda, cioè\\[10px]
		\begin{aligned}
			\hspace{80px}&h(v+v',w) = h(v,w) + g(v',w)\\
		&h(\alpha v,w) = \alpha h(v,w)\\
		&h(v,w + w') = h(v,w) + h(v,w')\\
			&h(v,\alpha w) = \overline{\alpha}h(v,w)\\
		\end{aligned}\\[10px]
		per ogni scelta di $v,w,v',w'\in V$ e $\alpha \in\mathbb{C}$
	\end{defi}
	\begin{defi}[Forma hermitiana]
		Una forma sesquilineare si dice hermitiana se
		\[
			h(v,w) = \overline{h(w,v)}
		.\] 
	\end{defi}
	\textbf{Osservazione}\\
	Se $h$ è hermitiana, $h(v,v)\in\R$, infatti deve risultare $h(v,v) = \overline{h(v,v)}$
	\begin{defi}[Forma antihermitiana]
		Una forma sesquilineare si dice antihermitiana se 
		\[
			g(v,w) = - \overline{h(v,w)}
		.\] 
	\end{defi}
	\textbf{Osservazione}\\
	In questo caso $h(v,v)\in\sqrt{-1}\R$\\	
	\begin{defi}
		Una forma hermitiana si dice semidefinita positiva se 
		\[
		h(v,v) \geq 0 \ \ \forall v\in V
		.\] 
	\end{defi}
	\begin{defi}
		Una forma hermitiana si dice definita positiva se 
		\[
		h(v,v)>0 \ \ \forall v \neq 0
		.\] 
		ovvero
		\[
			(h(v,v)\geq 0 \text{ e }h(v,v) = 0 \Rightarrow v=0)
		.\] 
	\end{defi}
	\textbf{Esempio}\\
	$V = \C^n$
	\[
		h\left( \icol{z_1\\ \vdots \\ z_n},\icol{w_1\\\vdots\\w_n}\right) = \sum^n_{i=1}z_i\overline{w_i}
	.\] 
	questo viene chiamato prodotto hermitiano standard su $\C^n$
 \[
		h\left( \icol{z_1\\ \vdots \\ z_n},\icol{z_1\\\vdots\\z_n}\right) = \sum^n_{i=1}z_i\overline{z_i} = \sum^n_{i=1}|z_i|^2
	\]
	\hline \ \\[10px]
	Dato $V$, consideriamo una base $B = \{v_1,\ldots,v_n\}$ di $V$.\\Se $h$ è una forma heritiana, diciamo che $H=(h_{ij}) = h(v_i,v_j)$ è la matrice che rappresenta $h$ nella base $B$.\\ 
	se  $u = \sum^n_{i=1}x_iv_i, \ \ \ w = \sum^n_{i=1}y_iv_i$\\
\begin{aligned}
	\hspace{30px}h(u,w) &= h(\sum^n_{i=1}x_iv_i,\sum^n_{i=1}y_iv_i) = \\
&= \sum^n_{i=1}x_ih_i(v_i,\sum^n_{i=1}y_iv_i) = \\
& = \sum^n_{i=1}x_i\overline{y_i}h(v_i,v_i) = \\
& = x^t H\overline{y}
\end{aligned}\\
Dato che $h$ è hermitiana, abbiamo $h(v,w) = \overline{h(w,v)}$ da cui segue\\[5px]
\begin{aligned}
	X^tHY &= \overline{Y^tHX}\\
	 &     = \overline{Y}^t \overline{H} \overline{X}\\
	 & = (\overline{Y}^t \overline{H} \overline{X})^t\\
	 & = \overline{X}^t \overline{H}^t \overline{Y} \ \ \ \ \Rightarrow \ \ \  H = \overline{H}^t
\end{aligned}
\begin{defi}
	Una matrice $M\in M_n(\C)$ si dice hermitiana se
	\[
		H = \overline{H}^t
	.\] 
\end{defi}
\newpage
\section{Operatori Unitari}
\begin{defi}
	Sia $T\in End(V)$ questo si dice unitario se
	\[
	\langle T(v), T(w) \rangle = \langle v,  \rangle 
	.\] 
\end{defi}
\begin{lemm}
$T\in End(V)$ operatore unitario\\
$1.$ Gli autovalori hanno modulo 1\\
$2.$ Autospazi relativi ad autovalori distinti sono ortogonali\\
\end{lemm}
\begin{dimo}
$1.$ Sia $v$ un autovettore di autovalore $\lambda$ 
\[
	\langle v, v \rangle = \langle Tv, Tv \rangle  = \langle tv, tv \rangle  = \lambda\overline{\lambda} \langle v, v \rangle = |\lambda|^2 \langle v, v \rangle 
.\] 
\[
v \neq 0 \Rightarrow  \ \ \ |\lambda|^2 = 1 \ \ \  \Rightarrow  \ \ \ |\lambda| = 1
.\] 
$2.$ Sia $v\in V_\lambda$, $w\in V_\mu$ \ \ $\lambda\neq\mu$
\[
	\langle v, w \rangle  = \langle Tv, Tw \rangle = \langle \lambda v, \mu w \rangle = \lambda\overline{\mu} \langle v, w \rangle 
.\] 
\[\text {Se } \langle v, w \rangle \neq 0  \Rightarrow \lambda \overline{\mu} = 1 \text{ Per il punto 1}\\
	.\]\[
	\lambda\overline{\lambda} =1\ \ \Rightarrow \ \ \overline{\lambda} = \overline{\mu} \ \ \Rightarrow \ \ \lambda = \mu \ \ \text{ assurdo}
.\] 
\end{dimo}
\begin{defi}
	Diciamo che $U\in M_n(\C)$ è unitaria se 
	\[
		U\overline{U}^t = Id
	.\] 
\end{defi}
\begin{prop}
	$T\in End(V)$ è unitario se e solo se la sua matrice in una base ortonormale è unitaria
\end{prop}
\begin{dimo}
	Sia $B = \{v_1,\ldots,v_n\}$ una base ortonormale di $V$ 
	\[
		\delta_{ij} = \langle v_i, v_j \rangle  = \langle Tv_i, Tv_j \rangle  = \langle Ae_i, Ae_j \rangle = e_i^tA^t\overline{A}e_j = A_i^t\overline{A}_j
	\] 
	dove abbiamo posto $A = (T)_B$ e $\{e_i\}$ è una base di $\C^n$.\\
	Abbiamo ottenuto quindi $A_i^t\overline{A}_j$ che è il prodotto hermitiano standard tra la $i$-esima e la $j$-esima colonna di $A$
\end{dimo}
\begin{teo}
	Sia $T\in End(V)$ un operatore unitario Esiste una base standard di autovettori per $T$
\end{teo} 
In particolare, per ogni matrice unitaria $A\in U(n)$ esiste $M\in U(n)$ tale che $M^{-1}AM$ è diagonale
\begin{nota}
	a volte si pone 
	\[
	 A^* = \overline{A}^t
	.\] 
	$A$ unitario $AA^* = Id$ \\
	$A$ hermitiano $A = A^*$ \\
	$A$ antihermitiano $A=-A^*$
\end{nota}
\begin{defi}[Operatore Aggiunto]
	Dato $T\in End(V)$, esiste unico $S\in End(V)$ tale che 
	\[
	\langle Tu, w \rangle = \langle u, Sw \rangle \ \ \forall u,w\in V
	.\] 
	Tale operatore è detto aggiunto hermitiano di $T$ e denotato con $T^*$
\end{defi}
\begin{defi}[operatore normale]
	Sia $V$ uno spazio vettoriale complesso dotato di prodotto hermitiano (forma hermitiana definita positiva), un operatore $L\in End(V)$ è normale se 
	\[
	L\circ L^* = L^*\circ L
	.\] 
\end{defi}
\textbf{Osservazione}\\
$L$ unitario, hermitiano, antihermitiano $ \Rightarrow$ $L$ diagonale
\begin{teo}
	Sono equivalenti le seguenti affermazioni:\\
	$1)$ $L$ è normale\\
	$2)$ esiste una base ortonormale di $V$ formata da autovettori di $L$
\end{teo}

%Inizio lezione 19
\newpage
\section{Diangonalizzazione unitaria di operatori normali}
	($\C^n$, prodotto hermitiano standard)\\
	Indichiamo $M^\star = \overline{M}^t$\\
	\begin{defi}
	 $M$ è normale se $MM^\star = M^\star M$\\
	\end{defi}
	\textbf{Nota}\\
	Sono normali le matrici\\ \begin{aligned}
		 \hspace{120px}&\text{unitarie} \ \ \ \ &MM^\star = Id\\
						       &\text{hermitiane} \ \ \ \ &M=M^\star\\
						       &\text{antihermitiane} \ \ \ &M = -M^\star
	 \end{aligned}\\
	\section{Classificazione delle isometrie}
	\begin{nome}
	$\cdot$ rotazioni\\
	$\cdot$ riflessioni\\
	$\cdot$ traslazioni\\
	$\cdot$ glissoriflessione $ = t_v\circ s_\alpha$ con $v\parallel \alpha^t$ (disegno de li mortacci sua)\\
$\cdot$ glissorotazioni $= t\circ R$ dove $v \parallel a$, $a$ asse di $R$ (altro disegno)\\
$\cdot$ riflessioni rotatorie $s_a\circ R$  $R$ rotazione di asse $\underline{a}$,  $s_\underline{a}$ è una riflessione rispetto ad una retta parallela ad $\underline{a}$
\end{nome}
\begin{teo}[Eulero 1776]
	Ogni isometria di $\mathbb{E}^3$ è di uno dei sei tipi sopra descritti
\end{teo}
\newpage
\section{Spazi Hermitiani}
	\begin{lemm}
		Sia $V$ uno spazio vettoriale su un campo $\R$\\
		Siano  $P,Q\in End(V)$ tali che $PQ=QP$. Allora, se  $V_\lambda$ è l'autospazio di autovalore $\lambda$ su $P$, risulta
		\[
		Q(V_\lambda)\subseteq V_\lambda
		.\] 
	\end{lemm}
	\begin{dimo}
		Sia $v\in V_ \lambda $ (cioè $P(v) = \lambda v)$. Dobbiamo vedere che $Qv\in V_ \lambda$.
		\[
		P(Q(v)) = (P\circ Q)(v) = (Q\circ P)(v) = Q( \lambda v) = \lambda Q(v)
		.\]
	\end{dimo}
	$(V,h)$ spazio Hermitiano (Spazio vettoriale complesso $h$ forma hermitiana definita positiva in $V$ )\\
	$\dim(V) < +\infty$
	 \begin{teo}
		Sia $(V,h)$ uno spazio hermitiano, $L\in End(V)$ operatore, sono equivalenti
		\begin{itemize}
			\item $L$ è normale (rispetto ad $h$)
			\item esiste una base ortonormale $B$ di  $V$ composta da autovettori per $L$
		\end{itemize}
	\end{teo}
	\begin{lemm}
		$(V,h)$ spazio hermitiano, $L\in End(V)$ normale\\
		sono equivalenti
		\begin{itemize}
			\item $Lv = \lambda v$
			\item  $L^\star v = \overline{ \lambda} v$
		\end{itemize}
		In particolare $ \lambda$ è l'autovalore per $L$ se e solo se $\overline{ \lambda}$ è autovalore per $ L^\star$ 
		\[
			V_ \lambda (L) = V_{\overline{ \lambda}}(L^\star)
		.\] 
	\end{lemm}
	\begin{dimo}
		Se $v = 0$ non c'è niente da dimostrare.\\
		Se $v \neq 0$ basta far vedere che se $v\in V_ \lambda (L)$ allora $v\in V_{\overline{ \lambda}}(L^\star)$. L'inclusione contraria segue da $L^{\star t} = L$
		 \[
		w\in V_ \lambda (L), \ \ v\in V_ \lambda (L)
		.\] 
		\begin{aligned}
			\hspace{80px}h(L^\star(v),w) &= h(v,L(w)) = h(v, \lambda w)\\
					&=\overline{ \lambda}h(v,w) = h(\overline{ \lambda}v,w)
		\end{aligned}\\
		\[
			h(L^\star(v) - \overline{ \lambda}v,w) = 0 \ \ \circledast
		\] 
		Per il lemma, siccome per ipotesi $L$ è normale, 
		 \[
		L^\star (v)\in V_\lambda (L), \ \ \overline{
		\lambda}v\in V_ \lambda (L)
		\] 
		\[
			\Rightarrow \ \ \ L^\star (v) - \overline{ \lambda} v\in V_ \lambda (L)
		\] 
		Quindi nella $\circledast$ posso prendere $w = L^\star (v) - \overline{ \lambda} v$, ottenendo 
		\[
			h(L^\star (v) - \overline{ \lambda} v, L^\star (v) - \overline{\lambda} v) = 0
		.\] 
		Poiché $h$ è definito positivo, segue\\
		\begin{aligned}
			&L^\star(v) - \overline{ \lambda}v = 0\\
			\text{cioè} \hspace{50px}  & L^\star (v) = \overline{ \lambda} v
		\end{aligned}\\
	\end{dimo}
	\textbf{Osservazione}\\
	Dal lemma segue $V_ \lambda(L) \perp V_\mu (L)$ se $ \lambda \neq \mu$
	\[
	v\in V_ \lambda, \ \ \ w\in V_\mu
	\] 
	\[
		\lambda h(v,w) = h( \lambda v,w) = h(Lv,w) = h(v,L^\star w) = h(v, \overline{\mu}w) = \mu h(v,w) \Rightarrow  h(v,w) = 0
	\]
	Dato che $ \lambda \neq \mu$
	\begin{nome}
		Chiamiamo $U(n)$ lo spazio delle matrici unitarie\\
	\end{nome}
	 \begin{teo}[Spettrale]
	 	$M$ è normale se e solo se $\exists A\in U(n) : \ A^tMA$ è ortogonale
	 \end{teo}
	\begin{dimo}[Teorema Spettrale]
		$1) \Rightarrow  2)$ Procediamo per induzione su $\dim V$,con base ovvia $\dim V = 1$ \\
		Supponiamo il teorema vero per gli spazi hermitiani di dimensione $\leq n-1$ e sia  $\dim_\C V = n$\\
		Sia  $v_1\in V$ un autovettore per $L$, che possiamo assumere di norma  $1$. Sia $V_1 = \C v_1, W = V_1^\perp$.\\
		Allora $V = V_1 \oplus W$.\\
		Poiché $V_1$ è $L$-invariante (per costruzione) e $L^\star$-invariante per il lemma precedente, lo stesso accade per  $W$.\\
		Inoltre $L|_W\in End(V)$ è normale.\\
		Per induzione, esiste una base $h|_W$-ortonormale formata da autovettori per $L|_W$, sia $\{v_2,\ldots,v_n\}.$ Allora $\{v_1,\ldots,v_n\}$ è una base $h$-ortonormale di $V$ formata da autovettori per $L$.\\
		$2) \Rightarrow 1)$. Sia $B = \{v_1,\ldots,v_n\}$ una base $h$-ortonormale di autovettori per $L$. Allora\\
		\begin{aligned}
			\hspace{80px}&[L]^B_B = \bigwedge = \matrice{ \lambda_1  &\ldots& 0\\
				 0  & \ddots & 0\\
			 0  &\ldots & \lambda_n}\\
			    &[L^\star]^B_B = \overline{[L]_B^B}^t = \overline{\bigwedge}\\
			    &[L\circ L^\star]^B_B = [L]^B_B[L^\star]^B_B = \bigwedge\overline{\bigwedge}=\overline{\bigwedge}\bigwedge = [L^\star]_B^B[L]^B_B = [L^\star \circ L]_B^B
		 \end{aligned} \\
		 Poiché la mappa $A \rightarrow [A]^B_B$ è un isomorfismo tra
		 $End(V)$ e $M_{nn}(\C)$, segue 
		 \[
		 L\circ L^\star = L^\star \circ L
		 .\] 
		 cioè $L$ è normale
	\end{dimo}
	 \hline \ \\[10px]
	 \textbf{Esercizio 1}\\[10px]
	$ L = \matrice{1&i\\-i&1}\ \ L^\star = \matrice{1&i\\-i&1} \Rightarrow $ $L$ matrice hermitiana\\
	Trovo ora il polinomio caratteristico\\
$t^2 - 2t = 0$ 
	che ha quindi autovalori $t = 0, t = 2$\\
	$v_0 = \C\matrice{1\\i}\ \ \ v_2 = \C\matrice{1\\-i}$\\
	$\langle \matrice{1\\i}, \matrice{1\\-i} \rangle  = 1 \cdot 2 + i\cdot i = 0$\\
	$ \langle \matrice{1\\i}, \matrice{1\\i} \rangle = 1\cdot 1 + i\cdot(-i) = 1-i^2 = 2$
	\[
		U = \matrice{1/\sqrt{2} &1/\sqrt{2}\\i/\sqrt{2}&-i/\sqrt{2}}\ \ \ \ \ \ U^-1LU = \matrice{0&0\\0&2}
	.\] 
	Il prodotto scalare standard è stato utilizzato per verificare che siano ortogonali, il secondo mi serve per normalizzare la matrice (di fatti divido per la radice del risultato) [I calcoli sono errati, guarda le slide]\ \\ \hline\newpage \ \\ 
	\textbf{Esempio 2}\\
		$L = \matrice{\sqrt{3}/2&-1/2\\1/2&\sqrt{3}/2}$ matrice ortogonale con determinante 1, quindi rotazione\\
		il polinomio caratteristico è $t^2 - \sqrt 3 t + 1$ gli autovalori sono quindi $t = \frac {\sqrt{3}\pm i}{2}$
		$v_{\frac {\sqrt{3}\pm i}{2}} = \C\matrice{i\\\pm 1}$
		 \[
			 U = \matrice{i/\sqrt 2 & i/\sqrt 2\\ 1/\sqrt 2 & - 1/\sqrt 2}
		.\] 
		\hline \ \\[10px]
		\textbf{Esempio 3}\\[8px]
			 $L = \matrice{1 +i&i\\-i&1+i} \ \ \ L^\star = \matrice{1-i&i\\-i&1-i}\\$
			 \[
				 LL^\star = \matrice{3&2i\\-2i&3} = L^\star L
			 .\] 
			 $t^2 - 2(i + 1) + 2i - 1 = 0\ \ t_1,t_2\\
			 v_{t_1} = \C\matrice{i\\1} \ \ v_{t_2} = \C\matrice{i\\-1}$\\
			 $U$ come nell'esercizio precedente\\ \ \hline \ \\
	\textbf{Osservazioni}\\
	1. È essenziale che $h$ sia definita positiva.\\
	\[
	 h(x,y) = x^tH\overline{y} \ \ \ M = \matrice{1&0\\0&-1}
	.\] 
	non è definita positiva $h(\icol{0\\1},\icol{0\\1}) = -1$
	\[
		L_A:\C^2 \rightarrow\C^2 \ \ A = \matrice{0&i\\i & -2}
	.\] 
	Dico che $L_A$ è autoaggiunto, quindi normale\\
	\begin{aligned}
		\hspace{80px}&h(L_AX,Y) = h(X,L_AY)\\
		&(L_AX)^tH\overline{Y} = X^tH\overline{L_AY}\\
		&X^tA^tH\overline{Y} = X^tH\overline{A}\overline{Y} \ \ \forall X,Y\\
		&A^tH = H\overline{A}\\
		&\matrice{0&u\\i&-2}\matrice{1&0\\0&-1} = \matrice{1&0\\0&-1}\matrice{0&-i\\-i&-2}\\
		&\hspace{47px}\matrice{0&-i\\i&2} = \matrice{0&-i\\i&2}
	\end{aligned}
	Calcolo il polinomio caratteristico di $A$ 
	\[
		\det\matrice{t& -i\\-i & t + 2} = t(t+2) + 1 = (t+1)^2
	.\] 
		Ma $A\neq \matrice{-1&0\\0&-1}$, in particolare non è diagonalizzabile\\
		2. Vediamo in dettaglio il fatto che $L|_W$ è normale\\
		Ritornando alla dimostrazione del teorema spettrlae, osserviamo che se $W$ è $L$-invariante è anche $L^\star$-invariante.\\
		Infatti, se $V = \bigoplus_\lambda V_ \lambda(L)$ (per esercizio da dimostrare)\\
		\begin{aligend}
			$W &= \bigoplus_ \lambda (V_ \lambda(L)\cap W)$\\
			   &=\bigoplus_ \lambda (V_{\overline{ \lambda}}(L^\star)\cap W)
			
		\end{aligend}\\
		$=> W$ è $L^\star$-invariante\\
		Adesso osservo che $(L|_W)^\star = (L^\star)|_W$\\
		\begin{aligend}
			&(L\left|_{W)}\circ(L\right|_W)^\star = (L|_W)\circ(L^star|_W) = \\
			&(L\circ L^\star)|_W = (L^\star\circ L)|_W = (L^\star|_W) \circ L|_W = (L|_W)^\star\circ L|_W
		\end{aligend}\\
		\hline \ \\ 
		\section{Richiami su spazi vettoriali duali}
		$V$ spazio vettoriale su $\K$ di dimensione finita
		 \[
		V^V = V^\star = Hom(V,\K)
		.\] 
		sia $A\leq V$
		 \[
			 Ann(A) = A^\# = \{f\in V^\star | f(a) = 0 \ \ \forall a \in A\}
		.\] 
		\textbf{Osservazioni}\\
		1) $A^\# $ è un sottospazio\\
		2) $A^{\#\#} = <A>$ \\
		\begin{aligned}
			\hspace{100px}&i:V \rightarrow V^{\star\star}\\
			&v\in V, \ \ f\in V^\star\\
			& i(v)(f) = f(v)
		\end{aligned}\\
	$V,W$ spazi vettoriali di dimensione finita $f\in Hom_\K(V,W)$, $f^\star\in Hom_\K(W^\star,V^\star)$, la trasposta di f è definita con $\phi\in W^\star$\\
	\hspace{40px}$f^\star(\phi) = \phi\circ f \\
	$\text{ }\hspace{100px}\includegraphics[scale=0.4]{funzione}\\
	\begin{defi}
	Definisco la  dualità standard su $V$ come 
	\[
	\langle \ , \  \rangle : V^\star\times V \rightarrow \K
	.\] 
	$\langle v, f \rangle = \langle f, v \rangle = f(v)$\\
	con questa proprietà
	\[
	\langle f(v), w^\star \rangle  = \langle v, f^\star(w^\star) \rangle 
	.\] 
\end{defi}
\hline \ \\
Ricordo che se $B = \{v_1,\ldots,v_n\}$ è una base di $V$ allora i funzionali $v_i^\star$ definiti da
 \[
	 \langle v_i^\star, v_j \rangle =\delta_{ij}
.\] 
per $1\leq i\leq n$ formano una base $B^\star$ di $V^\star$ detta base duale di $B$\\
Sia  $f:V \rightarrow W$ un'applicazione lineare, siano $B =\{v_1,\ldots,v_n\}, L = \{w_1,\ldots,w_m\}$ basi di $V,W$ consideriamo  $f^\star :W^\star \rightarrow V^\star$ Allora:\\
\begin{aligned}
	\hspace{120px}[f&]_B^B = [f^\star]^{B^\star}_{L^\star}^t	\\
		       & \storto{=} \ \ \ \hspace{18px} \storto{=}\\
		       &\hspace{-10px}(a_{ij}) \ \ \ \  (a^\star_{ij})
		        
\end{aligned}\\
\textbf{Tesi} \ \ $a_{ih} = a^\star_{hi}$\\
$f^\star(w^\star_i) = \sum^n_{i=1}a_{ij}^\starv_i^\star$\\
$f^\star(w_i^\star)(v_h) = \sum^n_{i=1}a^\star_{ij}v^\star_i(v_h) = \sum^n_{i=1}a_{ij}^\star\delta_{ih} = a^\star_{hi}$\\
\text{ }\ \ \storto{=}\\
$w_i^\star(f(w_h)) = w^\star_i(\sum^n_{i=1}a_{ih}w_i) = \sum^n_{i=1}a_{ih}w^\star_i(w_i)=$\\
$=\sum^n_{i=1}a_{ih}\delta_{ij}= a_{ih} $
\begin{teo}[Qualche proprietà importante]
	$f:V \rightarrow W$ lineare $\ \ f^\star : W^\star \rightarrow V^\star$\\
	$1) (Im f)^\# = \ker f^\star \\$
	 $2) (\ker f)^\# = Im f^\star$\\
	 $3) (\lambda f + \mu g)^\star = \lambda f^\star + \mu g^\star \ \ \ \ \ \ ( \lambda,\mu \in \K, g\in Hom(V,W)) $\\
	 $4) (h\circ f)^\star = f^\star\circ h^\star \hspace{40px} \ \ \ h:W \Rightarrow U $ lineare
\end{teo}
\begin{dimo}[Il punto 2, 3 e 4 vengono lasciati per esercizio]
	\begin{aligend}
	&1) $\emptyset\in (Im f)^\# $\\
	& \Leftrightarrow \forall w\in Imf \ \ \emptyset(w) = 0 \\
	& \Leftrightarrow \forall v \in V \emptyset(f(v)) = 0\\
	& \Leftrightarrow \emptyset \circ f = 0\\
	& \Leftrightarrow \emptyset \in kerf^\star
	\end{aligend}\\
	Quindi abbiamo visto che $(Imf)^\# = \ker F^\star$
\end{dimo}
\begin{prop}
	Sia $V$ uno spazio vettoriale di dimensione $n$ su $\K$ e $W$ un sottospazio. Allora
	\[
	\dim(W) + \dim W^\#  = n
	.\] 
\end{prop}
\begin{dimo}
	Da quanto visto, la mappa\\
	\begin{aligned}
		\hspace{80px}&Hom(V_1,V_2) \rightarrow Hom(V^star_2,V^star_1)\\
			     & \hspace{20px} \ f \ \ \ \ \ \ \ \  \rightarrow \ \ \  \ \ \ f^t
	\end{aligned}\\
	è un isomorfismo di spazi vettoriali. Inoltre $f$ è iniettiva (rispettivamente suriettiva) se e solo se $f^\star$ è suriettiva (rispettivamente iniettiva)\\
	Consideriamo la proiezione  $\pi:V \rightarrow V|_W :=U$ \\
	Poiché $\pi$ è suriettiva $\pi ^\star : U ^\star \rightarrow V^\star$ è iniettiva e 
	\[
	W^\#  = (\ker\pi)^\# = Im\pi^\star
	.\] 
	per cui 
	\[
	 \dim W^\# = \dim (Im \pi ^\star) = \dim U^\star = \dim V - \dim W
	.\] 
\end{dimo}
%lezione 2
\end{document}
