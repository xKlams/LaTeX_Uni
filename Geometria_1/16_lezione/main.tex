\documentclass[12px]{article}

\usepackage{amsmath}
\usepackage{amsthm}
\usepackage{mdframed}
\usepackage{amssymb}
\usepackage{nicematrix}
\usepackage{amsfonts}
\usepackage{tcolorbox}
\tcbuselibrary{theorems}
\usepackage{xcolor}
\usepackage{cancel}

\title{Lezione 15 Geometria I}
\date{2024-04-10}
\author{Federico De Sisti}
\newtheoremstyle{break}
  {1px}{1px}%
  {\itshape}{}%
  {\bfseries}{}%
  {\newline}{}%
\theoremstyle{break}
\newtheorem{theo}{Teorema}
\theoremstyle{break}
\newtheorem{lemma}{Lemma}
\theoremstyle{break}
\newtheorem{defin}{Definizione}
\theoremstyle{break}
\newtheorem{propo}{Proposizione}
\theoremstyle{break}
\newtheorem*{dimo}{Dimostrazione}
\theoremstyle{break}
\newtheorem*{es}{Esempio}

\newenvironment{dimo}
  {\begin{dimostrazione}}
  {\hfill\square\end{dimostrazione}}

\newenvironment{teo}
{\begin{mdframed}[linecolor=red, backgroundcolor=red!10]\begin{theo}}
  {\end{theo}\end{mdframed}}

\newenvironment{nome}
{\begin{mdframed}[linecolor=green, backgroundcolor=green!10]\begin{nomen}}
  {\end{nomen}\end{mdframed}}

\newenvironment{prop}
{\begin{mdframed}[linecolor=red, backgroundcolor=red!10]\begin{propo}}
  {\end{propo}\end{mdframed}}

\newenvironment{defi}
{\begin{mdframed}[linecolor=orange, backgroundcolor=orange!10]\begin{defin}}
  {\end{defin}\end{mdframed}}

\newenvironment{lemm}
{\begin{mdframed}[linecolor=red, backgroundcolor=red!10]\begin{lemma}}
  {\end{lemma}\end{mdframed}}

\newcommand{\icol}[1]{% inline column vector
  \left(\begin{smallmatrix}#1\end{smallmatrix}\right)%
}

\newcommand{\irow}[1]{% inline row vector
  \begin{smallmatrix}(#1)\end{smallmatrix}%
}

\newcommand{\matrice}[1]{% inline column vector
  \begin{pmatrix}#1\end{pmatrix}%
}

\begin{document}
	\maketitle
	\newpage
	\section{Ultima Parte teorica prima del compito}
	$O(2) = SO(2) \cup O(2)\setminus SO(2)$
	\[
		R_\theta = \matrice{cos\theta & -sin\theta\\ sin\theta& cos\theta}\ \ \ A_\theta = R_\theta A_\theta = \matrice{cos\theta& sin\theta\\ sin\theta & -cos\theta}
	.\] 
	\[
		R_\theta R_\varphi = R_{\theta + \varphi}
	.\] 
	\[
		A_\theta A_ \varphi = R_{\theta - \varphi}
	.\] 
	\begin{defi}[Riflessione]
		Isometria che fissa puntualmente una retta (detta asse della riflessione)
	\end{defi}
	$E$ piano euclideo
	$C\in E, r\subset E$ retta $\exists s,t$ rette passanti per $C$ tali che
	\[
	 R_{c,\theta} = \rho_r \circ\rho_s = \rho_t\circ\rho_r	.\] 
	"e viceversa"\\
	Possiamo fissare $c = 0 \ \ p_r = A_{o,\alpha}.$ Allora
	\[
		R_\theta = A_\alpha\circ A_{\alpha - \theta} = A_{\theta + \alpha}\circ A_\alpha
	.\] 
	dove $\rho_r = A_\alpha$ e $A_{\alpha - \theta} \equiv \rho_s$
\\
Il viceversa segue, sostituendo $c\equiv 0$, da $A_\alpha\circ A_\beta = R_{\alpha - \beta}$
 \[
	 R_{C,\theta}\circ R_{D,\varphi} \ \ \rightarrow \ \ \text{rotazione di angolo } \theta + \varphi \text{ Se } \theta + \varphi\neq 2k\pi,\ \ k\in \mathbb{Z}
.\] 
altrimenti è una traslazione (che è l'identità $ \LeftrightarrowC = D$)\\
Se $C = D$ chiaramente $R_{C,\theta}\circ R_{C, \varphi} = R_{C,\theta + \varphi}\\$
Se $C \neq D$ sia $r$ la retta per $C$ e $D$ Per la parte precedente possiamo scrivere
\[
	R_{C,\theta} = \rho_t\circ\rho_r, \ \ \ \ R_{D,\varphi} = \rho_r\circ\rho_s
.\] 
per certe rette $s,t$ 
\[
	T = R_{C,\theta}\circ R_{D,\varphi} = \rho_t\circ\rho_r\circ\rho_r\circ\rho_s
.\] 
Se $s,t$ sono incidenti allora per la parte precedente $T$ è una rotazione, altrimenti $s\parallel t$\\
\textbf{TODO disegno}\\
In coordinate rispetto ad un riferimetno cartesiano $Oe_1e_2$ Se $P\equiv\icol{x_1\\x_2}$ 
\[
	(R_{C,\theta}\circ R_{D,\varphi})(P) \ \ \ \text{ ha coordinate}
.\] 
\[
	R_\rho(R(x-d) + d-x) + x  
.\]
dove $c,d$ sono i vettori delle coordinate di $C,D $ rispettivamente\\
\begin{aligned}
	&\underline{R_{\theta + \varphi}(x - d)} + R_\theta(d-c) + c\\[-5px]
	&\hspace{-4px}\text{ parte lineare}
\end{aligend}\\[10px]
$T$ T è una translazione se e solo se $\theta + \varphi = 2k\pi, k\in\mathbb{Z}$ e in tal caso
\[
T(x) = x + R_\theta(d-c) = (d-c)
.\] 
che è l'identità se e solo se $d = c$ cioè $D=C$
\begin{defi}[Glissoriflessione]
	Una glissoriflessione è un'isometria di un piano euclideo ottenuta come composizione $t_v\circ\rho_r$ di una riflessione di asse $r$ con una traslazione $t_v\neq Id$ con $v\neq 0, v\parallel r$
\end{defi}
\textbf{TODO disegno}\\
\begin{teo}[Charles, 1831]
	Un'isometria di un piano euclideo che fissa un punto è una rotazione o una riflessione a seconda che sia diretta o inversa. Un'isometria senza punti fissi è una traslazione o una glissoriflessione a seconda che sia diretta o inversa
\end{teo}
\begin{dimo}
	Sia $f\in Isom(E)$\\ Se $f$ ha un punto fisso abbiamo già visto che $f$ è una rotazione se è diretta o una riflessione se $f$ è inversa
\end{dimo}
\end{document}
