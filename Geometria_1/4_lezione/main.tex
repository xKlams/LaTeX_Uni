\documentclass[12px]{article}

\usepackage{amsmath}
\usepackage{amsthm}
\usepackage{mdframed}
\usepackage{amssymb}
\usepackage{nicematrix}
\usepackage{amsfonts}

\title{Lezione 4 Geometria I}
\date{ 2024-03-16}
\author{Federico De Sisti}
\maketitle

\newtheoremstyle{break}
  {1px}{1px}%
  {\itshape}{}%
  {\bfseries}{}%
  {\newline}{}%
\theoremstyle{break}
\newtheorem{theo}{Teorema}
\theoremstyle{break}
\newtheorem{lemma}{Lemma}
\theoremstyle{break}
\newtheorem{defin}{Definizione}
\theoremstyle{break}
\newtheorem{propo}{Proposizione}
\theoremstyle{break}
\newtheorem*{dimo}{Dimostrazione}
\theoremstyle{break}
\newtheorem*{es}{Esempio}

\newenvironment{teo}
  {\begin{mdframed}\begin{theo}}
  {\end{theo}\end{mdframed}}

\newenvironment{lem}
  {\begin{mdframed}\begin{lem}}
  {\end{lem}\end{mdframed}}

\newenvironment{prop}
  {\begin{mdframed}\begin{propo}}
  {\end{propo}\end{mdframed}}

\newenvironment{def}
  {\begin{mdframed}\begin{defin}}
  {\end{defin}\end{mdframed}}

\newcommand{\icol}[1]{% inline column vector
  \left(\begin{smallmatrix}#1\end{smallmatrix}\right)%
}

\newcommand{\irow}[1]{% inline row vector
  \begin{smallmatrix}(#1)\end{smallmatrix}%
}

\begin{document}
\newpage
\section{Formula di Grassmann affine}
Richiami dalla scorsa lezione\\
Dati $\Sigma_i = p_i + W_i, \ \ i=1,2$ sottospazi affini (di $(A,V,+)$) allora: \\
\[\Sigma_1\cap\Sigma_2\neq\emptyset \Leftrightarrow \ \ \overrightarrow{p_1p_2}\in W_1 + W_2.\]
\[
	\Sigma_1\vee\Sigma_2 = p_1 + (W_1 + W_2 + <\overrightarrow{p_1p_2}>)
.\] 
Inoltre $\Sigma_1, \Sigma_2$ si dicono:\\
\textbf{incidenti} se $\Sigma_1\cap\Sigma_2\neq\emptyset$\\
\textbf{paralleli} se $W_1\subseteq W_2$ o $W_2\subseteq W_1$ \\
\textbf{sghembi} se $\Sigma_1\cap\Sigma_2 = \emptyset$ e $W_1\cap W_2 = \{0\}$
\begin{prop}[Fromula Grassmann per spazi affini]
	Siano $\Sigma_1, \Sigma_2 $ sottospazi affini di $A$, Allora 
	\[
	dim(\Sigma_1 \vee \Sigma_2) \leq dim\Sigma_1 + dim\Sigma_2 - dim(\Sigma_1\cap\Sigma_2)
	.\] 
	e vale l'uguaglianza se $\Sigma_1, \Sigma_2$ sono incidenti o sghembi\\
	si usa la notazione $dim(\emptyset) = -1$
\end{prop}
\begin{dimo}
	- Supponiamo $\Sigma_1,\Sigma_2$ incidenti, allora esiste
\begin{gather*}
	p_0 \in\Sigma_1\cap\Sigma_2 \\
	\Sigma_1 = p_0 + W_1, \Sigma_2 = p_0 + W_2\\
	\Sigma_1\cap\Sigma_2 = p_0 + W_1\cap W_2, \Sigma_1\vee\Sigma_2 = p_0 + W_1 + W_2
\end{gather*}
	dunque vale l'uguaglianza per Grassman vettoriale\\
	- Sia ora $\Sigma_1\cap\Sigma_2 = \emptyset$ allora $\Sigma_i = p_i + W_i$ ~ $i = 1, 2$\\
	risulta $\overrightarrow{p_1p_2}\notin W_1 + W_2$ (per lemma)\\
	\begin{gather*}
		dim(\Sigma_1\vee\Sigma_2) = dim(W_1 + W_2 + <\overrightarrow{p_1p_2}) = dim(W_1+W_2) + 1\leq \\ \leq dim(W_1) + dim(W_2) - (-1) = dim(W_1) + dim(W_2) + dim(\Sigma_1\cap\Sigma_2)
	\end{gather*}
	e vale l'uguaglianza se e solo se $dim(W_1) + dim(W_2) = dim(W_1 + W_2)$ ovvero $W_1\cap W_2 = {0}$ ovvero se $\Sigma_1, \Sigma_2$ sono sghembi $\hfill\square$	
\end{dimo}
\newpage
\begin{prop}
	siano $\Sigma_1, \Sigma_2$ sottospazi affini di $\mathbb{A}^n(\mathbb{K})$ definiti dai sistemi lineari \[
	A_iX = b_i \ i = 1,2
	.\]
	Allora:\\
	(a) $\Sigma_1, \Sigma_2$ sono incidenti se e solo se \[
		rk\begin{pNiceArray}{c|c}
			A_1 & b_1\\
			\hline
			A_2 & b_2 \\
			\end{pNiceArray} = rk \begin{pNiceArray}{c}
				A_1 \\
				\hline
				A_2
			\end{pNiceArray}
	.\] 
	detto r tale rango, $dim(\Sigma_1\cap\Sigma_2) = n-r$\\
	(b) $\Sigma_1, \Sigma_2$ sono sghembi se e solo se \[
		rk\begin{pNiceArray}{c|c}
			A_1 & b_1\\
			\hline
			A_2 & b_2 \\
			\end{pNiceArray} \geq rk \begin{pNiceArray}{c}
				A_1 \\
				\hline
				A_2
			\end{pNiceArray} = n
	.\] 
	(c) Se\[
		rk\begin{pNiceArray}{c|c}
			A_1 & b_1\\
			\hline
			A_2 & b_2 \\
			\end{pNiceArray} \geq rk \begin{pNiceArray}{c}
				A_1 \\
				\hline
				A_2
			\end{pNiceArray} = r < n
	.\] 
	allora $\Sigma_1$ (rispetto a $\Sigma_2$) contiene un sottospazio affine di dimensione $n-r$ parallelo a $\Sigma_2$ (rispetto a $\Sigma_1$)
\end{prop}
\begin{dimo}
	(a) $\Sigma_1 \cap\Sigma_2 \neq \emptyset \Leftrightarrow$ il  sistema è compatibile quindi tutto segue da Rochè-Capelli\\[10px]
	(b) la disuguaglianza tra i ranghi dice che $\Sigma_1\cap\Sigma_2 = \emptyset$;\\ il fatto che $rk \begin{pNiceArray}{c}
		A_1 \\
		\hline
		A_2
	\end{pNiceArray} = n$ implica che $W_1\cap W_2 = {0}$\\
(c) Di nuovo la disuguaglianza dei  ranghi implica $\Sigma_1\cap\Sigma_2 = \emptyset$;\\
Se ora $W_1\cap W_2 = W$ allora $dim(W_1\cap W_2) = n - r$\\
Scelto $p_1 \in \Sigma_1$ risulta\\
$p_1 + W \subset\Sigma_1$ \ ($W_1 \cap W_2 = W$ sottospazio di $W_1$)\\
e $W \subset W_2 \Rightarrow p_1 + W$ è parallelo a $\Sigma_2$ e $dim(p_1 + W) = dim(W) = n - r \hfill\square$
\end{dimo}
\begin{es}
	$\mathbb{A} \ \pi_1,\pi_2$ piani distinti\\
	$A_1,A_2$ vettori riga $(A_1 = (a_{11} \ a_{12} \ a_{13})\\
	C = \begin{pNiceArray}{c c}
		A_1 & b_1\\
		A_2 & b_2
	\end{pNiceArray} \in M_{2,4}(\mathbb{R})$\\
	piani distinti $\Rightarrow \ rk(C) = 2$\\
	$rg \begin{pNiceArray}{c}
		A_1 \\
		\hline
		A_2 \\
	\end{pNiceArray} = 2 \ \Rightarrow \pi_1\cap\pi_2$ è una retta\\
	$rg \begin{pNiceArray}{c}
		A_1 \\
		\hline
		A_2 \\
	\end{pNiceArray} = 1 \ \Rightarrow \pi_1\cap\pi_2 = \emptyset$ piani paralleli poiché $W_1 = W_2$\\
\end{es}
\newpage
$ \mathbb{A}^4, \pi_1\pi_2$ piani distinti tali che $rk(A_i|b_i) = 2$ \\
\[
C = \begin{pNiceArray}{c|c}
	A_1 & b_1 \\
	\hline
	A_2 & b_2 \\
\end{pNiceArray} \in M_{45} \ \ rk(C)\leq 4
.\]
\[
\begin{tabular}{|c|c|c|}
	\hline
	rk \icol{A_1 \\ \hline  A_2} & rk(C) & \pi_1\cap \pi_2 \\ 
	\hline
	4 & 4 & \{p\} \\
	3 & 4 & \emptyset \text{ e } W_1, W_2 \text{ hanno una direzione in comune} \\
	3 & 3 & r \\
	2 & 3 & \emptyset\\
	\hline
\end{tabular}
\]
\end{document}
