\documentclass[12px]{article}

\usepackage{amsmath}
\usepackage{amsthm}
\usepackage{mdframed}
\usepackage{amssymb}
\usepackage{nicematrix}
\usepackage{amsfonts}
\usepackage{tcolorbox}
\tcbuselibrary{theorems}
\usepackage{xcolor}
\usepackage{cancel}

\title{Lezione 14 Geometria 1}
\date{2024-04-04}
\author{Federico De Sisti}
\newtheoremstyle{break}
  {1px}{1px}%
  {\itshape}{}%
  {\bfseries}{}%
  {\newline}{}%
\theoremstyle{break}
\newtheorem{theo}{Teorema}
\theoremstyle{break}
\newtheorem{lemma}{Lemma}
\theoremstyle{break}
\newtheorem{defin}{Definizione}
\theoremstyle{break}
\newtheorem{propo}{Proposizione}
\theoremstyle{break}
\newtheorem*{dimo}{Dimostrazione}
\theoremstyle{break}
\newtheorem*{es}{Esempio}

\newenvironment{dimo}
  {\begin{dimostrazione}}
  {\hfill\square\end{dimostrazione}}

\newenvironment{teo}
{\begin{mdframed}[linecolor=red, backgroundcolor=red!10]\begin{theo}}
  {\end{theo}\end{mdframed}}

\newenvironment{nome}
{\begin{mdframed}[linecolor=green, backgroundcolor=green!10]\begin{nomen}}
  {\end{nomen}\end{mdframed}}

\newenvironment{prop}
{\begin{mdframed}[linecolor=red, backgroundcolor=red!10]\begin{propo}}
  {\end{propo}\end{mdframed}}

\newenvironment{defi}
{\begin{mdframed}[linecolor=orange, backgroundcolor=orange!10]\begin{defin}}
  {\end{defin}\end{mdframed}}

\newenvironment{lemm}
{\begin{mdframed}[linecolor=red, backgroundcolor=red!10]\begin{lemma}}
  {\end{lemma}\end{mdframed}}

\newcommand{\icol}[1]{% inline column vector
  \left(\begin{smallmatrix}#1\end{smallmatrix}\right)%
}

\newcommand{\irow}[1]{% inline row vector
  \begin{smallmatrix}(#1)\end{smallmatrix}%
}

\newcommand{\matrice}[1]{% inline column vector
  \begin{pmatrix}#1\end{pmatrix}%
}

\begin{document}
	\maketitle
	\newpage
	\section{Precisazione}
	Siano $S,T$ sottospazi affini in uno spazio euclideo $\delta$ di dimensione $n$. Diciamo che $S,T$ sono ortogonali se, posto $S = p + U, \ \ T= q + W, \ \ p\in S,q\in T$\\
	 $U,W$ sottospazi vettoriali di $V$,
	 \[
		 \langle U, W \rangle = 0\ \ \text{ se } dim(S) + dim(T) < n
	 .\] 
	 \[
		 \langle U^\perp, W^\perp \rangle  = 0 \ \ \text{ se } dim(S) + dim(T) \geq n
	 .\] 
	 \textbf{Esempi}\\
	 1. Due rette $r, s$ in $\mathbb{E}^3$ con vettori direttori $v_s, v_r$ \\
	 \textbf{COMPLETARE CON DISEGNI}\\
	2. retta e piano in $\mathbb{E}^3$\\
	\textbf{COMPLETARE CON DISEGNI}\\
	3. due piani in $\mathbb{E}^3$\\
	\textbf{COMPLETARE CON DISEGNI}\\
	sarò sincero, non si capisce un cazzo
	\section{Esercizi foglio 4}
	\textbf{es 3}\\[10px]
	\begin{aligned}
		&r: \begin{cases}
			x_1 - x_3 = 0\\
			x_2 - x_3 = 0
		\end{cases} \ \ r' = \begin{cases}
			x_1 + x_2 = 0\\
			x_3 = -1
		\end{cases}
	\end{aligned}\\[10px]
	Posizione reciproca\\
	La direzione di $r$ è $\mathbb{R}\icol{1\\1\\1},$ quella di $r'= \mathbb{R}\icol{1\\-1\\0}$\\
	Essendo tali vettori indipendenti, le rette non sono parallele\\
	$p'= \icol{1\\-1\\-1}\in r', \ \ O\in r\\
	\overrightarrow{Op'} = \icol{1\\-1\\-1}\nin \langle \icol{1\\1\\1}, \icol{1\\-1\\0} \rangle \\$
	quindi $r,r'$ sono sghembi\\
	$S = \pi\cap\pi'\ \ \ \pi$ piano per $r$ parallelo a $v\wedge v'$ \ \\
	$\pi'$ piano per $r'$ parallelo a $v\wedge v'$
	\[
		v\wedge v'\ \ \matrice{1& 1 & 1\\1 & -1&0} \ \ v\wedge v' = \icol{1\\1\\-2}
	.\] 
	\[
		\pi : \icol{x_1\\x_2\\x_3} = t\icol{1\\1\\1} + s\icol{1\\1\\-2}
	.\] 
	\[
		\pi' : \icol{x_1\\x_2\\x_3} = \icol{0\\0\\-1} + t\icol{1\\-1\\0} + s\icol{1\\1\\-2}
	.\] 
	trasformiamo in coordinate cartesiane\\
	$\pi \rightarrow det\matrice{x_1&x_2&x_3\\1&1&1\\1&1&-2} = 0 \rightarrow x_1 - x_2 = 0$\\
analogo per $\pi'$\\
 \textbf{es 4}\\
 proiezione ortogonale su $\pi$\\
 simmetria ortogonale di asse $\pi$ \\
\[
\pi : 2x_1 + x_2 - x_3 + 2 = 0
.\]
vettore normale a $\pi\ \ \ \ P_0\in\pi$\\
$p(P) = P_0 + \widetilde{p}(\overrightarrow{p_0p})\\
\sigma(P) = P_0 + \widetilde{\sigma}(\overrightarrow{p_0p})\\$
scelgo $p_0\in\pi$\ \ \ \ \ \ $P_0 = \icol{0\\0\\2}$\\
$W$ giacitura di $\pi = \langle \icol{1\\-2\\0}, \icol{0\\1\\1} \rangle  \ \ \ \ \mathbb{R}^3 = W\oplus W^\perp$\\
Dobbiamo decomporre $\overrightarrow{P_0P}$ rispetto a $W\oplus W^\perp$ \ \ $W^\perp = \mathbb{R}\icol{-1\\-1\\1}\\
\icol{x_1\\x_2\\x_3-2} = \alpha\icol{1\\-2\\0} + \beta\icol{0\\1\\1} + \gamma\icol{-2\\-1\\1}$\\
Questo poi è solo un sistema noioso da risolvere\\
\[
	p\icol{x_1\\x_2\\x_3} = (\text{ guarda le lavagnate, è un super vettore} ) 
.\] 
sulle lavagnate trovi anche il risultato della simmetria ma non lo svoglimento
\\
\textbf{es 5}\\
$\p \begin{cases}

\end{cases}
\end{document}
