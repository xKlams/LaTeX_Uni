\documentclass[12px]{article}

\title{Lezione 21 Geometria 1}
\date{2024-04-24}
\author{Federico De Sisti}

\input{../../setup.tex}

\begin{document}
	\maketitle
	\newpage
	\section{Nuove informazioni sulle forme bilineari}
	$V$ spazio vettoriale su $\R$ \\
	Ricordiamo che una forma bilineare è un'applicazione 
	\[
	b:V\times V \rightarrow \R
	.\] 
	Abbiamo già osservato che se $A = [b]_B$\\
	$X = [v]_B, \ \ Y=[w]_B$
	 \[
	b(v,w) = X^tAY
	.\] 
	Come cambia $[b]_B$ se cambio $B$ \\
	$B = \{v_1,\ldots,v_n\}$ \ \ \  X=[v]_B \ \ X'=[v]_B'\\
	$B' = \{v_1',\ldots,v_n'\}$ \ \ Y = [w]_B \ \ Y' = [w]_B'\\
	$A = [b]_B \ \ A' = [b]_{B'}$\\
	$$b(v,w) = X^tAY = X'^TA'Y'$$
	$X = MX', \ \ Y=MY'\ \ \ M = [Id_V]_B^B$\\
	 \begin{aligend}
		 \hspace{80px} &(MX')^tA(MY') = X'^t A'Y'\\
		 &X'M^t AMY' = X'^tA'Y'\\
		 & \ \ \ A' = M^tAM
	 \end{aligend}\\
\begin{defi}
	Diciamo che due matrici $A,B$ sono congruenti se esiste una matrice invertibile $M$ tale che  $B = M^tAM$
\end{defi}
\begin{prop}
	Due matrici rappresentano la stessa forma bilineare in basi diversi se e solo se sono congruenti
\end{prop}
\textbf{Osservazione}\\
1. La congruenza è una relazione di equivalenza\\
2. Il rango è invariante per la congruenza\\
3. Per matrici reali invertibili, il segno del determinante è invariante\\
4. Se $M$ è ortogonale\\
\[
	M^tAM = M^{-1}AM
.\] 
\hline \ \\
Se ho una forma bilineare $b:V\times V \rightarrow \mathbb{K}$ posso definire due applicazioni $V \rightarrow V^\star$ nel modo seguente.\\
Fissato $v\in V$, \ prendo \ \  \begin{aligned}
	&b_v(w) = b(v,w)\\
	&b_v'(w) = b(w,v)
\end{aligned}\\
È chiaro che $b_v, b'_v\in V^\star$ (usiamo il fatto che $b$ è bilineare)\\
Dunque ho due applicazioni $V \rightarrow V^\star$
\[
\delta_b(v) = b_v  \ \ \ \delta_b '(v) = b'_v
.\] 
\begin{defi}
	Il rango di una funzione bilineare è il rango di una qualsiasi matrice che la rappresenta
\end{defi}
\begin{defi}
	Una forma bilineare è non degenere se ha rango (massimo) $\dim V$
\end{defi}
\begin{prop}
Sia $V$ uno spazio vettoriale di dimensione finita,
\[
	b:V\times V \rightarrow \K \text{ una forma bilineare}
.\] 
Sono equivalenti 
\begin{itemize}
	\item $b$ è non degenere ovvero $b(v,v)= 0 \Leftrightarrow v = 0$
	\item $\forall v\in V, v\neq 0\ \ \exists w\in V : \ \ b(v,w)\neq 0$
	\item $\forall w\in V, \ w\neq 0 \ \ \exists v\in V: b(v,w) \neq 0$
	\item $\delta_b :V \rightarrow V^\star$ è un isomorfismo
	\item $\delta_b' : V \rightarrow V^*$ è un isomorfismo
\end{itemize}
\end{prop}
\begin{dimo}
	Sia $B = \{v_1,\ldots,v_n\}$ una base di $V$ e sia $A = [b]_B$\\
	$1) \Rightarrow  2)$ per ipotesi $\det A \neq 0$ se  $X = [v]_B \ \ X\neq 0  \Rightarrow  X^tA\neq 0$ \\
	quindi esiste $Y\in \K^n: X^tAY\neq 0$.\\ Se  $w\in V$ è tale che $[w]_B=Y$ ho dimostrato che  $b(v,w) = X^tAY\neq 0$ \\
	$2) \Rightarrow 1)$ Riscrivendo l'ipotesi in coordinate abbiamo\\
	\begin{aligned}
	\hspace{80px}&\forall X\neq 0 \ \exists Y : \ \ X^tAY\neq 0	\\
	& \Rightarrow X^t A\neq 0 \ \ \forall X\neq 0 \Rightarrow  A \text{ è invertibile}
	\end{aligned}\\
	$1) \Leftrightarrow 3)$ è come sopra\\
	$2) \Rightarrow 4)$ Poiché $\dim V = \dim V^\star$ basta vedere che $\delta_b$ è iniettava, cioè  $\ker\delta_b=\{0\}$\\
	 \begin{aligend}
		&v\in \ker\delta_b \ \ \Rightarrow \ \ \delta_b(v) = b_v \ \text{ è il funzionale nullo, cioè}\\
		&b_v(w) = 0 \ \ \forall w\in V\\
		&b_v(w) = b(v,w) \Rightarrow  v = 0 
	\end{aligend}\\
	$4) \Rightarrow  2)$ Dato $v\neq 0$,  $\delta_b(v) = b_v\neq 0$ perché  $\delta_b$ è un isomorfismo, \\quindi esiste  $w\in V:$\\
	\[b(v,w) = b_v(w)\neq 0\]
	$3) \Leftrightarrow 5)$ è simile a $2) \Leftrightarrow 4)$
\end{dimo}
\section{Caso Simmetrico}
\[
b(v,w) = b(w,v)
.\] 
\textbf{Osservazione}\\
$b$ è simmetrica se e solo se lo è qualsiasi matrice che la rappresenta.
\textbf{Dato} $S\subset V$ definiamo
\[
	S^\perp = \{v\in V| b(v,s) = 0 \ \ \forall s\in S\}
.\] 
\textbf{Esercizio} $S^\perp$ è un sottogruppo e, $S^\perp = <s>^\perp$
\begin{defi}
	Due sottospazi $U,W$ si dicono ortogonali se\\
	\[
	Y\subseteq W^perp \ \( \Leftrightarrow W\subset U^\perp )\\\]
\end{defi}
\begin{defi}
	$v\in V$ si dice isotropo se $b(v,v) = 0$\\
\end{defi}
\begin{defi}
	$\ker b = \{ v\in V|b(v,w)=0 \ \ \forall w\in V\} = V^\perp$
\end{defi}
 \textbf{Osservazione}\\
 $b$ è non degenere se e solo se $\ker b = \{0\}$\\
 \begin{prop}
 	Sia $b$ non degenere, $W\subseteq V$ sottospazio,\\
	Allora, se $\delta_b:V \rightarrow V^\star$ è l'isomorfismo canonico indotto da $b$, $\delta_b(W^t) = W^\star.$ In particolare risulta sempre $\dim W + \dim W^\perp = \dim V$
 \end{prop}
 \textbf{Nota}\\
 Non è vero, anche nel caso non degenere, che $V = W\oplus W^\perp$
\begin{dimo}
	$w\in W^\perp\ \ \ \delta_b(w) = b_w$ Voglio vedere che\\
	$b_w\in W^\# \ \ \ b_w(w')=b(w,w')=0 \ \ \forall w'\in W$ \\
	Quindi $\delta_b(W^\perp)\subseteq W^\#$\\
	Prendo ora  $f\in W^\# $; poiché  $b$ è non degenere, $\delta_b$ è un isomorfismo, quindi esiste $v\in V$
	 \[
	f = \delta_b(v) = b_v \ \ b(v,w) = b_v(w) = 0 \ \ \forall w \Rightarrow v\in W^\perp
	.\] 
	quindi $f = \delta(b_v)$ con $v\in W^\perp$
\end{dimo}
\begin{prop}
	Sia $V$ spazio vettoriale, $W\subset V$ sottospazio, $b\in Bi(V).$ Sono equivalenti:
	\begin{itemize}
		\item $V = W\oplus W^\perp$ 
		\item $b|_W$ è non degenere
	\end{itemize}
\end{prop}
\begin{lemm}
	$\ker b|_W = W\cap W^\perp$
\end{lemm}
\begin{dimo}[lemma]
	$w\in \ker b|_W \Leftrightarrow b(w,w') = 0 \ \ \forall w'\in W \Leftrightarrow w \in W\cap W'$
\end{dimo}
\begin{dimo}[proposizione]
	$1) \Rightarrow 2)$segue dal lemma perché dall'ipotesi $W\cap W^\perp = \{0\}$ \\
	$2) \Rightarrow  1)$ Sia $\{w_1,\ldots,w_s\}$ una base di $W$ \\
	Per ipotesi $A = (b(w_i,w_j))$ è invertibile, in particolare dato $v\in V$, il sistema lineare
	\[
		* \ \ A\matrice{x_1\\\vdots\\x_s} = \matrice{b(v,w_1_\\\vdots\\b(v,w_s)}
	\]
	ha soluzione unica. Poniamo 
	\[
	w = v - \sum^s_{h=1}x_jw_j
	.\] 
	Notiamo che * significa
	\[
	\sum^s_{h=1}b(v_h,w_j)x_h=b(v,w_j) \ \ 1\leq j \leq s
	.\] 
	Calcoliamo \\
	\begin{aligned}
		&b(w,w_i) = b(v - \sum^s_{h=1}x_hw_h, w_j) =  b(v,w_j) - \sum^s_{h=1}x_hb(w_h,w_j) = b(v,w_j) = \\
		&=b(v,w_i) - b(v,w_i) = 0
	\end{aligned}\\
	Poiché i $\{w_i\}$ sono una base di $W$, risulta $b(w,u) = 0\ \ \ \forall u\in W$, cioè  $w\in W^\perp$ Allora
	\[
	v = w + \sum^s_{h=1}x_hw_h
	.\] 
	Pertanto $V = W + W^\perp$, per ipotesi  $W\cap W^\perp = \ker b|_W = \{0\}$, quindi $V = W\oplus W^\perp$
\end{dimo}
\end{document}
