\documentclass[12px]{article}

\title{Lezione 8 Geometria I}
\date{2024-03-18}
\author{Federico De Sisti}

\usepackage{amsmath}
\usepackage{amsthm}
\usepackage{mdframed}
\usepackage{amssymb}
\usepackage{nicematrix}
\usepackage{amsfonts}
\usepackage{tcolorbox}
\tcbuselibrary{theorems}
\usepackage{xcolor}
\usepackage{cancel}

\newtheoremstyle{break}
  {1px}{1px}%
  {\itshape}{}%
  {\bfseries}{}%
  {\newline}{}%
\theoremstyle{break}
\newtheorem{theo}{Teorema}
\theoremstyle{break}
\newtheorem{lemma}{Lemma}
\theoremstyle{break}
\newtheorem{defin}{Definizione}
\theoremstyle{break}
\newtheorem{propo}{Proposizione}
\theoremstyle{break}
\newtheorem*{dimo}{Dimostrazione}
\theoremstyle{break}
\newtheorem*{es}{Esempio}

\newenvironment{dimo}
  {\begin{dimostrazione}}
  {\hfill\square\end{dimostrazione}}

\newenvironment{teo}
{\begin{mdframed}[linecolor=red, backgroundcolor=red!10]\begin{theo}}
  {\end{theo}\end{mdframed}}

\newenvironment{nome}
{\begin{mdframed}[linecolor=green, backgroundcolor=green!10]\begin{nomen}}
  {\end{nomen}\end{mdframed}}

\newenvironment{prop}
{\begin{mdframed}[linecolor=red, backgroundcolor=red!10]\begin{propo}}
  {\end{propo}\end{mdframed}}

\newenvironment{defi}
{\begin{mdframed}[linecolor=orange, backgroundcolor=orange!10]\begin{defin}}
  {\end{defin}\end{mdframed}}

\newenvironment{lemm}
{\begin{mdframed}[linecolor=red, backgroundcolor=red!10]\begin{lemma}}
  {\end{lemma}\end{mdframed}}

\newcommand{\icol}[1]{% inline column vector
  \left(\begin{smallmatrix}#1\end{smallmatrix}\right)%
}

\newcommand{\irow}[1]{% inline row vector
  \begin{smallmatrix}(#1)\end{smallmatrix}%
}

\newcommand{\matrice}[1]{% inline column vector
  \begin{pmatrix}#1\end{pmatrix}%
}

\newcommand{\C}{\mathbb{C}}
\newcommand{\K}{\mathbb{K}}
\newcommand{\R}{\mathbb{R}}


\begin{document}
	\maketitle
	\newpage
	\section{Complementi}
	$ \mathbb{A} $ spazio affine reale con associato spazio vettoriale $V$\\
	\begin{defi}[Semiretta]
		Possiamo definire la semiretta di origine $Q\in \mathbb{A} $ e direzione $v\in V\setminus \{0\}$
		\[
		P = Q + tv, t\geq 0\ \ (\overrightarrow{QP} = tv, t\geq 0)
		.\] 
	\end{defi}
	\begin{defi}[Segmento]
		Possiamo definire il segmento di estremi $A, B\in \mathbb{A} \ \ (A\neq B) $
		\[
		P = A + t \overrightarrow{AB} \ \ \ \ 0\leq t\leq 1
		.\] 	
	\end{defi}
	i punti $p_1,\ldots,p_t$ che dividono il segmento $AB$ in $t$ parti uguali sono dati, cioè
\[
	\overrightarrow{AP_1} = \overrightarrow{p_1p_2} = \overrightarrow{p_2p_3} = \ldots = \overrightarrow{p_{t-1}B}
.\] 
sono dati da \[
	\overrightarrow{AP_i} = \frac{i}{t}\overrightarrow{AB} \ \ 1 \leq i\leq t-1
.\] 
	In un riferimento affine $Oe_1\ldots,e_n$, in cui
	\[
		A = \icol{a_1\\ . \\.\\.\\ a_n}, \ \ B = \icol{b_1\\ .\\.\\.\\b_n} \ \ P_i = \icol{x_1\\.\\.\\.\\ x_n}
	.\] 
	\[
		\icol{x_1^i - a_1\\.\\.\\.\\x_n^i - a_n} = \frac{i}{t}\icol{b_1-a_1\\.\\.\\.\\ b_n = a_n}
	.\] 
	\[
		\icol{x_1^i\\.\\.\\x_n^i} = \frac{1}{t}\icol{ib_1 (t - i)a_1\\.\\.\\.}
	.\] 
	in particolare, il punto medio del segmento $AB$ ha coordinate
	\[
		\icol{\frac{a_1+b_1}{2}\\.\\.\\.\\\frac{a_n + b_n}{2}}
	.\] 
	$A,B,C$ non allineati
	\[
	\overrightarrow{AP} = t\overrightarrow{AB} + k\overrightarrow{AC}
	.\] 
	se $t,n \geq 0$ e $t + n\leq 1$ allora abbiamo un triangolo ABC\\
	se $0\leq t,n \leq 1$ abbiamo il parallelogramma individuato da $A,B,C$\newpage
	\textbf{Osservazione}\\
	Questo procedimento funziona in ogni dimensione, Ad esempio se A,B,C,D sono quattro punti indipendenti
	\[
	\overrightarrow{AP} = t\overrightarrow{AB} + k\overrightarrow{AC} + v\overrightarrow{AD}
	.\] 
	se $0\leq,t,n,v\leq 1$ tetraedro di vertici ABCD\\
	se $n,t,v\geq 0$ e $n+t+v\leq 1$ si ha un parallelogramma\\
	in generale dati $p_0,\ldots,p_k$ punti indipendenti:\\
	\[
	\overrightarrow{p_0p} = \sum^k_{i=1}t_ip_0p_i, \ \ \sum^k_{i=1}t_i \leq 1
	.\] 
	definisce il \textbf{$k$-simplesso di vertici $p_0,\ldots,p_k$}\\
	\begin{defi}[Sottosineime Convesso]
	$S\subseteq \mathbb{A} $ si dice Convesso se per ogni $A,B\in S$ il segmento $AB$ è contenuto in $S$
\end{defi}
\
\section{Cambiamenti di riferimento affine}
Sia $(A,V,+)$ uno spazio affine $n$-dimensionale
\[
	R = Ee_1,\ldots,e_n;\ \ \ \ R'= Ff_1,\ldots,f_n \ \ \ \text{due riferimenti affini}
.\] 
\[
	\varepsilon = \{e_1,\ldots,e_n\}, \ \ \Gamma = \{f_1,\ldots,f_n\}
\]
\[
\overrightarrow{EP} = \sum^n_{i=1}x_ie_i\ \ \ \overrightarrow{FE} = \sum^n_{i=1}b_ie_i \ \ \ \overrightarrow{FP} = \sum^n_{i=1}y_if_i
.\] 
\[
	A = (e_{ij}) = \prescript{}{\varepsilon}{(Id_V)}_\Gamma
.\] 
\begin{gather}
\overrightarrow{FP} = \overrightarrow{FE} + \overrightarrow{EP} = -\overrightarrow{EF} + \overrightarrow{EP} = -\sum^n_{i=1}b_ie_i + \sum^n_{i=1}x_ie_i\\
\overrightarrow{FP} = \sum^n_{i=1}y_if_i = \sum^n_{i=1, j=1}y_ia_{ij}-e_i
\end{gather}
Comparando $(1),(2)$ troviamo \[
X = AY + b
.\] 
\[
	\left(\frac{1}{X}\right) = \begin{pNiceArray}{c|c}
		1 & 0 \\
		\hline
		b & A \\ 
	\end{pNiceArray} = \left(\frac{1}{Y}\right)
.\] 
\[
	Y = A^{-1}X - A^{-1}b
.\] 
\section{Esercizi}
Trovare l'affinità $F: \mathbb{A} ^2 \rightarrow \mathbb{A} ^2$ tale che
\[
	f\icol{1\\3} = \icol{1\\1} \ \ f(r) = r', \ \ f(s) = s
.\] 
dove $r: x = 0, \ \ s: 2x - y = 0 \ \ r': x - 2y = 1$\\
\hline \ \\
$f$ è del tipo $f\icol{x_1\\x_2} = \icol{e & b\\ c & d}\icol{x_1\\x_2} + \icol{\alpha_1\\ \alpha_2}$ con $ad + bc \neq 0$\\
\[
\begin{pNiceArray}{ccc}
	1 & 0 & 0\\
	\alpha_1 & a & b \\
	\alpha_2 & c & d \\
\end{pNiceArray} \ \ \ \ \ 
\left(\frac{1}{f\icol{x_1\\x_2}}\right) = \begin{pNiceArray}{ccc}
	1 & 0 & 0\\
	\alpha_1 & a & b \\
	\alpha_2 & c & d \\
\end{pNiceArray} \icol{1\\x_1\\x_2}
.\]
Imponiamo le condizioni del testo
\[
	f\icol{1\\ 3} = \icol{1\\1} \rightarrow \begin{equations}
		\begin{cases}
			a + \alpha_1 + 3b = 1\\
			\alpha_2 + c + 3d = 1
		\end{cases}
	\end{equations}
.\] 
Un punto in $r (x_1 = 0)$ è del tipo $\icol{0\\t} \Rightarrow f\icol{0\\t}\in r\ \ \ \forall t\in \R$
\[
\icol{\akpha_1 + bt\\\alpha_2 + dt}\in v'
.\] 
\[
x_1 - 2x_2 = 1 \Leftrightarrow \alpha_1 + bt - 2 )\alpha_2 + dt) = 1 \ \ \forall t
.\] 
\[
\Rightarrow \alpha_1 - 2\alpha_2 = 1
.\]
\[
b - 2d = 0
.\] 
Sicuramente il punto di S ha coordinate \\
\begin{gather*}
	\icol{t \\2t}, \ \ \text{e imponiamo} \ \ f\icol{t\\2t}\in s \\
	f\icol{t\\2t} = \icol{\alpha_1 + at + 2bt\\\alpha_2 + ct + 2dt} \\
	2(\alpha_1 + at + 2bt) = \alpha_2 + ct + 2dt \\
	\Rightarrow 
	\begin{equations}
		\begin{cases}
			2\alpha_1 - alpha_2 = 0\\
			2a + 4b - c - 2d = 0 
		\end{cases}
	\end{equations}\\
	a = - \frac{2}{3} \ \ b = \frac{2}{3} = c \ \ d = \frac{1}{3} \ \ \alpha_1 = - \frac{1}{3} \ \ \alpha_2 = - \frac{2}{3} \\
	f\icol{x_1 \\ x_2} = \icol{-\frac{1}{3} -\frac{2}{3}x_1 + \frac{2}{3}x_2\\ - \frac{2}{3} + \frac{2}{3}x_1 + \frac{1}{3}c_2}
\end{gather*}
\\
\hfill \ \\
\[
	\pi_1 = \icol{0 \\ 0\\ 0\\ 0 } + <\icol{1 \\ 0\\0\\0}> \ \ \pi_2: \icol{0\\0\\1\\0} + <\icol{1 \\ 0 \\0 \\0},\icol{0\\0\\0\\1}>
.\] 
Dire se sono incidenti, paralleli o sghembi\\
\hfill
\\
\pi_1: \begin{equations}
	\begin{cases}
		x_3 = 0 \\
		x_4 = 0
	\end{cases}
\end{equations} \ \ \ \ 
\pi_2: \begin{equations}
	\begin{cases}
		x_2 = 0\\
		x_3 = 1
	\end{cases}
\end{equations}
\\
$Dalle equazioni cartesiane è chiaro che $pi_1\cap pi_2 = \emptyset$, quindi $\pi_1\pi_2$ non sono incidenti la giacitura di $\pi_1,\pi_2$ sono $W_1 = \R e_1 + \R e_2, \ \ W_2 = \R e_1 + \R e_2$\\
dunque \textbf{APPUNTI DA RECUPERARE}
\\
\\
\\
\begin{align*}
	f:\  &A \rightarrow A\ \\
	   & R_1 \ \ \ \ R_1'\\
	   & R_2 \ \ \ \ R_2'
\end{align*} 
\[
	[f]^{R_1'}_{R_1} = \begin{pNiceArray}{ c | c}
	1 & 0 \\
	\hline
	b & A \\
\end{pNiceArray} 
.\] 
\[
	[f]^{'R_2}_{R_2} = [Id]^{R_2}_{R'_1}[f]^{R_1'}_{R_1}[Id]^{R_1}_{R_2}
.\] 
Troviamo l'affinità che manda ordinatamente $A, B,C$ in $A',B',C'$ ove \[
	A = \icol{1\\1} \ \ B = \icol{3\\a} \ \ C = \icol{4\\4} 
.\] 
\[
	A\ = \icol{-1\\0} \ \ B'\icol{-3\\2} \ \ C'= \icol{1\\4}
.\]
$R_1 = \{ A, \{\overrightarrow{AB},\overrightarrow{AC}\}\}$ è un riferimento affine\\
$R_2 = \{A',\{\overrightarrow{A'B'}, \overrightarrow{A'C'}\}\}$ è un riferimento affine
\[
	[F]^{R_2}_{R_1} = \begin{pNiceArray}{c | c c}
	1 & 0 & 0 \\
	\hline
	0 & 1 & 0 \\
	0 & 0 & 1 \\
\end{pNiceArray} 
.\]
\[
	f(\overrightarrow{AB}) = \alpha \overrightarrow{A'B'} + \beta \overrightarrow{B'C'}
.\] 
\[
	R = \{\icol{0\\0},\icol{1\\0},\icol{0\\1}\}
.\] 
\[
	[f]^R_R = [Id]^R_{R_2}[f]^{R_2}_{R_1}[Id]^{R_1}_R
.\]
\[
	R_2 = \{A', \{\overrightarrow{AB}, \overrightarrow{A'C'}\}\} = \{\icol{-1\\0},\{\icol{-4\\2}, \icol{8\\4}\}\}
.\] 
Quindi la matrice del cambio di base da $R$ a $R_2$ è\\[5px]
\[
	[Id]^R_{R_2} = \begin{pNiceArray}{c | c c}
		1 & 0 & 0 \\
		-1 & -4 & 8 \\
		0 & 2 & 4\\
	\end{pNiceArray} 
.\]\\
Analogamente si fa con $R_1 = \{\icol{1\\1},\{\icol{2\\1},\icol{3\\3}\}\}$\\
\hline
\newpage
\section{Forme Bilineari e Simmetriche}
$V$ Spazio vettoriale su $  \mathbb{K}$
\begin{defi}
	Una funzione $g:VxV \rightarrow \mathbb{K}$ Si dice \textbf{Forma bilineare} se è lineare in ciascuna variabile fissata l'altra
\end{defi}
in altre parole:\\
\[
	g(\alpha v_1 + \betaa v_2,v_3) = \alpha g(v_1,v_3) + \beta g(v_2, v_3) \ \ \forall \alpha,\beta\in\mathbb{K}\ \ \forall \alpha,\beta\in V\ \ \forall v_1,v_2,v_3\in V
.\]
\begin{defi}
	g si dice \textbf{simmetrica} se 
	\[
	g(v_1,v_2) = g(v_2,v_1)\ \ \forall v_1,v_2\in V
	.\] 
\end{defi}
\begin{es}
	Sia A una atrice quadrata $nxn$
	\[
		\text{Allora} \ \ \ g_A(x,y) = X^tAY
	.\] 
	è una forma bilineare su \mathbb{K}^n
\end{es}
\begin{es}
	$g_A$ è bilineare con 
	\begin{gather*}
		A = \icol{2& 1 \\ -1 & 3} \\
		f\left(\icol{x_1\\x_2},\icol{y_1\\y_2}\left) = \irow{x_1 x_2}\icol{2y_1 + y_2\\-y_1 + 3y_2} = x_1(2y_1 + y_2) + x_2(-y_1 + ey_2)= \\ = 2x_1y_1 + x_1y_2 - x_2y_1 + 3x_2y_2
	\end{gather*}
\end{es}
\textbf{Osservazione} \\
$g_A$ è simmetrica se e solo se $A$ è simmetrica \\
\begin{es}[Importante]
	in $\mathbb{K}^n$ prendiamo $A = I_n$
	\[
		g_{I_m}(X,Y) = X^tY = \sum^n_{i=1}x_iy_i
	.\] 
\end{es}
Se $g$ è una forma bilineare simmetrica su $V$ e $B = \{v_1,\ldots,v_n\}$ è una base di V, definisco la matrice di $g$ rispetto a $B$ come \[
	[g]_B \rightarrow a_{ij} = g(v_i,v_j) \ \ \ \ 1 \leq i,j \leq n
.\] 
\[
	g(v,w) = g(\sum^n_{i=1}x_iv_i, \ \ \sum^n_{i=1}y_iv_i) = \sum^n_{i,j=1} x_iy_ig(v_i,v_j) = \sum^n_{i,j=1} x_iy_ia_{ij} = X^tAY
.\] 
\textbf{Ricorda:}
$X^t$ è la matrice trasposta di X
\newpage
\section{Prodotto Scalare}
V spazio vettoriale Reale
\begin{defi}[Prodotto Scalare]
	Un prodotto scalare su $V$ è una forma bilineare simmetrica \\$< , >: V\bigtimes V \rightarrow \mathbb{R}$ tale che \\
	\begin{gather*}
		<v,v> \geq 0 \ \ \forall v\in V\\
		<v,v> = 0 \Leftrightarrow v=0
	\end{gather*}
\end{defi}
\begin{nome}
	$1. v,w\in V$ si dicono \textbf{ortogonali} se \[
	<v,w> = 0
	.\] 
	$2.$ $|| v || = \sqrt{<v,v>}$ è la norma di v\\
	$3.$ In $\mathbb{R}^n$, $<\icol{x_1\\.\\.\\.\\x_n},\icol{y_1\\.\\.\\.\\y_n}> = \sum^n_{i=1}x_iy_i$
	è detto \textbf{prodotto scalare standard}
	\[
		||\icol{x_1\\.\\.\\x_n}|| = \sqrt{\sum^n_{i=1}x_i^2}
	.\] 
\end{nome}
\newpage
\begin{prop}[Disuguaglianza di Schwarz]
\[v,w\in V \ \ \ \ <v,w>^2 \leq <v,V><w,w>.\]
e vale l'uguaglianza se e solo se v,w sono dipendenti
\end{prop}
\begin{dimo}
	Se w=0 la disuguaglianza è ovvia, quindi possiamo assumere $w\neq 0$. Per $v,w\inV, a,b\in\R$
	\begin{gather*}
	0\leq <av + bw, av + bw> = a<v,av + bw> + b<w,av + bw> =\\
	= a(a<v,v> + b<v,w>) + b(a<w,v> + b<w,w>)i =\\
	= a^2<v,v> + 2ab <v,w> + b^2 <w,w>
\end{gather*} 
Dove abbiamo utilizzato la simmetria del prodotto scalare $<v,w> = <w,v>$\\
Notiamo che vale l'uguaglianza solo se $av + bw = 0$, cioè $v,w$ sono paralleli.\\
La relazione
\[
a^2<v,v> + 2ab <v,w> + b^2 <w,w> \geq 0
.\] 
vale per ogni scelta di $a,b$. \\
Prendo $a = <w,w>$ e $b = -<v,w>$\\
\[
0\leq <w,w>^2<v,v>-2<w,w><v,w>^2 + <v,w>^2<w,w>
.\]
Poiché $W \neq 0, \ \ <w,w> > 0$ quindi posso dividere la relazione precedente per $< w,w>$, per altro senza cambiare verso dato che il prodotto scalare è definito positivo
\[
0\leq <w,w><v,w> - <v,w>^2
.\] 
ovvero \[
<v,w>^2 \leq <v,v><w,w>
.\] 
\end{dimo} \\
\textbf{Osservazione}\\
$|<v,w>|\leq ||v||||w||$\\
\textbf{Proprietà della lunghezza}\\
1. $\forall v\in V \ \ ||v|| \geq 0$ e $||v|| = 0 \Leftrightarrow v = 0$\\
2. $||\alpha v|| = |\alpha| ||v||\ \ \ \forall \alpha\in \R, \ \ \forall v\in V$ \\
3. $||v+w|| \leq ||v|| + ||w|| \ \ \ \forall v,w\in V$
\end{document}
