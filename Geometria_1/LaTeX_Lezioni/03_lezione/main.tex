\documentclass[12px]{article}


\title{Lezione 3 Geometria I}
\date{2024-03-21}
\author{Federico De Sisti}

\usepackage{amsmath}
\usepackage{amsthm}
\usepackage{mdframed}
\usepackage{amssymb}
\usepackage{nicematrix}
\usepackage{amsfonts}
\usepackage{tcolorbox}
\tcbuselibrary{theorems}
\usepackage{xcolor}
\usepackage{cancel}

\newtheoremstyle{break}
  {1px}{1px}%
  {\itshape}{}%
  {\bfseries}{}%
  {\newline}{}%
\theoremstyle{break}
\newtheorem{theo}{Teorema}
\theoremstyle{break}
\newtheorem{lemma}{Lemma}
\theoremstyle{break}
\newtheorem{defin}{Definizione}
\theoremstyle{break}
\newtheorem{propo}{Proposizione}
\theoremstyle{break}
\newtheorem*{dimo}{Dimostrazione}
\theoremstyle{break}
\newtheorem*{es}{Esempio}

\newenvironment{dimo}
  {\begin{dimostrazione}}
  {\hfill\square\end{dimostrazione}}

\newenvironment{teo}
{\begin{mdframed}[linecolor=red, backgroundcolor=red!10]\begin{theo}}
  {\end{theo}\end{mdframed}}

\newenvironment{nome}
{\begin{mdframed}[linecolor=green, backgroundcolor=green!10]\begin{nomen}}
  {\end{nomen}\end{mdframed}}

\newenvironment{prop}
{\begin{mdframed}[linecolor=red, backgroundcolor=red!10]\begin{propo}}
  {\end{propo}\end{mdframed}}

\newenvironment{defi}
{\begin{mdframed}[linecolor=orange, backgroundcolor=orange!10]\begin{defin}}
  {\end{defin}\end{mdframed}}

\newenvironment{lemm}
{\begin{mdframed}[linecolor=red, backgroundcolor=red!10]\begin{lemma}}
  {\end{lemma}\end{mdframed}}

\newcommand{\icol}[1]{% inline column vector
  \left(\begin{smallmatrix}#1\end{smallmatrix}\right)%
}

\newcommand{\irow}[1]{% inline row vector
  \begin{smallmatrix}(#1)\end{smallmatrix}%
}

\newcommand{\matrice}[1]{% inline column vector
  \begin{pmatrix}#1\end{pmatrix}%
}

\newcommand{\C}{\mathbb{C}}
\newcommand{\K}{\mathbb{K}}
\newcommand{\R}{\mathbb{R}}


\begin{document}
	\maketitle
	\newpage
	\section{Nella lezione precedente...}
	Abbiamo introdotto i sottospazi affini di $(A,V)$ come i sottospazi del tipo \[
		p + W \ \ \ \ W\subseteq V \ \ \text{sottospazio vettoriale}
	.\] 
	Ricordiamo anche che $p + W = q + W \Leftrightarrow \overrightarrow{PQ}\in W$
	\section{Nuova lezione}
	\textbf{Osservazione} \\
	Se $\Sigma_1 = p_1 + W_1, \ \ \Simga_2 = p_2 + W_2$ sono sottospazi affini , la loro intersezione, se non vuota, è un sottospazio affine. Infatti $p\in\Sigma_1\cap\Sigma_2$ 
	\[
	\Sigma_1\cap\Sigma_2=p + W_1\cap W_2
	.\] 
	\begin{lemm}
		$\emptyset\neq S\subset A \ \ \ \ p,q\in S$\\
		$H_p = \{\overrightarrow{px}\ \ |\ \ x\in S\} \ H_q =\{ \overrightarrow{qy}\ \ |\ \ y\in S\}$\\
		Allora $<H_p> = <H_q>$ e $p + <H_p> = q + <H_q>$ \\(sottospazio generato da $S$)
	\end{lemm}
	\begin{dimo}
		\begin{aliged}
		&v_0 = \overrightarrow{pq} \ \ v_0\in H_p \ \ -v_0 = \overrightarrow{qp}\in H_q \\
		&H_p\ni \overrightarrow{px} = \overrightarrow{pq} + \overrightarrow{qx} = v_0 + \overrightarrow{qx}\in <H_q>\\
		& H_p \ \subseteq \ <H_q>\ \  \Rightarrow\ \ \ <H_p>\ \subseteq \ <H_q>\\
		&H_q\ni \overrightarrow{qy} = \overrightarrow{qp} + \overrightarrow{py}\in <H_q>\ \Rightarrow  \ <H_q>\ \subseteq \ <H_p>
		\end{aliged} \\
		Quindi \begin{aligend}
			&<H_p> = <H_q> \\
			&\overrightarrow{pq}\in<H_p> = <H_q>\\
			&p + <H_p> = q + <H_q>
		\end{aligend}
	\end{dimo}
	\begin{nome}
		$\Sigma_1,\Sigma_2$ sottospazi affini \[
			\Sigma_1 \vee \Sigma_2 := \text{sottospazio generato da } \Sigma_1\cup\Sigma_2
		.\] 
	\end{nome}
	\newpage
	\begin{lemm}
		Siano $\Sigma_i = p_i + W_i, \ \ \ i = 1,2$ sottospazi affini. Allora\\
		(a) $\Sigma_1\cap\Sigma_2\neq\emptyset \Leftrightarrow \overrightarrow{p_1p_2}\in W_1+W_2\\$
		(b) $\Sigma_1\vee\Sigma_2 = p_1 + (W_1 + W_2 + <\overrightarrow{p_1p_2}>)$
	\end{lemm}
	\begin{dimo}
		(a) $p_0\in\Sigma_1\cap\Sigma_2$ allora
		\begin{aligend}
			&\Sigma_1 = p_0 + W_1 \ \ \Simga_2 = p_0+W_2\\
			&\exists w_i\in W_i, \ \ i = 1,2 \ \ \text{ t.c }\\
			&p_1 = p_0 + W_1, p_2 = p_0 + W_2\\
			&\overrightarrow{p_1p_2} = w_2-w_1\in W_1 + W_2\\
		\end{aligend}
		Viceversa, se $\overrightarrow{p_1p_2} = w_1 + w_2, w_1\in W_1, w_2\in W_2 $\\
		\begin{aligned}
			&p_2 = p_1 + \overrightarrow{p_1p_2} = p_1 + w_1 + w_2\\
			&p_2 - w_2 = p_1 + w_1 \in \Sigma_1\cap\Sigma_2
		\end{aligned}
		(2) Dato $x\in\Sigma_1\cup\Sigma_2$, risulta \\
		$\overrightarrow{p_1x}\in W_1$ se $x\in\Sigma_1$ \\ 
		oppure
		\[
		\overrightarrow{p_1x}\in\overrightarrow{p_1p_2}+ W_2 \ \ (\overrightarrow{p_1x} = \overrightarrow{p_1p_2} + \overrightarrow{p_2x})
		.\] 
		Dunque la giacitura di $\Sigma_1\vee\Sigma_2$ è \[
		W_1 + W_2 + <\overrightarrow{p_1p_2}>
		.\] 
	\end{dimo}
	\section{Posizioni Reciproche di sottospazi affini}
	\begin{defi}
		Siano $\Sigma_1,\Sigma_2$ sottospazi affini di $(A,V)$ di giacitura rispettivamente $W_1,W_2$ Diciamo che \\
		1) $\Sigma_1,\Sigma_2$ sono \textbf{incidenti}, se \Sigma_1\cap\Sigma_2\neq\emptyset\\
		2) $\Sigma_1,\Sigma_2$ sono \textbf{paralleli} se $W_1\subseteq W_2$ p $W_2\subseteq W_1$\\
		3) $\Sigma_1,\Sigma_2$ sono \textbf{sghembi} se $\Sigma_1\cap\Sigma_2 = \emptyset$ e $W_1\cap W_2 = \{0\}$
	\end{defi}
	\textbf{Osservazione} \\
	Queste posizioni non sono mutuamente esclusive e non costituiscono tutte le possibilità\\
	\newpage
	\section{Esercizi Elementari}
	\textbf{Esercizio 1} \\
	Dire se $p = \icol{1\\0\\2}$ appartiene alla retta per $\icol{1\\5\\4}$
	e direzione $\icol{1\\1\\1}$
	\textbf{Svolgimento}
	Scriviamo l'equazione parametrica della retta
	\[
		\icol{x_1\\x_2\\x_3} = \icol{1\\5\\4} + t\icol{1\\1\\1}
	.\]
	\[
		\icol{1\\0\\2} = \icol{1\\5\\4} + t\icol{1\\1\\1} \ \ \ \ \ \begin{cases}
			t = 0\\
			t = -5
		\end{cases}
	.\] 
	alternativamente avrei potuto cercare le coordinate cartesiane\\
	\hline \ \\
	\textbf{Esercizio 2} \\
	Scrivere le equazioni parametriche e cartesiane per il piano contenente 
	\[
		A = \icol{1\\1\\1}, B = \icol{1\\1\\0},C = \icol{1\\0\\0}
	.\]
	\begin{aligned}
	&P = A + t\overrightarrow{AB} + s\overrightarrow{AC}\\
	&\icol{x_1\\x_2\\x_3} = \icol{1\\0\\0} + t\left(\icol{1\\1\\0}-\icol{1\\0\\0}\right) + s\left(\icol{1\\1\\1} - \icol{1\\0\\0}\right)\\
	&\icol{x_1\\x_2\\x_3} = \icol{1\\0\\0}+ t\icol{0\\1\\0} + s\icol{0\\1\\1}\\
	&\icol{x_1-1\\x_2\\x_3}\in <\icol{0\\1\\0},\icol{0\\1\\1}>\\
	&det\icol{0&1&0\\0&1&1\\x_1-1&x_2&x_3} = 0  \ \ \ \Rightarrow \ \ x_1 = 1
	\end{aligned} \\[10px]
	\hline \ \\
	\textbf{Esercizio 3} \\
	Scrivere equazioni per il piano identificato dalla retta \\ $\icol{x_1\\x_2\\x_3} = \icol{1\\2\\3} + t\icol{0\\1\\0}$ e dal punto $\icol{0\\0\\0}$ \\[10px]
	\textbf{Svolgimento} \\ 
	modo 1, scelgo due punti distinti sulla retta e riduco al punto precedente\\
	modo 2, sia $q = \icol{1\\2\\3}, \ v = \icol{0\\1\\0}$ e $O = \icol{0\\0\\0}$\\ considero il piano $P = q + tv + s\overrightarrow{Oq}$ \\
	\begin{aligend}
		&\icol{x_1\\x_2\\x_3}=\icol{1\\2\\3} + t\icol{0\\1\\0} + s\icol{1\\2\\3}\\
		&det\matrice{x_1-1&x_2-2&x_3-3\\0&1&0\\1&2&3} = 0 \\
		&3(x_1-1) - (x_3-3) = 0\\
		&3x_1 -x_3 = 0
	\end{aligend}\\
	Fascio di piani di asse una retta $r$ è l'insieme dei piani che contengono $r$
	\[
	r = \begin{cases}
		a_1x_1+a_2x_2+a_3x_3+a_4 = 0\\
		b_1x_1+b_2x_2+b_3x_3+b_4=0
	\end{cases}
	.\] 
	Equazione del fascio
	\[
	\lambda(a_1x_1+a_2x_2+a_3x_3+a_4) + \mu(b_1x_1+b_2x_2+b_3x_3+b_4) = 0 \ \ \ \lambda,\mu\in \mathbb{K}
	.\] 
	Ogni piano del fascio si ottiene con una coppia $(\lambda,\mu)\in \mathbb{K}^2.$ Coppie proporzioneali per un fattore non nulla invidiano lo stesso piano
\end{document} 
