\documentclass[12px]{article}

\title{Lezione 7 Geometria I}
\date{2024-03-14}
\author{Federico De Sisti}

\usepackage{amsmath}
\usepackage{amsthm}
\usepackage{mdframed}
\usepackage{amssymb}
\usepackage{nicematrix}
\usepackage{amsfonts}
\usepackage{tcolorbox}
\tcbuselibrary{theorems}
\usepackage{xcolor}
\usepackage{cancel}

\newtheoremstyle{break}
  {1px}{1px}%
  {\itshape}{}%
  {\bfseries}{}%
  {\newline}{}%
\theoremstyle{break}
\newtheorem{theo}{Teorema}
\theoremstyle{break}
\newtheorem{lemma}{Lemma}
\theoremstyle{break}
\newtheorem{defin}{Definizione}
\theoremstyle{break}
\newtheorem{propo}{Proposizione}
\theoremstyle{break}
\newtheorem*{dimo}{Dimostrazione}
\theoremstyle{break}
\newtheorem*{es}{Esempio}

\newenvironment{dimo}
  {\begin{dimostrazione}}
  {\hfill\square\end{dimostrazione}}

\newenvironment{teo}
{\begin{mdframed}[linecolor=red, backgroundcolor=red!10]\begin{theo}}
  {\end{theo}\end{mdframed}}

\newenvironment{nome}
{\begin{mdframed}[linecolor=green, backgroundcolor=green!10]\begin{nomen}}
  {\end{nomen}\end{mdframed}}

\newenvironment{prop}
{\begin{mdframed}[linecolor=red, backgroundcolor=red!10]\begin{propo}}
  {\end{propo}\end{mdframed}}

\newenvironment{defi}
{\begin{mdframed}[linecolor=orange, backgroundcolor=orange!10]\begin{defin}}
  {\end{defin}\end{mdframed}}

\newenvironment{lemm}
{\begin{mdframed}[linecolor=red, backgroundcolor=red!10]\begin{lemma}}
  {\end{lemma}\end{mdframed}}

\newcommand{\icol}[1]{% inline column vector
  \left(\begin{smallmatrix}#1\end{smallmatrix}\right)%
}

\newcommand{\irow}[1]{% inline row vector
  \begin{smallmatrix}(#1)\end{smallmatrix}%
}

\newcommand{\matrice}[1]{% inline column vector
  \begin{pmatrix}#1\end{pmatrix}%
}

\newcommand{\C}{\mathbb{C}}
\newcommand{\K}{\mathbb{K}}
\newcommand{\R}{\mathbb{R}}


\begin{document}
	\maketitle
	\newpage
	\section{Esercizi Vari}
	\textbf{Piccola definizione per esercizio}\\
	$ \mathbb{A}^n_ \mathbb{K}$   Aff$ (\mathbb{A} ^n_ \mathbb{K}) = \{f_{a,b}| A\in GL(n, \mathbb{K}), b\in \mathbb{K}^n\}$
\begin{gather*}
	f_{A,b}(X) = AX + b \\
	\begin{pNiceArray}{c|c}
		1 & 0 \\
		\hline
		b & A \\
	\end{pNiceArray} \\
\end{gather*}
\textbf{Esercizio 1}\\
\hline \ \\ 
\begin{align*}
	&f: \mathbb{A}^3_{\mathbb{R}} \rightarrow \mathbb{A}^3_{\mathbb{R}}\\
	&f\icol{1\\2\\0} = \icol{2\\-1\\1}, \ \ \ f\icol{1\\3\\1} = \icol{3\\-1\\0} \\
	&\varphi(e_1) = e_1+e_3, \ \ \varphi(e_2) = e_1-e_2 \\
	& \text{e chiamiamo} \\
	&p_0 = \icol{1\\2\\0}, \ \ q_0 = \icol{2\\-1\\1}, \ \ p_1 = \icol{1\\3\\1}, \ \ q_1 = \icol{3\\-1\\0}
\end{align*}
Dove $\varphi$ è la parte lineare di f.\\
Trovare l'espressione di f in coordinate affini canoniche \\e trovare i punti fissi  di $f$.\\
\textbf{Svolgimento}
\begin{align*}
	&\varphi\icol{1\\0\\0} = \icol{1\\0\\1} \ \ \ \varphi\icol{0\\1\\0} = \icol{1\\-1\\0} \ \ \varphi( \overrightarrow{p_0p_1}) = \overrightarrow{q_0q_1}\\
	&	\varphi(\icol{1\\3\\1} - \icol{1\\2\\0}) = \icol{3\\-1\\0} - \icol{2\\-1\\1} \\
	&\varphi\icol{0\\1\\1} = \icol{1\\0\\-1} \\
	&\varphi\icol{0\\0\\1} = \varphi(\icol{0\\1\\1}-\icol{0\\1\\0}) = \varphi\icol{0\\1\\1} - \varphi\icol{0\\1\\0} = \icol{1\\0\\-1}-\icol{1\\-1\\0} = \icol{0\\1\\-1}
\end{align*}
Se $\varepsilon = \{e_1,e_2,e_3\}$ è la stessa base standard di $\mathbb{R}^3$
\[
	[\varphi]^\varepsilon_ \varepsilon = \icol{1 & 1 & 0\\ 0 &-1&1\\1&0&-1}
.\] 
\begin{align*}
	&f\icol{x_1\\x_2\\x_3} = \icol{2\\-1\\1} + \icol{1 & 1 & 0\\ 0 &-1&1\\1&0&-1}\icol{x_1-1\\x_2-2\\x_3}\\
	&f\icol{x_1\\x_2\\x_3} = \icol{1 & 1 & 0\\ 0 &-1&1\\1&0&-1}\icol{x_1\\x_2\\x_3} + \icol{-1\\1\\0}\\
	&f\icol{x_1\\x_2\\x_3} = \icol{x_1+x_2-1 \\-x_2+x_3+1\\x_1-x_3}
\end{align*}
Dove abbiamo utilizzato il fatto che $F(p) = f(p_0) + \varphi(\overrightarrow{p_0p}) = q_0 + \varphi(\overrightarrow{p_0p})$ \\
\textbf{Cerchiamo ora i punti fissi}\\
\begin{gather*}
	\matrice{x_1+ x_2- x_1 \\ -x_2+x_3+1\\x_1-x_3} = \matrice{x_1\\x_2\\x_3}\\
		\begin{sistema}
			x_2 - 1 = 0\\
			-2x_2 + x_3 = -1\\
			x_1 - 2x_3 = 0\\
		\end{sistema} =
		\begin{sistema}
			x_1 = 2\\
			x_2 = 1\\
			x_3 = 1 \\
		\end{sistema}
\end{gather*}
\hline \ \\ 
\textbf{Esercizio 2} \\
Dimostrare che un'affinità di piano affine che ha tre punti fissi non allineati è l'identità\\
\textbf{Svolgimento}\\
Osservo che in un piano affine tre punti $p_0,p_1,p_2$ sono non allineati se e solo se $\overrightarrow{p_0p_1}, \overrightarrow{p_0p_2}$ sono linearmente indipendenti, ovvero $p_0,p_1,p_2$ sono affinamente indipendenti. D'altra parte, un'affinità è univocamente determinatta dall'immagine di tre punti indipendenti. L'identità è un'affinità con (almeno) tre punti fissi. Per l'unicità si ha $f=Id$.\\
\hline \ \\
\textbf{Esercizio 3} \\
In $ \mathbb{A} ^2_ \mathbb{R}$ consideriamo la retta $r: x + y =1$\\
i. Determinare le affinità che fissano tutti i punti di r\\
ii. Tra le affinità determinate in (i), trovare quelle che mandano $ \icol{1\\3} $ in $\icol{2\\-2}$\\
iii. Tra le affinità determinate in i, trovare le traslazioni\\
\textbf{Svolgimento} \\
\[
	f\matrice{x\\y} = \matrice{e & b \\ c & d}\matrice{x \\ y} + \matrice{\alpha \\ \beta} \ \ \ ad-bc\neq 0
.\] 
Basta scegliere $f(p) = p$ per due punti distinti $p\in r$. Posso scegliere $\icol{1\\0}, \icol{0\\1}$ \\
\begin{aligned}
&\begin{sistema}
	\matrice{a & b \\ c & d}\matrice{1 \\ 0} + \matrice{\alpha \\ \beta} = \matrice{1\\0} \\
	\matrice{a & b \\ c & d}\matrice{1 \\ 0} + \matrice{\alpha \\ \beta} = \matrice{1\\0} \\
\end{sistema} \hspace{80px}
\begin{sistema}
	a + \alpha = 1\\
	c + \beta = 0\\
	d + \alpha = 0\\
	d + \beta = 1
\end{sistema}\\
&\begin{sistema}
	a = 1 -\alpha\\
	b = -\alpha\\
	c = -\beta\\
	d = 1-\beta
\end{sistema} \ \ \ \ \  \ \ 
	f\matrice{x\\y} = \matrice{1-\alpha & -\alpha \\ -\beta & 1-\beta}\matrice{x\\y} + \matrice{\alpha\\\beta}
\end{aligned}\\
\begin{aligned}
	&(1-\alpha)(1-\beta) - \alpha\beta = 1 - \alpha - \beta\neq 0 \ \ \ \alpha + \beta \neq 1 \\
	&\text{ii} \ \ \matrice{1 - \alpha & -\alpha \\ -\beta & 1 - \beta}\matrice{1\\3} + \matrice{\alpha\\\beta} = \matrice{2\\-2} \\
	&\begin{sistema}
		1-\lapha - 3\alpha + \alpha = 2 \\
		-\beta + 3 - 3\beta + \beta = -2
	\end{sistema} \ \ \ 
	\begin{sistema}
		-3\alpha = 1\\
		-3\beta = -5
	\end{sistema} \ \ \
	\begin{sistema}
		\alpha = -\frac{1}{3}\\
		\beta = \frac{5}{3}
	\end{sistema} \\
	&\text{iii} \begin{pNiceArray}{c| c c}
		1 & 0 & 0 \\
		\hline
		\alpha & 1-\alpha & - \alpha \\
		\beta  & - \beta &1-\beta
	\end{pNiceArray} \rightarrow \matrice{1 - \alpha &\alpha \\ -\beta & 1 - \beta} = \matrice{1 & 0 \\ 0 & 1} \Rightarrow \alpha=\beta=0
\end{aligned} \\ 
e quindi l'unica traslazione è l'identità\\
\hline \ \\
\textbf{Nota}\\
\begin{align*}
	&f_{A,b} = AX + b \ \ \ f_{A,b}\circ f_{C,b} = f_{AC,Ad + b} \\
	& \begin{pNiceArray}{c | c}
		1 & 0 \\
		\hline
		b & A \\
	\end{pNiceArray} \in M_{(n+1)\times(n+1)}( \mathbb{K}) \ \ \ \ A\in M_{n\times n}( \mathbb{K}) \\ 
	&\begin{pNiceArray}{c | c}
		1 & 0 \\
		 \hline
		b  & A\\
	\end{pNiceArray} 
	\begin{pNiceArray}{c | c}
		1 & 0 \\
		 \hline
		d  & C \\
	\end{pNiceArray}  = 
	\begin{pNiceArray}{c | c}
		1 & 0 \\
		 \hline
		Ad + b  & AC \\
	\end{pNiceArray} \\
	&\begin{pNiceArray}{c|c c}
		1 & 0 & 0 \\
		\hline
		b_1 & a_{11} & a_{12} \\
		b_2 & a_{21} & a_{22} \\ 
	\end{pNiceArray} 
	\begin{pNiceArray}{c|c c}
		1 & 0 & 0 \\
		\hline
		d_1 & c_{11} & c_{12} \\
		d_2 & c_{21} & c_{22} \\ 
	\end{pNiceArray}  = \\
	&= 
	\begin{pNiceArray}{c | c c}
		1 & 0 & 0 \\
		\hline
		b_1  + a_{11}d_1 + a_{12}d_2 & a_{11}c_{11} + a_{12}c_{21} & a_{11}c_{12} + a_{12}c_{22} \\
		b_2 + a_{21}d_1 + a_{22} d_2 & a_{21}c_{11} + a_{22}c_{22} & a_{21}c_{12} + a_{22}c_{22} \\ 
	\end{pNiceArray} 
\end{align*} \\
\hline \ \\
\textbf{Esercizio 4} \\
In $ \mathbb{A} ^4_\mathbb{Q} \ \ \ L: \begin{sistema}
	x_1 - 3x_3 = 2 \\
	x_2 + x_4 = 1
\end{sistema} \ \ W = <\matrice{2\\0\\1\\0},\matrice{0\\1\\0\\-2}>$ \\
Scrivere le matrici delle proiezioni su $L$ parallela a $W$ e la matrice della simmetria di asse $L$ e direzione $W$ \\[5px]
\textbf{Svolgimento}\\
\begin{aligned}
	&L= P+W_1 \ \ V = W_1\oplus W_2 \\
	& p_L^{W_2}(X) = P + \pi_L^{W_2}(\overrightarrow{px}) \\
	& s_L^{W_2}(X) = P + \sigma_L^{W_2}(\overrightarrow{px})
\end{aligned}
Cerco ora l'equazione parametrica di L\\
\begin{aligned}
	& \begin{sistema}
		x_1-3x_3=2\\
		x_2=x_4=1
	\end{sistema} \ \  \rightarrow
	\begin{sistema}
		x_1 = 2 + 3t \\
		x_2 = 1 - s\\
		x_3 = t \\
		x_4 s
	\end{sistema} \ \ \ \rightarrow \ \ \  \matrice{x_1\\x_2\\x_3\\x_4} = \matrice{2\\1\\0\\0} + t\matrice{3\\0\\1\\0} + s\matrice{0\\-1\\0\\1}\\
	&V + W: \matrice{x_1-2\\x_2-1\\x_3\\x_4} = \alpha\matrice{2\\0\\1\\0} + \beta\matrice{0\\1\\0\\-2} + \gamma\matrice{3\\0\\1\\0} + \delta\matrice{0\\-1\\0\\1}
\end{aligned}\\[10px]
Qui il professore utilizza la sacra formula di Antani per giungere al seguente risultato \\
\begin{aligned}
	&\gamma = -2 +x_1-2x_3 \\
	&\delta = -2+2x_2 + x_4 \ \ \ p_L^W\icol{x_1\\x_2\\x_3\\x_4} = \icol{2\\1\\0\\0} + \gamma\icol{3\\0\\1\\0} + \delta\icol{0\\1\\0\\1} \\
	&
\begin{NiceArray}{c| c c c c}
	1 & 0 & 0 & 0 &0 \\
	\hline
	-4 & 3  & 0 & -6 & 0 \\
	-1 & 0 & 2 & 0 & 1 \\
	-2 & 1 & 0 & -2 & 0 \\
	2 & 0 & -2 & - & -2 \\
\end{NiceArray} \\
	&s_L^W\icol{x_1\\x_2\\x_3\\x_4} = \icol{2\\1\\0\\0} - \alpha\icol{2\\0\\1\\0} - \beta\icol{0\\1\\0\\-2} + \gamma\icol{3\\0\\1\\0} + \delta\icol{0\\1\\0\\-1}\\
	&\begin{pNiceArray}{c | c c c c c}
		1 & 0 & 0&0&0\\
		-8&5&0&-12&0\\
		-2&0&3&0&2\\
		-4&2&0&-5&0\\
		4&0&-4&0&-3\\
	\end{pNiceArray} 
\end{aligned} \\
\end{document}
