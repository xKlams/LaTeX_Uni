\documentclass[12px]{article}

\title{Lezione 7 Geometria 2}
\date{2025-03-17}
\author{Federico De Sisti}

\input{../../../setup.tex}

\begin{document}
	\maketitle
	\newpage
	\subsection{Cotntinuo sulla topologia prodotto}
	\begin{teo}
		Siano $P,Q$ spazio topologico, sia $P\times Q$ con topologia prodotto
		\begin{enumerate}
			\item $B = \{ U\times V \ | \ U\subseteq P\ \ \text{ aperto },  V\subset Q$ aperto  $\}$ è una base della topologia prodotto.
			\item Per ogni  $x_0\in P , y_0\in Q$ le applicazioni
				\begin{center}
					\begin{aligned}
						
						p|_{P\times \{y_0\}} : P&\times\{y_0\} \rightarrow P\\
									&(x,y_0) \rightarrow x
					\end{aligned}
				\end{center}
				\begin{center}
					\begin{aligned}
						
						p|_{\{x_0\}\times Q} : &\{x_0\} \times Q\rightarrow P\\
									&(x_0,y) \rightarrow y
					\end{aligned}
				\end{center}
				sono omomorfismi (dove $P\times\{y_0\}$ e $\{x_0\}\times Q$  hanno topologia di sottospazio)\\
			\item le proieizoni $p : P\times Q \rightarrow P$\\
				$q: P\times Q \rightarrow Q$\\
				sono aperte
			\item Sia $X$ spazio topologico $f : X \rightarrow P\times Q$\\
				allora $f$ è continua se e solo se lo sono le sue componenti $p\circ f $ e $q\circ f$
		\end{enumerate}
	\end{teo}
	\begin{dimo}
		1) Dimostriamo prima di tutto che esiste una topologia $T$ su $P\times Q$ che ha  $B$ per base.\\
		Verifichiamo le condizioni date in una proposizione precedente\\
		a)  $P\times Q$ dev'essere unione di  elementi di  $B$, è vero perché  $P\times Q\in B$ \\
		b) Siano $U,U'\subseteq P$ aperti,  $V,V'\subseteq Q$ aperti, allora l'intersezione.\\
		\[
			(U\times V)\cap (U'\times V')
		.\] 
		(è l'intersezione di due elementi qualsiasi di $B$) si deve poter scrivere come unione di elementi di  $B$:
		 \[
			 (U\cap U')\times (V\cap V')
		.\] 
		quindi questa intersezione è un elemento di $B$.\\
		Abbiamo dimostrato che esiste  $T$ che ha $B$ per base.\\
		Confrontiamo $T$ con la topologia con la topologia prodotto. Prima cosa: dimostriamo che $p$ e $q$ sono continue se su $P\times Q$ mettiamo $T$.\\
		Vediamo $p:P\times Q \rightarrow P$\\
		sia $A\subseteq P$ aperto, allora $p^{-1}(A) = A\times Q$\\
		è un aperto di $T$, Quindi $p$ è continua.\\
		Analogamente $q$ è continua.\\
		Segue $T$ è più fine della topologia prodotto (per definizione della topologia prodotto).\\
		$T \spuseteq $ topologia prodotto\\
		Dimostriamo $T\subseteq $ topologia prodotto\\
		Dimostriamo che $B\subseteq $ topologia prodotto\\
Siano $U\subseteq P$ aperto,  $V\subseteq Q$ aperto\\
quindi  $Y\times V\in B$ allora:
 \[
 U\times Q = p^{-1}(U)
.\] 
dev'essere aperto anche in topologia prodotto.\\
Anche $P\times V = q^{-1}(V)$ dev'essere aperto in topologia prodotto.\\
 \[
 U\times V = (U\times Q) \cap (P\times V)
 .\] 
 unione arbitraria di aperti è aperta, quindi $T\subseteq$ topologia prodotto.\\
 Quindi $B$ è base della topologia prodotto\\[10px]
 2) Dimostriamo che 
 \[
	 p|_{P\times\{y_0\}} P\times\{y_0\} \rightarrow P
 .\] 
 è omeomorfismo.\\
 Quest'applicazione è biettiva, è continua perché è restrizione (su un sottospazio) di un'applicazione continua\\
 Dobbiamo dimostrare che l'inversa è continua\\
 \begin{center}
	 \begin{aligned}

		 $\varphi :& P \rightarrow P\times \{y_0\}\\
			  &x \rightarrow (x,y_0)$
	 \end{aligned}
 \end{center}
Basta verificare che le controimmagini di elementi della base sono aperti (esercizi settimanali).\\
Inoltre una base del sottospazio $P\times \{y_0\}$ è ottenuta intersecando gli elementi elementi dalla base $B$ al sottospazio (esercizi settimanali). Sia $U\times V \in B$ ( $U\subseteq P$ aperto, $V\subseteq Q$ aperto) e considero
\[
	A = (U\times V) \cap (P\times \{y_0\})
.\] 
%TODO inserisci immagine che non hai
Abbiamo $A = \begin{cases}
	U\times \{y_0\} \ \ \text{ se } y_0\in V\\
	\emptyset \ \ \ \  \ \ \ \ \ \ \ \ \text{ se } y_0\not\in V
\end{cases}$ \\
allora $ \varphi^{-1}(A) = \begin{cases}
	U \ \ \ \ \text{se} \ y_0\in V\\
	\emptyset \ \ \ \ \ \text{se} \ y_0\not\in V
\end{cases}$ 
In entrambi i casi ho un aperto di $P$\\
segue che  $p|_{R\times\{y_0\}}$ è un omeomorfismo. Analogamente lo è anche $q|_{\{x_0\}\times Q}$ \\[20px]
3)Dimostriamo che $p,q$ sono aperti\\
%TODO aggiungi immagini che non hai 
Su $A\subseteq P\times Q$ aperto considero
 \[
 A = \bigcup_{y_0\in Q}A\cap (P\times  y_0\})
.\] 
 \[
 p(A) = \bigcup_{y_0\in Q}p(A\cap (P\times  y_0\}))
.\] 
\[
	= \bigcup_{y_0\in Q} p|_{P\times\{y_0\}} (A\cap (P\times\{y_0\}))
.\] 
Ora l'insieme $A\cap (P\times\{y_0\})$ è aperto nel sottospazio $P\times \{y_0\}$, e $p|_{P\times\{y_0\}}$ è omemorfismo. \\
quindi $p|_{P\times\{y_0\}}(A\cap(P\times\{y_0\}))$ è aperto di $P$.\\
Segue:  $p(A)$ aperto in $P$. Cioè $p$ è aperta analogamente q è aperta \\
4) Abbiamo 
%TODO aggiungi immagine 4 26
Se $f$ è continua allora lo sono le mappe $p\circ f, q\circ f$\\
Viceversa, supponiamo  $p\circ f$ continua. ALlora dimostriamo  $f$ continua. Di nuovo usiamo $B$, quindi  $U = \subseteq P$ aperto, $V\subseteq Q$ aperto, dimostriamo che $f^{-1}(U\times V)$ è aperto in $X$. Abbiamo
 \[
 f^{-1}(U\times V) = (p\circ f)^{-1}(U)\cap (q\times f)^{-1}(V)
.\] 
è aperto per continuità di $p\circ f$ e $q\circ f$\\
\end{dimo}
 \textbf{Osservazione}\\
 Siano $P,Q$ spazi topologici, siano  $B_P$ base della topologia di  $P$ e  $B_Q$ base della topologia di $Q$ allora
  \[
	  \{U\times V| U\in B_P, \ \ V\in B_Q\}
 .\] 
 è una base della topologia prodotto.\\
 \textbf{Esempi}\\
 1) $P = Q = \R$ con topologia euclidea prendiamo le basi  $B_P = B_Q = \{]a,b[\ |\ a,b\in \R \ a< b\}$ della topologia euclidea su  $\R$\\
 Per l'osservazione  $\{]a,b[\times]c,d[\ | a,b,c,d\in\R \ \ a < b,\ c < d\}$\\
 è base della topologia prodotto su  $\R^2 = \R\times \R$\\
 Sappiamo anche che questa è una base della topologia euclidea su  $\R^2$ quindi questa è la topologia prodotto.\\
 ANalogamente, la topologia euclidea su  $\R^n$ è la topologia prodotto su 
  \[
	  \R^n = \R^{n-1}\times\R
 .\] 
 2) Considero $\R $ con topologia di Zarinksi, allora la topologia prodtto su
 \[
	 \R^2 = \R\times \R
 .\] 
 dove ogni $\R$ ha la topologia di Zarinski non è la topologia di Zarinksi su $\R^2$\\
 \begin{defi}[Spazi di Hausdoff]
 	Uno spazio topologico $X$ si dice di Hausdoff (o T2) se $\forall x,y\in X$ con  $x\neq y, \ \exists U$  intorno di  $x, V $  intorno di $y$  t.c.  $U$ \cap V = \emptyset.
 \end{defi}
 \textbf{Esempi}\\
 1) Ogni spazio metrico è T2, basta prendere $U = B_{d(x,y)/2}(x)$ $V = B_{d(x,y)/2}(y)$\\
 2) $X = \emptyset$ di Hausdoff\\
3) $X$ qualsiasi con topologia banale  allora:
\begin{itemize}
	\item se $|X|\leq 1$ allora  $X$ è T2
	\item se  $|X| \geq 2$ allora  $X$ non è T2
\end{itemize}
4) Se $X$ ha topologia cofinita.
\begin{itemize}
	\item se $X$ è un insieme finito allora la topologia è discreta e $X$ è T2
	\item se $X$ è infinito allora $X $ non è T2.
\end{itemize}
\textbf{Osservazione}\\
Dati $x,y\in X$ con  $x\neq y$\\
se esistono intorni  $U$  di $x, V$ di  $y$ con $U\cap V = \emptyset$ allora esistono aperti  $(x\in)A(\subseteq U)$ e  $(y\in )B(\subseteq V)$ e sono disgiunti.\\
Quindi  $X$ è T2 se e solo se $\forall x,y\in X$ con $x\neq y$  $\exists U $ intorno aperto di $x$ $V$ intorno aperto di $y$ con $U\cap V = \emptyset$
 \begin{lemm}
	Se $X$ spazio topologico è T2, tutti i suoi sottoinsiemi finiti sono chiusi.
\end{lemm}
\begin{dimo}
	Sia $x\in X$ per ogni  $y\in X$ scegliamo intorni aperti 
	 \[
	U\ni x, V\ni y
	.\] 
	con $U\cap V= \emptyset$ $V\not\ni x$, quindi  $V\subseteq (X\setminus \{x\})$ \\
	Cioè $X\setminus \{x\}$ è intorno di ogni suo punto.\\
	Segue  $\{x\}$ è chiuso.\\
	Allora tutti i sottoinsiemi finiti sono chiusi
\end{dimo}
\begin{prop}
	Sottospazi e prodotti di spazi di Hausdoff sono di Hausdoff
\end{prop}
\begin{dimo}
	Sia $X$ T2 sia $Y\subseteq X$ sottospazio.
	%TODO Aggiungi immagine 17 18\\
	Siano $y,y'\in Y$ con $y\neq y'$\\
	Scegliamo  $U\ni y, U'\ni y'$ aperti in  $X$ e disgiunti $U\cap U' = \emptyset$\\
	allora  $U\cap Y$ e $U'\cap Y$ sono aperti in  $Y, $ disgiunti, e contengono rispettivamente $y$ e $y'$, Allora  $Y$ è T2.\\
	Siano ora $P,Q$ spazi topologici, entrambi T2, siano  $(a,b)\neq (c,d)\in P\times Q$\\
	Supponiamo  $a\neq c$ \\
siano $U\ni a, U'\ni x$ aperti in  $P$, $U\cap U' = \emptyset$. Allora  $U\times Q$ e  $U'\times Q$ sono aperti disgiunti contenenti $(a,b)$ e $(c,d)$ rispettivamente.\\
Se $a  = c$ allora $b\neq d$  e la dimostrazione è analoga con spazi del tipo $P\times U, P\times U' $
\end{dimo}
\begin{teo}
	Sia $X$ spazio topologia, considero $X\times X$ con topologia prodotto e la diagonale  $\Delta = \{(a,a)\in X\times X \ | \ a\in X\}$\\
	Vale:  $X$ T2 se e solo se  $\Delta$ è chiusa in $X\times X$
\end{teo}
Idea intuitiva dell'enunciato, parte  $\Rightarrow$ .\\
sia $x\in X$ un punto che "si muove" (ad esempio è un termine di una successione).\\
Supponiamo  $x$ "tende" ad un limite $a\in X$, cioè entra progressivamente in ogni intorno di  $a$. Se  $x$ "tende" anche a $b\in X$ e  $X$ è T2, allora $a = b$ (perché se $a\neq b$ allora hanno intorni disgiunti).\\
Allora potrò dire che $(x,x)$ "tende" alla coppia $(a,b)$ e la proprietà T2 implica $a = b$, cioè la diagonale  $\{(x,x) \ | \ $ è chiusa $\}$\\
\begin{dimo}
	$ \Rightarrow $ suppongo $X $ T2, dimostriamo che $(X\times X)\setminus\Delta$ è aperto in topologia prodotto\\
	Sia $(x,y)\in (X\times X)\setminus \Delta,$ cioè  $x\neq y$\\
	Siano  $U, V$ aperti di $X$ disgiunti con  $U\ni x, V\ni y$, allora  $U\times V$ è aperto in $X\times X$, contiene  $(x,y)$\\
	Inoltre  $(U\times V)\cap \Delta = \emptyset$, perché 
	 \[
		 (U\times V)\cap \Delta = \{(z,z)\in X\times X \ | \ z \in U \ \ z\in V\}
	.\] 
	è vuoto perché $z$ apparterrebbe a $U\cap V = \emptyset$\\
	Quindi  $(X\times X)\setminus \Delta$\\ è intorno di ogni suo punto, cioè è chiuso\\
	$ (\Leftarrow)$ Suppongo $\Delta$ chiuso, cioè $(X\times X)\setminus\Delta$ aperto di $X\times X$. Siano  $x\neq y$ di $X$, allora $(x,y)\in (X\times X)\setminus\Delta$\\
	Per la base  $B$ vista per la topologia prodotto esiste  $U\times V\subseteq(X\times X)\setminus\Delta$\\
	tale che  $U\times V$ aperto di $X, U\times V\ni (x,y)$. Allora  $U\cap V = \emptyset$ \\
	(ragionamento di prima, non esistono punti come z). Inoltre $x\in U$,  $y\in V$. Segue  $X$ è T2.
\end{dimo}
\textbf{Osservazione}\\
Ricordo $C = \{(x,y)\in \R^2\ | \ xy = 1\}$\\
è chiusa perché  $ C = f^{-1}(\{1\})$ \\
dove 
\begin{center}
	\begin{aligned}
		$f: \R^2 &\rightarrow\R\\
		(x,y) &\rightarrow xy$
	\end{aligned}
\end{center}
Pi	u in generale siano $X,Y$ spazi topologici  $f :X \rightarrow Y$ continua. Suppongo $Y$ di Hausdoff e sia $y\in Y$\\
Allora  $\{y_0\}$ è chiuso in $Y$, quindi
 \[
	 \{x\in X\ | \ f(x) = y_0\} = f^{-1}(\{y_0\}) \ \ \text{ è chiuso}
.\] 
\begin{coro}
	$X,Y$ spazi topologici.\\
	Siano $f,g: X \rightarrow Y$ continue, e l'insieme
	\[
	 C = \{x\in X\ | \ f(x) = g(x)\}
	.\] 
	Se $Y$ è $T2$ allora  $C$ è chiuso in $X$.
\end{coro}
\begin{dimo}
Consideriamo
\begin{center}
	\begin{aligned}
		$\varphi : X & \rightarrow Y\ \ \times \ \ Y\\
			    x & \rightarrow (f(x), g(x)) $
	\end{aligned}
\end{center}
Per il primo teorema della lezione, questa è continua. Allora $C = \varphi^{-1}(\Delta)$ che è la diagonale in $Y\times Y$\\
Ma la diagonale è chiusa quindi  $C$ è chiuso
\end{dimo}
\end{document}
