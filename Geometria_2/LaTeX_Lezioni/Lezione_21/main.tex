\documentclass[12px]{article}

\title{Lezione N+5 Geometria 2}
\date{2025-05-27}
\author{Federico De Sisti}

\usepackage{amsmath}
\usepackage{amsthm}
\usepackage{mdframed}
\usepackage{amssymb}
\usepackage{nicematrix}
\usepackage{amsfonts}
\usepackage{tcolorbox}
\tcbuselibrary{theorems}
\usepackage{xcolor}
\usepackage{cancel}

\newtheoremstyle{break}
  {1px}{1px}%
  {\itshape}{}%
  {\bfseries}{}%
  {\newline}{}%
\theoremstyle{break}
\newtheorem{theo}{Teorema}
\theoremstyle{break}
\newtheorem{lemma}{Lemma}
\theoremstyle{break}
\newtheorem{defin}{Definizione}
\theoremstyle{break}
\newtheorem{propo}{Proposizione}
\theoremstyle{break}
\newtheorem*{dimo}{Dimostrazione}
\theoremstyle{break}
\newtheorem*{es}{Esempio}

\newenvironment{dimo}
  {\begin{dimostrazione}}
  {\hfill\square\end{dimostrazione}}

\newenvironment{teo}
{\begin{mdframed}[linecolor=red, backgroundcolor=red!10]\begin{theo}}
  {\end{theo}\end{mdframed}}

\newenvironment{nome}
{\begin{mdframed}[linecolor=green, backgroundcolor=green!10]\begin{nomen}}
  {\end{nomen}\end{mdframed}}

\newenvironment{prop}
{\begin{mdframed}[linecolor=red, backgroundcolor=red!10]\begin{propo}}
  {\end{propo}\end{mdframed}}

\newenvironment{defi}
{\begin{mdframed}[linecolor=orange, backgroundcolor=orange!10]\begin{defin}}
  {\end{defin}\end{mdframed}}

\newenvironment{lemm}
{\begin{mdframed}[linecolor=red, backgroundcolor=red!10]\begin{lemma}}
  {\end{lemma}\end{mdframed}}

\newcommand{\icol}[1]{% inline column vector
  \left(\begin{smallmatrix}#1\end{smallmatrix}\right)%
}

\newcommand{\irow}[1]{% inline row vector
  \begin{smallmatrix}(#1)\end{smallmatrix}%
}

\newcommand{\matrice}[1]{% inline column vector
  \begin{pmatrix}#1\end{pmatrix}%
}

\newcommand{\C}{\mathbb{C}}
\newcommand{\K}{\mathbb{K}}
\newcommand{\R}{\mathbb{R}}


\begin{document}
	\maketitle
	\newpage
	\subsection{Zenobobi}
	\begin{defi}[Parametrizzazione di Monge]
		$f: V \rightarrow\R$ differenziabile aperto di $\R^{n-1} \Rightarrow \begin{aligned}
			\varphi : &V \rightarrow U = Im( \varphi)$  \\
				  & $(a_1,\ldots,a_{n-1}) \rightarrow (a_1,\ldots,a_{n-1},f)$
		\end{aligned}\\
		è una parametrizzazione.
	\end{defi}
	\begin{teo}
		Sia $S\subseteq\R^3$\\
		una superficie differenziabile immersa allora $\exists$ una parametrizzazione di Monge per ogni punto di  $S$
	\end{teo}
	\begin{dimo}
		$p\in S$\\ 
\begin{aligned}
	$\psi: &V \rightarrow U$\\
	       & $q \rightarrow p$
\end{aligned}
		  parametrizzazione\\
		$d\psi$ è iniettivo,  $q = \psi^{-1}(p)$\\
		ed è dato della matrice Jacobiana\\
		GUARDA 17 25\\
		\[
		\pi: \R^3 \rightarrow\R^2
		.\] 
		proiezione su $(x,y)$\\
		 $\pi\circ\psi$ ha differenziale in  $q$ che è isomorfismo\\
		 $ \Rightarrow  $ a meno di restringere l'aperto $U$ abbiamo $\pi\circ \psi : U \rightarrow W = \pi(U)$  invertibile con inversa $C^\infty $ \\
INSERISCI IMMAGINE 5 28\\
Otteniamo la parametrizzazione di Mange definita da
\[
	\begin{aligned}
		\tilde \psi = &\psi\circ (\pi\circ\psi)^{-1}:W \rightarrow U\\
			      &(x,y) \rightarrow (x,y,f(x,y))

	\end{aligned}
.\] 
È $C^\infty$  sui punti di $W$ componibile con l'inversa della restrizione $\pi|_U$
	\end{dimo}
	\subsection{Applicazioni differenziabile tra superfici}
	\begin{defi}
		\begin{enumerate}
			\item Sia $S$ una superficie differenziabile \\
				Una funzione  $f: S \rightarrow \R$ si dice differenziabile in $x\in S$ se  $\exists $ un intorno coordinato  $U$ di $x$ è carta  $ \varphi: U \rightarrow V\subseteq \R^2$\\
				tale che $f\circ \varphi^{-1}: V \rightarrow \R$ differenziabile in $ \varphi(x)$\\
				$f$ si dice differenziabile se è differenziabile in $x\ \ \forall x\in S$
			\item  $f: S \rightarrow \R^n$\\
				è differenziabile se lo sono tutte le sue componenti
			\item $S_1, S_2$ superfici differenziabili\\
				$f: S_1 \rightarrow S_2$ è differenziabile su $x\S_1$ se $\exists \varphi: U \rightarrow V$  carta locale intonro ad $x$\\
				$ \varphi': U' \rightarrow V'$ carta locale intorno ad $f(x)$ tale che  $ \varphi'\circ f\circ \varphi^{-1}: V \rightarrow V'$ è differenziabile
			\item $f: S_1 \rightarrow S_2$ è un diffeomorfismo se è iniettivo, differenziabile e $f^{-1}: S_2 \rightarrow S_1$ è differenziabile
		\end{enumerate}
	\end{defi}
	\textbf{Esempi}\\
	\begin{enumerate}
		\item $ \begin{aligned}
				f: &S \rightarrow \R$\\
				   &$x \rightarrow \|x-u\|^2$ differenziabile
		\end{aligned}
	\item $S_1 = \{x^2 + y^2 + z^2 = 1\}$\\
		$S_2 = \{(x,y,z) \ | \ x^2+ y^2 = 1\}$\\
		$\tilde S_1 = S_2\setminus\{(0,0,1),(0,0,1)\}$\\
		$f: \tilde S_1 \rightarrow S_2$\\
		$(x,y,z) \rightarrow (\frac{x}{\sqrt{x^2 + y^2}}, \frac{y}{\sqrt{x^2 + y^2}}, z)$
	\end{enumerate}
	\textbf{Esercizio}\\
	Dimostrare che $f$ è differenziabile\\
	\textbf{Suggerimento}\\
	usare la parametrizzazione
	\[
		\begin{aligned}
			\psi: &V \rightarrow \tilde S_1\\
			      & (\theta,\rho) \rightarrow (\cos(\rho)\cos(\theta),\cosa(\rho)\sin(\theta)\sin(\rho))

		\end{aligned}
	.\] 
		\[
		\begin{aligned}
			\psi^{-1} : &V' \rightarrow S_{2}\\
				    &(\theta, t) \rightarrow (\cos\theta, \sin\theta, t)
		\end{aligned}
		.\] 
		Sia $f : A \rightarrow \R^m$ differenziabile \\ 
		$A$ aperto di $\R^n$, $p\in A$\\
		 $d_p f : \R^n \rightarrow \R^m$ ed è definito\\
		 $\ \ \ v \rightarrow \mathcal J f_p\cdot v$, dove $\mathcal J$ è la Jacobiana\\
		 Sia  $\alpha : I \rightarrow \R^n$ curva con $\alpha(t_0) = p$ e $\alpha'(t_0) = v$ per $t_0\in I$\\
		 Allora $df_p(v) = \beta'(0)$\\
		  $\beta(t) = f(\alpha(t))$
\end{document}
