\documentclass[12px]{article}

\title{Lezione 3 Geometria 2}
\date{2025-03-04}
\author{Federico De Sisti}

\usepackage{amsmath}
\usepackage{amsthm}
\usepackage{mdframed}
\usepackage{amssymb}
\usepackage{nicematrix}
\usepackage{amsfonts}
\usepackage{tcolorbox}
\tcbuselibrary{theorems}
\usepackage{xcolor}
\usepackage{cancel}

\newtheoremstyle{break}
  {1px}{1px}%
  {\itshape}{}%
  {\bfseries}{}%
  {\newline}{}%
\theoremstyle{break}
\newtheorem{theo}{Teorema}
\theoremstyle{break}
\newtheorem{lemma}{Lemma}
\theoremstyle{break}
\newtheorem{defin}{Definizione}
\theoremstyle{break}
\newtheorem{propo}{Proposizione}
\theoremstyle{break}
\newtheorem*{dimo}{Dimostrazione}
\theoremstyle{break}
\newtheorem*{es}{Esempio}

\newenvironment{dimo}
  {\begin{dimostrazione}}
  {\hfill\square\end{dimostrazione}}

\newenvironment{teo}
{\begin{mdframed}[linecolor=red, backgroundcolor=red!10]\begin{theo}}
  {\end{theo}\end{mdframed}}

\newenvironment{nome}
{\begin{mdframed}[linecolor=green, backgroundcolor=green!10]\begin{nomen}}
  {\end{nomen}\end{mdframed}}

\newenvironment{prop}
{\begin{mdframed}[linecolor=red, backgroundcolor=red!10]\begin{propo}}
  {\end{propo}\end{mdframed}}

\newenvironment{defi}
{\begin{mdframed}[linecolor=orange, backgroundcolor=orange!10]\begin{defin}}
  {\end{defin}\end{mdframed}}

\newenvironment{lemm}
{\begin{mdframed}[linecolor=red, backgroundcolor=red!10]\begin{lemma}}
  {\end{lemma}\end{mdframed}}

\newcommand{\icol}[1]{% inline column vector
  \left(\begin{smallmatrix}#1\end{smallmatrix}\right)%
}

\newcommand{\irow}[1]{% inline row vector
  \begin{smallmatrix}(#1)\end{smallmatrix}%
}

\newcommand{\matrice}[1]{% inline column vector
  \begin{pmatrix}#1\end{pmatrix}%
}

\newcommand{\C}{\mathbb{C}}
\newcommand{\K}{\mathbb{K}}
\newcommand{\R}{\mathbb{R}}


\begin{document}
	\maketitle
	\newpage
	\subsection{Parte interna, chiusura,intorni}
	\begin{defi}
		Sia $X$ spazio topologico, sia $D\subseteq X$ un sottoinsieme, la parte interna di  $D$ è 
		\[
		 D^o = \bigcup_{A\subseteq D, A\text{ aperto } } A
		.\] 
		La chiusura di $D$ è 
		\[
			\overline D = \bigcap_{C\supseteq D, C \text{ chiuso}} C
		.\] 
		la frontiera di $D$ è
		\[
		 \partial D = \overline D \setminus D^o
		.\] 
		I punti di $D^o$ si dicono interni a $D$, quelli di $\overline D$ si chiamano aderenti a $D$.
	\end{defi}
	\textbf{Osservazioni}\\
	1) $D^o$ è un aperto e  $\overline D$ è un chiuso (posso vederlo come l'intersezione tra  $\overline D$ e il complementare di $D^o$, che è chiuso)\\
	 2) Anche $\partial D$ è chiuso perché\\
	 $\parial D = \overline D\cap (X\setminus D^o)$ dove  $(X\setminus D^o)$ è un chiuso\\
	  \textbf{Esempio}\\
	  1) $X = \R$ con topologia euclidea. Sia $D = [0,1]$.\\
	  Allora  $D^o = ]0,1[$, verifica:\\
	  $D^o \supseteq ]0,1[: $\\
	  La parte interna di  $D$ contiene tutti gli aperti, dato che  $]0,1[$ è un aperto è contenuto.\\A
	  $D^o\subseteq ]0,1[$:\\
	  supponiamo per assurdo che  $D^o\not \subseteq ]0,1[$ , allora $0\in D^o$ oppure  $1\in D^o$ ( mi limito a considerare i punti di  $D$ perché $D^o\subseteq D$).\\
	  Supponiamo  $0\in D^o$, allora esiste  $A\subseteq \R$ aperto t.c.  $A\subseteq D, A\ni 0$ (è uno degli A della definizione).\\
	  Allora  esiste $\varepsilon > 0$ t.c.  $]0 - \varepsilon , 0 + \varepsilon[\subseteq a \subseteq D$.\\
	  assurdo, Analogamente  $1\not\in D^o$ quindi vale  $\subseteq$\\
	  2)  $X = \R$ con topologia cofinita $D = [0,1]$. Allora  $D^o = \bigcup_{A\subseteq D, A \text{ aperto}}$ \\
	  Sia  $A$ aperto.\\
	  $A\subseteq D$ abbiamo\\
	  $A = \emptyset $ oppure $A = \R\setminus\{\text{insieme finito}\}$\\
	  Ma questa ultima è impossibile\\
	  allora  $D^o = \emptyset$ in questa topologia (con questo $D$)\\
	  esercizio: calcolare $\overline D$\\
	  3)  $X = \R$, $T = $ topologia per cui $A$ è aperto $ \Leftrightarrow A = \emptyset$ oppure $A\ni 0$ \\
	  Considero $\overline{\{1\}} = \{1\}$, questo insieme non contiene lo zero, quindi $\{1\}$ è esso stesso un chiuso.\\
	  Però $\overline{\{0\}} = ?$ \\
	  I chiusi in  $T$ sono $\R$ e i sottoinsiemi che non contengono lo  $0$. Quindi l'unico insieme chiuso che contiene $\{0\}$ è $\R$, allora $\overline{\{0\}} = \R$
	   \begin{defi}
	  	Sia $X$ spazio topologico, un sottoinsieme di $D\subseteq X$ si dice denso se $\overline D = X$
	  \end{defi}
	  \textbf{Esempio}\\
	  $X = \R $ con topologia euclidea,\\
	  $D = \Q$. Dimostriamo che è denso\\
	  L'unico chiuso che contiene  $D$ è $X$ stesso.\\
	   Sia $C\subseteq \R$ chiuso con  $C\supseteq \Q$\\
	   sia $a\in\R\setminus C$ aperto\\
	   allora  $\exists \varepsilon > 0 \ | \ ]a - \varepsilon, a + \varepsilon[ \subseteq \R\setminus C$ \\
	   allora $] a - \varepsilon, a + \varepsilon[\cap \Q = \emptyset$ \\
	   assurdo.\\
	   Allora $a$ non esiste e $C = \R$. 
	   \textbf{Osservazione:}\\
	   1) Sia  $D\subseteq X$ spazio topologico\\
	   vale:\\
	    \[
	   X\setminus(\overline D) = (X\setminus D)^o
	   .\] 
	   \begin{dimo}
	   	Usando direttamente la definizione:\\
		\[
			X\setminus(\overline D) = X\setminus (\bigcap_{C\supseteq D, C \text { chiuso}}C) = \bigcup_{C\supset D, C \text { chiuso}}(X\setminus C) 
		.\] 
		(ultima eguaglianza per esercizio)
		\[
			= \bigcup_{A = X\setminus C, C\supset D, C\text{ chiuso}}A = \bigcup_{A\text{ aperto}, X\setminus A\supset D} A= \bigcup_{A\text{ aperto}, A\subseteq X\setminus D}A
		.\] 
	   \end{dimo}
	   2) $D$ denso\\
	   $\text{}\ \ \ \ \ \ \ \ \ \ \ \ \ \storto \Leftrightarrow$\\
	   $D$ interseca ogni aperto non vuoto (esercizio)
	   \begin{defi}
	   	Sia $X$ spazio topologico,\\
		$U\subseteq X, x\in U^o$\\
		Allora  $U$ si dice intorno di $x$.\\
		Equivalentemente, un sottoinsieme $U\subseteq X$ si dice intorno di $x\in X$ se esiste  $A\subseteq X$ aperto t.c. $x\in A\subseteq U$
	   \end{defi}
	   \textbf{Esempio}\\
	   $X = \R$ topologia euclidea,  $x = 0, U = ]-1,1[$ è intorno di  $x$ (si prende ad esempio $A = U, $ o anche $A = ]-1/2,1/2[$\\
	   Anche $V = [-1,1]\cup \{5\}$ è un intonro di 0, ad esempio  $A = ]-1/2, 1\16[\cup]3/16,7/16[$\\
	    \textbf{Osservazione}\\
	    $U\subseteq X$ è aperto  $ \Leftrightarrow U = U^o \Leftrightarrow U$ è  un intorno di ogni suo punto.
	    \begin{lemm}
Siano $X$ spazio topologico, $x\in X$ $D\subseteq X.$ Allora $x\in\overline D \Leftrightarrow \forall U$ intorno di $x$ vale $U\cap X \neq\emptyset$
	    \end{lemm}
	    \begin{dimo}
	    Supponiamo $x\in\overline D$ sia $U$ intorno di $x$\\
	    per assurdo suppongo  $D\cap U = \emptyset$
	    %TODO se vuoi aggiungi foto marzo 4 6:27
	    Considero  $A  \subseteq X$ aperto con $x\in A\subseteq U$\\
	    Considero il chiuso  $X\setminus A = C$\\
	    Abbiamo  $C\supseteq D$ perché  $D\cap U = \emptyset$ e allora anche $D\cap A = \emptyset$A\\
	    Abbiamo  $C\supset D$ perché  $D\cap U = \emptyset$ e allora anche  $D\cap A = \emptyset $ e allora anche $D\cap A = \emptyset$. Cioè $C$ compare nella definizione di $\overliene D $ e $C\not\ni x$ perché  $x\in A$\\
	    Ma  $x\in\overline D$ quindi  $x$ è in tutti i chiusi che contengono $D$, assurdo\\
	    Viceversa, supponiamo $D$ intorno di $x$, per assurdo però $x\neq \overline D$, Allora esiste un chiuso  $C$ che contiene $D$ ma non $x$.\\
	    Considero  $A = X\setminus C$ è un aperto contenente  $x$. Cioè $A$ è un intorno di $x$ e $A$ non interseca $D$; assurdo. Quindi $x\in D$ 
	    %TODO se vuoi aggiungi immaigne 4 mar 6:36
	    \end{dimo}
	    \begin{defi}[Famiglia degli intorni, sistema fondamentale]
	    	Sia $X$ spazio topologico e $x\in X$ La famiglia di tutti gli intorni di  $x$ si denota con $I(x)$.\\
		Un sottoinsieme $J\subseteq I(x)$ è detto sistema fondamentale di intorni di $x$ (o base locale in $x$) se $\forall U\in I(x)\ \  \exists V\in J \ |\ V\subseteq U $ 
	    \end{defi}
	    \textbf{Esempi:}\\
	    $X = \R$ con topologia euclidea.\\
	     $x\in\R$ qualsiasi\\
	     $J = \{]x - \varepsilon, x+ \varepsilon[ \ |\ \varepsilon > 0,  \ \varepsilon \in\R\} $\\
	     è sistema fondamentale di intorni di  $x$\\
	     $J'\{[x- \frac 1n, x + \frac 1n] \ | \ n\geq 1, \  n\in\N\}$ \\
	     è un sistema fondamentale di interni di $x$\\
	     \[
		      J''= \{[x-\frac 1n, x + \frac 2n[\cup \{x + \frac 3n \} | n\geq 1, n\in\N\}\\
	     .\] 
	     è un sistema fondamentale di riferimento
	      \[
		      J'''= \{]x-\frac 1n, x + \frac 1n[\cup \{10 \} | n\geq 1, n\in\N\}\\
	     .\] 
	     $(10\neq x)$\\
	     non è un sistema fondamentale di riferimento\\
	     \subsection{Applicazioni continue}
	     \begin{defi}
		     Siamo $X, Y$ spazio topologico $f : X \rightarrow Y$ un'applicazione. $f$ si dice continua se $f^{-1} (A)$ è aperto (in $X$) $\forall A$ aperto ($Y$)
	     \end{defi}
	     \textbf{Nota (per la tesi)}\\
	     non iniziare mai una frase con un simbolo, è facile fare errori  (lui può ma solo per essere veloce)\\
	     \textbf{Esempi:}\\
	     1) Se $X$ ha topologia discreta, ogni $f$ è continua (qualsiasi sia $Y$)\\
	     2) Se  $Y$ ha una topologia banale, allora $f^{-1}(\emptyset) =\emptyset, f^{-1}(Y) = X$\\
	     quindi ogni $f$ è continua.\\
	     3) Supponiamo $X,Y$ con topologia cofinita e $f:X \rightarrow Y$ iniettiva\\
	     $f^{-1}(\emptyset) = \emptyset, \ $\\
	     gli altri aperti sono del tipo $Y\setminus\{$ insieme finito  $\} = Y\setminus\{y_1,\ldots, y_n\}$\\
	     allora:\\
	     $f^{-1}(Y\setminus\{y_1,\ldots,y_n\}) = X\setminus\{f^{-1}(y_1)\cup\ldots\cup f^{-1}(y_n)\}$
	     e tutte le controimmagini di questi elementi sono o vuote o di cardinalità 1, quindi è aperto.


\end{document}
