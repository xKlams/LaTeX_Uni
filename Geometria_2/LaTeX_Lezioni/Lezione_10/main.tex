\documentclass[12px]{article}

\title{Lezione 10 Geometria 2}
\date{2025-03-25}
\author{Federico De Sisti}

\input{../../../setup.tex}

\begin{document}
	\maketitle
	\newpage
	\subsection{Altre informazioni sui compatti}
	\textbf{Ricorda:}\\
	Un chiuso in un compatto è compatto, un compatto in un $T2$ è chiuso
	\begin{coro}
		Sia $\R$ con topologia euclidea, $Y\subseteq \R$ sottospazio allora  $Y$ compatto $ \Leftrightarrow Y$ chiuso e limitato.
	\end{coro}
	\begin{dimo}
		$ \Rightarrow \R$ è $T2$ , se $Y$ è compatto allora $Y$ è chiuso in  $\R$. Inoltre  $Y$ è limitato perchè il ricoprimento $R = \{Y\cap ]-n,n[\ |\ n\in\Z\}$ ha sottoricoprimenti finiti.\\
		$ \Leftarrow$ Suppongo $Y$ chiuso e limitato, quindi in $n\in\Z_{\geq 0}$ tale che  $Y\subseteq [-n,n]$.\\
		L'intervallo  $[-n,n] $ è omeomorfo a  $[0,1]$ quindi è compatto.\\
		Inoltre $Y$ è chiuso in $[-n,n]$ quindi è compatto
	\end{dimo}
	\begin{teo}
		Sia $f:X \rightarrow Y$  applicazione continua fra spazi topologici, se $X$ è compatto allora $f(X)$ è compatto.
	\end{teo}
	\begin{dimo}
		Considero la restrizione
		\[
		\tilde f : X \rightarrow f(X)
		.\] 
		è continua. $\R$ ricoprimento aperto di $f(X)$, allora
		 \[
			 \tilde R = \{f^{-1}(A)\ |\ A\in R\}
		.\] 
		è un ricoprimento aperto di $X$, quindi esiste un sottoricoprimento finito
		\[
			\{f^{-1} (A_1),\ldots,f^{-1}(A_n)\}
		.\] 
		Abbiamo, $f(f^{-1}(A)) = A\cap f(X) = A$\\
		Quindi  $\{A_1,\ldots, A_n\}$ è sottoricoprimento finito di $R$
	\end{dimo}
	\begin{coro}
		Siano $X$ spazio topologico e $f: X \rightarrow \R$ continua, se $X$ è compatto e diverso dal vuoto allora $f$ ammette massimo e minimo.
	\end{coro}
	\begin{dimo}
		$f(x)(\subseteq \R)$ è compatto, $f(x)\neq 0$ e limitato, quindi  $\sup f(x) \in\R\ni\inf f(x)$. $f(X)$ è anche chiuso, quindi $\sup f(x) = \max f(x)$ e  $\inf f(x) = \min f(x)$
	\end{dimo}
	\subsection{Come trovare omeomorfismi}
	Trovare esplicitamente un omeomorfismo è a volte molto rogonoso, di seguito troviamo degli strumenti per facilitare il lavoro.\\
\begin{coro}
	Sia $f: X \rightarrow Y$ applicazione continua fra spazi topologici. Se $X$ è compatto e $Y$ è $T2$ allora $f$ è chiusa
\end{coro}
\begin{dimo}
	$Y$ è $T2$, quindi $f(C)$ è chiuso in  $Y$.
\end{dimo}
Otteniamo ora un fatto utilissimo.
\begin{coro}
	Sia $f: X \rightarrow Y $ applicazione continua fra spazi topologici. Se $X$ è compatto, $Y$ è $T2$, e $f$ è biettiva, allora $f $ è omeomorfismo
\end{coro}
\begin{dimo}
	Dal corollario precedente, $f$ è chiusa. Allora è un continua, biettiva, chiusa, segue: omeomorfismo.
\end{dimo}
\begin{prop}
	Sia $f : X \rightarrow Y$ applicazione continua fra spazi topologici.\\
	Supponiamo
	\begin{enumerate}
		\item $f$ suriettiva.
		\item $Y$ compatto.
		\item $f^{-1}(y)$ compatto $\forall y\in Y$.
		\item  $f$ chiusa.
	\end{enumerate}
	Allora $X$ è compatto.
\end{prop}
\textbf{Osservazione}\\
La condizione analoga (4.) $f$ aperta, non è sufficiente a garantire la compattezza di $X$. (fogli di esercizi per controesempio)
 \begin{dimo}
	 Definiamo $A\csubseteq X$ aperto, un insieme  $A'\subseteq Y:$
	  \[
		  A' = \{y\in Y\ | \ f^{-1}(y)\subseteq A\}
	 .\] 
\textbf{Osservazione }\\
$A'$ potrebbe essere vuoto. Però A' è aperto in $Y$, verifichiamo che $Y\setminus A'$ è chiuso in $Y:$\\
 \[
	 Y\setminus A' = \{z\in Y\ | \ f^{-1}(z)\not\subseteq A\} = \{z\in Y\ | \ \exists b\in X\setminus A \ | \ f(b) = z\} = f(X\setminus A).
.\] 
Allora $Y\setminus A'$ è chiuso perché immagine di $X\setminu A$  tramite $f$ chiusa.\\
Sia $R$ ricoprimento aperto di $X$.\\
Sia  $y\in Y$, considero  $f^{-1}(y)$ è compatto. Allora esistono  $A_1,\ldots,A_n\in R$ tale che $f^{-1}(y)\subseteq A_1\cup\ldots\cup A_n$, ogni $A_i$ dipende da $y$.\\
Definiamo  $B_y = A_1\cup\ldots\cup A_n$ ($B_y$ dipende da $y$, definito come $A'$) aperto in $X$\\
Considero $B'_y$ è non vuoto e contiene $y\in Y$ \\
Segue:
\[
	\{B_y'\ | \ y\in Y\}\text{ è un ricoprimento aperto di } X
.\] 
Per compattezza possiamo estrarre un sottoricoprimento finito $B_1_{y_1},\ldots,B_{y_n}'$\\
Segue $\{B_{y_1},\ldots,B_{y_n}\}$ è ricoprimento finito aperto di $X$, Ciascun  $B_y$ è unione di un numero finito di elementi di $R$, quindi è un ammette un sottoricoprimento finito.
\end{dimo}
\begin{prop}
	Siano $P,Q$ spazi topologici. Se  $P$ è compatto allora la proiezioni $P\times Q \rightarrow Q$ è chiusa.
\end{prop}
\textbf{Osservazione:}\\
La proposizione in realtà è un'equivalenza: $P$ è compatto $ \Leftrightarrow p :P\times Q  \rightarrow Q$ è chiusa $\forall Q $ spazio topologico.
\begin{dimo}
	Sia $C\subseteq P\times Q$ chiuso in topologia prodotto.\\
	Allora $(P\times Q)\setminus C$ è aperto, vogliamo dimostrare che $Q\setminus q(C)$ è aperto, cioè che $Q \setminus q(C)$ è intorno di ogni suo punto $y\in Q\setimnus q(C)$ \\
	$P\times \{y\} (\subseteq P\times Q)$\\
	è omeomorfo a  $P$, quindi compatto.\\
	Consideriamo la solita base della topologia prodotto\\$(P\times Q)\setminus C = \bigcup^{}_{i\in I}U_i\times V_i$ \\
	Per compattezza $P\times\{y\}$ è contenuto nell'unione.\\
$P\times \{y\}\subseteq (U_{i_1}\times V_{i_1})\cup\ldots\cup(U_{i_n}\times V_{i_n})\subseteq (P\times Q)\setminus C$\\
Considero $V = V_{i_1}\cap\ldots\cap V_{i_n}$\\
 \[
	 P\times \{y\}\subseteq (U_{i_1}\times V)\cup\ldots\cup ( U_{i_n}\times V)\subseteq (P\times Q)\setminus C
.\] 
Allora poniamo $U = U_{i_1}\cup\ldots\cup U_{i_n}$\\
\[
	P\times\{y\}\subseteq U\times V \subseteq (P\times Q)\setminus C
.\] 
Segue: nessun punto di $V$ è la seconda coordinata di alcun punto di $C$ cioè  $y\in V\subseteq Q\setminus q(C)$\\
Segue  $q(C)$ è chiuso.
\end{dimo}
\textbf{Osservazione}\\
La dimostrazione assomiglia a quella su $T2 \Leftrightarrow \triangle$ chiusa nel prodotto (vedi teorema di Wallace sul Manetti).\\
\begin{coro}
	Se $P$ e $Q$ sono spazi topologici compatti allora $P\times Q$ compatto.
\end{coro}
\begin{dimo}
	Applichiamo a $q: P\times Q \rightarrow Q$ la proposizione che da condizioni sufficienti alla compattezza del dominio.\\
	Abbiamo $q$ continua, suriettiva, codominio $Q$ compatto, controimmagini $P\times \{y\}$ compatte, $q$ chiusa, per la proposizione precedente, Segue $P\times Q$ compatto
\end{dimo}
\textbf{Esempio}\\
$[0,1]\times[0,1] = [0,1]^2$ è compatto, e in generale $[0,1]^n\ \ \forall n\geq 1$  è compatto.\\
 \textbf{Osservazione}\\
 A questo  punto si dimostra facilmente $Y\subseteq \R^n$ è compatto $ \Leftrightarrow Y$ è chiuso e limitato.
 \subsection{Identificazioni}
 \begin{defi}
	 Un'applicazione $f: X \rightarrow Y$ fra spazi topologici si dice identificazione se
	 \begin{enumerate}
		 \item $f$ è continua e suriettiva
		 \item Un sottoinsieme $A\subseteq Y$ qualsiasi è aperto se e solo se  $f^{-1}(A)$ è aperto in $X$
	 \end{enumerate}
 \end{defi}
 \textbf{Osservazione}\\
Se $f $ è un'identificazione allora la topologia su $Y$ è determinata da $f$ e dalla topologia su $X.$\\
 \begin{lemm}
	Sia $f: X \rightarrow Y$ applicazione continua fra spazi topologici.
	\begin{enumerate}
		\item Se $f$ è suriettiva e aperta allora è un'identificazione.
		\item Se $f$ è suriettiva e chiusa allora è un'identificazione.
	\end{enumerate}
\end{lemm}
\begin{dimo}
	$1)$ Supponiamo  $f$ suriettiva, aperta e continua.\\
	Sia $A\subseteq Y$ supponiamo  $f^{-1}(A)$ aperto in $X$. Dimostriamo che $A$ è aperto in $Y$.\\
	Considero  $f(f^{-1}) = A\cap f(X) = A\cap Y = A$ aperto perché  $f$ è aperta\\
	Il punto 2 della dimostrazione è lasciata per esercizio ma è del tutto analoga.
\end{dimo}
\textbf{Esempi}\\
\begin{enumerate}
	\item Ogni omomorfismo è identificazione.
	\item Le proiezioni $p : P\times Q \rightarrow P$ e $q : P\times Q \rightarrow Q$ sono identificazioni.
	\item $f:[0,1] \rightarrow \ \ \ \ \ \ S^1$\\
		\text{}\ \ \ \ \ \ \ \ \ $t \rightarrow (\cos(2\pi t),\sin (2\pi t))$\\
		è suriettiva e continua.\\
		è anche chiusa perché $[0,1]$ è compatto e $S^1 $ è $T2$.
\end{enumerate}
\end{document}
