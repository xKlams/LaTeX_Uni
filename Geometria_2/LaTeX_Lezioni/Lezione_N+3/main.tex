\documentclass[12px]{article}

\title{Lezione N+3 Geometria 2}
\date{2025-05-20}
\author{Federico De Sisti}

\input{../../../setup.tex}

\begin{document}
	\maketitle
	\newpage
	\subsection{dimostrazione ultimo corollario}
	\begin{coro}
		$\pi_1(\pro_\R^n) \cong \frac{\Z}{2\Z}$ $\forall n\geq 2$
	\end{coro}
	\begin{dimo}
		Negli esercizi settimanali è dato un rivestimento
		\[
		S^n \rightarrow \pro_\R^n
		.\] 
		di grado $2 \ \ \forall n\geq 1$\\
		Per il teorema di ieri:
		 \[
			 p^{-1}(x) \leftrightarrow p_*(\pi_1(S^1))\backslash \pi_1(\pro^n_\R)
		.\] 
		dove $x =\pro^n_\R$ Se  $n\geq 2$ allora  $\pi_1(S^n)$ è banale, quindi $p_*(\pi(S^n))$ è banale, e $p_*(\pi_1(S^1))\backslash \pi_1(\pro^n_\R)$ è in biezione con $\pi_1(\pro^n_\R)$\\
		Quindi $\pi_1(\pro^n_\R)$ ha solo due elementi da cui $\cong \frac{\Z}{2\Z}$
	\end{dimo}
	\section{Geometria differenziale}
	\subsection{Varietà topologiche e differenziali}
	\begin{defi}
		Sia  $X$ spazio topologico. Esso si dice una varietà topologica di dimensione $n\in\Z_{\geq 0}$ se
		\begin{enumerate}
			\item $X$ di Hausdoff
			\item  $\forall x\in X$ esistono un intorno aperto  $U\subseteq X$ di $x$, un aperot  $V\subseteq \R^n$ e un omeomorfismo  $ \varphi: U \rightarrow V$ detto carta locale
			\item $X$ è $2^o$-numerabile.
		\end{enumerate}
		Una collezione di triple $(U,V, \varphi)$ tali che i sottoinsiemi $U$ ricoprono  $X$ è detta atlante.
\end{defi}
\textbf{Esempi:}
\begin{enumerate}
	\item $\R^n$ è una varietà topologica di dimensione  $n$ (basta una carta locale  $U = V = \R^n$,  $Id_{\R^n} = \varphi)$ 
	\item $S^n$ è varietà topologica di dimensione  $n$, per esempio posso prendere l'atlante\\
		\[
		U_1 = S^n\setminus\{(0,\ldots,0,1)\} \]\[
			U_2 = S^n\setminus\{(0,\ldots,0,-1)\}
		.\] 
		$V_1 = V_2 = \R^n$ \\
		$U_1 \rightarrow V_1$ proiezione stereografica 
	\item $T = $ toro in  $\R^3$\\
		AGGIUNGI IMMAGINE 5 39\\
		(il toro è omeomorfo a  $S^1\times S^1$ e al quoziente di un quadrato)
	\item Ciambelle con tanti buchi\\
		sono varietà di dimensione 2\\
		AGGIUNGI Immagine
	\item  $\pro^n_\R$ è varietà topologica di dimensione  $n$ :\\
		carte locali $U_i$ dove 
		 \[
			 U_i = \{[x_0,\ldots,x_n]\ | \ x_i\neq 0 \}
		.\] 
		$ \varphi : U_i \rightarrow V_i = \R^n$\\
		$[x_0,\ldots,x_n] \rightarrow (\frac {x_0} {x_i},\ldots,  \frac{x_{i-1}}{x_i},\frac{x_{i+1}}{x_i},\ldots, \frac {x_n} {x_i})$\\
		è ben definita, \underline{continua} (verifica per esercizio)\\
	 ed è omeomorfismo perchè ha inversa
	 \[
	 \begin{aligned}
		 V_i &\rightarrow U_i\\
		 (y_0,\ldots, y_n)& \rightarrow [y_0, \ldots, y_{i-1}, 1, y_{i+1},\ldots y_n]
	 \end{aligned}
	 .\] 
 \item $\C^n$ e  $\pro^n_\C$ sono varietà topologiche di dimensione  $2n$
\end{enumerate}
	  \begin{defi}
	 	Sia $A$ atlante di una varietà topologica di dimensione  $n.$  $A$ si dice  $C^\infty$ se per ogni
		 \[
			 (U_1,V_1, \varphi_1), \ \ (U_2,V_2, \varphi_2)\in A
		.\] 
		la composizione 
		\[
			\varphi_2\circ \varphi_1^{-1}|_{ \varphi_1( U_1\cap U_2)}: \varphi_1(U_1\cap U_2) \rightarrow \varphi_2(U_1\cap U_2)
		.\] 
		è di classe $C^\infty$ se $U_1\cap U_2\neq \emptyset$
	 \end{defi}
\textbf{Esempi}
\begin{enumerate}
	\item Gli atlanti visti prima per $\R^n, S^n, T, \pro^n_\R, \C^n, \pro^n_\C,$ sono tutti $C^\infty$
	\item Se $X $ ha un atlante fatto da due sole carte locali, allora questo atlante è $C^\infty$
\end{enumerate}
\begin{defi}
	Siano $A, B$ due atlanti $C^\infty$ di una stessa varietà topologica si dicono compatibili se $A\cup B$ è  $C^\infty$
\end{defi}
\textbf{Osservazione}\\
Si verifica facilmente che la compatibilità è una relazione d'equivalenza.
\begin{defi}
	Una varietà differenziale di dimensione $n$ è una varietà topologica di dimensione  $n$ con una classe di equivalenza di atlanti  $C^\infty$
\end{defi}
\textbf{Esempio}\\
$X =\R$\\
$A = \{(U_1,V_1, \varphi_1)\}$ $U_1 = X$, $V_1 = \R$ $ \varphi_1 = Id_X$ è atlante $C^\infty$\\
$B = \{(U_2, V_2, \varphi_2)\}$ $U_2 = X$ $V_2 = \R$ \\
\begin{aligned}
	$ \varphi_2: X = \R &\rightarrow \R$\\
	$x & \rightarrow x^3$
\end{aligned}\\
omeomorfismo\\
$A$ e $B$ non sono compatibili, i cambi di coordinate sono  \[
	\begin{aligned}
		\varphi_2\circ \varphi_1^{-1}: V_1 &\rightarrow V_2\\
						   x& \rightarrow x^3
	\end{aligned}
.\] 
\[
	\begin{aligned}
		\varphi_1\circ \varphi_2^{-1}: V_2 &\rightarrow V_1\\
		x& \rightarrow \sqrt[3]x 
	\end{aligned}
.\] 
è continua, biettiva non $C^\infty$\\
\subsection{Varietà differenziabili immerse in  $\R^N$ }
\begin{defi}[Varietà differenziabile immersa in $\R^N$]
	Sia $X\subseteq\R^N$ sottospazio topologico $(N\geq 0)$. $X$ è detta varietà differenziabile immersa in  $\R^N$ di dimensione  $m\in\Z_{\geq 0}$ se $\forall x\in X$ esistono  $U\subseteq X$ intorno aperto di  $x, V\subseteq\R^m$ aperto,
	 \[
	\psi : V \rightarrow U\subseteq \R^N
	.\] 
	tale che 
	\begin{enumerate}
		\item 	$\psi$ è omeomorfismo
		\item $\psi $ è  $C^\infty$ come applicazione  $V \rightarrow \R^N$ con $V\subseteq \R^m$ aperto 
		\item $\forall q\in V$ il differenziale di  $d\psi_q$ è iniettivo
				 $(d\psi_q: \R^m \rightarrow \R^n$ è l'applicazione lineare di matrice canonica Jacobiana di $\psi$ in $q)$
	\end{enumerate}
	le $\psi$ si dicono parametrizzazione\\
	Gli aperti  $U$ si dicono aperti coordinati.\\
\end{defi}
\textbf{Nota}\\
	In letteratura spesso "carte locali" e "parametrizzazioni" sono sinonimi.\\
	Invece di "immerse" si dice spesso "immerse regolarmente", e in inglese questo "immerse" corrisponde a "embedded"
\textbf{Esempi:}\\
\begin{enumerate}
	\item $S^1$ è varietà differenziabile immersa in $\R^2$\\
		 \[
			 \begin{aligned}
				 \psi_1: ]0,2\pi[ &\rightarrow U_1 = S^1\setminus\{(1,0)\}\\
						  t & \rightarrow (\cos(t), \sin (t))
			 \end{aligned}
		.\] 
		 \[
			 \begin{aligned}
				 \psi_2: ]-\pi,\pi[ &\rightarrow U_1 = S^1\setminus\{(-1,0)\}\\
						  t & \rightarrow (\cos( t), \sin (t))
			 \end{aligned}
		.\] 
		Si verifica facilmente che $\psi_1, \psi_2$ sono continue, biettive, con inversa continua,\\
		La matrice Jacobiana di $\psi_1$ è
		\[
			(J\psi_1) = \matrice{-\sin(t)\\\cos(t)}
		.\] 
		matrice di un'applicazione lineare
		\[
		\R \rightarrow\R^2
		.\] 
		ha rango $1\ \ \forall t$ quindi  $(d\psi_1)_t$ è iniettiva $\forall t$.\\
		Quindi la definizione è soddisfatta.
	\item  Sia  $V\subseteq \R^m$ aperto. Sia  $f: V \rightarrow \R$ $C^\infty$\\
		il grafico di  $f$ in $\R^{m+1}$ è una varietà grafico immersa in  $\R^{m+1}$ di dimensione $m$.\\
		Infatti  $\Gamma = \{(x_1,\ldots, x_m, f(x_1,\ldots,x_n))\ | \ (x_1,\ldots,x_n)\in V\}$\\
		mettiamo la singola parametrizzazione
		\[
		\begin{aligned}
			$\psi : V & \rightarrow U = \Gamma$\\
			(x_1,\ldots,x_m) & \rightarrow (x_1,\ldots,x_m, f(x_1,\ldots,x_m))
		\end{aligned}
		.\] 
		$\psi$ è biettiva, continua, $\psi^{-1}$ è la restrizione a $\Gamma$ alle prime  $m$ coordinate, quindi $\psi^{-1}$ è continua\\
		 $\psi$ è  $C^\infty$ perché lo sono le sue componenti.\\
		 La matrice Jacobiana è IMMAGINE\\
		 ha rango $m$ quindi il differenziale è iniettiva.\\
		 \textbf{Esercizio}\\
		 Trovare parametrizzazione che rendano $S^n$ una varietà differenzaibile immersa in $\R^{n+1}$ (Suggerisco di usare parametrizzazione come quella del grafico)
		 \textbf{Esempio:}\\
		 Nella definizione non è sufficiente richiedere $\psi$cibtubya biettiva,  $C^\infty$, un differenziabile iniettivo in ogni punto. Cioè da queste ipotesi non segue $\psi^{-1}$ continua
\end{enumerate}

\end{document}
