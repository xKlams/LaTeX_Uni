\documentclass[12px]{article}

\title{Lezione 12 Geometria 2}
\date{2025-04-01}
\author{Federico De Sisti}

\usepackage{amsmath}
\usepackage{amsthm}
\usepackage{mdframed}
\usepackage{amssymb}
\usepackage{nicematrix}
\usepackage{amsfonts}
\usepackage{tcolorbox}
\tcbuselibrary{theorems}
\usepackage{xcolor}
\usepackage{cancel}

\newtheoremstyle{break}
  {1px}{1px}%
  {\itshape}{}%
  {\bfseries}{}%
  {\newline}{}%
\theoremstyle{break}
\newtheorem{theo}{Teorema}
\theoremstyle{break}
\newtheorem{lemma}{Lemma}
\theoremstyle{break}
\newtheorem{defin}{Definizione}
\theoremstyle{break}
\newtheorem{propo}{Proposizione}
\theoremstyle{break}
\newtheorem*{dimo}{Dimostrazione}
\theoremstyle{break}
\newtheorem*{es}{Esempio}

\newenvironment{dimo}
  {\begin{dimostrazione}}
  {\hfill\square\end{dimostrazione}}

\newenvironment{teo}
{\begin{mdframed}[linecolor=red, backgroundcolor=red!10]\begin{theo}}
  {\end{theo}\end{mdframed}}

\newenvironment{nome}
{\begin{mdframed}[linecolor=green, backgroundcolor=green!10]\begin{nomen}}
  {\end{nomen}\end{mdframed}}

\newenvironment{prop}
{\begin{mdframed}[linecolor=red, backgroundcolor=red!10]\begin{propo}}
  {\end{propo}\end{mdframed}}

\newenvironment{defi}
{\begin{mdframed}[linecolor=orange, backgroundcolor=orange!10]\begin{defin}}
  {\end{defin}\end{mdframed}}

\newenvironment{lemm}
{\begin{mdframed}[linecolor=red, backgroundcolor=red!10]\begin{lemma}}
  {\end{lemma}\end{mdframed}}

\newcommand{\icol}[1]{% inline column vector
  \left(\begin{smallmatrix}#1\end{smallmatrix}\right)%
}

\newcommand{\irow}[1]{% inline row vector
  \begin{smallmatrix}(#1)\end{smallmatrix}%
}

\newcommand{\matrice}[1]{% inline column vector
  \begin{pmatrix}#1\end{pmatrix}%
}

\newcommand{\C}{\mathbb{C}}
\newcommand{\K}{\mathbb{K}}
\newcommand{\R}{\mathbb{R}}


\begin{document}
	\maketitle
	\newpage
	\subsection{Nuovo sito del prof}
	Sito del corso www.sites.google.com/uniroma1.it/guidopezzini/\\
	\subsection{Gruppi topologici}
	\begin{defi}
		Un gruppo topologico è un gruppo $G$ che è anche uno spazio topologico\\
		tale che l'operazione di gruppo
		\[
		\begin{aligned}
			&G\times G \rightarrow G\\
			& (g,h) \rightarrow g\cdot h
		\end{aligned}
		.\] 
		e l'inverso
		\[
		\begin{aligned}
			&G \rightarrow G\\
			& g \rightarrow g^{-1}
		\end{aligned}
		.\] 
		sono applicazioni continue
	\end{defi}
	\textbf{Esempi}
	\begin{enumerate}
		\item Se $G$ è un gruppo qualsiasi diventa un gruppo topologico con topologia banale o discreta.
		\item $(\R^n,+)$ $\R^n$ con topologia euclidea è un gruppo topologico.
		\item  $GL(n,\R), GL(n,\C)$ con prodotto di matrici\\
			Identifichiamo \\
			$Mat_n(\R)$ con $\R^{n^2}$ e  $Mat_N(\C)$ con $\R^{2n^2}$\\
			mettiamo su  $GL(n)$ la topologia di sottospazio. Allora $GL(n,\R)$ e  $Gl(n,\C)$ sono gruppi topologici 
		\item Anche sottogruppi noti quali $Sl(n), So(n), U(n),\ldots$ \\
			sono gruppi topologici.
	\end{enumerate}
	\textbf{Esercizio}(difficile)\\
	Sia  $G$ un gruppo topologico $T2$, Sia $H\subseteq G$ un sottogruppo chiuso. Consideriamo
	 \[
		 G/H = \{\text{classi laterali }gH \text{ con } g\in G\}
	.\] 
	Ricordo: $G/H = G/\sim$\\
	dove  $g_1\sim g_2 \Leftrightarrow g_1H = g_2H$\\
	Dimostrare che $G/H$ è $T2$ (con la topologia quoziente)\
	\newpage
\subsection{Proprietà di numerabilità}
\begin{defi}
	Sia $X$ spazio topologico
	 \begin{enumerate}
		 \item $X$ si dice $1^o$-numerabile se ogni punto ha un sistema fondamentale di intorni numerabili
		 \item $X$ si dice $2^o$-numerabile se  la sua topologia ha una base numerabile.
		 \item $X$ si dice separabile se $X$ ha un sottoinsieme denso e numerabile.
	\end{enumerate}
\end{defi}
\textbf{Esempi}
\begin{enumerate}
	\item $\R^n\ni p,$ sistema fondamentale di intorni e  $\{B_{\frac 1n}(p)\ | \ n\in \Z_{\geq 1}\}$\\
		quindi $\R^n$ è $1^o $-numerabile\\
		Base numerabile
		\[
			\{B_{\frac 1n}(q)\ | \ n\in \Z_{\geq 1 }, q\in \Q^n\}
		.\] 
		quindi $\R^n$  è $2$-numerabile. Inoltre $\Q^n$ è denso in $\R^n$ ed è numerabile quindi  $\R^n$ è separabile.
	\item Ogni spazio topologico finito è $1^o$-numerabile, $2^o$-numerabile, separabile.
	\item Ogni spazio topologico numerabile è separabile
	\item Sia $X$ spazio topologico discreto, di qualsiasi cardinalità: è $1^o$-numerabile, Per ogni $x\in X$\\
		$\{x\}$ è intorno di $x$, $\{\{x\}\}$ è un sistema fondamentale di intorni.

\end{enumerate}
\begin{lemm}
	Ogni spazio topologico $2^o$-numerabile è $1^o$ numerabile
\end{lemm}
\begin{dimo}
	Sia $B$ base numerabile, Sia $x\in X$ e consideriamo  $J = \{C\in B\ | C \ni x\}$.\\
	Allora  $J$ è sistema fondamentale di intorni. Infatti sia $U$ intorno di x,\\
	sia $A\subseteq X$ aperto tale che $x\in A\subseteq U$ scriviamo  $A \bigcup^{}_{i\in I}B_i$\\
	con $B_i\in B.$ Esiste $i_0\in I\ | \ B_{i_0}\ni x$ allora $B_{i_0}\in J$ e $B_{i_0}\subseteq U$. Segue $J$ è sistema fondamentale di intorni
\end{dimo}
\begin{prop}
	Ogni spazio metrico:
	\begin{enumerate}
	\item  è $1^o$-numerabile
	\item se è separabile allora è $2^o$-numerabile
	\end{enumerate}
\end{prop}
\begin{dimo}
	Procediamo per ogni punto
	\begin{enumerate}
		\item Sia $p\in X$ $(X$ spazio metrico$)$\\
			allora $\{B_{\frac 1n}(p)\ |\ n\in \Z_{\geq 1}\}$ è sistema fondamentale d'intorni 
		\item Sia $E$ sottoinsieme denso numerabile di $X$, 
			\[
				B = \{B_{\frac 1n}(e)\ | \ e\in E\ \ n\in \Z_{\geq 1}\}
			.\] 
			Verifichiamo che è una base. Sia $A\subseteq X$ aperto.\\
			Dato $a\in A$\\
		scegliamo  $n(a)\in \Z_{\geq 1}$ tale che $B_{\frac{2}{n(a)}}(a)\subseteq A}$\\
		Consideriamo $B_{\frac{1}{n(a)}}(a)$ e un punto  $e\in E$ tale che 
		 \[
			 e\in B{\frac 1 {n(a)}}(a) \ 
		.\] 
		ricordiamo: $B_{\frac 2{n(a)}}(a)\subseteq A.$ Allora\\
		$B_{\frac {1}{n(a)}}(e)\ni a$\\
		Segue  $a\in B_{\frac{1}{n(a)}}(e)\subseteq B_{\frac{2}{n(a)}}(a)\subseteq A$ per la disuguaglianza triangolare.\\
		Chiamiamo  $e = e(a)$ Abbiamo
		\[
			A = \bigcup^{}_{a\in A} B_{\frac 1 {n(a)}} (e(a))
		.\] 
		Segue $B$ è base.
	\end{enumerate}
\end{dimo}
\begin{lemm}
	Ogni spazio topologico $2^o$-numerabile è separabile.
\end{lemm}
\begin{dimo}
	Sia $B$ base numerabile, $B = \{A_1,A_2,\ldots\}$\\
	basta scegliere $e_i\in A_i \ \ \forall i$ e porre  $E = \{e_1,e_2,\ldots\}$
\end{dimo}
\textbf{Esempio:}\\
\underline{Non è vero} che ogni spazio topologico $1^o$-numerabile è separabile e $2^o$-numerabile. Ad esempio $\E$ con topologia di Sorgenfey\\
Una base è $B = \{[a,b[\ | \ a<b\}$\\
Quindi  $\Q$ è denso in $\R$ anche con questa topologia, quindi $\R $ è separabile.\\
Per ogni $a\in \R$ la famiglia  $\{[a,a + \frac 1n[\ | \ n\in \Z_{\geq 1}\}$\\
è un sistema di intorni di  $a$. Ma $\R$ con questa topologia \undelrine{non} è $2^o$-numerabile.\\
Veridica per esercizio (suggerimento, considero $[x,x + 1[ \ \ \forall x\in \R$ e usare il fatto che $x $ è il minimo di qualsiasi sottoinsieme $A $ tale che $x\in A\subseteq[x,x+1[$\ )\\
Segue anche che questo spazio topologico non è metrizzabile (ma è  T2)\\
\subsection{Successioni}
\begin{defi}
	Sia $X$ spazio topologico.\\
	Una successione in $X$ è un'applicazionea
	\[ \begin{aligned}
		a:&\Z_{\geq 1} \rightarrow X\\
		& n \rightarrow a_n
	\end{aligned}\]
	Si dice che $a$ converge a $p\in X$ se $\forall U$ intorno di $p$ $\exists N\in\Z_{\geq 1}\ | \ a(n)\in U\ \ \ \foralli n\geq N$
\end{defi}
\textbf{Osservazione:}\\
Se $X$ ha topologia banale ogni successione converge a ogni $p\in X$\\
Se $X$ è $T2$ allora i limiti delle successioni sono unici se $a$ (succ.) converge a $p$ e $q$ allora $p=q$ \\
\begin{prop}
	Sia $X$ spazio topologico $1^o$-numerabile, Siano $A\subseteq X, p\in X$ sono equivalenti:
	 \begin{enumerate}
		 \item esiste una successione in $A$ che converge a $p$ 
		 \item $p\in\overline A$
	\end{enumerate}
\end{prop}
\begin{dimo}
	$1) \Rightarrow  2)$ \\
	Se vale 1) ogni intorno di $p$ interseca $A$ quindi $p\in \overline A$ \\
	$2) \Rightarrow 1)$ \\
	Supponiamo $p\in \overline A$.\\
	Sia  $\{U_n\}$ sistema fondamentale di intorni numerabile.\\
	Considero $\forall n\geq 1$  $U_1\cap\ldots\cap U_n$\\
	è intorno di $p$, scelgo  $a(n)\in A\cap U_1\cap\ldots\cap U_n$\\
	Allora la successione 
	 \[
		 \begin{aligned}
		 a:&\Z_{\geq 1} A \rightarrow A\\
		   & n \rightarrow a(n)
		   \end{aligned}
	.\] 
	converge a $p$. Verifica sia $U$ intorno di  $p$ sia $N$ tale che $U_n\subseteq U$\\
	per ogni $n\geq N$ abbiamo
	 \[
		 a(n)\in U_1\cap \ldots\cap U_n\cap\ldots\cap U_n
	.\] 
	quindi $a(n)\in U$ e la successione converge a $p$.
\end{dimo}
\begin{defi}[Sottosuccessioni]\
	\text{ }
	\begin{enumerate}
		\item Una sottosucceisone di una successione $a: \Z_{\geq 1} \rightarrow X$\\
			è una successione del tipo
			\[
			b(n) = a(f(n))
			.\] 
			dove $f:\Z_{\geq 1} \rightarrow \Z_{\geq 1}$ \\
			è strettamente crescente.
		\item Uno spazio topologico è compatto per successioni se ogni successione ha sottosuccessioni convergenti.
	\end{enumerate}
\end{defi}
\textbf{Osservazioni}\\
La compattezza e la compattezza per successioni non sono equivalenti.\\
Esistono compatti per successioni ma non compatti. (esempio della linea lunga, long line)\\
Esistono spazi topologici compatti ma non compatti per successioni\\
(prodotti di infiniti spazi topologici).
\end{document}
