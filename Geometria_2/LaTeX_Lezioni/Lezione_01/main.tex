\documentclass[12px]{article}

\title{Lezione 1 Geometria 2}
\date{2025-02-27}
\author{Federico De Sisti}

\input{../../../setup.tex}

\begin{document}
	\maketitle
	\newpage
	\section{Informazioni pratiche}
	Giovedì esercitazioni\\
	Ci sono gli esercizi settimanali! Alcuni di questi sono da sapere per l'orale\\
	Se vogliamo essere avvertiti per urgenze possiamo mandare una mail\\
	C'è il sito del corso\\
	Per la maggior parte del corso di studia su Topologia di Marco Manetti\\
	\textbf{Esami:}\\
	Ci sono 2 esoneri\\
	L'esame è scritto e orale\\
	\textbf{Prerequisiti}\\
	1) Familiarità con le funzioni continue\\
	2) Un po' di teoria dei gruppi\\
	3) Derivate di applicazioni in più variabili\\[10px]
	Il corso è diviso in 3 parti:\\
	1)Topologia generale\\
	2) Topologia algebrica\\
	3) Geometria differenziale\\
	\section{Topologia Generale}
	\subsection{Introduzione}
	Nasce per studiare sottoinsiemi di $\R^n$, cosa posso fare con un sottoinsieme di  $R^n$ con un applicazione continua?\\
	Studieremo:\\
	1) Proprietà dei sottoinsiemi di $\R^n$, come ad esempio la compattezza, da un punto di vista astratto.\\
	2) Applicheremo le stesse proprietà ad insiemi dotati di "geometria" meno intuitiva\\
	Ad esempio la topologia generale si applica in 
	\begin{itemize}
		\item Analisi
		\item Algebra
		\item Logica
	\end{itemize}
	\textbf{Esempio}\\
	In $\R^2$ prendiamo \\
	 \[
		 X = \R\times\{0,1\}
	.\] 
	%TODO inserisci grafico delle rette parallele all'asse x passanti per 00 e 01\\
	Poniamo, in maniera informale, questa relazione d'equivalenza:
	\[
		(x,0)\sim (x,1) \ \ \ \ \forall x\in\R
	.\] 
	Il quoziente "assomiglia" ad una sola retta, gli elementi equivalenti vengono "appiccicati".\\
Seconda relazione d'equivalenza:
\[
	(x,0)\sim(x,1) \text{   solo per   } x\leq 0
.\] 
$X/\sim$ in questo caso assomiglia a :\\
%TODO aggiungi immagine diapasol
Terza relazione d'equivalenza:
\[
	(x,0)\sim (x,1) \text{    solo per    } x < 0
.\] 
%TODO aggiungi immagine pisello (due punti nello stesso posto)
Una specie di analogo della figura precedente, ma il punto $[0,0]$ è raddoppiato
\begin{defi}[Funzioni continue]
	Dati $X\subseteq \R^n$ e  $Y\subseteq \R^m$ insiemi qualsiasi, si definisce continua un'applicazione  $f:X \rightarrow Y$ se 
	\[
		\forall p\in X \ \ \forall \epsilon >0\ \ \exists\delta>0\ \  \text {se} \ \ x\in X\ \  \text {soddisfa}
	.\] 
	\[
		||x-p||<\delta \ \ \text{allora} \ \ ||f(x)-f(p)||<\epsilon
	.\] 
\end{defi}
\begin{defi}[Omeomofrismo]
	Data $f: X \rightarrow Y$ si dice omeomorfismo se è biettiva, continua,\\ e $f^{-1} :Y \rightarrow X$ è continua.
\end{defi}
\textbf{Osservazione}\\
In topologia generale gli omeomorfismo hanno un ruolo analogo agli isomorfismi in algebra e algebra lineare.\\
\textbf{Esempio:}\\
1) $[0,1]$ (in  $\R$) è omeomorfo ad $[a,b] \ \ \forall a,b\in \R$ con $a<b$\\
ad esempio\\
\begin{center}
\begin{aligned}
	f:[0,1] \rightarrow [a,b]\\
	  &t \rightarrow (1-t)a + tb
\end{aligned}
\end{center}
è biettiva, continua e $f^{-1}$ è continua.\\
2) $S^1 = \{(x,y)\in \R^2|x^2 + y^2 = 1 \}$ e $Q = \{(x,y)\in\R^2 | max\{|x|,|y|\} = 1\}$\\
%TODO inserisci immagine quadrato
Per esempio possiamo normalizzare i punti del quadrato
\begin{center}
\begin{aligend}
	&Q \rightarrow S^1\\
	&p \rightarrow\frac {p}{||p||}
\end{aligend}
\end{center}
che è continua, biettiva e ha inversa continua.\\
3)$[0,1]\cup]2,3]$ non è omeomorfo a  $[0,2]$\\
ad esempio \\\begin{aligned}
	f:[0,1]&\cup]2,3] \rightarrow[0,2] \\
	&x \rightarrow \begin{cases}
		x \ \ \ \ \text{se} \ x\leq 1\\
		x - 1 \ \ \text{ se} \ x> 1
	\end{cases}
\end{aligned}\\
Con l'analisi matematica si dimostra facilmente che non esiste alcuna biezione con inversa continua.\\
\textbf{Osservazione}\\
In algebra se $f$ è un omomorfismo biettivo allora $f^{-1}$ è un omomorfismo.\\
In topologia se $f$ è continua e biettiva, $f^{-1}$ non è sempre continua.\\
4) $]0,1[$ è omeomorfo a $]a,b[$  $\forall a,b\in\R$  $a < b$\\
Inoltre  $]0,1[$ è omeomorfo a $]0,+\infty[$\\
ad esempio tramite\\
 \begin{aligned}
	 ]0,&+\infty[ \rightarrow ]0,1[\\
	    & x \rightarrow e^{-x}
\end{aligned}\\
5)$]0,+\infty[$ è omeomorfo a $\R$, ad esempio tramite $ x \rightarrow log(x)$\\
6) $]0,1[$ non è omeomorfo a $[0,1]$\\
7) $S^n = \{p\in\R^{n+1} |\ \  ||p|| = 1\}$\\
$S^n\setminus\{(0,\ldots,0,1)\}$ è omeomorfo a $\R^n$
%TODO inserisci disegno circonferenza senza un punto
Ad esempio tramite la proiezione stereografica (esercizio: vedere la formula)\\
8) Ci sono molti esempi di figure omeomorfe fra loro, ma un omeomorfismo esplicito è difficile, ad esempio.\\
un $l$-agono regolare qualsiasi (in  $\R^n$) e un  $r$-agono qualsiasi sono omeomorfi ($\forall\ l,r\geq 3)$ \\
9)  $\R^n$ e $\R^m$ sono omeomorfi se e solo se  $n = m$ (è un teorema difficile, nel corso vedremo la dimostrazione per qualche esponente specifico, $n\leq 2$)\\
Vediamo due riformulazioni della continuità.
\begin{defi}
$A\subseteq \R^n$ si dice aperto se 
\[
	\forall p\in A \ \ \exitst \varepsilon > 0 \ \ | \ \ \text{se}\ \  x\in \R^n
.\] 
\[
	\text{soddisfa } \  ||x-p|| <\espilon \ \text{ allora } x\in A
.\] 
\end{defi}
\begin{nota}
	$B_\varepsilon (p) = \{x\in\R^n| \ ||x-p|| < \epsilon\}$\\
	è la palla aperta di centro $p$ e raggio $\varespilon$
\end{nota}
\begin{defi}
	Sia $X\subseteq \R^n$ e  $p\in \R^n$ p si dice aderente a $X$ se  $\forall \espilon>0 \ \exists x\in X \  | \ ||x-p||<\epsilon$
	%TODO aggiungi disegno palla storta 
	(chiaramente se $p\ini X$  allora è aderente a $X$, basta prendere  $x = p$)
\end{defi}
\textbf{Esempio:}\\
$1\in\R$ è aderente a $X = [0,1[$\\
 \begin{prop}
	Sia $f:\R^n \rightarrow\R^m$ Sono equivalenti:
	\begin{enumerate}
		\item $f$ è continua
		\item $\forall Z\subseteq \R^n \ \ \forall p\in \R^n$ se $p$ è aderente a $Z$\\ allora  $f(p)$ è aderente a $f(Z)$ 
		\item $\forall A\subseteq \R^m$ se  $A$ è aperto, allora $f^{-1}(A)$ è aperto.
	\end{enumerate}
\end{prop}
\begin{dimo}
	$1) \Rightarrow  2)$ \\
	Siano $Z\subseteq\R^n, p\in\R^n$ suppongo $p$ aderente a  $Z$, dimostriamo che  $f(p)$ è aderente a $f(Z)$\\
	%TODO aggiungi immagine palla in quadrato
	Sia  $\epsilon > 0 $ qualsiasi, sia $\delta > 0 $, dalla continuità di  $f$ in $p$. Dato che $p$ è aderente a $Z$ esiste  $z\in Z$  tale che  $||z-p|| < \delta$.\\
	Allora  $f(z)$ è un punto di $f(Z)$ e  $||f(p)-f(z)|| < \varepsilon$.\\
	 $2) \Rightarrow 3)$ \\
	 Sia $A\subseteq \R^m$ aperto, dimostriamo che $f^{-1}(A)$ è aperto, dimostriamo che $f^{-1}(A)$ è aperto in $\R^m$\\
	 %TODO aggiungi immagine\\
	 Sia  $p\in f^{-1}(A)$ sia $f(p)\in A$\\
	 Per assurdo supponiamo.\\
	 $\forall \varespilon > 0 \exists q\in\R^n$ fuori da  $f^{-1}(A)$ ma  $||p-q|| < \varepsilon$.\\
	 Allora $p$ è aderente a  $\R^m\setminus f^{-1}(A)$.\\
	 Segue, per 2),  $f(p)$ è aderente a $f(\R^n\setminus f^{-1}(A))$\\
	 quindi per ogni  $\eta > 0$ esistono punti a distanza  $< \eta$ non in  $f(\R^n\setminus f^{-1}(A))$ punti che non vanno in $A$\\
	 allora  $f(p)$ è aderente a $\R^m\setminus A$. Questo è assurdo perché  $A$ è aperto\\
	 La terza implicazione la vediamo lunedì
\end{dimo}



\end{document}
