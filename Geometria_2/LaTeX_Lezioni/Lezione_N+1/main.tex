\documentclass[12px]{article}

\title{Lezione N+1}
\date{2025-05-13}
\author{Federico De Sisti}

\usepackage{amsmath}
\usepackage{amsthm}
\usepackage{mdframed}
\usepackage{amssymb}
\usepackage{nicematrix}
\usepackage{amsfonts}
\usepackage{tcolorbox}
\tcbuselibrary{theorems}
\usepackage{xcolor}
\usepackage{cancel}

\newtheoremstyle{break}
  {1px}{1px}%
  {\itshape}{}%
  {\bfseries}{}%
  {\newline}{}%
\theoremstyle{break}
\newtheorem{theo}{Teorema}
\theoremstyle{break}
\newtheorem{lemma}{Lemma}
\theoremstyle{break}
\newtheorem{defin}{Definizione}
\theoremstyle{break}
\newtheorem{propo}{Proposizione}
\theoremstyle{break}
\newtheorem*{dimo}{Dimostrazione}
\theoremstyle{break}
\newtheorem*{es}{Esempio}

\newenvironment{dimo}
  {\begin{dimostrazione}}
  {\hfill\square\end{dimostrazione}}

\newenvironment{teo}
{\begin{mdframed}[linecolor=red, backgroundcolor=red!10]\begin{theo}}
  {\end{theo}\end{mdframed}}

\newenvironment{nome}
{\begin{mdframed}[linecolor=green, backgroundcolor=green!10]\begin{nomen}}
  {\end{nomen}\end{mdframed}}

\newenvironment{prop}
{\begin{mdframed}[linecolor=red, backgroundcolor=red!10]\begin{propo}}
  {\end{propo}\end{mdframed}}

\newenvironment{defi}
{\begin{mdframed}[linecolor=orange, backgroundcolor=orange!10]\begin{defin}}
  {\end{defin}\end{mdframed}}

\newenvironment{lemm}
{\begin{mdframed}[linecolor=red, backgroundcolor=red!10]\begin{lemma}}
  {\end{lemma}\end{mdframed}}

\newcommand{\icol}[1]{% inline column vector
  \left(\begin{smallmatrix}#1\end{smallmatrix}\right)%
}

\newcommand{\irow}[1]{% inline row vector
  \begin{smallmatrix}(#1)\end{smallmatrix}%
}

\newcommand{\matrice}[1]{% inline column vector
  \begin{pmatrix}#1\end{pmatrix}%
}

\newcommand{\C}{\mathbb{C}}
\newcommand{\K}{\mathbb{K}}
\newcommand{\R}{\mathbb{R}}


\begin{document}
\maketitle
\newpage
\subsection{Sollevamenti di cammini}
\textbf{Esempi}
\begin{enumerate}
	\item $\rho : \R \rightarrow S^1$ \\
		solito rivestimento,\\
		\begin{aligned}
			$\alpha : [0&,1] \rightarrow S^1$\\
				  &$t \rightarrow (\cos(2\pi t), \sin (2\pi t))$
		\end{aligned}\\
		$\alpha\in \Omega(S^1, (1,0),(1,0))$\\ 
		I numeri $t\in \R$ t.c. $\rho(t) = a$ sono gli interi.\\
		Possiamo sollevare  $\alpha$ partendo da 
		\[
			\begin{aligned}
				\alpha^\uparrow_0: [0,1] \rightarrow &\R \rightarrow S^1\\
						   &t \rightarrow  (\cos(2\pi t), \sin(2\pi t))\\
				t \rightarrow ?&
			\end{aligned}
		.\] 
		dove "?" è tale che composto con $\rho$ fa $\alpha$\\
		quindi  $\alpha_0^\uparrow (t) = t$ \\
		Posso partire da qualunque $n\in \Z$:\\
		 $\alpha_3^t (t) = t + 3$\\
		 Composto con $\rho$ da  $\alpha$ e parte da $3$\\
		 Potrei usare 
		  \[
		 \begin{aligned}
			 \beta : &[0,1] \rightarrow S^1\\
				 &t \rightarrow (\cos(-6 \pi t), \sin (-6\pi t))
		 \end{aligned}
		 .\] 
		 Esempi di sollevamento:\\
		 $\beta^\uparrow_5(t) = 5 - 3t$
\end{enumerate}
\begin{teo}[Sollevamento delle omotopie di cammini]
	Sia $p: E \rightarrow X$ un rivestimento, $F:[0,1]\times [0,1] \rightarrow X$ continua,\\
	$e\in E$ tale che  $p(e) = F(0,0)$.\\
	Allora esiste un unico sollevamento  $g: [0,1]\times [0,1] \rightarrow E$ di $F$ tale che  $G(0,0) = e$.
\end{teo}
\begin{dimo}
	L'unicità segue dal teorema di unicità dei sollevamenti (quello di $Y \xrightarrow{f} X$ è $Y$ connesso). Dimostriamo l'esistenza di $G$.\\
	Considero  $F(-,0)$ è un cammino  $[0,1] \rightarrow X$ e anche $F(0,-)$ è un cammino  $[0,1] \rightarrow X$ \\
	Solleviamo partendo da $e$, otteniamo i sollevamenti\\
	\[
		\alpha: [0,1] \rightarrow E
	.\] 
	\[
		\beta :[0,1] \rightarrow E
	.\] 
	Soddisfano \\
	$p(\alpha(t)) = F(t,0)$\\
	$p (\beta(t)) = F(0,t)$ \\
	e coincidono per $t =0 $ \\
	Definiamo $L: ([0,1]\times\{0\} ) \cup (\{0\}\times [0,1])\subseteq Q = [0,1]\times[0,1]$\\
	Definiamo\\
	 \[
	\begin{aligned}
		g: &L \ \ \ \ \rightarrow \ \ \ E\\
		   & (t,s) \rightarrow \begin{cases}
		   \alpha(t) \ \ \ \text { se } s = 0\\
			   \beta(s) \ \ \text{ se } t = 0
		   \end{cases}
	\end{aligned}
	.\] 
	$g$ è continua e solleva\\
	\[
	F|_L : L \rightarrow X
	.\] 
	Quindi vogliamo dimostrare che esiste $G$ sottoinsieme di $F$ che coincide con $g$ su $L\subseteq Q$\\
	Passo 1:\\
	Supponiamo l'immagine di  $F$ contenuta in un aperto banalizzante $V \subseteq X.$\\
	Sia $p^{-1}(V) = \bigcup_{i \in I}U_i$ come nella definizione.\\
 $L$ è connesso, $g(L)$ è connesso, e contenuto in $p^{-1}(V)$ \\
 Per connessione esiste un unico $i_0\in I$ tale che $g(L)\subseteq U_{i_0}$\\
 Sia $s: V \rightarrow U_{i_0}$ la sezione locale, poniamo $G = s\circ F$ questa solleva  $F$\\
  Coincide con $g$ su  $L$ per l'unicità dei sollevamenti. \\
 Passo 2: caso generale.\\
 Non supponiamo $Im(F)$ contenuta in un aperto banalizzante.\\
 Dal teorema del numero di Lebesgue esiste  $n\in \Z_{\geq 1}$ tale che \\
  \[
	  \left[\frac{i-1}n,\frac in\right]\times\left[\frac{i-1}n,\frac in\right] = Q_{i,j}
 .\]
 un singolo aperto banalizzante $V_{i,h}\subseteq X$\\
 Se sollevo ogni quadratino in ordine con continuità, posso "appiccicarlo" a quelli vecchi, così che la mia funzione non abbia "salti" e sia quindi discontinua \\
 Per il passo 1, posso sollevare $F|_{Q_{i,j}}$\\
 Per  assicurare che questi sollevamenti si incollino, ordiniamo i $Q_{i,j}$ la coppia definiamo $(i,j) \leq (h,k) \Leftrightarrow \begin{cases}
 	i + k < h+ k \ \\
	i + j = h + k, \ \ i\ \leq k
 \end{cases}$
 Aggiungi foto 6:00 13 maggio.\\
 Vale: i lati inferiore e sinistro di $Q_{i,j}$ sono contenuti in 
  \[
	  L\cup \bigcup^{}_{(a,b) < (i,j)} Q_{a,b}
 .\]
 Patiamo da $Q_{1,1}:$ per il passo $1$ esiste, un sollevamento:
  \[
	  \tilde G_{1,1}: Q_{1,1} \rightarrow E
 .\] 
 tale che $\tilde G (0,0) = e, $ e  $\tilde G$ solleva  $F|_{Q_{1,1}}: Q_{1,1} \rightarrow X$ \\
 Per l'unicità dei sollevamenti, $g$ e  $\tilde G_{1,1}$ si incollano a un sollevamento\\  $G_{1,1}:L\cup Q_{1,1} \rightarrow E$ di $F|_{L\cup Q_{1,1}}$\\
 Il quadrato successivo è  $Q_{2,1}$, per il passo  $1$ esiste $\tilde G_{2,1}: Q_{2,1} \rightarrow E$ che solleva $F|_{G_{2,1}}$ e tale che  $\tilde G_{2,1} \left(\frac 1n, 0) = G_{1,1}(\frac 1n, 0)$\\
	 Di nuovo $\tilde G_{2,1} e G_{1,1}$ si incollano a un sollevamento \\
	  \[
		  G_{2,1}: L \cup Q_{1,1} \cup Q_{2,1} \rightarrow E
	 .\] 
	 che solleva:\\
	 $F|_{L\cup Q_{1,1}\cup Q_{2,1}}$\\
	 iterando sollevo $F|_{Q_{i,j}}$ a un'applicazione  $G_{i,j} : Q_{i,j} \rightarrow E$\\
	 che si incolla alla precedente ottenendo
	 \[
		 G_{i,j}: L\cup \left( \bigcup^{}_{(a,b) \leq (i,j)}Q_{(a,b)} \right) \rightarrow E
	 .\] 
	 Il sollevamento richiesto di $F$ è $G_{n,n} : Q \rightarrow E.$
\end{dimo}
\begin{teo}
	Sia $p : E \rightarrow X$ un rivestimento, scegliamo $a,b\in X$ e $\alpha,\beta\in \Omega(X,a,b)$ scegliamo  $e\in E$ tale che  $p(e) = a$ considero i sollevamenti  $\alpha^\uparrow_e, \beta^\uparrow_e$.\\
	Allora sono equivalenti
	 \begin{enumerate}
		 \item $\alpha \sim \beta$
		 \item  $\alpha^\uparrow_e(1) = \beta^\uparrow_e (1)$ e  $\alpha^\uparrow_e\sim\beta^\uparrow_e$
	\end{enumerate}
\end{teo}
\begin{dimo}
	$(2) \Rightarrow  (1)$ è facile\\
	se $\alpha^\uparrow_e(1) = \beta^\uparrow_e(1)$  ed esiste un omomorfismo di cammini $G$ di $\alpha^\uparrow_e$ a $\beta^\uparrow_e$\\
	allora  $p\circ G = F: [0,1]\times[0,1] \rightarrow X$\\
	è un omotopia di cammini da $\alpha$ a $\beta$\\
	(verifica per esercizio)\\
	 $(1) \Rightarrow (2)$\\
	 Sia $F$ omotopia di cammini in $X$ da $\alpha$ a $\beta$.\\
	 Per il teorema precedente posso sollevare $F$ a  $G : [0,1]\times [0,1] \rightarrow E$\\
	 tale che $G(0,0) = e.$\\
	 Dobbiamo dimostrare che $G$ è omotopia di cammini da $\alpha_e^\uparrow$ a  $\beta_e^\uparrow$, e che  $\alpha_e^\uparrow(1) = \beta^\uparrow_e(1)$. \\
	 Foto 6:40\\
	 $A)\ F(-,0) = \alpha, \ G(-,0)$ è sollevamento di  $F(-,0) = \alpha$\\
	 e parte da  $G(0,0) = e$\\
	 segue  $G(-,0) = \alpha^\uparrow_e$ per l'unicità dei sollevamenti.\\
	 B) $G(0,-)$ solleva $F(0,-)$ partendo da  $G(0,0) = e$\\
	 Ma  $F(0,-) = 1_a$ perché  $F$ è omotopia di cammino.\\
	 Quindi $G(0,-)$ solleva  $1_a$ partendo da  $e$, ma anche  $1_e$ solleva  $1_a $ partendo da  $e$ \\
	 Per l'unicità $G(0,-) = 1_e$\\
	 Analogamente $F(1,-) = 1_b$\\
	 e il cammino  $G(1,-)$ parte da  $G(1,0) = \alpha^\uparrow_e(1)$ e solleva  $1_b$ come prima  $G(1,-)$ è costante e vale $G(1,0) = 1_{\alpha^\uparrow_e(1)}$\\
	 C)  $G(-,1)$ solleva  $F(-,1) = \beta$, parte da  $G(0,1)= $ punto finale di $G(0,-)=1_e$ quindi  $G(-,1)$ pare da  $e$, quindi per l'unicità  $G(-,1) = \beta^\uparrow_e$\\
	 Segue:\\
	 $\beta_e^\uparrow (1) = G(1,1) = 1_{\alpha^\uparrow_e(1)}(1)= \alpha_e^\uparrow(1)$\\
	 inoltre  $G$ è omotopia di cammini da $\alpha^\uparrow_e$ a  $\beta^\uparrow_e$
\end{dimo}
Usiamo subito questo teorema per calcolare il primo gruppo fondamentale non banale, quello di $S^1$
 \begin{coro}
	$\pi_1(S^1) \cong \Z$.\\
	Precisamente, sia $a = (1,0)$ definiamo \\
	\begin{aligned}
		$\alpha^{(n)}:&[0,1] \rightarrow S^1$ \\
			      & $t \rightarrow (\cos(2\pi n t), \sin(2 \pi  n t))$\\
	\end{aligned}\\
		Definiamo\\
		\begin{aligned}
			$\Sigma: &Z \rightarrow \pi(S^1, a)$\\
				      & $n \rightarrow [a^{(n)}]$
		\end{aligned}\\
		Allora $\Sigma$ è isomorfismo di gruppi
\end{coro}
	\begin{dimo}
		Assumiamo $\Sigma $ omomorfismo, dimostriamo che è iniettivo, siano  $n,m\in \Z$, assumiamo  $\Sigma(n)= \Sigma(m)$\\
		cioè  $\alpha^{(n)}\sim\alpha^{(m)}.$\\
		Considero il rivestimento solito  $\rho : \R \rightarrow S^1$\\
		solleviamo $\alpha^{(n)}$ e  $\alpha^{(m)}$ partendo da $0\in\R$\\
		 \[
			 (\alpha^{(n)})^\uparrow_0(t) = nt
		.\] 
		\[
			(\alpha^{(m)})^\uparrow_0(t) = mt
		.\] 
		Per il teorema, questi hanno stesso punto finale: $n\cdot 1 = m\cdot 1$ cioè  $n = m$
	\end{dimo}

\end{document}
