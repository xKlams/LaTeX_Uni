\documentclass[12px]{article}

\title{Lezione 6 Geometria 2}
\date{2025-03-11}
\author{Federico De Sisti}

\usepackage{amsmath}
\usepackage{amsthm}
\usepackage{mdframed}
\usepackage{amssymb}
\usepackage{nicematrix}
\usepackage{amsfonts}
\usepackage{tcolorbox}
\tcbuselibrary{theorems}
\usepackage{xcolor}
\usepackage{cancel}

\newtheoremstyle{break}
  {1px}{1px}%
  {\itshape}{}%
  {\bfseries}{}%
  {\newline}{}%
\theoremstyle{break}
\newtheorem{theo}{Teorema}
\theoremstyle{break}
\newtheorem{lemma}{Lemma}
\theoremstyle{break}
\newtheorem{defin}{Definizione}
\theoremstyle{break}
\newtheorem{propo}{Proposizione}
\theoremstyle{break}
\newtheorem*{dimo}{Dimostrazione}
\theoremstyle{break}
\newtheorem*{es}{Esempio}

\newenvironment{dimo}
  {\begin{dimostrazione}}
  {\hfill\square\end{dimostrazione}}

\newenvironment{teo}
{\begin{mdframed}[linecolor=red, backgroundcolor=red!10]\begin{theo}}
  {\end{theo}\end{mdframed}}

\newenvironment{nome}
{\begin{mdframed}[linecolor=green, backgroundcolor=green!10]\begin{nomen}}
  {\end{nomen}\end{mdframed}}

\newenvironment{prop}
{\begin{mdframed}[linecolor=red, backgroundcolor=red!10]\begin{propo}}
  {\end{propo}\end{mdframed}}

\newenvironment{defi}
{\begin{mdframed}[linecolor=orange, backgroundcolor=orange!10]\begin{defin}}
  {\end{defin}\end{mdframed}}

\newenvironment{lemm}
{\begin{mdframed}[linecolor=red, backgroundcolor=red!10]\begin{lemma}}
  {\end{lemma}\end{mdframed}}

\newcommand{\icol}[1]{% inline column vector
  \left(\begin{smallmatrix}#1\end{smallmatrix}\right)%
}

\newcommand{\irow}[1]{% inline row vector
  \begin{smallmatrix}(#1)\end{smallmatrix}%
}

\newcommand{\matrice}[1]{% inline column vector
  \begin{pmatrix}#1\end{pmatrix}%
}

\newcommand{\C}{\mathbb{C}}
\newcommand{\K}{\mathbb{K}}
\newcommand{\R}{\mathbb{R}}


\begin{document}
	\maketitle
	\newpage
	\subsection{boh}
	\textbf{Osservazione:}\\
	Sia $X$ spazio topologico, $Y\subseteq X$ con topologia di sottospazio $T_Y$ .\\
	Considero l'inclusione di $Y$ in $X$ come applicazione

	\begin{center}
		\begin{aligned}
			i : &Y \rightarrow X\\
			    &y \rightarrow y
		\end{aligned}
	\end{center}
	$i$ è costruita (mettendo su $Y$ la topologia $T_Y$). Verifica: sia  $B\subseteq X$ aperto\\
	la controimmagine è $i^{-1}(B)$ Questo è aperto in topologia di sottospazio.\\
	Sia  $T$ una topologia su $Y$ ( non necessariamente = $T_Y$ ), suppongo che $i: Y \rightarrow X $ sia continua anche usando $T$ come topologia su $Y$ \\
	Allora $\forall B\subseteq X$ aperto, $i^{-1}(B)$ è aperto in $Y$ cioè $i^{-1}(B)\in T$. \\
	Al variare di $B$ aperto in $X$, gli insiemi $i^{-1}(B)$ formano  $T_Y$, quindi $T_y \subseteq T$.\\
	Possiamo considerare la famiglia di tutte le topologie su  $Y$ per cui l'inclusione è continua. L'intersezione di esse è contenuta in $T_Y$  perché $ T_Y$ è una di esse, e contiene $T_Y$ perché ogni $T$ siffatta contiene $T_Y$. \\
	Quindi $T_Y$ è la topologia meno fine fra quelle per cui $i$ è continua. 
	\begin{prop}
		Sia $f: X \rightarrow Z$ applicazione continua fra spazi topologici, sia $Y\subseteq X$ con topologia di sottospazio, allora $f|_Y:Y \rightarrow Z$ è continua
	\end{prop}
	\begin{dimo}
		Usiamo l'inclusione $i:X \rightarrow Y$ e osserviamo $f|_Y:Y \rightarrow Z$ concateno con\\
		 \[
			 f\circ i: Y \xrightarrow{i} X \xrightarrow{f} Z
		.\] 
		$f$ e $i$ sono continue, lo è anche $f\circ i$
	\end{dimo}
	\begin{prop}
		Siano $X$ spazio topologico, $Y\subseteq X$ con topologia di sottospazio, $Z$ spazio topologico e $f : Z \rightarrow Y$.\\
		Consideriamo l'estensione del codominio di $f$ da $Y$ a $X$ che è l'applicazione  $i\circ f :Z\xrightarrow{f} Y \xrightarrow{i} X$\\
		Allora $f$ è continua se e solo se $i\circ f$ è continua.
	\end{prop}
	\begin{dimo}
		%TODO disegno 11 marzo 5:44
		$ ( \Rightarrow )$ ovvio poiché $i\circ f$ è composizione di applicazioni continue\\
		$ ( \Leftarrow)$ Sia $A\subseteq Y$ aperto, scegliamo $B\subseteq X$ aperto tale che $B\cap Y = A$.\\
		Allora  $f^{-1}(A) = (i\circ f)^{-1}(B)$ \\
		poiché chiedere che $z\in Z$ vada in $A$ tramite $f$ è equivalente a richiedere che vada in $B$.\\
		Allora $f^{-1}(A)$ è aperto per continuità di $i\circ f$
	\end{dimo}
	\textbf{Osservazione}\\
	Data in generale $f: Z \rightarrow X$ spesso la si restringe all'immagine
	\begin{center}
		\begin{aligned}
			\tilde f: &Z \rightarrow Im(f)\\
				  & z - f(z)
		\end{aligned}
	\end{center}
	vale $f$ continua  $ \Leftrightarrow \tilde f$ continua, perché posso considerare l'inclusione 
	\[
	 i: Im(f) \rightarrow X
	.\]  
	e allora $f = i \circ \tilde f$\\
	 \textbf{Esempio:}\\
	 $X = \R$ con topologia euclidea.\\
	 $Y = [0,1[$ con topologia di sottospazio
	 \[
	  Z = ]0,1[ \ \ (\subseteq Y)
	 .\] 
	 Sia verifica facilmente (esercizio) che la chiusura di $Z$ in $Y$ è $[0,1[$e la chiusura di $Z$ in $X$ è $[0,1]$ \\
	 Le chiusure sono diverse, ma
	 \[
	  [0,1[ = [0,1]\cap Y
	 .\] 
	 dove il primo intervallo è in $Y$ e il secondo intervallo in $X$\\
	 Questo si generalizza.
	 \begin{lemm}
	 	Sia $X$ spazio topologico, $Y\subseteq X$ con topologia di sottospazio, $Z\subseteq Y$ la chiusura di $Z$ in $Y$ è uguale a $Y$ intersecato la chiusura di $Z$ in $X$
	 \end{lemm}
	 \begin{dimo}
		 Chiusura di $Z $ in $Y = \displaystyle \bigcap_{\substack{C\subseteq Y,\\ C\text{ chiuso in } Y,\\ C\supseteq Z}}C = \ldots $ \\
		 Per ogni tale $C$ scelgo un $D\subseteq X$ chiuso in $X$ tale che $C = D\cap Y$\\
		  \[
			  \ldots = \bigcap_{\substack{C\subset U,\\ C\text{ chiuso in } Y,\\ C\supseteq Z,\\ D\subseteq X,\\ D \text{ chiuso in } X \\ t.c. \ D\cap Y = C}}D\cap Y 
		 .\] 
		 \[
			 = \bigcap_{\substack{D'\subseteq X,\\ D'\text{ chiuso in } X,\\ D'\supset Z}}D'\cap Y
		 .\] 
		 L'ultima uguaglianza vale perché ogni $D$ della prima intersezione compare fra i $D$ della seconda intersezione, Per ogni $D'$ della seconda seconda intersezione considero $C = D'\cap Y$ che è in  $Y$, chiuso in $Y$, contenente $Z$, quindi compare fra i  $C$ della prima intersezione; ad esso corrisponde un $D$ della prima intersezione, che soddisfa $D\cap Y = C =D'\cap Y.$\\
		 Quindi per ogni  $D'$ della seconda intersezione esiste un $D$ della prima con la stessa intersezione con $Y$, ovvero  $D\cap Y = D'\cap Y$, Quindi vale l'uguaglianza.\\
		 L'uguaglianza prosegue:\\
		  \[
			  = \left( \bigcap_{\substack{D'\subsetteq Z,\\ D'\text{ chiuso in } X,\\ D'\supseteq Z}} D' \right)\cap Y
		 .\] 
		 dove la parentesi è la chiusura di $Z$ in $X$
	 \end{dimo}
	 \textbf{Osservazione}\\
	 Attenzione: non vale un enunciato analogo con la parte interna.\\
	 Ad esempio $X = \R$ cn topologia euclidea $Y = \Z$  $Z= \{0\}$\\
	 La parte interna di  $Z$ in $X$ è vuota, perché $Z$ non contiene alcun aperto di $\R$\\
	 Invece la topologia di sottospazio su  $Y$ è la topologia discreta e $Z$ è aperto in $Y$.\\
	 Quindi  $Z$ è la propria parte interna come sottoinsieme di $Y$.
	 \begin{defi}
	 	Sia $f: X \Rightarrow Y $ un'applicazione continua fra spazi topologici, $f$ è un'inversione topologica se la restrizione
		\begin{center}
			\begin{aligned}
				\tilde f: &X \rightarrow f(X)\\
					  &x \rightarrow f(x)
			\end{aligned}
		\end{center}
		è un omeomorfismo, dove su $f(X)\subseteq Y$ metto la topologia di sottospazio.
	 \end{defi}
	 \textbf{Esempio}\\
	 1) Considero 
	 \begin{center}
	 	\begin{aligned}
			&\R \rightarrow \R \\
			&x \rightarrow (x,0)
	 	\end{aligned}
	 \end{center}
	 (qui $\R, \R^2$con topologia euclidea) è un immersione, la verifica è per esercizio.\\
	 2) 
	 \begin{center}
	 	\begin{aligned}
			$f : $&$[0,2\pi[ \rightarrow \C$\\
			   & $t \rightarrow e^{it}$
	 	\end{aligned}
	 \end{center}
	 Su $[0,2\pi[\subseteq \R$\\
	 metto la topologia di sottospazio indotta dalla topologia euclidea su  $\R,$ su $\C = \R^2$ metto la topologia euclidea\\
	 È continua, iniettiva e $f([0,2\pi[) = S^1 = \{z\in\C \ | \ |z|=1\}$\\
	 Questa  $f$ non è un'immersione, infatti $[0,\pi[$ è aperto nel dominio, ma $f([0,2[)$ non è aperto in $S_1$ con topologia di sottospazio.
	 %TODO aggiungi figura 6:40
	 quel chiuso dovrebbe essere interesezione tra la circonferenza e un aperto di $\R^2$, Ciò non è possibile perchè ci sarebbe un intorno su un estremo della circonferenza.\\
	 \subsection{Prodotti topologici}
	 Siano $P,Q$ spazi topologici.\\
	 vogliamo definire una topologia "naturale" su $P\times Q$.\\
	  \textbf{Esempio:}\\
Considero $P = Q = \R$ con topologia euclidea
 \[
 P\times Q = \R\times \R = \R^2
.\] 
La topologia su $\R^2$ sarà quella euclidea. Considero ad esempio
\[
	U \subseteq \R\text{ aperto}, \ \ V\subseteq\R\text{ aperto}
.\] 
il prodotto $U\times V$ sarà aperto in  $\R^2$, posso pensare che questa sia quindi la mia topologia, ma vediamo qualche esempio con la topologia euclidea.\\
Ad esempio  $U = ]a,b[ \ \ \ V = ]c,d[$, allora  $U\itimes V = ]a,b[\times]c,d[$ è un rettangolo aperto\\
Anche un disco aperto in  $\R^2$ è aperto in topologia euclidea, ma non riesco a scriverlo con questo prodotto  $U\times V$ con $U\subseteq \R, \ V\subseteq \R$\\
Potrei prendere
 \[
	 B = \{U\times V \ | \ \ \substack{U\subseteq \R \text{ aperto },\\ \hspace{-3px}V\subseteq\R\text{ aperto }}\}
.\] 
come base per la topologia su $\R\times\R$
 \begin{defi}
	Siano $P,Q$ spazi topologici, la topologia prodotto su $P\times Q$ è la meno fine fra quelle per cui le proiezioni:
	\begin{center}
		\begin{aligned}
			p:& P\times Q \rightarrow P\\
			  &(a,b) \rightarrow a\\
			q:&P\times Q \rightarrow Q\\
			  &(a,b) \rightarrow b
		\end{aligned}
	\end{center}
	Sono continue.
\end{defi}
\textbf{Osservazione}\\
Esistono topologie su $P\times Q$ tali che  $p$ e $q$ sono continue, per esempio la topologia discreta su $P\times Q$\\
La topologia prodotto è l'intersezione di tutte le topologia per cui  $p$ e $q$ sono continue.
\begin{teo}
	
\end{teo}

\end{document}
