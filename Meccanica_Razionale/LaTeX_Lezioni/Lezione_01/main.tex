\documentclass[12px]{article}

\title{Lezione 1 Meccanica Razionale}
\date{2025-02-28}
\author{Federico De Sisti}

\input{../../../setup.tex}

\begin{document}
	\maketitle
	\newpage
	\section{Introduzione al corso}
	\subsection{Contatti}
	sergio.simonella@uniroma1.it, stanza 17, mercoledì alle ore 11:00\\ 
	\textbf{Testi consigliati}\\
	Truesdell 1974 Essays of the history of mechanics (punto di vista critico sull'inizio della meccanica analitica)\\
	Buttà-Negrini "Note del corso di meccanica razionale"\\
	Arnold metodi matematici della meccanica classica
	\subsection{Cosa è la meccanica razionale}
	È la teoria del moto (meccanica), cerca di collegare risultati sperimentali con prove matematiche (raizonale).\\
	\subsection{Assomi (leggi del moto)}
	\begin{ass}
Every body continues in its state of rest, or of uniform motion straight ahead, unless it is compelled to change that state by forces impressed upon it. 
	\end{ass}
	Questa è la prima legge, essenzialmente risale a Galileo 
	\begin{ass}
		The change of motion is proportional to the motive force impressed, and it takes place along the right line in which that force is impressed
	\end{ass}
	\begin{ass}
		To each action there is always a contrary and equal reaction, or the mutual action of two bodies are always equal and directed to contrary parts
	\end{ass}
	1 + 2 + 3 determinano la legge del moto (intesa come evoluzione temporale del corpro)\\
	\[
	 t \rightarrow x(t)
	.\] 
	dove t è il tempo (assoluto)\\
	$x(t)=$ configurazione spaziale (uno o più punti nello spazio fisico)\\
	\textbf{Dato di fatto}\\
	$ x \rightarrow x(t)$ non è esplicita (non abbiamo $x(t)$ polinomio o integrale, a meno di casi abbastanza semplici)\\
	Sistema meccanico $ = \begin{cases}
		\text{non integrabile (solito)}\\
		\text{integrabile (eccezioni)}
	\end{cases}$ \\
	qualitativamente:\\
	moto "caotico"\\
	moto "regolare"\\
	Poincaré diceva che essenzialmente ci interessava capire le loro proprietà geometriche\\
	i moti regolari sono 16 e sono fondamentali\\
	\begin{defi}[Sistema meccanico]
		Sistema di $N$ "particelle" ($N\in\N$) o "punti materiali" $p_0,p_1,\ldots,p_N$ in $\R^3$ di "masse inerziali"  $m_1,m_2,\ldots, m_N \ \ m_i>0 \ i=1,\ldots,N$
	\end{defi}
	\begin{nota}
		Moto $t \rightarrow P(t)$\\
		$P(t) = (P_1(t), P_2(t), \ldots,P_N(t))$\\
		configurazione (in coordinate cartesiane)\\
		$x(t) = (x^{(1)}(t),\ldots,x^{(n)}(t))$\\
		$x^{(k)}(t)\in \R^3\ \ x^{(k)}(t) = (x_1^{(k)}(t), x_2^{(k)},x_3^{(k)})\ \ k = 1,\ldots, N$\\
		velocità \ \ $v(t) = (v^{(1)}(t), \ldots, v^{(n)}(t))$ \ \  $v^{(k)}(t) = (v_1^{(k)}(t), v_2^{(k)}(t), v_3^{(k)}(t))$\\
		 \[
			 v(t) = \dot x(t)  = \lim_{\varepsilon \rightarrow 0}\frac{x(t_0 + \varepsilon) - x(t)}{\varepsilon}
		.\] 
		accellerazione a(t) = \ldots\\
		a(t) = \dot v(t)  = \ddot x(t) = \frac{d}{dt}v(t) = \frac{d^2}{dt^2}x(t)
	\end{nota}
	\textbf{Esempi:}\\
	1)Caduta libera:\\
	$N = 1\ \ m =1 \ \ g > 0$\\
	 $x\in \R^3 \ \ \ \ddot x(t) = -g(0,0,1)$\\
	 Condizioni iniziali $(x(0),\cdot x(0)) = (x_0,\cdot x_0)\ \ \ \ x_0,\cdot x_0 \in \R$\\
	 \[
	 \begin{cases}
	 	\ddot x_1(t) = 0\\
	 	\ddot x_2(t) = 0\\
	 	\ddot x_3(t) = 0\\
	 \end{cases} \ \ 
	 \begin{cases}
	 	\dot x_1(t) = \dot x_1(0)\\
	 	\dot x_2(t) = \dot x_2(0)\\
	 	\dot x_3(t) =\dot x_3(0) - gt\\
	 \end{cases}
	 \begin{cases}
	 	x_1(t) = x_1(0) +\dot x_1(0)t\\
	 	x_2(t) = x_2(0) +\dot x_2(0)t\\
	 	x_3(t) =x_3(0) +\dot x_3(0)t - \frac 12gt^2\\
	 \end{cases}
	 .\] 
	 2) Molla attaccata alla parete\\
	 % TODO inserisci disegno molla attaccata al muro
	 $\ddot x(t) = -\alpha x (t) \ \ \alpha > 0$\\
	 $\frac{m}{m'} = \frac{\alpha'}{\alpha}$\\
		  $m\ddot x(t) = -m\alpha x(t) = -kx(t)$ $ k > 0 $ costante\\
		  da qui possiamo ricavare l'equazione del moto (oscillatore armonico)\\
		  $x(t) = x(0)\cos(\sqrt\alpha t) + \frac{x(0)}{\sqrt \alpha}\sin(\sqrt\alpha t)$\\
	3) Pianeti $N= 2$\\
	%aggiungi immagine pianeti
	configurazione  $x^{(1)}(t), x^{(2)}(t)$\\
	\begin{cases}
		
	$m_1\ddotx^{(1)}(t) = F_{12}$\\
	$m_2\ddotx^{(2)}(t) = F_{21}$
	\end{cases}
	\ \ \ 	$\displaystyle F_{12} = -F_{21} = -Gm_1m_2\frac{x^{(2)} - x^{(1)}}{|x^{(2)}-x^{(1)}|^3}$\\

	\textbf{Osservazione}\\
	I tre esempi hanno struttura "$F = ma$" con $F = -\triangledown U$\\
	Caduta libera:\\
	$U_g = mg x_3$\\
	Molla: \\
	$U_k (x) = \frac 12 k x^2$ \\
	Pianeti\\
	$U(x) = U(x^{(1)}, x^{(2)}) = -G\frac{m_1m_2}{|x^{(2)}-x^{(1)}}$\\
	$F_{12} = -\triangledown_{x^{(1)}}U_G = Gm_1m_2 - \frac{1}{x^{(2)} - x^{(1)}}\frac{x^{(2)} - x^{(1)}}{|x^{(2)} - x^{(1)}|}$ \\
	$F_{21} = -\triangledown_{x^{(2)}}U_G(x^{(1)},x^{(2)})$ \\
	\textbf{Nota:}
	Indice in alto indica il punto materiale, in basso indica la coordinata\\



\end{document}
