\documentclass[12px]{article}

\title{Fisica 2 Lezione 1}
\date{2025-09-30}
\author{Federico De Sisti}

\input{../../setup.tex}

\begin{document}
	\maketitle
	\newpage
	\subsection{Introduzione}
	Il martedì si fanno gli esercizi, giovedì ci da il testo.
	\subsection{Qualche formula}
	\[
		F_{e_{21}} = k_0\frac{q_1q_2}{r^2_12}\hat r_{12}
	.\] 
	È la forza elettrica tra 2 particelle di carica $q_1, q_2$\\
	$k_0 = 9\cdot 10^9 \frac {Nm^2}{C^2}, \ \frac{1}{4\pi \e_0}$ dove $\e_0$ è la costante dielettrica nel vuoto e rappresenta la capacità di accogliere i campi magnetici nel vuoto (supercazzola ?)\\
	$\e_0 = 8,854\cdot 10^{-12}\frac{C^2}{m^2N}$\\
	\[
		\frac{| \overrightarrow{F_e}|}{F_g} = \frac {k_0e^2}{Gm_pm_e} = \frac{9\cdot 10^9 \cdot(1,6\cdot 10^{-18})^2}{6,67\cdot 10^{-11}\cdot 1840(9,1\cdot 10^{-31})^2} = 2,3\cdot 10^{39}
	.\] 
	Caso di un atomo di idrogeno, le due masse sono un protone e un elettrone.\\
	\[
	\overrightarrow{a}\times \overrightarrow{b} = - \overrightarrow{b}\times \overrightarrow{a}
	.\] 
	\[
		\det\matrice{i & j & k\\ a_x & a_y & a_z\\ b_x & b_y & b_z}
	.\] 
	\[
		( \overrightarrow{a}\times \overrightarrow{b})\cdot \overrightarrow{c} = ( \overrightarrow{c}\times \overrightarrow{a} ) \cdot\overrightarrow{b} = ( \overrightarrow{b}\times \overrightarrow{c} ) \overrightarrow{a}
	.\] 
	\[
		( \overrightarrow{a}\times \overrightarrow{b})\times \overrightarrow{c} = ( \overrightarrow{a} \overrightarrow{c}) \overrightarrow{b} - ( \overrightarrow{b} \overrightarrow{c}) \overrightarrow{a}
	.\] 
	\[
		( \overrightarrow{a}\times \overrightarrow{b})\cdot ( \overrightarrow{c}\times \overrightarrow{d}) = ( \overrightarrow{a} \overrightarrow{c})( \overrightarrow{b} \overrightarrow{d}) - ( \overrightarrow{a} \overrightarrow{d})( \overrightarrow{b} \overrightarrow{c})
	.\] 


	
\end{document}
