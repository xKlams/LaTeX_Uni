\documentclass[12px]{article}

\title{Lezione 07 Analisi Reale}
\date{2025-03-18}
\author{Federico De Sisti}

\usepackage{amsmath}
\usepackage{amsthm}
\usepackage{mdframed}
\usepackage{amssymb}
\usepackage{nicematrix}
\usepackage{amsfonts}
\usepackage{tcolorbox}
\tcbuselibrary{theorems}
\usepackage{xcolor}
\usepackage{cancel}

\newtheoremstyle{break}
  {1px}{1px}%
  {\itshape}{}%
  {\bfseries}{}%
  {\newline}{}%
\theoremstyle{break}
\newtheorem{theo}{Teorema}
\theoremstyle{break}
\newtheorem{lemma}{Lemma}
\theoremstyle{break}
\newtheorem{defin}{Definizione}
\theoremstyle{break}
\newtheorem{propo}{Proposizione}
\theoremstyle{break}
\newtheorem*{dimo}{Dimostrazione}
\theoremstyle{break}
\newtheorem*{es}{Esempio}

\newenvironment{dimo}
  {\begin{dimostrazione}}
  {\hfill\square\end{dimostrazione}}

\newenvironment{teo}
{\begin{mdframed}[linecolor=red, backgroundcolor=red!10]\begin{theo}}
  {\end{theo}\end{mdframed}}

\newenvironment{nome}
{\begin{mdframed}[linecolor=green, backgroundcolor=green!10]\begin{nomen}}
  {\end{nomen}\end{mdframed}}

\newenvironment{prop}
{\begin{mdframed}[linecolor=red, backgroundcolor=red!10]\begin{propo}}
  {\end{propo}\end{mdframed}}

\newenvironment{defi}
{\begin{mdframed}[linecolor=orange, backgroundcolor=orange!10]\begin{defin}}
  {\end{defin}\end{mdframed}}

\newenvironment{lemm}
{\begin{mdframed}[linecolor=red, backgroundcolor=red!10]\begin{lemma}}
  {\end{lemma}\end{mdframed}}

\newcommand{\icol}[1]{% inline column vector
  \left(\begin{smallmatrix}#1\end{smallmatrix}\right)%
}

\newcommand{\irow}[1]{% inline row vector
  \begin{smallmatrix}(#1)\end{smallmatrix}%
}

\newcommand{\matrice}[1]{% inline column vector
  \begin{pmatrix}#1\end{pmatrix}%
}

\newcommand{\C}{\mathbb{C}}
\newcommand{\K}{\mathbb{K}}
\newcommand{\R}{\mathbb{R}}


\begin{document}
	\maketitle
	\newpage
	\subsection{$\sigma$-algebra}
	\begin{defi}
		$X$ insieme non vuoto, Una famiglia $\eta\subseteq P(X)$ si dice  $\sigma$-algebra su $X$ se 
		 \begin{enumerate}
			 \item $\emptyset, X\in \eta$
			 \item  $E\in \eta \Rightarrow E^c\in\eta$ 
			 \item $\{E_i\}\subseteq \eta \Rightarrow \bigcup^{+\infty}_{i = 1}E_i\in\eta$
		 \end{enumerate}
	\end{defi}
	\textbf{Osservazione}\\
	Se $\eta$ è $\sigma$-algebra e $\{E_i\}\subseteq \eta \Rightarrow \bigcap_{i=1}^{+\infty}E_i\in\eta$ \\
	$E_i\in\eta \rightarrow E_i^c\in \eta \ \ \forall i$\\
$ \Rightarrow  \bigcup^{+\infty}_{i = 1}E_i^c\in \eta \Rightarrow ( \bigcup^{+\infty}_{i=1}E_i^c)^c\in \eta\\$

Una misura individua una $\sigma$-algebra\\
\textbf{In generale}\\
se $\mu$ è una misura su  $X$\\
 \[
	 \eta_\mu = \{R\subseteq X \ : \ E \text{ e } \mu-\text{misurabile}\}
.\] 
è una $\sigma$-algebra\\
In particolare in $\R$ c'è la $\sigma$-algebra di Lesbegue $=$ famiglia degli insiemi misurabili secondo Lesbegue\\
\begin{defi}
	Sia $X$ un insieme non vuoto e sia $F\subset P(X)$ si chiama  $\sigma$-algebra generata da $F$ la $\sigma$-algebra data da 
	\[
		\Sigma_F = \bigcap_{\substack{\eta\text{ è algebra}\\ F\subseteq \eta}}\eta
	.\] 
	la più piccola $\sigma$-algebra che contiene $F$
\end{defi}
\begin{defi}
	Se $(X,\iota)$ è uno spazio topologico la  $\sigma$-algebra generata da $\iota$ si dice $\sigma$-algebra di Borel
\end{defi}
Vogliamo ora indagare sulla $\sigma$-algebra di Lesbegue in $\R$ $\eta_m = \eta$
 \begin{prop}
	Se  $I\subseteq \R$ è un intervallo\\
	 $ \Rightarrow  I \in\eta $ (è misurabile secondo Lesbegue)
\end{prop}
\begin{dimo}
	$I\subseteq \R$ intervallo  $ \Rightarrow \ \ \forall F\subseteq \R \ \ m(F)\geq m(F\cap I) + m(F\setminus I)$ 
\textbf{Primo caso}\\
Supponiamo $I = (a,+\infty) \ \ a\in\R$\\
Sia  $F\subseteq \R, $ con $m(F) < +\infty$ \\
$\forall \varepsilon > 0 \ \ \exists \{I_i\}$ successione di intervalli tale che  $F\subseteq \bigcup^{+\infty}_{i=1}I_i$ e $m(F)\leq \sum^{+\infty}_{i=1}|I_i| < m(F) + \varepsilon$
\[
m(F) + \varepsilon > \sum^{+\infty}_{i =1}|I_i| = \sum^{+\infty}_{i=1}(|I_i\cap I| + |I\setminus I|) 
\] 
\[
= \sum^{+\infty}_{i=1}|I_i\cap I| + \sum^{+\infty}_{i=1}|I_i\setminus I| \geq m(F\cap I) + m(F\setminus I)
.\] 
e per $\varepsilon \rightarrow 0$ si ha $m(F) \geq m(F\cap I) + m(F\setminus I)$
$I_i  = (\alpha_i, \beta_i)$ \ \ $|I_i| = \beta_i -\alpha_i = \beta_i - a + a - \alpha_i = |I_i\cap I| + I_i\cap I|$\\
$F\subseteq \bigcup^{+\infty}_{i = 1}I_i$\\
$F\cap I\subseteq  \bigcup^{+\infty}_{i =1}(I_i\cap I)$\\
$F\setminus I \subseteq \bigcup^{+\infty}_{i =1}(I_i\setminus I)$ \\
\textbf{Quindi:}\\
Intervalli del tipo $I= (a,+\infty)\in \eta \rightarrow I = (-\infty, a]\in\eta$\\
$ \rightarrow (a,b]\in\eta$\\
$[a,+\infty) = \bigcap_{n=1}^{+\infty}(a - \frac 1n, +\infty)\in \eta$\\
$ \Rightarrow (-\infty , a)\in \eta$ \\
$ \Rightarrow (a,b)\in \eta$ \\
$ \Rightarrow $ vale anche per gli intervalli chiusi\\
tutti gli intervalli sono misurabili, quindi anche tutti gli aperti.
\end{dimo}
\begin{teo}
	Ogni aperto $a\subseteq\R$ è unione al più numerabile di intervalli aperti disgiunti
\end{teo}
\begin{coro}
	$\sigma$-algebra di Borel in $\R$ = $\beta\subseteq\eta = $  $\sigma$-algebra di Lesbegue
\end{coro}
L'inclusione puo essere stretta perché  $F$ insieme misurabili con Lesbegue e non con Borel\\
Quindi in $\R$ si ha:\\
$B\subseteq \eta\subsetneq P(R)$, che è stretta perché e come controesempio abbiamo l'insieme di Vitali perchè non vale l'additività  $V\in P(\R)\setminus\eta$\\
\begin{teo}[Caratterizzazione sui misurabili secondo Lesbegue]
	Sia $E\subseteq \R$ sono equivalenti
	 \begin{enumerate}
		 \item $E\in\eta$
		 \item  $\forall \varepsilon > 0 \ \ \exists A_\varepsilon\subseteq\R$ aperto $t.c.$ $E\subseteq A_\varepsilon$ e $m(A_\varepsilon\setminus E)< \varepsilon$
		 \item  $\exists \ F\in B$ ($F$ è intersezione numerabile di aperti) tali che $E\subseteq F$ e $m(F\setminus E) = 0$
		 \item  $\forall \varepsilon > 0 \ \exists \ C_\varepsilon$ chiuso tale che $C_\varepsilon\subseteq E$ e $m(E\setminus C_\varepsilon) < \varepsilon$
		 \item  $\exists G\in B$ ( $G$ è unione numerabile di chiusi) tale che $G\subseteq E$ e $m(E\setminus G) = 0$
	\end{enumerate}
\end{teo}
\begin{dimo}
	$1) \Rightarrow 2)$ \\
	$Hp \ E\in \eta$\\
	Primo caso:  $m(E) < +\infty$\\
	$\forall\varepsilon > 0 \exists\{I_i\}$ successione di intervalli aperti tali che $E\subseteq \bigcup^{+\infty}_{i = 1}I_i^\varepsilon$ (l'insieme $A_\varepsilon$ aperto) e $ \sum^{+\infty}_{i = 1}|I_i^\varepsilon| < m(E) + \varepsilon$\\
	$E\in \eta \Rightarrow  m(A_\varepsilon) = m(A_\varepsilon\cap E) + m(A_\varepsilon\setminus E) = \\
	=m(E) + m(A_\varepsilon\setminus E)$ \\
	$ \Rightarrow m(A_\varepsilon) - m(E) = m(A_\varepsilon\setminus E)$ \\
	quindi\\
	$(A_\varepsilon\setminus E) = m(A_\varepsilon) - m(E)\leq \sum^{+\infty}_{i = 1} m(I_i^\varepsilon) - m(E) = \sum^{+\infty}_{i = 1}|I_i^\varepsilon | -m(E) < \varepsilon $|\
	Secondo caso: $m(E) = +\infty$ \\
	$E = \bigcup^{+\infty}_{n= 1}E\cap (-n,n)$\\
	$E\in\eta \Rightarrow \ \forall n E\cap (-n,n) = E_n \in \eta, \eta(E_n) < +\infty$ \\
	applicando il primo caso \\
	$\forall n, \ \forall\varepsilon$\\
	 $\exists A_n^\varepsilon$ aperto tale che $A_n^\varepsilon \geq E_n$ e  $m(A_n^\varepsilon \setmins E_n) < \varepsilon$\\
	  $A_\varepsilon = \bigcup^{+\infty}_{n=1}A_n^\varepsilon\supseteq \bigcup^{+\infty}_{n = 1}E_n = E$\\
	  $m(A_\varepsilon\setminus E) = m(\displaystyle \bigcup^{+\infty}_{n = 1}(A_n^\varepsilon \setminus E)) \leq m( \bigcup^{+\infty}_{n = 1}(A_n^\varepsilon\setminus E_n))\leq \sum^{+\infty}_{n = 1}m(A_n^\varepsilon\setminus E_n) \leq \sum^{+\infty}_{n=1}\frac{\varepsilon}{2^n} = \varepsilone$ \\
	  Se è misurabile c'è un aperto che lo contiene e la misura della differenza è finita\\[10px]
	  $2) \Rightarrow  3)$ \\
	  $Hp \ \forall \vaerpsilon > 0, \exists A_\varepsilon $ aperto, $A_\e\supseteq E$  e $m(A_\e\setminus E)< \e$\\
	  Th  $\exists F\in B$ tale che $F \supseteq E$ e  $m(F\setminus E) = 0$\\
	  Per  $\e = \frac 1n, \ \forall n\geq 1 \ \ \exists A_n$ aperto $t.c. \ A_n\supseteq E$ e $m(A_n\setminue E)< \frac 1n$\\
	  $F = \bigcap^{+\infty}_{n=1}A_n\in B,\ \ F\supseteq E$ e $m(F\setminus E)\leq m(A_n\setminus E) \leq \frac 1n$\\
	  $ \Rightarrow  n \rightarrow +\infty \ \ m(F\setminus E) = 0$\\[10px]
	   $3) \Rightarrow 1)$ \\
	   Hp $\exists F\in B: \ F\supseteq E $ e $m(F\semtinus E) = 0$\\
	    $ E = F\semtinus(F\setminus E) = F\cap (F\setminus E)^c\in \eta$\\
	     $1) \Rightarrow 4)$ \\
	     $E\in \eta \Rightarrow E^c \in \eta \Rightarrow \forall \e > 0 \exists A_\e$ aperto\\
	     tale che $A_\e\supseteq E^c$ e $m(A_\e\setminus E^c)<\e$\\
	      $C_\e = A_\e^c$ è chiuso\\
	      $E^c\subseteq A_\e \ \Rightarrow \ E\supseteq A_\e^c = C_\e$ \\
	      $m(E\setminus C_\e) = m(E\cap C^c) = m(E\cap A_\e) = m(A_\e\setminus E^c)<\e$\\[10px]
	      $4) \Rightarrow  5)$ \\
	      Per $\e = \frac 1n \ \ \forall\n\in \N$\\
	       $\exists C_n$ chiuso, $C_n \subseteq E$ tale che  $m(E\setminus C_n)< \frac 1n$\\
	       \[
	         G = \bigcup^{+\infty}_{n = 1}C_n\in B,  \ \ G\subseteq E
	       .\]
	       $m(E\setminus C) \leq m(E\setminus C_n)\leq \frac 1n \ \ \ \forall n$\\
	       per $n \rightarrow +\infty m(E\setminus G) = 0$ \\[10px]
	       $5) \Rightarrow  1)$ \\
	       Hp: $\exists G \in B$ tale che $G\subseteq E$ e $m(E\setminus G) = 0$\\
	        $ \Rightarrow E = G\cup (E\setminus G)\in \eta$ perché unione di misurabili



\end{dimo}
	
\end{document}
