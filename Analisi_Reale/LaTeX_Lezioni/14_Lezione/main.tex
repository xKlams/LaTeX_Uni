\documentclass[12px]{article}

\title{Lezione 14 Analisi Reale}
\date{2025-04-09}
\author{Federico De Sisti}

\usepackage{amsmath}
\usepackage{amsthm}
\usepackage{mdframed}
\usepackage{amssymb}
\usepackage{nicematrix}
\usepackage{amsfonts}
\usepackage{tcolorbox}
\tcbuselibrary{theorems}
\usepackage{xcolor}
\usepackage{cancel}

\newtheoremstyle{break}
  {1px}{1px}%
  {\itshape}{}%
  {\bfseries}{}%
  {\newline}{}%
\theoremstyle{break}
\newtheorem{theo}{Teorema}
\theoremstyle{break}
\newtheorem{lemma}{Lemma}
\theoremstyle{break}
\newtheorem{defin}{Definizione}
\theoremstyle{break}
\newtheorem{propo}{Proposizione}
\theoremstyle{break}
\newtheorem*{dimo}{Dimostrazione}
\theoremstyle{break}
\newtheorem*{es}{Esempio}

\newenvironment{dimo}
  {\begin{dimostrazione}}
  {\hfill\square\end{dimostrazione}}

\newenvironment{teo}
{\begin{mdframed}[linecolor=red, backgroundcolor=red!10]\begin{theo}}
  {\end{theo}\end{mdframed}}

\newenvironment{nome}
{\begin{mdframed}[linecolor=green, backgroundcolor=green!10]\begin{nomen}}
  {\end{nomen}\end{mdframed}}

\newenvironment{prop}
{\begin{mdframed}[linecolor=red, backgroundcolor=red!10]\begin{propo}}
  {\end{propo}\end{mdframed}}

\newenvironment{defi}
{\begin{mdframed}[linecolor=orange, backgroundcolor=orange!10]\begin{defin}}
  {\end{defin}\end{mdframed}}

\newenvironment{lemm}
{\begin{mdframed}[linecolor=red, backgroundcolor=red!10]\begin{lemma}}
  {\end{lemma}\end{mdframed}}

\newcommand{\icol}[1]{% inline column vector
  \left(\begin{smallmatrix}#1\end{smallmatrix}\right)%
}

\newcommand{\irow}[1]{% inline row vector
  \begin{smallmatrix}(#1)\end{smallmatrix}%
}

\newcommand{\matrice}[1]{% inline column vector
  \begin{pmatrix}#1\end{pmatrix}%
}

\newcommand{\C}{\mathbb{C}}
\newcommand{\K}{\mathbb{K}}
\newcommand{\R}{\mathbb{R}}


\begin{document}
	\maketitle
	\newpage
	\subsection{Integrali dipendenti da un parametro}
		$(X,\mu)$ spazio di misura
		 \[
			 f : I\times X \rightarrow [-\infty,+\infty]
		.\] 
		$I$ intervallo di $\R$, tale che 
		 \begin{itemize}
			 \item per quasi ogni $x\in X$\\
				  $t\in I \rightarrow f(t,x), f(\cdot, x)$ continua
			  \item $\forall t\in I\ \ f(t,\cdot)\in L^1(X,\mu)$
				   \[
				  h(t) = \int_Xf(t,x)d\mu(x)
				  .\] 
		\end{itemize}
		\begin{teo}
			Sia $(X,\mu)$ spazio di misura, $I\subseteq \R$ intervallo e $f: I\times X \rightarrow [-\infty, +\infty]$
			\begin{enumerate}
				\item se  $f(\cdot,x)$ è continua su  $I$ per quasi ogni $x\in X\ \ f(t,\cdot)\in L^1(X)\ \ \forall t\in I$  e  $\exists g\in L^1(X)$ t.c.  $|f(t,x)|\leq g(x) \ \ \ \forall t\in I$ per quasi ogni  $x\in X$\\
			 $ \Rightarrow  $ la funzione $h(t) = \int_Xf(t,x)d\mu$ è continua su $I$
		 \item Se per quasi ogni $x\in X, \ \ f(\cdot, x)$ è derivabile su $I$ e se $\exists g_1\in L^1(X)$ tale che 
			 \[
				 |\frac{\partial f}{\partial t}(t,x)|\leq g_1(x)\ \ \ \text{ per q.o. }x\in X\ \ \forall t\in I
			 .\] 
			 $ \Rightarrow h$ è derivabile e 
			 \[
				 h'(t) = \int_X\frac{\partial f}{\partial t}(t,x)d\mu(x)
			 .\] 
			\end{enumerate}
			stiamo dicendo che la derivata dell'integrale è l'integrale della derivata.
		\end{teo}
		\begin{dimo}
			Da recupearare (Chat con Alberto Agostinelli)
		\end{dimo}
		\textbf{Osservazione}\\
		$f: I\times X \rightarrow [-\infty, +\infty]$ \\
		nelle ipotesi del teorema continua in $t_0\in I$ e  $|f(t,x)| \leq g(x) \ \ \forall t\in I$ quasi ovunque in  $X$.\\
		Posso considerare non tutti gli  $t\in I$ ma selezionar e i $t$ in sottointervalli di $I$ quindi in un intorno di $t_0$per avere la continuità in $ t_0$\\
		\subsection{Assoluta continuità dell'integrale}
			Ricordiamo che $s$ funzione semplice $s\geq 0$\\
			 $\forall E\in M$ e definiamo
			  \[
				  \mu_s(E) = \int_Esd\mu \Rightarrow \mu_S \text{ è una misura}
			 .\]  
			 adesso se $f\in L^1(X)$\\
			 $\mu_f(E) = \int_E|f|d\mu$ è una misura su  $X$\\
			  $\mu_f(E) = 0\ \ \forall E\in M$  tale che  $\mu(E) = 0$
		\begin{teo}[Assoluta continutià dell'integrale]
sia $f\in L^1(X)$\\
 $ \Rightarrow  \forall \e > 0 \ \ \exists \ \ \delta = \delta(\e, f) > 0$ tale che $\int_E|f|d \mu < \e \ \ \forall E\in M, \ \ \mu(E) < \delta$\\
 enunciato più "suggestivo":
 \[
 \lim_\mu(E) \rightarrow 0 \ \ \int_E|f|d\mu = 0
 .\] 

		\end{teo}
		il punto sta nel fatto che sti cazzi di chi è $E$ basta che la sua misura tenda a $0$
		 \begin{dimo}
			non la scrivo, mancano 3 giorni all'esonero, parla con Alberto Agostinelli.
		\end{dimo}
		\textbf{Osservazione}\\
		Sia $\{f_n\}\subset L^1(X)$\\
		 $ \Rightarrow  \forall \e > 0 \exists \delta = \delta(\e,f_n) = \delta (\e,n)$ tale che $\int_E |f_n|d\mu < \e $ se  $\mu(E) < \delta (\e ,n)$\\
		 Se  $\exists f\in L^1(X)$ tale che \\
		 $f_n \rightarrow f$ in $L^1(X)$ ( $\int_X|f_N-f|d\mu \rightarrow 0)$\\
		 $ \Rightarrow  $ la proprietà di assoluta continuità dell'integrale e`verificata uniformemente rispetto a $n$\\
		  $\forall \e > 0 \ \ \exists \delta_f(\e)$\\
		  tale che $\int_E|f|d\mu < \e \ \ $ se  $\mu(E) < \delta_f$\\
		   $\displaystyle\int_E|f_n|d\mu \leq \int_E|f_n-f|d\mu + \int_E|f|d\mu\leq \int_X|f_n-f|d\mu + \int_E|f|d\mu < \e + \e$\\
		   $ \displaystyle\Rightarrow $  per $\delta = \min\{\delta_{f_1},\ldots,\delta_{f_{n_\e}}, \delta_f\} > 0 $\\
		   $\displaystyle\int_E |f|d\mu < 2\e$ se  $\mu(E) < \delta \forall n\in \N$\\[10px]
		   Piccolo conto apparentemente poco utile:\\
		   \textbf{Riscrittura delle convergenza quasi ovunqe}\\
		   $(X,\mu)$ spazio di misura\\
		   $f_n,f: X \rightarrow [-\infty , + \infty]$ finite  quasi ovunque $f_n \rightarrow f $quasi ovunque\\
		   $ \Leftrightarrow \exists N\subset X, \mu(N) = 0 $ tale che $f_n(x) \rightarrow f(x)\ \ \forall x\in X\setminus N$\\
		   $ \Leftrightarrow \exists N\subset X\ \ \mu(N)$ tale ceh $\forall \e > 0 \ \ \exists k = k(x,\e)$ tale che 
		    \[
		   |f_n(x) - f(x)| < \e \ \ \ \forall n\geq k \ \ \ \forall x\inb X\setminus B
		   .\] 
		   $ \Leftrightarrow \exists N\subset X, \mu(N) = 0$ tale che 
		   \[
			   X\setminus N\subseteq \bigcup^{+\infty}_{k = 1} \bigcap^{+\infty}_{n = k}\{|f_n-f| < \e\} 
		   .\] 
		   $ \Leftrightarrow \e > 0 $
		   \[
			   \left( \bigcup^{+\infty}_{k= 1} \bigcap^{+\infty}_{n = k }\{|f_n-f| < \e\} \right)^c\subseteq N
		   .\] 
		   ovvero
		   \[
			   \left( \bigcap^{+\infty}_{k= 1} \bigcup^{+\infty}_{n = k }\{|f_n-f| \geq \e\} \right)\subseteq N
		   .\] 
		   quindi tutta sta roba ha misura nulla poiché contenuta in $N$ che ha misura nulla.\\

		 

	
\end{document}
