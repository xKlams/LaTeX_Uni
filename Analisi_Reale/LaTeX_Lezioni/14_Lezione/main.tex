\documentclass[12px]{article}

\title{Lezione 14 Analisi Reale}
\date{2025-04-29}
\author{Federico De Sisti}

\usepackage{amsmath}
\usepackage{amsthm}
\usepackage{mdframed}
\usepackage{amssymb}
\usepackage{nicematrix}
\usepackage{amsfonts}
\usepackage{tcolorbox}
\tcbuselibrary{theorems}
\usepackage{xcolor}
\usepackage{cancel}

\newtheoremstyle{break}
  {1px}{1px}%
  {\itshape}{}%
  {\bfseries}{}%
  {\newline}{}%
\theoremstyle{break}
\newtheorem{theo}{Teorema}
\theoremstyle{break}
\newtheorem{lemma}{Lemma}
\theoremstyle{break}
\newtheorem{defin}{Definizione}
\theoremstyle{break}
\newtheorem{propo}{Proposizione}
\theoremstyle{break}
\newtheorem*{dimo}{Dimostrazione}
\theoremstyle{break}
\newtheorem*{es}{Esempio}

\newenvironment{dimo}
  {\begin{dimostrazione}}
  {\hfill\square\end{dimostrazione}}

\newenvironment{teo}
{\begin{mdframed}[linecolor=red, backgroundcolor=red!10]\begin{theo}}
  {\end{theo}\end{mdframed}}

\newenvironment{nome}
{\begin{mdframed}[linecolor=green, backgroundcolor=green!10]\begin{nomen}}
  {\end{nomen}\end{mdframed}}

\newenvironment{prop}
{\begin{mdframed}[linecolor=red, backgroundcolor=red!10]\begin{propo}}
  {\end{propo}\end{mdframed}}

\newenvironment{defi}
{\begin{mdframed}[linecolor=orange, backgroundcolor=orange!10]\begin{defin}}
  {\end{defin}\end{mdframed}}

\newenvironment{lemm}
{\begin{mdframed}[linecolor=red, backgroundcolor=red!10]\begin{lemma}}
  {\end{lemma}\end{mdframed}}

\newcommand{\icol}[1]{% inline column vector
  \left(\begin{smallmatrix}#1\end{smallmatrix}\right)%
}

\newcommand{\irow}[1]{% inline row vector
  \begin{smallmatrix}(#1)\end{smallmatrix}%
}

\newcommand{\matrice}[1]{% inline column vector
  \begin{pmatrix}#1\end{pmatrix}%
}

\newcommand{\C}{\mathbb{C}}
\newcommand{\K}{\mathbb{K}}
\newcommand{\R}{\mathbb{R}}


\begin{document}
	\maketitle
	\newpage
	\subsection{Boh}
	\begin{prop}
		$(X,\mu)$ spazio di misura; $f_n, f: X \rightarrow[-\infty,+\infty]$ finite quasi ovunque; $f_n \rightarrow f \ \ q.o. \Leftrightarrow \ \forall \e>0\ \ \ \mu( \bigcap_{k=1}^{+\infty} \bigcup^{+\infty}_{n=k}\{|f_n-f|\geq \e\})=0$
	\end{prop}
	\begin{dimo}
		$ ( \Rightarrow )$ Già visto \\
		$( \Leftarrow)$  $\forall y \ \ \mu( \bigcap^{+\infty}_{n = 1} \bigcup^{+\infty}_{n=k}\{|f_n-f|\geq \frac 1y\}) = 0 \\\Rightarrow  \mu( \bigcup^{+\infty}_{y = 1} \bigcap^{+\infty}_{k =1} \bigcup^{+\infty}_{n = k}\{\ |f_n-f|\geq \frac 1y\}) := \mu(N) = 0$\\
		$x\in X\setminus N \Leftrightarrow x\in \bigcap^{+\infty}_{y = 1} \bigcup^{+\infty}_{k = 1} \bigcap^{+\infty}_{n=k}\{|f_n-f| <\frac 1y\}$\\
		Vuol dire che $\forall y \ \ \exists k_y$ (dipendente da  $y$) tale che $|f_n(x) -f(x)| < \frac 1y$\\
		 $\forall n\geq k_y \Rightarrow  f_n(x) \rightarrow f(x) \Rightarrow  f_n \rightarrow f$ quasi ovunque
	\end{dimo}
	Se io so che $\forall \e > 0 \ \ \mu(\{f_n-f|\geq\e\}) \xrightarrow{ n \rightarrow + \infty} 0 \Leftrightarrow ?$\\
	$\forall \e > 0 \ \ \mu( \bigcup^{+\infty}_{n=k}\{|f_n-f|\geq \e\}) \xrightarrow{n \rightarrow +\infty} 0 \Leftrightarrow ?$\\
	È una condizione più forte o debole?
	Osserviamo che $ \bigcup^{+\infty}_{n = k}\{|f_n-f|\geq \e\}$ forma una successione decrescente $( F_1\supseteq F_2\supseteq\ldots)$\\
	$\mu( \bigcap^{+\infty}_{n=1}F_n^\e = \mu ( \bigcap^{+\infty}_{n=1} \bigcup^{+\infty}_{n=k}\{|f_n-f|\geq \e\})\leq \lim_{ n \rightarrow +\infty}\ \mu( \bigcup^{+\infty}_{n = k}\{|f_n-f|\geq \e\})$\\
		Se poi diventano di misura finita (da un certo punto in poi) vale $=$\\
		 \begin{defi}
			 $f_n, f : X \rightarrow[-\infty,+\infty]$ misurabili finite quasi ovunque; $f_n \rightarrow f$ in misura se $\forall \e>0 \ \ \ \ \mu(\{|f_n-f|\geq\e\}) \xrightarrow{n \rightarrow +\infty} 0$
		\end{defi}
		\begin{prop}
			Sia $(X,\mu)$ spazio di misura finita $(\mu(X) < +\infty)$. Se $f_n, f : X \rightarrow [-\infty, + \infty]$ misurabili, finite quasi ovunque $ \Rightarrow $ $f_n \rightarrow f$ in misura
		\end{prop}
		\begin{dimo}
			Per la proposizione precedente:\\ $\forall \e > 0 , f_n \rightarrow f$ q.o. $ \Leftrightarrow \mu( \bigcap^{+\infty}_{k = 1} \bigcup^{+\infty}_{n = k}\{|f_n-f|\geq\e\}) = 0$\\
			ma questo per ipotesi è uguale a
			\[
				\limsup_{n \rightarrow +\infty}\mu(\{|f_n-f|\geq\e\})\leq\lim_{ k \rightarrow+\infty }\mu( \bigcup^{+\infty}_{n = k} \{|f_n-f|\geq \e\}) = 0
			.\]  
			se il $\limsup = 0 \Rightarrow  \lim = 0$  Quindi $\mu(\{f_n-f|\geq \e\}) \rightarrow 0 $
		\end{dimo}
		\begin{prop}
			Se $f_n, f \in L^1(X,\mu); f_n \rightarrow f$ in $L^1(X) \Rightarrow  f_n \rightarrow f$ in misura
		\end{prop}
		\begin{dimo}
			$\forall \e > 0$
			\[
				
				\e\mu(\{|f_n-f|\geq\e\})\leq\int_{\{|f_n-f|\geq\e\}}|f_n-f|d\mu\leq\int_X|f_n-f| d\mu \rightarrow 0
			.\]
			$ \Rightarrow  f_n \rightarrow f$ in misura
		\end{dimo}
		$f_n \rightarrow f$ in misura $\substack{?\\ \Rightarrow }f_n \rightarrow f$ quasi ovunque\\
		$\mu(\{|f_n-f|\geq\e\}) \rightarrow 0$ \\
		definiamo $g_n := |f_n-f|$,  $f_n : X \rightarrow [0,+\infty] \Rightarrow \mu(\{g_n > \e\}) \rightarrow 0 \substack{?\\ \Rightarrow}  g_n \rightarrow 0 $ quasi ovunque\\
		L'insieme di sopralivello può muoversi sull'asse $x$. Non è detto che fissata $x$ allora le successioni di funzioni tendano a $0$.\\
		 \textbf{Esempio}\\\
		 $\forall n$ dividiamo  $[0,1] $ in  $2^n$ intervalli di ampiezza  $\frac{1}{2^n}$\\
		 $I_{n,m} = [\frac{k-1}{2^n},\frac{k}{2^n}) \ \ 1\leq k\leq 2^n$\\
		 $\{\chi_{I_{k,n}}\}_{1\leq k\leq 2^n, n\geq 1}$ successione di funzioni misurabili secondno $Lebesgue$ su  $[0,1]$. Cosa succede per  $ n \rightarrow +\infty$ C'è convergenza solo a $0$\\
		 $\forall \e > 0, n(\{X_{I_{n,k}}\geq \e\}) = \begin{cases}
				 \emptyset\ \ \ \text{ se } \e > 1\\
				 \frac 1 {2^n} = m(I_{k,n})\ \  \ \text{ se } 0 < \e < 1
			 \end{cases} \xrightarrow{ n \rightarrow +\infty}$\\
			 $ \Rightarrow  \chi_{I_{n,k}} \xrightarrow{ n \rightarrow +\infty} 0$ in misura.\\
		 $\forall x\in [0,1), \chi_{I_{k,n}}(x) = 1$ per infiniti indici \\
		 \[
			 \int_{[0,1)}|\chi_{I_k,n}|dm = m(I_{k,n}) = \frac {1}{2^n} \xrightarrow{ n \rightarrow +\infty} 0

			 
		 .\] 
		 \[
			 \chi_{I_{k,n}} \rightarrow 0 \ \ \ \text{ in } L^1([0,1])
		 .\] 
	Quindi c'è anche la convergenza in $L^1$\\
	Quindi la convergenza in misura (e in  $L^1$ ) $\cancel{ \Rightarrow  }$ convergenza puntuale (quasi ovunque)\\
	Ma $\{\chi_{I_{n,m}}$ ha un'estratta che converge puntualmente a  $0$
		 \[
			 \chi_{I_{n,m}} = \chi_{[0,\frac{1}{2^n})} \rightarrow 0 \ \ \text{q.o. in } [0,1)
		.\] 
		\begin{teo}
			Siano $f_n, f: X \rightarrow [-\infty, +\infty]$ misurabili, finite quasi ovunqe. Se $f_n \rightarrow f$ in misura $ \Rightarrow  \exists $ sottosuccessione $\{f_{n_k}\}$ tale che  $f_{n_k} \rightarrow f$ quasi ovunque
		\end{teo}
		\begin{dimo}[Errata]
			$a_n\geq 0, a_n \rightarrow 0\ \ \sum^{+\infty}_{n =1}a_n < +\infty$ no!\\
			$\e > 0 $ fissato, $\forall k \ \exists n_k$ tale che  $\mu(\{|f_n-f|\geq \e\})< \frac{1}{2^k}$\\
			$\displaystyle \Rightarrow  \mu(\{|f_n-f|\geq \e\}) < \frac 1 {2^n} \Rightarrow  \mu( \bigcup^{+\infty}_{n = 1}\{|f_{n_k}-f|\geq\e\})\leq \sum^{+\infty}_{n=1}\mu(\{|f_{n_k}-f|\geq\e\})\leq \sum^{+\infty}_{n=1}\frac{1}{2k} < + \infty$ \\
			\[
				\mu( \bigcup^{+\infty}_{n =j }\{|f_{n_k}-f|\geq \e\})\leq \sum^{+\infty}_{k=j}\frac{1}{2^k} \rightarrow 0
			.\] 
			\[
				\Rightarrow \mu( \bigcap^{+\infty}_{j = 1} \bigcup^{+\infty}_{n = j}\{|f_{n_k}-f|\geq \e\}) = \lim_{j \rightarrow +\infty}\mu( \bigcup^{+\infty}_{n = j}\{|f_{n_k}-f|\geq\e\}) =0 
			.\] 
			Dov'è l'errore? Qui l'estratta dipende da $\e$, invece io voglio un'estratta che valga $\forall \e > 0$. In questo caso presa un'estratta questa vale  $\forall \e'\geq \e$. Quindi voglio sostituire  $\e$ con una cosa infinitesima. La dimostrazione è lasciata per esercizio.
	\end{dimo}
	\begin{coro}
		Se $f_N, f\in L^1(X,\mu), f_n \rightarrow f$ in $L^1 \Rightarrow \exists$ estratta $\{f_{n_k}\}$ tale che $f_{n_k} \rightarrow f$ quasi ovunque.
	\end{coro}
	\begin{dimo}
		$f_n \rightarrow$ in $L^1 \Rightarrow  f_n \rightarrow f $ in misura \hfill (disuguaglianza di Chebychev (?))\\
		$ \Rightarrow  \exists \{f_{n_k}\}: f_{n_k} \rightarrow f$ quasi ovunque.
	\end{dimo}
	\textbf{Osservazione}\\
	In generale non tutta $\{f_n\}$ converge puntualmente. Per esempio  $\chi_{I_{n,k}}$\\
	 \textbf{Osservazione}\\
	 Se $\mu(X) = +\infty, f_n \rightarrow f$ quasi ovunque  $ \Rightarrow  f_n \rightarrow $ in misura?\\
	 $\mu( \bigcap^{+\infty}_{k= 1} \bigcup^{+\infty}_{n = k}\{|f_n-f|\geq\e\}) =0$\\
Può essere che si formino tutte misure finite e l'intersezione faccia $0$.\\
In generale non vale per esempio  $f_n = \chi_{[n,+\infty)}, f_n(x) \rightarrow 0 \ \ \forall x\in \R$\\
Ma $m(\{f_n>\e\}) = +\infty$ per  $\e < 1\ \ \Rightarrow  \ \ f_n \cancel \rightarrow 0 $ in misura\\
Ma quindi bisogna che i sottoinsiemi vadano a $+\infty$, quindi di un ambiente di misura infinita.\\
Se  $\mu(X)< +\infty, f_n \rightarrow f$  quasi ovunque $ \Leftrightarrow \mu( \bigcap^{+\infty}_{k = 1} \bigcup^{+\infty}_{n = k}\{|f_n-f|\geq\e\}) = 0 = \lim_{ n \rightarrow +\infty} \mu ( \bigcup^{+\infty}_{n = k}\{|f_n-f|\geq\e\})$\\
$ \Rightarrow  \forall \delta > 0 \ \exists k = k(\delta, \e)$ tale che $\mu ( \bigcup^{+\infty}_{n = k}\{|f_n-f|\geq\e\})<\delta$\\
$x\in X\setminus F_{\delta, \e} \Leftrightarrow x\in \bigcap^{+\infty}_{n=k}\{|f_n-f|<\e\} \Leftrightarrow |f_n(x)-f(x)| < \e \ \ \forall n\geq k$\\
Abbiamo trovato il $k$ per cui l'ultima disequazione è piccola. Questo è vero $\forall x$. Ma allora  $\sup_{X\setminus F_{\delta,\e}}|f_n-f| < \e \ \ \forall n\geq k \Rightarrow  f_n \rightarrow f$ uniformemente in $X\setminus F_\delta$ 
\begin{defi}
	Siano $(X,\mu)$ uno spazio di misura  e  $f_n , f: X \rightarrow [-\infty, +\infty]$ misurabili finita quasi ovunque si dice $f_n \rightarrow f$ quasi uniformemente se $\forall \delta > 0 \ \ \exists F_\delta \subset X, F_\delta$ misurabile,  $\mu(F_\delta) < \delta$ tale che  $f_n \rightarrow f$ uniforme in $X\setminus F_\delta$.
\end{defi}
\textbf{Esempio}\\
$f_n(x) = x^n, \ \ x\in [0,1]$\\
$f_n(x) \xrightarrow{ n \rightarrow+\infty} \begin{cases}
	0 \ \ \ 0\leq x < 1\\
	1 \ \ \ \text{se } x = 1
\end{cases}=f(x)$\\
$\sup_{[0,1]}|f_n-f| = \sup_{0\leq x\leq 1}|f_n-f| = 1$\\
Se tolgo  $1, \sup_{[0,1)}|f_n-f| = \sup_{[0,1)}x^n = 1$ comunque le cose vanno male!\\
Dobbiamo togliere un intorno di  $1 \Rightarrow \sup_{[0,1-\delta]}|f_n-f| = \sup_{[0,1-\delta]}x^n = (1-\delta)^n \xrightarrow{n \rightarrow +\infty}0 $ \\
$ \Rightarrow  f_n \rightarrow 0 $ uniformemente in $[0,1-\delta]\forall \delta > 0 $
\begin{teo}[Erogv]
	Se $\mu(X) < + \infty$, $f_n \rightarrow f$ quasi uniformemente $ \Leftrightarrow f_n \rightarrow f $ quasi ovunque
\end{teo}
È per questo che non si può passare al limite sotto l'integrale(serve la convergenza uniforme), si ha al di fuori di un insieme di misura piccola $ \Rightarrow  $ serve almeno una dominazione
\end{document}
