\documentclass[12px]{article}

\title{Lezione 14 Analisi Reale}
\date{2025-04-29}
\author{Federico De Sisti}

\usepackage{amsmath}
\usepackage{amsthm}
\usepackage{mdframed}
\usepackage{amssymb}
\usepackage{nicematrix}
\usepackage{amsfonts}
\usepackage{tcolorbox}
\tcbuselibrary{theorems}
\usepackage{xcolor}
\usepackage{cancel}

\newtheoremstyle{break}
  {1px}{1px}%
  {\itshape}{}%
  {\bfseries}{}%
  {\newline}{}%
\theoremstyle{break}
\newtheorem{theo}{Teorema}
\theoremstyle{break}
\newtheorem{lemma}{Lemma}
\theoremstyle{break}
\newtheorem{defin}{Definizione}
\theoremstyle{break}
\newtheorem{propo}{Proposizione}
\theoremstyle{break}
\newtheorem*{dimo}{Dimostrazione}
\theoremstyle{break}
\newtheorem*{es}{Esempio}

\newenvironment{dimo}
  {\begin{dimostrazione}}
  {\hfill\square\end{dimostrazione}}

\newenvironment{teo}
{\begin{mdframed}[linecolor=red, backgroundcolor=red!10]\begin{theo}}
  {\end{theo}\end{mdframed}}

\newenvironment{nome}
{\begin{mdframed}[linecolor=green, backgroundcolor=green!10]\begin{nomen}}
  {\end{nomen}\end{mdframed}}

\newenvironment{prop}
{\begin{mdframed}[linecolor=red, backgroundcolor=red!10]\begin{propo}}
  {\end{propo}\end{mdframed}}

\newenvironment{defi}
{\begin{mdframed}[linecolor=orange, backgroundcolor=orange!10]\begin{defin}}
  {\end{defin}\end{mdframed}}

\newenvironment{lemm}
{\begin{mdframed}[linecolor=red, backgroundcolor=red!10]\begin{lemma}}
  {\end{lemma}\end{mdframed}}

\newcommand{\icol}[1]{% inline column vector
  \left(\begin{smallmatrix}#1\end{smallmatrix}\right)%
}

\newcommand{\irow}[1]{% inline row vector
  \begin{smallmatrix}(#1)\end{smallmatrix}%
}

\newcommand{\matrice}[1]{% inline column vector
  \begin{pmatrix}#1\end{pmatrix}%
}

\newcommand{\C}{\mathbb{C}}
\newcommand{\K}{\mathbb{K}}
\newcommand{\R}{\mathbb{R}}


\begin{document}
	\maketitle
	\newpage
	\subsection{Boh}
	\begin{prop}
		$(X,\mu)$ spazio di misura; $f_n, f: X \rightarrow[-\infty,+\infty]$ finite quasi ovunque; $f_n \rightarrow f \ \ q.o. \Leftrightarrow \ \forall \e>0\ \ \ \mu( \bigcap_{k=1}^{+\infty} \bigcup^{+\infty}_{n=k}\{|f_n-f|\geq \e\})=0$
	\end{prop}
	\begin{dimo}
		$ ( \Rightarrow )$ Già visto \\
		$( \Leftarrow)$  $\forall y \ \ \mu( \bigcap^{+\infty}_{n = 1} \bigcup^{+\infty}_{n=k}\{|f_n-f|\geq \frac 1y\}) = 0 \\\Rightarrow  \mu( \bigcup^{+\infty}_{y = 1} \bigcap^{+\infty}_{k =1} \bigcup^{+\infty}_{n = k}\{\ |f_n-f|\geq \frac 1y\}) := \mu(N) = 0$\\
		$x\in X\setminus N \Leftrightarrow x\in \bigcap^{+\infty}_{y = 1} \bigcup^{+\infty}_{k = 1} \bigcap^{+\infty}_{n=k}\{|f_n-f| <\frac 1y\}$\\
		Vuol dire che $\forall y \ \ \exists k_y$ (dipendente da  $y$) tale che $|f_n(x) -f(x)| < \frac 1y$\\
		 $\forall n\geq k_y \Rightarrow  f_n(x) \rightarrow f(x) \Rightarrow  f_n \rightarrow f$ quasi ovunque
	\end{dimo}
	Se io so che $\forall \e > 0 \ \ \mu(\{f_n-f|\geq\e\}) \xrightarrow{ n \rightarrow + \infty} 0 \Leftrightarrow ?$\\
	$\forall \e > 0 \ \ \mu( \bigcup^{+\infty}_{n=k}\{|f_n-f|\geq \e\}) \xrightarrow{n \rightarrow +\infty} 0 \Leftrightarrow ?$\\
	È una condizione più forte o debole?
	
\end{document}
