\documentclass[12px]{article}

\title{Lezione 02}
\date{2025-02-28}
\author{Federico De Sisti}

\usepackage{amsmath}
\usepackage{amsthm}
\usepackage{mdframed}
\usepackage{amssymb}
\usepackage{nicematrix}
\usepackage{amsfonts}
\usepackage{tcolorbox}
\tcbuselibrary{theorems}
\usepackage{xcolor}
\usepackage{cancel}

\newtheoremstyle{break}
  {1px}{1px}%
  {\itshape}{}%
  {\bfseries}{}%
  {\newline}{}%
\theoremstyle{break}
\newtheorem{theo}{Teorema}
\theoremstyle{break}
\newtheorem{lemma}{Lemma}
\theoremstyle{break}
\newtheorem{defin}{Definizione}
\theoremstyle{break}
\newtheorem{propo}{Proposizione}
\theoremstyle{break}
\newtheorem*{dimo}{Dimostrazione}
\theoremstyle{break}
\newtheorem*{es}{Esempio}

\newenvironment{dimo}
  {\begin{dimostrazione}}
  {\hfill\square\end{dimostrazione}}

\newenvironment{teo}
{\begin{mdframed}[linecolor=red, backgroundcolor=red!10]\begin{theo}}
  {\end{theo}\end{mdframed}}

\newenvironment{nome}
{\begin{mdframed}[linecolor=green, backgroundcolor=green!10]\begin{nomen}}
  {\end{nomen}\end{mdframed}}

\newenvironment{prop}
{\begin{mdframed}[linecolor=red, backgroundcolor=red!10]\begin{propo}}
  {\end{propo}\end{mdframed}}

\newenvironment{defi}
{\begin{mdframed}[linecolor=orange, backgroundcolor=orange!10]\begin{defin}}
  {\end{defin}\end{mdframed}}

\newenvironment{lemm}
{\begin{mdframed}[linecolor=red, backgroundcolor=red!10]\begin{lemma}}
  {\end{lemma}\end{mdframed}}

\newcommand{\icol}[1]{% inline column vector
  \left(\begin{smallmatrix}#1\end{smallmatrix}\right)%
}

\newcommand{\irow}[1]{% inline row vector
  \begin{smallmatrix}(#1)\end{smallmatrix}%
}

\newcommand{\matrice}[1]{% inline column vector
  \begin{pmatrix}#1\end{pmatrix}%
}

\newcommand{\C}{\mathbb{C}}
\newcommand{\K}{\mathbb{K}}
\newcommand{\R}{\mathbb{R}}

\usepackage{yfonts}

\begin{document}
	\maketitle
	\newpage
	\subsection{Prima scheda informazioni}
	parte da recuperare\\
	\subsection{Misure}
	$X$ insieme non vuoto\\
	$2^X = $  insieme delle parti di  $X$ =  $\{$ sottoinsiemi $E\subseteq X\}$\\
	$\phi,X\in 2^X = \{\chi:X \rightarrow\{0,1\}\}$\\
	$\chi \leftrightarrow E = \{\chi = 1\}$\\

	%TODO aggiungi immagine funzione successiva
	 $\chi_E(x) = \begin{cases}
		 1 \ \ \text { se } x\in E\\
		 0 \ \ \text { se } x\in X\setminus E
	 \end{cases}$
	 \begin{defi}
		 Sia $X$ non vuoto. Una misura è una funzione $\mu : 2^X \rightarrow [0,+\infty]$ che soddisfa le due proprietà:
		 \begin{enumerate}
			 \item $\mu(\emptyset) = 0$
			 \item  per ogni famiglia numerabile di sottoinsiemi $E,\{E_i\}_{i\in\N^+} \subseteq X$\\
				 $\displaystyle E\subseteq\bigcup_{i=1}^\infty E_i \Rightarrow \mu(E)\leq \sum^\infty_{i=1}\mu(E_i)$
		 \end{enumerate}
		 La seconda proprietà viene chiamata sub-additività numerabile
	 \end{defi}
\textbf{Commenti:}\\
1) numerabile $ \Leftrightarrow $ al più numerabile\\
$\{E_i\}_{i\in\N^+}$ possono essere finite: $\{E_1,E_2,\ldots,E_n\}$ 
$\N^+ = \{1,2,3,\ldots\}$\\
2) Proprietà di monotonia: $E\subset F \Rightarrow \mu(E)\leq \mu (F)$\\
Segue da (ii) prendendo $E_1 = F, E_2 = \emptyset, E_3=\emptyset, \ldots$\\
3) Gli insiemi $\{E_i\}$ non sono necessariamente disgiunti\\
4) In generale in (ii) non vale l'uguaglianza neanche se:\\ $E = E_1\cup E_2 $ con $E_1\cap E_2= \emptyset$\\
Può accadere che $E\cap F = \emptyset$\\
 \[
\mu(E\cup F) < \mu(E) + \mu (F)
.\] 
5) Comunemente quello che noi chiamiamo misura sono dette misure esterne\\
\hline\ \\
Esempi di misure:
\begin{itemize}
	\item La misura che conta: $X$\\
		$\mathbb H^0: 2^X \rightarrow [0,+\infty]$\\
		$\mathbb H^0(E) = \begin{cases}
			0 \ \ E =\emptyset\\
			n \ \ E \text{ ha n elementi}\\
			+\infty \ \ E \text { infinito}
		\end{cases}$
	\item Misura delta di Dirac:\\
		$X, \ x_0\in X$\\
		$\delta_{x_0}: 2^X \rightarrow[0,+\infty]$\\
		$\delta_{x_0}(E) = \begin{cases}
			1 \ \ \text{ se } x_0\in E\\
			0 \ \ \text { se } x_0 \not\in E
		\end{cases}$
\end{itemize}
\textbf{Verifica}\\
$\delta_{X_0}$ è una misura\\
\textbf{Osservazione}\\
Se $X$ è infinito allora $H^0(X) = +\infty$\\
Viceversa  $\delta_$ da finire\\
\subsection{Insiemi misurabili}\\
$X \not = \emptyset, \mu$ misura su $X$\\
 \textbf{Osservazione}\\
 Possono esistere $E, F$ t.c.\\
  \[
	  E\cap F = \emptyset \text{ ma } \mu(E\cup F) < \mu (E) + \mu (F) 
 .\] 
 \begin{defi}[Caratheodory]
 	Sia $X\not = \emptyset$ e  $\mu$ misura su $X$\\
	Un insieme  $E\subseteq X$ si dice misurabile se vale:
	 \[
	\mu(A) = \mu(A\cap E) + \mu (A\setminus E) \ \ \ \forall A \subseteq X
	.\] 
 \end{defi}
 \textbf{Commenti:}\\
 1) $A = X$
  \[
  \mu(X) = \mu(E)  + \mu (E^c)
 .\] 
 2) Vale sempre 
 \[
 \mu(A) \leq \mu(A\cap E) + \mu(A\setminus E)
 .\] 
 \begin{center}
 \begin{aligned}
	 E \text { è mi}&\text{surabile}\\
	   & \storto\Leftrightarrow\\
	 \mu(A) \geq  \mu(A\cap &E) + \mu (A\setminus E)

\end{aligned}
 \end{center}
 \newpage
 \begin{teo}
 	Sia $X \neq \emptyset$ e $\mu$ misura.
	\begin{enumerate}
		\item la classe degli insiemi misurabili è una $\sigma$-algebra:\\
			1) $\emptyset, X\in M$ \\
			2) $E\in M \Rightarrow E^c\in M$ \\
			3) $\{E_i}_i{i\in\N^+}\subseteq M \Rightarrow \bigcup^\infty_{i=1} E_i\in M$
		\item $\mu$ è numerabilmente additiva su M: se $\{E_i\}_{i\in N^+}$ sono disgiunti a coppie $(E_i\cap E_j = \emptyset \ \ \forall i\neq j)$ allora
			\[
				\left(\bigcup^\infty_{i=1}E_i \right) = \sum^\infty_{i=1}\mu(E_i)
			.\] 
	\end{enumerate}
 \end{teo}
 %TODO controlla IL TEORMA DALGI APPUNTI
 \textbf{Commenti}\\
 1) M è chiuso anche per intersezioni numerabili: $E_i\in M$\\
  \[
	  \left(\bigcap_i E_i \right)^c = \bigcup_i E_i ^c\in M \Rightarrow \bigcap_i E_i\in M
 .\] 
 2)  $\lim sup_{i \rightarrow\infty} E_i := \bigcap_{N\in\N}\bigcup_{i\geq N} E_i$\\
  $\lim inf_{i \rightarrow\infty} E_i := \bigcup_N\in\N\bigcap_{i\geq N} E_i$

 \begin{dimo}
 	Passo 1: $M$ è un algebra\\
		\cdot $\emptyset\in M, X\in M$\\
	Vado a verificare che $\forall A\subseteq X$ vale
	\[
	\mu(A) = \mu(A\cap \emptyset) + \mu (A\setminus\emptyset) = \mu(\emptyset) + \mu(A)
	.\] 
	dove sappiamo che $\my(\emptyset) = 0$\\
	Per  $X:$
	 \[
	\mu(A) = \mu(A\cap X) + \mu(A\setminus X) = \mu(A) + \mu(\emptyset)
	.\] 
	$\cdot E \in M \Rightarrow  E^c\in M$\\
	Vado a verificare che per ogni $A\subseteq X$ vale le proprietà di Caratheodory:
	$\mu(A) = \mu(A\cap E^c) + \mu (A\setminus E^c) = \mu (A\setminus E ) + \mu(A\cap E)$\\
	$\cdot \ E_2, E_2\in M \Rightarrow E_1\cup E_2\in M$
	Considero un insieme test $A\subseteq X:$\\
	 $\mu(A) = \mu(A\cap E_1) + \mu(A\setminus E_1)$ 
	 \[
	 1)\ \ \mu(A) = \mu(A\cap E_1) + \mu(A\setminus E_1)
	 \] 
	 \[
	  \mu(A\cap E_1) + \mu ((A\setminus E_1)\cap E_2) + \mu((A\setminus E_1)\setminus E_2)
	 \] 
	 il risultato è ottenuto applicando Caratheodory al secondo termine della somma (1)\\
\[
\geq \mu((A\cap E_1)\cup(A\setminus E_1)\cap E_2) + \mu(A\setminus(E_1\cup E_2))
\] 
\[
  \mu(A)\geq \mu(A\cap (E_1\cup E_2)) + \mu (A\setminus (E_1\cup E_2)) \Rightarrow E_1\cup E_2\in M
\] 
Passo 2: finita additività di $\mu$ in $M$\\
 $E_1, E_2\in M, E_1\cap E_2 = \emptyset$\\
 Per ogni $A\subseteq X$:\\
  \[
 \mu(A\cap (E_1\cup E_2)) = \mu(A\cap (E_1\cup E_2)\cap E_1) + \mu (A\cap (E_1\cup E_2)\setminus E_1)
 \] 
 Ottenuto sempre per Caratheodory
 \[
 \mu(A\cap E_1) + \cap (A\cap E_2)
 .\] 
 Iterando questo passaggio:\\
 $E_1, \ldots, E_n\in M$ allora:\\
 \[
	 \mu(A\cap \bigcup^N_{i=1} E_i) = \sum^N_{i=1} \mu (A\cap E_i)
 .\] 
 \textbf{Spiegazione passaggio precedente}
 \[
  \mu (A\cap \bigcup^N_{i=1} E_i) = \mu (A\cap E_1) + \mu (A\cap \bigcup^N_{i=2} E_i) = \ldots = \sum^N_{i=1} \mu (A\cap E_i)
 .\] 
 Passo 3: mostriamo le proprietà di $\sigma$-algebra e numerabile additività\\
 Siano $\{E_i\}_{i\in\N^+}\subseteq M$\\
 Consideriamo gli insiemi:\\
 $F_1:= E_1,\ \ \ \ \  F_2 = E_2\setminus E_1$\\
 $F_3 := E_3\setminus(E_1\cup E_2)$\\
 quindi definiamo ricorsivamente:
 $F_k : = E_k\setminus\bigcup_{i=1} ^{k-1}$\\
 Allora  $F_i$ sono disgiunti a coppie 
 \[
	 (F_i\cap F_j = \emptyset \ \ \forall i\neq j)
 .\] 
 \[
	 \bigcup_{i=1}^\infty F_i = \bicup_{i=1}^\infty E_i
 .\] 
 $\cdot F_i\in M$\\
 Fissiamo il test di Caratheodory $A\subseteq X, \ F_i\in M$, Passo 1: $M$ algebra\\
   \[
	  \mu(A) = \mu(A\cap\bigcup^N_{i=1}F_i) + \mu(A\setminus \bigcup_{i=1}^N F_i)
 .\] 
 Usando il passo 2: finita additività
 \[
	 = \sum^N_{i=1}\mu(A\cap F_i) + \mu (A\setminus\bigcup^N_{i=1}F_i)
 \] 
 \[
	 \geq \sum^N_{i=1} \mu(A\cap F_i) + \mu(A\setminus\bigcup^\infty_{i=1}F_i)
 .\] 
 Passiamo al limite $ N \rightarrow\infty$
\[\mu (A) \geq \lim_{N \rightarrow\infty} \sum^N_{i=1}\mu(A\cap F_i) + \mu(A\setminus \bigcup^\infty F_i) = \sum^\infty_{i=1} \mu(A\cap F_i) + \mu(A \setminus \bigcup F_i)\]
\begin{center}
\begin{aligend}
	
	&\dicentertyle\geq \mu ( A\cap \bigcup^\infty_{i=1} F_i) + \mu (A\setminus \bigcup^\infty_{i=1}F_i)\\
	&= \mu(A\cap \bigcup^\infty E_i) + \mu(A\setminus \bigcup^\infty E_i)\\
	& \Rightarrow \bigcup^\infty_{i=1}E_i\in M
\end{aligend}
\end{center}
Se prendiamo come test $A = \bugcup^\infty_{i=1}F_i$, allora $\mu(\bigcup^\infty_{i=1}F_i)\geq \sum^\infty_{i=1}\mu(E_i)\geq \mu(\bigcup^\infty_{i=1}F_i)$\\
$ \Rightarrow \mu(\bigcup^\infty F_i) = \sum^\infty_{i=1}\mu(F_i)$ |\
$F_i$ sono disgiunti a coppie
 \end{dimo}


	
\end{document}
