\documentclass[12px]{article}

\title{Lezione 15 Analisi Reale}
\date{2025-04-16}
\author{Federico De Sisti}

\usepackage{amsmath}
\usepackage{amsthm}
\usepackage{mdframed}
\usepackage{amssymb}
\usepackage{nicematrix}
\usepackage{amsfonts}
\usepackage{tcolorbox}
\tcbuselibrary{theorems}
\usepackage{xcolor}
\usepackage{cancel}

\newtheoremstyle{break}
  {1px}{1px}%
  {\itshape}{}%
  {\bfseries}{}%
  {\newline}{}%
\theoremstyle{break}
\newtheorem{theo}{Teorema}
\theoremstyle{break}
\newtheorem{lemma}{Lemma}
\theoremstyle{break}
\newtheorem{defin}{Definizione}
\theoremstyle{break}
\newtheorem{propo}{Proposizione}
\theoremstyle{break}
\newtheorem*{dimo}{Dimostrazione}
\theoremstyle{break}
\newtheorem*{es}{Esempio}

\newenvironment{dimo}
  {\begin{dimostrazione}}
  {\hfill\square\end{dimostrazione}}

\newenvironment{teo}
{\begin{mdframed}[linecolor=red, backgroundcolor=red!10]\begin{theo}}
  {\end{theo}\end{mdframed}}

\newenvironment{nome}
{\begin{mdframed}[linecolor=green, backgroundcolor=green!10]\begin{nomen}}
  {\end{nomen}\end{mdframed}}

\newenvironment{prop}
{\begin{mdframed}[linecolor=red, backgroundcolor=red!10]\begin{propo}}
  {\end{propo}\end{mdframed}}

\newenvironment{defi}
{\begin{mdframed}[linecolor=orange, backgroundcolor=orange!10]\begin{defin}}
  {\end{defin}\end{mdframed}}

\newenvironment{lemm}
{\begin{mdframed}[linecolor=red, backgroundcolor=red!10]\begin{lemma}}
  {\end{lemma}\end{mdframed}}

\newcommand{\icol}[1]{% inline column vector
  \left(\begin{smallmatrix}#1\end{smallmatrix}\right)%
}

\newcommand{\irow}[1]{% inline row vector
  \begin{smallmatrix}(#1)\end{smallmatrix}%
}

\newcommand{\matrice}[1]{% inline column vector
  \begin{pmatrix}#1\end{pmatrix}%
}

\newcommand{\C}{\mathbb{C}}
\newcommand{\K}{\mathbb{K}}
\newcommand{\R}{\mathbb{R}}


\begin{document}
	\maketitle
	\newpage
	\subsection{Convergenze varie (alberto agostinelli)}
	\begin{defi}[Convergenza quasi uniforme]
		$f_n \rightarrow f$ quasi uniformemente se $\forall \delta > 0 \ \ \exists F_\delta\subseteq X, F_\delta$ misurabile  $\mu(F_\delta) < \delta$\\
	tale che   $\sup_{X\setminus F_\delta} |f_n - f| \rightarrow 0$ $(f_n \rightarrow f$ uniformemente in $X\setminus F_\delta)$
	\end{defi}
	\begin{prop}
		$f_n \rightarrow f$ quasi uniformemente $ \Leftrightarrow \forall \e > 0 \mu ( \bigcup^{+\infty}_{n = k}\{|f_n - f | \geq \e \}) \xrightarrow{ k \rightarrow + \infty} 0$
	\end{prop}
	\begin{dimo}
		$ ( \Rightarrow )$ \\
		$\forall \delta > 0 \exists F_\delta \subset X, \mu(F_s) < \delta $ tale che,  $f_n \rightarrow f$ uniformemente in $X\setminus F_\delta \\\Leftrightarrow \forall \delta > 0 \exists F_\delta \subset X, \mu (F_\delta)<\delta$ \\
		tale che $\forall \e > 0 \exists k = k(\e,\delta) \ : \ |f_n(x) - f(x)| <\e \ \ \ \forall n\geq k \ \ \ \forall x\in X\semtinsu F_\delta$\\
		$ \Leftrightarrow \forall \delta > 0 \exists F_\delta\subset X, \mu(F_\delta) < \delta$ tale che $\forall \e>0 \exists k = k(\delta,\e)$ \\
		\[
		X\setminus F_\delta \subseteq \bigcap_{n=k}^{+\infty}\{|f_n-f| < \e\}
		.\] 
		$ \Leftrightarrow \forall \delta > 0 \ \ \exists F_\delta \subset X \ \ \mu(F_\delta)<\delta$\\
		tale che $\forall \e > 0 \ \ \exists k = k(\e, \delta)$\\
		 \[
			 \left( \bigcap_{n=k}^{+\infty}\{f_n-f| < \e\})^c = \bigcup^{+\inftu}_{n=k}\{f_n-f|\geq \e\}\subseteq F_\delta
		.\] 
		$ \Rightarrow  \forall \delta > 0  \ \ \forall \e > 0 \ \ \ \exists k = k(\delta,\e)$ \\
		 \[
			 \mu( \bigcup^{+\infty}_{n=k}\{|f_n - f|\geq \e\})<\delta \Rightarrow \mu( \bigcup^{+\infty}_{n = k}\{|f_n-f|\geq \e\}) \xrightarrow{k \rightarrow +\infty} 0
		.\] 
		$( \Leftarrow)$ \\
		$\foralll \e>0 \ \ \forall \delta > 0 \ \ \ \exists k = k(\e,\delta)$ tale che \\
		 \[
			 \mu( \bigcup^{+\infty}_{n = k } \{|f_n-f|\geq \e\})<\delta
		.\] 
		$\forall j\in \N$ per  $\e = \frac 1 j, \ \ \delta = \frac\nu {2^j}, \nu > 0$ fissato\\
		$ \Rightarrow \exists k_j = k_j(j,\nu)$ tale che $\mu ( \bigcup^{+\infty}_{n = k_j}(\{f_n-f|\geq \frac 1 j\}) < \frac \nu {2^j}$\\
		\[
			\Rightarrow \mu ( \bigcup^{+\infty}_{j = 1} \bigcup^{+\infty}_{n = k_j} \{ |f_n - f|\geq \frac 1 j\}) \leq \sum^{+\infty}_{j = 1}\frac \nu {2^j} = \nu
		.\] 
		$x\in X\setminus F_\nu$ (dove  $F_\nu$ è l'argomento della misura precedente)\\
		$ \Rightarrow x\in \bigcap^{+\infty}_{j = 1} \bigcap^{+\infty}_{n = k_j}\{|f_n-f| < \frac 1 j\}$ \\
		$\Rightarrow  \forall j \ \ \ \exists k_j$ tale che $|f_n(x) - f(x)| < \frac 1 j\ \ \ \forall n \geq k_j$|\
		$ \Rightarrow  \sup_{X\setminus F_\nu}|f_n - f| \xrightarrow{n \rightarrow +\infty} 0 \Rightarrow f_n \rightarrow f$ uniformemente su $X\setminus F_\nu$\\
		Abbiamo caratterizzato la convergenza quasi uniforme con la misura dei sopralivelli  $\forall \e > 0 $\\ 
		conseguenza:\\
		\[
			f_n \rightarrow f \text{ q.u.} \Rightarrow  \begin{cases}
				f_n \rightarrow f \ \text{ q.u.}\\
				f_n \rightarrow f \ \text{ in misura}
			\end{cases}
		\] 
		\[
		\storto \Leftrightarrow
		\] 
		\[
			\forall \e > 0 \ \ \mu \left( \bigcup^{+\infty}_{n = k}\{|f_n - f| \geq e\} \right) \xrightarrow{k \rightarrow +\infty} 0
		.\] ma allora
		\[
		0 = \lim_{k \rightarrow +\infty}\mu ( \bigcup^{+\infty}_{n = k }\{|f_n - f| \geq \e\} ) = \mu ( \bigcap^{+\infty}_{k = 1} \bigcup^{+\infty}_{n = k}\{|f_n-f|\}\geq \e\})
		.\] 
		\[
			\forall k \ \ \mu(\{f_k -f| \geq \e\}\leq \mu( \bigcup^{+\infty}_{n= k}\{\f_n - f| \geq \e\}) \rightarrow 0
		.\] 
		 $ \Rightarrow f_n \rightarrow f$  in misura

	\end{dimo}
	\begin{teo}[Egorov]
		Sia $(X,\mu)$ spazio di misura finita (  $ \mu(X) < +\infty$ )\\
		Allora:
		\[
		f_n \rightarrow f \ \ \ \ q.o.\ \ \  \Leftrightarrow \ \ \ f_n \rightarrow f \ \ \ \ q.u.
		.\] 
	\end{teo}
	\begin{teo}[Vitali]
		Sia $(X,\mu)$ uno spazio di misura  finita e siano  $f_n, f\in L^1(X)$ tale che  $f_n \rightarrow f$ quasi ovunque quasi ovunque\\
		allora $f_n \rightarrow f $ in $L^1 \Leftrightarrow \{f_n\}$ equi-assolutamente integrabili\\ $\forall \e> 0 \ \exists \delta = \delta(\e) > 0 $ tale che
		\[
			\int_E |F_n|d\mu < \e \text{ se } E\in M \ \ \forall n, \mu(E) < \delta
		.\] 
	\end{teo}
	\begin{dimo}
		$ ( \Rightarrow  )$ già visto\\
		$ ( \Leftarrow )$\\
		$f_n \rightarrow f$ quasi ovunque + $\mu(X) < +\infty$\\
		 $ \Rightarrow  $ (per Egorov)\\
		 $\forall \delta > 0 \exists f_\delta\in M, \mu (F_\delta) < \delta$ tale che  $f_n \rightarrow f$ uniformemente in $X\setminus f_\delta$\\
		 Sia  $\e > 0$ fissato\\
		  $ \Rightarrow  $ sia $\deta = \delta ( \e)$ dato dall'ipotesi di equi-assoluta integrabilità\\
		  e sia  $f_\delta\in M$ dato dal teorema di Egorov
		   \[
			   \Rightarrow  \int_X |f_n - f| d\mu = \int_{X\setminus F_\delta}|f_n - f| d\mu  + \int_{F_\delta}|f_n - f| d\mu 
		  .\] 
		  \[
		  \leq \sup_{X\setminus F_\delta} |f_n-f| \mu (X\setminus F_\delta) + \int_{F_\delta}|f_n|d\mu + \int_{F_\delta}|f|d\mu
		  .\] 
		  \[
			  \leq (\sup_{X\setminus F_\delta}|f_n-f|)\mu(X) + \e + \e \ \  (\text{dato che } \mu(F_\delta) < \delta )
		  .\] 
		  \[
			  \Rightarrow \limsup_{n \rightarrow +\infty}\int_X |f_n - f| d\mu \leq 2\e \ \ \forall \e > 0 
		  .\] 
		  \[
		  \Rightarrow  \int_X \ |f_n - f | d\mu \rightarrow 0
		  .\] 
	\end{dimo}
	$f: \R \rightarrow \R \ \ (\R, m)$\\
	$f$ continua $ \Rightarrow  f$ misurabile\\
	$f$ continua quasi ovunque $ \Rightarrow  f$ misurabile\\
	MANCA UNA PARTE\\
	\[
		D_f = \{x\in \R \ | f \text{ è discontinua in } X\}\\
	.\] 
	$m\mu(D_f) = 0$  $f$ è misurabile, infatti:
	\[
		\forall t\in \R \ \ \{f > t\} = \{f > t\}\cap D_f\cup \{f > t \} \setminus D_f
	.\] 
	$ \Rightarrow  $ ha misura nulla $ \Rightarrow  $ è misurabile\\
	$x\in \{f > t\} \setminus D_f$\\
	$\lim_{y ->x} f(y) =  \Rightarrow  f(x) > t$ e $f$ è continua in $X$\\ 
	$ \Rightarrow  \exists \delta_x > 0 \ : \ f(y) > t \ \ \ \forall y\in (x - \delta_x, x + \delta_x)$ \\
	$ \Rightarrow \{f > t\}\setminus D_f = \bigcup^{}_{x\in\{f > t\}\setminus D_f}(x-\delta_x, x + \delta_x)\setminus D_f = \bigcup^{}_{x\in\{f>t\}\setminus D_f}(x-\delta_x,x+\delta_x)\detminus D_f$ aperto è misurabile\\
	$f : \R \rightarrow \R$\\
	se $\exists g\in C(\R)$\\
	tale che  $f = g$ quasi ovunque  $ \Rightarrow  f$ misurabile\\
	$\exists N\subset \R, m(N) = 0$\\
	tale che  $f = g$ in  $\R\setminus N$ \ \\
	$\forall t\in \R \ \ \{f > t\} = \{f > t\}\setminus N \cup \{f > t\} \setminus N$ è misurabile\\
	 $f = \chi_\Q = 0$ quasi ovunque\\
	  $ f = g$ quasi ovunque  $\exists N, \mu (N) \ \ f = g $ in  $\R\setminus N$\\
	  $x\in \R \setminus N$ \ \  $\lim_{y \rightarrow x} f(y)$\\
	  $f = \chi_[0,1]$ è continua quasi ovunque ma non  può essere ugguale quasi ovunque ad una funzione continua
	  \begin{teo}
	  	Sia $f: \R \rightarrow R$ misurabile $  \Rightarrow  \forall \delta > 0 \ \ \exists g_\delta \in C(\R)$\\
		tale che $m(\{f\neq g\}) < \delta$\\
	e	$\sup_{\R}|g_\delta| \leq \sup_\R|f|$
	  \end{teo}
\end{document}
