\documentclass[12px]{article}

\title{Lezione 03 (La prima con la Leoni)}

\date{2025-03-04}
\author{Federico De Sisti}

\input{../../../setup.tex}

\begin{document}
	\maketitle
	\newpage
	\section{Misura di Lebegque}
	\subsection{Porprietà dellle afunzioen lunghezza di intervalli}
	$I$ intervallo in $\R$ \\
	$|I| = \begin{cases}
		+\infty \ \ \ \ \ \ \ \ \ \ \ \ \ \ \ \ \ \ \ \text { se } I \text { è illimitato}\\
		sup I - inf I  \ \ \ (b - a)  \ \ \ \text{Se } I \text{ è limitato di estremi } a < b
	\end{cases}$\\
\textbf{Esempi di intervallo}\\
$\emptyset = (a,a) \ \forall a \in \R$\\
 $\R = (-\infty, \infty)$|\
 $\{x\} = [x,x] \ \ \ \forall x\in \R$ \\
 \textbf{Proprietà:}
 \begin{enumerate}
	 \item $|\emptyset  | = 0$
	 \item monotonia\\
		 $I\subseteq J \Rightarrow  |I| \leq |J|$
	 \item finita additività\\
		  $\displaystyle I = \bigcup^n_{i = 1} I_i \ \ I_i$ interevallo\\
		  $I_i\cap I_j = \emptyset \ \ \forall i\neq j$\\
		  $  \displaystyle \Rightarrow |I| = \sum^n_{i=1} |I_i|$
  \end{enumerate}
  \textbf{Nota}\\
se $I$ illimitato\\
$ \Rightarrow  \exists 1\leq i\leq n $ tale che $I_i$ illimitato\\
 $ \Rightarrow |I| = +\infty = |I_i| = \sum^n_{i=1}|I_k|$ \\
 Se $I$ limitato $ \Rightarrow  I_i$ limitato $\forall i = 1,\ldots, n$\\
$|I| = \sum^n_{i=1}|I_i|$
\begin{enumerate}
	\item[4.] $I$ intervallo\\
		$\displaystyle |I| = \sum_{n\in\Z}|I\cap [n,n+1)|$\\
		$\displaystyle = |I| = \sum^\infty_{n=0}|I\cap [n, n + 1)| + \sum^{-\infty}_{n=0}|I\cap [n, n + 1)|$
\end{enumerate}
\textbf{Nota}\\
Se $I$ illimitato \\
$ \Rightarrow  I\cap[n,n+1] = [n,n+1) $ per infiniti indici $n\in\Z$\\
$ \Rightarrow |I| = +\infty = \sum^{n\in\Z}|I\cap[n,n+1)|$ per infiniti n\\
Se $I$ limitato\\
$ \Rightarrow  I = \bigcup^k_{n = l}I\cap [n,n+1]$ per $l,k\in\Z$\\
\begin{enumerate}
	\item[5.] Numerabile subadditività\\
		Se $I$ intervallo, $\{I_i\}$ successione di intervalli tale che\\
		$I\subseteq\bigcup^\infty_{i=1} I_i$\\
		$ \Rightarrow  |I|\leq \sum^\infty_{i=1} |I_i|$
\end{enumerate}
\textbf{Dimostrazione 5.}\\
Si può assumere $I_i$ limitato $\forall i$\\
1) caso,  $I$ compatto, $I_i$ aperti $\forall i$\\
$I = [a,b], I_i = (a_i,b_i)$ \ \  $a_i<b_i$\\
$I$ compatto, $\{I_i\}$ ricoprimento aperto\\
$ \Rightarrow \exists$ sottoricoprimento finito
\[
	I = [a,b]\subseteq\bigcup^n_{k=1}I_{k}
.\] 
Dato che sono un numero finito di intervalli dico che $I_1$ è quello con l'estremo più a sinistra di tutti.\\
si può supporre che $a_1 < a < b_1$ se $b_1 > b$ $ \Rightarrow  I\subseteq I_1 \Rightarrow |I| \leq |I_2|\leq \sum^\infty_{i=1}|I_i|$\\
Reiterando trovo l'aperto contenente $a_1$, se questo contiene anche $b$ mi fermo sennò continuo.\\
abbiamo quindi rinumerato $I_1,\ldots,I_n$ in modo che  $a_{i+1} < b_i < b_{i+1} \ \ \ \forall 1\leq i\leq n$\\
 $ \sum^n_{i=1}|I| = \sum^n_{i=1}b_i -a_i = b_1-a_1 + \ldots + b_n-a_n$\\
 notiamo che $b_1 > a_2$ quindi $b_1 - a_2 > 0$, procedendo così per ogni coppia otteniamo
  \[
 \geq b_n - a_1\geq b - a = |I|
 .\] 
 2) caso $I$ limitato, $I_i$ limitati\\
 $\forall\varepsilon > 0 \exists I^\varepsilon$ chiuso, $I^\varepsilon \subset I$ tale che  $ |I^\varepsilon|  = (1-\varepsilon)|I|$\\
  $\forall i \ \ \exists I^\varepsilon_i$ aperto tale che  $I_i\subset I^\varepsilon_i$ e $| \sum^\varepsilon_{i}| = (1-\varepsilon)|I_i|$\\
  $I^\varepsilon\subset I\subset\bigcup^\infty_{i = 1} I_i\subseteq\cup^\infty_{i=1}I^\varepsilon_i$\\
  $I_i = \frac{1}{1-\varepsilon}|I^\varepsilon| \leq \sum^\infty_{i=1}|I^\varepsilon| = \frac{1 + \varepsilon}{1-\varepsilon} \sum^\infty_{i=1}|I_i|$\\
  Quindi $|I|\leq \sum^\infty_{i=1}|I_i|$\\
  3) caso $I$ illimitato, $I_i$ limitati $n\in\Z$\\
  $n\in\Z$ \ \ $ I\cap [n, n + 1)\subseteq \bigcup^ \infty_{n=1}(I_i\cap[n,n+1))$\\
  Quindi ho un intervallo limitato coperto da intervalli limitati\\
  per il 2 caso
   \[
  |I\cap[n,n+1)|\leq \sum^\infty_{n=1} |I\cap[n,n+1)|
  \] 
  Per la 4)
  \[
  \sum_{n\in\Z} |I\cap[n,n+1)|\leq \sum_{n\in\Z} \sum^\infty_{i=1}|I_i\cap[n,n+1)|
  \]
  \[
	  = \sum^\infty_{i=1} \sum_{n\in\Z} |I_i\cap[n,n+1)| 
  \] 
  \[
   \sum^\infty_{i=1} |I_i|
  .\] 
\begin{enumerate}
	\item[6.] numerabile additività\\
		$I = \bigcup^\infty_{i=1} I_i, \ \ I_i\cap I_j = \emptyset\ \ \ \forall i\neq j$ \\
		 $ \Rightarrow |I| = \sum^\infty_{i=1}|I_i|$ \\
\end{enumerate}
\begin{dimo}
	$|I| \leq \sum^\infty_{i=1}|I_i|$ vero per la 5)\\
	Ci basta dimostrare l'altro verso della disuguaglianza\\
	 se $I$ limitato,  (con estremi $a<b$)\\
	  $\forall k\geq 1$ consideriamo  $I_1,I_2,\ldots,I_k$ sono contenuti in $I$ e disgiunti\\
	  questi possono essere rinumerati in modo che $a_1 < b_1\leq a_2 < b_2 \leq\ldots\leq a_k < b_k$\\
	  $ \sum^k_{i=1}\leq b -a $\\
	  $ \sum^k_{i=1}|I_i| = \sum^k_{i=1}(b_i - a_i)$ \\
	  $= b_1 - a_1 + b_2 - a_2 + \ldots + b_k - a_k\leq b_k-a_1 \leq b - a= |I|$\\
	  per lo stesso ragionamento di prima, possiamo formare coppie positive\\
	  \[
	  |I| \geq \sum^k_{i=1} |I_i| \ \ \forall k\geq 1
	  .\] 
	  \[
	   \Rightarrow |I|\geq \sum^\infty_{i=1}|I_i|
	  .\] 
	  Se $I$ illimitato\\
	  $I = \bigcup^\infty_{i=1}I_i, I_i\cap I_j = \emptyset $ se $i\neq j$\\
	  $I\cap [n,n+1) = \bigcup^\infty_{i=1}I_i\cap[n,n+1)$\\
	   $ \Rightarrow |I\cap[n,n+1)| = \sum^\infty_{i=1}|I_i\cap[n,n+1)|$ \\
	   $ \Rightarrow \sum_{n\in\Z}|I\cap[n,n+1)| = \sum_{n\in\Z} \sum^\infty_{i=1}|I_i\cap[n,n+1)| = \sum^\infty_{i=1}|I_i|$\\
\end{dimo}
	   \begin{itemize}
	   \item[7.]	$I$ intervallo, $x\in\R$\\
		    $I + x$ traslato di $I$ \\
		    $|I+x| = |I|$\\
		    (invarianza per traslazioni)
	   \end{itemize}
	   \begin{defi}[Misura esterna]
	Sia $E\subseteq \R$\\
	Si definisce misura ( esterna) di Lebesgue di $E$ 
	\[
		m(E) = \inf\lbrace\sum ^\infty_{i=1} |I_i|\  :\ E\subseteq \bigcup^\infty_{i=1} I_i, I_i \text { intervalli}\rbrace
	.\] 
	$M:P(\R) = 2^\R \rightarrow [0,+\infty]$
\end{defi}
\textbf{Osservazione}\\
Se $D\subset \E$ è un insieme nullo:
\[
	\forall \varepsilon >0\ \ \exists \{I_i\} \text{ successione di intervalli tale che }  D\subseteq\bigcup^\infty_{i=1}I_i \text { e } \sum^\infty_{i=1}|I_i| < \varepsilon
.\]
\[
	\Leftrightarrow m(D) = 0
.\] 
Ricordiamo che tra gli insiemi di misura nulla, ci sono gli insiemi numerabili\\
2) Per definire $m$ si usano ricoprimenti numerabili ma anche i ricoprimenti finiti sono ammessi,
\[ E\subseteq \bigcup^n_{i=1}I_i = \bigcup^n_{i=1}I_i\cup\bigcup^\infty_{i = n+1}\emptyset\]
C'è una differenza enorme tra considerare ricoprimenti finiti o ricoprimenti numerabili\\
$E\subseteq \R$\\
$\inf\{ \sum^\infty_{i=1}, E\subseteq\bigcup^\infty_{i=1}I_i \text { intervallo}\}\leq \inf\{ \sum^n_{i=1}|I_i|, E\subseteq\bigcup^n_{i=1}I_i, I_i\ \text{intervalli}\}$\\
La disuguaglianza può essere stretta\\
\textbf{Esempio}\\
$E = \Q\cap[0,1]$\\
è numerabile  $ \Rightarrow m(E) = 0$ \\
Sia $\{I_1,\ldots,I_n\}$ ricoprimento finito di $E$ con intervalli\\
$\displaystyle E  = \Q\cap[0,1]\subseteq \bigcup^n_{i=1}I_i \Rightarrow \sum^n_{i=1}|I_i|\geq 1$ \\
Infatti\\
$R = \Q\cap[0,1]\subseteq\bigcup^n_{i=1}I_i \Rightarrow [0,1]\subseteq^n_{i=1}I_i$ \\
$ \Rightarrow [0,1]=\overline{\Q\cap[0,1]}$ \\
$\displaystyle\leq (\bigcup^n_{i=1})\leq \bigcup^n_{i=1}\overline{I_i}$
$ \Rightarrow 1 = |[0,1]| \leq \sum^n_{i=1} |\overline I_i| = \sum^n_{i=1}|I_i|$ \\
$ \Rightarrow \inf\{ \sum^n_{i=1}|I_i|, E\subseteq \bigcup_{i=1}^n I_i, I_i $ intervallo $\} \geq 1$ ma  $E\subseteq [0,1]$\\
$ \Rightarrow \inf\{\sum^n_{i=1}|I_i|, E\subseteq \bigcup_{i=1}^n I_i\}\leq 1$ \\
Se avessi ricoprimenti finiti $\Q$ avrebbe misura 1, e questo non ci piace perché è un insieme numerabile, questo è il motivo per cui ammetto ricoprimenti infiniti.


\end{document}
