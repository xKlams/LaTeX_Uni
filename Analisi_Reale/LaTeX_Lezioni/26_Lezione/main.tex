\documentclass[12px]{article}

\title{Lezione 26 Analisi Reale}
\date{2025-05-28}
\author{Federico De Sisti}

\usepackage{amsmath}
\usepackage{amsthm}
\usepackage{mdframed}
\usepackage{amssymb}
\usepackage{nicematrix}
\usepackage{amsfonts}
\usepackage{tcolorbox}
\tcbuselibrary{theorems}
\usepackage{xcolor}
\usepackage{cancel}

\newtheoremstyle{break}
  {1px}{1px}%
  {\itshape}{}%
  {\bfseries}{}%
  {\newline}{}%
\theoremstyle{break}
\newtheorem{theo}{Teorema}
\theoremstyle{break}
\newtheorem{lemma}{Lemma}
\theoremstyle{break}
\newtheorem{defin}{Definizione}
\theoremstyle{break}
\newtheorem{propo}{Proposizione}
\theoremstyle{break}
\newtheorem*{dimo}{Dimostrazione}
\theoremstyle{break}
\newtheorem*{es}{Esempio}

\newenvironment{dimo}
  {\begin{dimostrazione}}
  {\hfill\square\end{dimostrazione}}

\newenvironment{teo}
{\begin{mdframed}[linecolor=red, backgroundcolor=red!10]\begin{theo}}
  {\end{theo}\end{mdframed}}

\newenvironment{nome}
{\begin{mdframed}[linecolor=green, backgroundcolor=green!10]\begin{nomen}}
  {\end{nomen}\end{mdframed}}

\newenvironment{prop}
{\begin{mdframed}[linecolor=red, backgroundcolor=red!10]\begin{propo}}
  {\end{propo}\end{mdframed}}

\newenvironment{defi}
{\begin{mdframed}[linecolor=orange, backgroundcolor=orange!10]\begin{defin}}
  {\end{defin}\end{mdframed}}

\newenvironment{lemm}
{\begin{mdframed}[linecolor=red, backgroundcolor=red!10]\begin{lemma}}
  {\end{lemma}\end{mdframed}}

\newcommand{\icol}[1]{% inline column vector
  \left(\begin{smallmatrix}#1\end{smallmatrix}\right)%
}

\newcommand{\irow}[1]{% inline row vector
  \begin{smallmatrix}(#1)\end{smallmatrix}%
}

\newcommand{\matrice}[1]{% inline column vector
  \begin{pmatrix}#1\end{pmatrix}%
}

\newcommand{\C}{\mathbb{C}}
\newcommand{\K}{\mathbb{K}}
\newcommand{\R}{\mathbb{R}}


\begin{document}
	\maketitle
	\newpage
	\subsection{Ricapitolando}
	\[
		\int_{X\times Y} \chi_P(x,y)d(\mu\times\nu) =\mu\times\nu(P) = \int_X\int_Y\chi_P(x,y)d\mu d\nu = \int_X\int_Y\chi_P d\nu d\mu
	.\] 
	In particolare se $P = A\times B \Rightarrow  \mu\times\nu(A\times B) = \mu(A)\cdot \nu(B)$ \\
	$\forall E \subseteq X\times Y\ \ \mu\times\nu(E) = \inf\{\mu\times\nu(P), P$ plurirettangolo,  $E\subseteq P\}$ \\
	$E\subseteq P \Rightarrow \mu\times\nu (E) \leq \mu\times\nu (P) \Rightarrow \mu\times\nu(E)\leq\inf\{\mu\times\nu(P), P\supseteq E\}$ \\
	Se $\displaystyle E\subseteq \bigcup^{+\infty}_{i=1}A_i\times B_i = P \Rightarrow  \sum^{+\infty}_{k=1}\mu(A_i)\nu(B_i)\geq  \mu\times\nu(P) \geq \inf\{\mu\times\nu(P), \ P\supseteq E\}$ \\
	$\mu\times\nu(E)\gqe \inf\{\mu\times\nu(P), P\supseteq E\}$\\
	$ \Rightarrow  P$ plurirettangolo $ \Rightarrow  P\in M_{\mu\times\nu}$ \\
	$\forall E\subseteq X\times Y\ \ \ \exists P_\infty = \bigcap^{+\infty}_{k=1}P_k, P_k$ plurirettangolo.\\
	tale che $P_\infty\supseteq E$ e  $\mu\times\nu(P_\infty) =\mu\times\nu(E)$ \\
	Quindi integrare secondo la misura prodotto è equivalente a integrare prima su una misura e poi sull'altra.\\
\begin{lemm}
	Sia $\{P_k\}$ una successione decrescente di plurirettangoli di misura  $\mu\times\nu$ finita.\\
	 $P_1\supseteq P_2\supseteq\ldots, \ \mu\times\nu(P_k) < +\infty\ \ \forall k\ $  e sia $P_\infty = \bigcap^{+\infty}_{k=1}P_k$\\
	 Allora $\forall y\in Y, \chi_{P_\infty}(\cdot, y)$ è $\mu$-misurabile,  $y \rightarrow \int_X \chi_{P_\infty}(x,y)d\mu$ è $\nu$-misurabile \\
	 $\forall x\in X, \chi_{P_\infty}(X,\cdot)$ è  $\nu$-misurabile, $x \rightarrow \int_X\chi_{P_\infty} (x,y)d\nu$ è $\mu$-misurabile.\\
	 Inoltre
	 \[
		 \mu\times\nu(P_\infty) = \int_Y\int_X\chi_{P_\infty}(x,y)d\mu d\nu = \int_X\int_Y \chi_{p_\infty}d\nu d\mu
	 .\] 
\end{lemm}
\textbf{Osservazione}\\
Quindi ciò che vale in generale per i plurirettangoli vale anche per il limite dell'intersezione.\\

\begin{dimo}
	$\displaystyle\mu\times\nu(P_\infty) = \lim_{ k \rightarrow +\infty}\mu\times\nu (P_k) = \lim_{k \rightarrow +\infty}\int_Y\int_X\chi_{P_k}(x,y)d\mu d\nu$\\
	Il primo passaggio è giustificato dal fatto che $\{P_k\}$ è una successione decrescente di misurabili di misura finita.\\
	 \[
		 \chi_{P_\infty}(x,y)  = \lim_{k \rightarrow +\infty} \chi_{p_k}(x,y)\ \ \forall x\in X, \ \forall y\in Y
	 .\] 
	 $\chi_{P_\infty}(\cdot, y)$ è  $\mu$-misurabile $\forall y\in Y$ \\
	 $\chi_{P_\infty}(x,\cdot)$ è  $\nu$-misurabile $\forall x\in X$\\
 Per passare al limite all'interno dell'integrale devo dimostrare che la prima funzione (la più grande) abbia integrale finito.\\
  \[
	  \mu\times\nu(P_1) = \int_Y\int_X\chi_{P_1}(x,y)d\mu d\nu < +\infty
 .\] 
 $ \Rightarrow  $ per quasi ogni $y$ \ \ $\displaystyle\int_X\chi_{P_1}(x,y)d\mu < +\infty$ \\
 \[
	 \lim_{k \rightarrow +\infty}\int_X\chi_{P_k}(x,y)d\mu = \int_X\chi_{P_\infty}(x,y)d\mu
 .\] 
 per $\nu$-quasi ovunque $y\in Y$ \\
 \[
	 \psi_k(y) = \int_X\chi_{P_k}(x,y)d\mu
 .\] 
 $\psi_k(y) \xrightarrow{k \rightarrow +\infty} \psi(y) = \int_X \chi_{P_\infty}(x,y)d\mu$  per $\nu$ quasi ogni  $y$ \\
 $\displaystyle\lim_{k \rightarrow +\infty}\int_Y \psi_k(Y)d\nui = \int_Y\psi(y)d\nu\hfill $ per convergenza monotona
 \[
	 \int_Y\psi(y)d\nu =\int_Y\int_X \chi_{P_\infty}(x,y)d\mu d\nu
 .\] 
\end{dimo}
\begin{lemm}
	Sia $N\subseteq X\times Y$ tale che  $\mu\times\nu (N) = 0$\\
	 $ \Rightarrow $ per $\nu$-quasi ogni  $y\in Y$ fissato,  $\chi_N(\cdot, y)$ è  $\mu$-misurabile $y \rightarrow\int_X\chi_N(x,y)d\mu$ è $\nu$-misurabile (definita q.o.)\\
	 per $\mu$-quasi ogni $x\in X$ fissato,  $\chi_N(x,\cdot)$ è $\nu$-misurabile e $x \rightarrow \int_Y \chi_N (x,y)d\nu$  è $\mu$-misurabile (definita q.o.)\\
	 e inoltre $\int_Y\int_X\chi_{N(x,y)}d\mu d\nu = \int_X\int_Y\chi_N (x,y) d\nu d\mu = 0 = \mu\times\nu(N)$
 \end{lemm}
 \begin{dimo}
	 $\mu\times\nu(N) = 0 = \inf\{\mu\times\nu (P), P$ plurirettangolo,  $P\supseteq N\}$ \\
	 $\exists \{P_k\}$ successione di plurirettangoli con  $P_k\supseteq N$ tale che  $\mu\times\nu (P_k) < \frac 1k \ \ \forall k$\\
	 Eventualmente sostituendo
	  \[
		  P_k \ \ \text{con} \ \ P'_k = \bigcap^{k}_{j=1}P_j
	 .\]  
	 si può supporre $P_1\supseteq P_2\supseteq\ldots$\\
	 $\displaystyle N\subseteq P_k \ \ \forall k \Rightarrow  N\subseteq \bigcap^{+\infty}_{k-1}P_k = P_\infty$ con $\displaystyle\mu\times\nu (P_\infty) = 0= \int_Y\int_X \chi_{P_\infty}(x,y)d\mu d\nu = \int_X\int_Y\chi_{P_\infty}(x,y)d\nu d\mu$\\
 $\chi_N(x,y)\leq \chi_{P_\infty}(x,y)\hfill (N\subseteq P_\infty)$ \\
 per $y\in Y$ fissato
  \[
	  \chi_N(\cdot, y)\leq \chi_{P_\infty}(\cdot, y)
 .\] 
 \[
	 \int_Y\int_X \chi_{P_\infty}(x,y)d\mu d\nu \Rightarrow \int_X\chi_{P_\infty}(x,y)d\mu = 0
 .\]  Per $\nu$-quasi ogni $y\in Y$\\
  \[
	  \Rightarrow \chi_{P_\infty}(\cdot, y) =0 
 .\] 
 \[
  \Rightarrow  \chi_N(\cdot, y) =0 \ \ q.o.\ \ in \ \ X
 .\] 
 per $\nu$-quasi ogni $y\in Y$\\
  $ \Rightarrow $ per $\nu$-quasi ogni $y\in Y$,  $\chi_N(\cdot, y)$ è  $\mu$-misurabile ($ =0$ quasi ovunque)\\
  e  $\displaystyle\int_X\chi_N(x,y) d\mu = 0 $ per  $\nu$-quasi ogni $y\in Y$\\
   $\displaystyle \Rightarrow  \int_Y(\int_X\chi_N(x,y)d\mu)d\nu = 0$
 
 \end{dimo}
 \begin{prop}
 	Sia $E\subseteq X\times Y$ tale che  $E$ è  $\mu\times\nu$-misurabile e  $\mu\times\nu(E) < +\infty$\\
	Allora:
	\begin{itemize}
		\item 	per $\nu$-q.o. $y\in Y$ fissato,  $\chi_E(\cdot,y)$ è  $\mu$-misurabile e $y \rightarrow \int_X\chi_E(x,y)d\mu$ è $\nu$-misurabile
		\item per $\mu$-q.o. $x\in X$ fissato,  $\chi_E(x,\cdot)$ è $\nu$-misurabile e $ x \rightarrow \int_Y\chi_E(x,y)d\nu$ è  $\mu$-misurabile
	\end{itemize}
	Inoltre $\displaystyle\int_Y\int_X\chi_E(x,y)d\mu d\nu = \int_X\int_Y\chi_E(x,y)d\nu d\mu \\= \mu\times\nu(E) = \int_{X\times Y}\chi_E(x,y)d(\mu\times\nu)$
 \end{prop}
\begin{dimo}
	$\mu\times\nu(E) = \inf\{\mu\times\nu(P), E\subseteq P, P$ plurirettangolo$\}$\\
	$ \Rightarrow  \exists P_\infty = \bigcap^{+\infty}_{k=1}P_k$, $\{P_k\}$ pulrirettangi  $P_k\subseteq P_{k+1}$\\
	tale che  $E\subseteq P_{\infty}$ e  $\mu\times\nu(E) = \mu\times\nu(P_\infty)< +\infty$\\
	 $E \ \ \mu\times\nu$-misurabile  $ \Rightarrow  \mu\times\nu (P_\infty\setminus E) =\mu\times\nu (P_\infty) - \mu\times\nu(E) = 0$ \\
	 $ \Rightarrow  \chi_E(x,y) = \chi_{P_\infty} - \chi_{P_\infty\setminus E}(x,y)$ \\
	 La prima verifica la tesi per lemma $1$ e la seconda parte verifica il la tesi per lemma $2$
\end{dimo}
\begin{defi}
	Uno spazio di misura $(X,\mu)$ si dice  $\sigma$-finita se $X = \bigcup^{+\infty}_{i=1}X_i$ con $X_i$ misurabili, $X_i\cap X_j = \emptyset \ \ \ $ se  $i\neq j$ e  $\mu(X_i) < \infty \ \forall i$
\end{defi}
\begin{teo}[Tonelli]
	Siano $(X,\mu), (Y,\nu)$ spazi di misura  $\sigma$-finiti, sia $f: X \times Y \rightarrow [0,+\infty)$\\ $\mu\times\nu$-misurabile.\\
	Allora:
	\begin{enumerate}
		\item per $\nu$-q.o. $y\in Y, \ f(\cdot ,y)$ è  $\mu$-misurabile e \\
			$y \rightarrow \int_X f(x,y)d\mu$ è $\nu$-misurabile
		\item per $\mu$-q.o, $\xin X$, $f)x,\cdot)$ è $\nu$-misurabile e\\
			$x \rightarrow\int_Y f(x,y)d\nu$  è $\mu$-misurabile
	\end{enumerate}
	Inoltre 
	 \[
	\int_{X\times Y}f(x,y)d(\mu\times\nu) = \int_Y\int_X f(x,y)d\mu d \nu = \int_X\int_Y f(x,y)d\nu d\mu
	.\] 
\end{teo}
\begin{dimo}
Primo caso: $\mu(X) < +\infty, \mu(Y) < +\infty$\\
$f \geq 0, $ misurabile  $ \Rightarrow  \exists \{s_k\}$ successione di funzioni semplici $s_k(x,y) \xrightarrow{ k \rightarrow +\infty} f(x,y) \ \ \forall (x,y)\in X\times Y \ \ s_k\leq s_{k+1}\ \ \forall k$\\
\[
	\int_{X\times Y}f(x,y) d (\mu\times\nu) = \lim_{ k \rightarrow+\infty} \int_{X\times Y}s_k (x,y)d (\mu\times\nu)
.\] 
verifica la tesi perché combinazione lineare finita di funzioni caratteristiche.\\
secondo caso $X,Y\ \ \sigma$-finiti\\
$X = \bigcup^{+\infty}_{i=1}X_i,\ \ \mu(X_i) < + \infty \ \forall i$
$Y = \bigcup^{+\infty}_{j=1}Y_j,\ \ \mu(Y_j) < + \infty \ \forall j$\\
$f(x,y) = \sum^{+\infty}_{i,j=1}f(x,y)\chi_{X_i}(x)\chi_{Y_j}(y)$\\
$= \sum^{+\infty}_{i,j=1}f(x,y)\chi_{X_i\times Y_j}(x,y)$\\
Quindi è combinazione di funzioni che verificano la tesi, quindi anche $f$ verifica la tesi

\end{dimo}
\begin{teo}[Fubini]
	Siano $(X,\mu), (Y,\nu)$ spazi di misura  $\sigma$-finiti\\
	Se $f\in L^1(X\times Y, \mu\times \nu)$\\
	 $ \Rightarrow $ stessa tesi del teorema di Tonelli 
\end{teo}
\begin{dimo}
	$f = f^+ - f^-$ si applica teorema di Tonelli a  $f^+$ e $f^-$
\end{dimo}
\textbf{Osservazione}\\
Non è vero per funzioni non in $L^1$ oppure non non negative.
	
\end{document}
