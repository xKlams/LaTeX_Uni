\documentclass[12px]{article}

\title{Lezione 25 Analisi Realae}
\date{2025-05-27}
\author{Federico De Sisti}

\usepackage{amsmath}
\usepackage{amsthm}
\usepackage{mdframed}
\usepackage{amssymb}
\usepackage{nicematrix}
\usepackage{amsfonts}
\usepackage{tcolorbox}
\tcbuselibrary{theorems}
\usepackage{xcolor}
\usepackage{cancel}

\newtheoremstyle{break}
  {1px}{1px}%
  {\itshape}{}%
  {\bfseries}{}%
  {\newline}{}%
\theoremstyle{break}
\newtheorem{theo}{Teorema}
\theoremstyle{break}
\newtheorem{lemma}{Lemma}
\theoremstyle{break}
\newtheorem{defin}{Definizione}
\theoremstyle{break}
\newtheorem{propo}{Proposizione}
\theoremstyle{break}
\newtheorem*{dimo}{Dimostrazione}
\theoremstyle{break}
\newtheorem*{es}{Esempio}

\newenvironment{dimo}
  {\begin{dimostrazione}}
  {\hfill\square\end{dimostrazione}}

\newenvironment{teo}
{\begin{mdframed}[linecolor=red, backgroundcolor=red!10]\begin{theo}}
  {\end{theo}\end{mdframed}}

\newenvironment{nome}
{\begin{mdframed}[linecolor=green, backgroundcolor=green!10]\begin{nomen}}
  {\end{nomen}\end{mdframed}}

\newenvironment{prop}
{\begin{mdframed}[linecolor=red, backgroundcolor=red!10]\begin{propo}}
  {\end{propo}\end{mdframed}}

\newenvironment{defi}
{\begin{mdframed}[linecolor=orange, backgroundcolor=orange!10]\begin{defin}}
  {\end{defin}\end{mdframed}}

\newenvironment{lemm}
{\begin{mdframed}[linecolor=red, backgroundcolor=red!10]\begin{lemma}}
  {\end{lemma}\end{mdframed}}

\newcommand{\icol}[1]{% inline column vector
  \left(\begin{smallmatrix}#1\end{smallmatrix}\right)%
}

\newcommand{\irow}[1]{% inline row vector
  \begin{smallmatrix}(#1)\end{smallmatrix}%
}

\newcommand{\matrice}[1]{% inline column vector
  \begin{pmatrix}#1\end{pmatrix}%
}

\newcommand{\C}{\mathbb{C}}
\newcommand{\K}{\mathbb{K}}
\newcommand{\R}{\mathbb{R}}


\begin{document}
	\maketitle
	\newpage
	\subsection{Prodotto di spazi di Misura}
$(X,\mu), (Y,\nu)$ spazi di misura ( $\mu, \nu$ misure esterne)\\
 $M_\mu$ misurabili in  $X$ rispetto a $\mu$\\
 $M_\nu$ misurabili in  $Y$ rispetto a $\nu$ \\
Vogliamo definire una misura in $X\times Y$\\
che sia il prodotto delle due misure  $\mu\times \nu$\\
ovvero si vuole che:\\
se  $A\in M_\mu, B\in M_\nu$\\
 $\mu\times\nu (A\times B)  = \mu(A)\nu(B)$\\
 \textbf{Nota}\\
 Non ci basta questo perché non tutti gli insiemi in  $X\times Y$ sono di forma  $A\times B$. Si può pensare ad una circonderenza in  $\R\times \R$, che non è il prodotto delle proiezioni. \\
 \begin{defi}
 	La misura prodotto $\mu\times\nu$ su $X\times Y$ è definita da: $\forall E\subseteq X\times Y \ $
	 \[
		 \mu\times\nu(E) = \inf\left\{ \sum^{+\infty}_{i=1}\mu(A_i)\nu(B_i), \ A_i\in M_\mu, \ B_i\in M_\nu, \ E\subseteq \bigcup^{+\infty}_{i = 1}A_i\times B_i\right\}
	.\] 
	Se $A\in M_\mu, B\in M_\nu$  $ \Rightarrow R= A\times B$ rettangolo (misurabile) e $ \bigcup^{+\infty}_{i =1}A_i\times B_i$ si dice plurirettangolo.
 \end{defi}
 \textbf{Osservazione}\\
  \begin{enumerate}
	  \item $\mu\times\nu$ è una misura (esterna), infatti:
		   \begin{itemize}
			   \item $\mu \times \nu : X\times Y \rightarrow [0,+\infty]$ 
			   \item $0\leq \mu\times\nu (\emptyset) \leq \sum^{}_{i}\mu(\emptyset)\nu(\emptyset) =0$
			   \item Se $E\subseteq \sum^{+\infty}_{j = 1}E_j$\\
				   Se $\exists j$ tale che  $\mu\times\nu(E_j) = +\infty \Rightarrow \mu\times \nu(E)\leq \sum^{+\infty}_{j = 1}\mu\times\nu(E_j)$ \\
				   Se $\mu\times\nu(E_j) < +\infty \ \ \forall j$\\
				   $ \Rightarrow  \forall j, \forall \e$ \ $\exists \{A_i^j\times B_i^j\}_i$\\
				   tale che  $E_j\subseteq \bigcup^{+\infty}_{i= 1}A_i^j\times B_i^j$\\
				   $\mu\times\nu(E_j) \leq \sum^{+\infty}_{i=1}\mu(A^j_i)\nu(B_i^j) < \mu\times \nu(E_j + \frac{\e}2^i$\\
				   $E\subseteq \bigcup^{+\infty}_{j = 1}E_j\subseteq \bigcup^{+\infty}_{j=1} \bigcup^{+\infty}_{i = 1}A_i^j\times B^j_i$ è un plurirettangolo\\
				   $ \Rightarrow \mu\times\nu(E)\leq \sum^{+\infty}_{jj=1} \sum^{+\infty}_{u = 1}\mu(A_i^j)\times\nu(B_i^j)$\\
				   $\leq \sum^{+\infty}_{j=1}(\mu\times\nu(E_j) + \frac{\e}{2^j})$\\
				   $= \sum^{+\infty}_{j=1}\mu\times\nu(E_j) + \e$ per $\e \rightarrow 0$\\
				   $\mu\times\nu(E)\leq \sum^{+\infty}_{j=1}\mu\times\nu(E_j)$
		  \end{itemize}
	  \item La famiglia dei plurirettangoli è chiusa per unioni numerabili e per intersezioni finite, siano $A_i,B_i,C_i,D_i$ misurabili rispettivamente per  $\mu$ e per  $\nu\ \ \forall i$
		  \[
		  P = \bigcup^{+\infty}_{i=1}A_i\times B_i, \ Q = \bigcup^{+\infty}_{j=1}C_j\times D_j
		  .\] 
		  \[
		  P\cap Q = \bigcup^{+\infty}_{i=1} \bigcup^{+\infty}_{j =1 }A_i\times B_i\cap C_j\times D_j  = \bigcup^{+\infty}_{i=1} \bigcup^{+\infty}_{j=1}(A_i\cap C_j)\times (B_i\cap D_j)
		  	.\] 
		\item La differenza di due rettangoli è un plurirettangolo\\
			$R = A\times B, \ \ S= C\times D$\\
			 $R\setminus S = A\setminus C\times B \cup A\cap C\times V\setminus D$
		 \item Ogni plurirettangolo si può scrivere come unione numerabile di rettangoli disgiunti.
			  $P = \bigcup^{+\infty}_{i=1}A_i\times B_i = A_1\times B_1 \cup (A_2\times B_2\setminus A_1\times B_1) \cup A_3\times B_3\setminus \bigcup^{2}_{i=1}A_i\times B_i)$\\
			  $\displaystyle P = A_1\times B_1 \cup\bigcup^{+\infty}_{i=2}(A_i\times B_i \setminus \bigcup^{i-1}_{j=1}A_j\setminus B_j)$
 \end{enumerate}
 $P$ plurirettangolo\\
  $P = \bigcup^{+\infty}_{i=1}A_i\times B_i$ \\
  $\chi_P(x,y)$\\
   $\forall y\in Y$ fissato\\
    $x\in X \rightarrow \chi_P(x,y) = \begin{cases}
    	1 \ \ se \ \ (x,y)\in P\\
    	0 \ \ se \ \ (x,y)\not\in P
    \end{cases} = \begin{cases}
    	0 \ \ se \ \ y\not\in \bigcup^{+\infty}_{j = 1}B_j\\
	1 \ \ se \ \ x\in A_i\ \ \forall i\ \ \text{tale che } y\in B_i
    \end{cases}$\\
    $ = \begin{cases}
    	0 \ \ se \ \ y\not\in \bigcup^{+\infty}_{j = 1}B_j\\
	\chi_{ \bigcup^{}_{\substack{i\\ y\in B_i}}A_i}(x)
    \end{cases}$\\
    $ \Rightarrow  \forall y\in Y \ \chi_P(\cdot,y)$ è $\mu$-misurabile\\
    Se $P = \bigcup^{+\infty}_{i=1}A_i'\times B_i'$ disgiunti\\
    $\forall y\not\in \bigcup^{}_{i}B_i' \Rightarrow  \chi_p(\cdot, y) \equiv 0$ \\
    Se $y\in \bigcup^{}_{i}B'_i \Rightarrow  \exists! i$  tale che $y\in B_i$ e $\chi_p(\cdot, y) = \chi_{A'_i}$\\
     $ \Rightarrow  \int_X \chi_p(x,y)d\mu = \begin{cases}
     	0\ \ se \ \ y\not\in \bigcup^{+\infty}_{i =1}B'_i\\
	\mu( \bigcup^{}_{\substack{i\\y\in B_i'}}A'_i)
\end{cases} = \sum^{+\infty}_{i=1}\mu(A'_i)\chi_{B'_i}(y)$ è $\nu$-misurabile\\
$y\in B_i'\cap B_j' \Rightarrow A'_i\cap A'_j = \emptyset$
\begin{prop}
	Se $P\subset X\times Y$ plurirettangolo\\
	 $ \Rightarrow  \mu\times\nu(P)= \int_Y \left(\int_X \chi_P(x,y)d\mu \right)d\nu = \int_X \left(\int_Y\chi_P(x,y)d\nu \right)d\mu$
\end{prop}
\begin{dimo}
	$P\subsetetq \bigcup^{+\infty}_{i=1}A_i\times B_i, A_i\in M_\mu, B_i\in M_\nu$\\
	$ \Rightarrow  \chi_P(x,y)\leq \chi_{ \bigcup^{+\infty}_{i=1}A_i\times B_i} (x,y)$ e sono funzioni caratteristiche di plurirettangoli\\
	$\displaystyle\int_Y \left(\int_X \chi_P(x,y) d\mu\right)d\nu\leq \int_Y \left(\int_X \chi_{ \bigcup^{+\infty}_{i=1}A_i\times B_i}(x,y)d\mu \right) d\nu$ \\
	$\displaystyle\leq \int_Y \left(\int_X \sum^{+\infty}_{i=1}\chi_{A_i\times B_i} (x,y)d\mu \right) d\nu$\\
	$\displaystyle = \sum^{+\infty}_{i=1}\int_Y \left(\int_X \chi_{A_i}(x)\chi_{B_i}(y)d\mu \right)d\nu$\\
	$\displaystyle = \sum^{+\infty}_{i=1}\mu(A_i)\nu(B_i)$ \\
	$\displaystyle \Rightarrow  \int_Y \left( \int_X \chi_P(x,y)d\mu)d\nu\leq \inf\{ \sum^{+\infty}_{i=1}\mu(A_i)\nu(B_i), P\subseteq \bigcup^{+\infty}_{i=1}A_i\times B_i\} = \mu\times \nu(P)$\\
		$\displaystyle P = \bigcup^{+\infty}_{i=1}A_i'\times B_i'$\\
$A_i'\times B_i'\cap A'_j\times B_j' = \emptyset$ se $i\neq j$ \\
$\displaystyle\int_Y \left(\int_X \chi_{ \bigcup^{+\infty}_{i=1}A'_i\times B_i'}d\mu \right)d\nu\\
=\int_Y \left(\int_X \sum^{+\infty}_{i=1}\chi_{A'_i\times B'_i}d\mu \right) d\nu\\
=\sum^{+\infty}_{i=1}\int_Y \left(\int_X \chi_{A'_i\times B'_i}d\mu \right)d\nu = \sum^{+\infty}_{i=1}\mu(A'_i)\nu(B'_i)\geq \mu\times\nu(P) $
\end{dimo}
\begin{lemm}
	Se $E\subseteq X\times Y$\\
	 \[
		 \mu\times\nu(E) = \inf \{\mu\times\nu (P), P \text{ plurirettangolo } E\subseteq P\}
	 .\] 
\end{lemm}
\begin{dimo}
	$E\subseteq P \Rightarrow  \mu\times\nu (E)\leq \mu\times \nu (P)$ \\
	$ \Rightarrow \mu\times\nu(E)\leq \mu\times \nu(P)$ \\
	$ \Rightarrow \mu\times\nu(E)\leq \inf\{\mu\times\nu(P), \ P\ \text{ plurirettangolo } E\subseteq P\}$ \\
	Se  $P = \bigcup^{+\infty}_{i=1}A_i'\times B_i'$ disgiunti\\
	$ \Rightarrow  \mu\times \nu (P) = \sum^{+\infty}_{i=1}\mu(A'_i)\nu(B_i')\geq \mu\times \nu(E)$
\end{dimo}
\begin{prop}
	$P\subseteq X\times Y$ plurirettangolo  $ \Rightarrow P$ è $\mu\times\nu$-misurabile
\end{prop}
\begin{dimo}
	Basta dimsotrare che $A\in M_\mu, B\in M_\nu \Rightarrow  A\times B\in M_{\mu\times\nu}$ \\
	Th.  $\forall E\subseteq X\setminus Y\ \ \mu\times\nu(E\cap A\times B) + \mu\times \nu (E\setminus A\times B) = \mu\times \nu(E)$\\
	Sia  $Q$ plurirettangolo, $Q\geq E$\\
	 \[
	 \mu\times \nu(Q\cap A\times B) + \mu\times\nu(Q\setminus A\times B)
	 .\] 
	 $\displaystyle = \int_Y \left(\int_X \chi_{Q\cap A\times B} (x,y)d\mu \right)d\nu + \int_Y \left(\int_X \chi_{Q\setminus A\times B} \right)d\nu$\\
	 $\displaystyle = \int_Y\int_X(\chi_{Q\cap A\times B} + \chi_{Q\setminus A\times B} )d\mu\dnu = \int_Y\int_X \chi_Qd\mu\dnu = \mu\times \nu (Q)$ \\
	 $\displaystyle\mu\times\nu (E\cap A\times B) + \mu\times \nu (E\setminus A\times B)$\\ Prendendo l'inf rispetto a  $Q$ \[
		 \Rightarrow  \mu\times\nu(E\cap A\times B) + \mu\times \nu (E\setminus A\times B) \leq\mu\times\nu(E) \Rightarrow  A\times B \text{ misurabile}
	 .\] 
\end{dimo}
\begin{prop}
	$\mu\times \nu$ è regolare, ovvero  $\forall E\subset X\times Y\ \ \exists F\in M_{\mu\times\nu}$ tale che  $E\subseteq F$ e  $\mu\times\nu(E) = \mu\times\nu(F)  \Rightarrow  A\times B$ misurabile.
\end{prop}
\begin{dimo}
	per esercizio.
\end{dimo}

	
\end{document}
