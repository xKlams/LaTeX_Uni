\documentclass[12px]{article}

\title{Lezione 20 Analisi Reale}
\date{2025-05-13}
\author{Federico De Sisti}

\input{../../../setup.tex}

\begin{document}
	\maketitle
	\newpage
	\subsection{Funzioni a supporto compatto}
	\begin{defi}
	Sia $f: \R \rightarrow \R$ si dice a  supporto compatto se $f$ è nulla fuori di un compatto\\
	ovvero:\\
	$\exists K$ compatto tale che  $supp(f)\subset K \Leftrightarrow f=0 $ in $\R$
	 \[
	C_c(\R)\subset L^p(\R) \ \ \forall 1\leq p < +\infty
	.\] 
infatti se $f\in C_c(\R) \Rightarrow  \exists K$ compatto tale che $f =0 $ in  $\R\setminus L \Rightarrow  f$  limitata in $K, $  $f$ è limitata in $\R\setminus K \Rightarrow  f\in L^\infty(\R)$ \\
inoltre $\int_\R|f|^p d m = \int_K|f|^pd m\leq \|f\|_\infty^p m(K) < +\infty $
	\end{defi}
	\begin{teo}
		$C_c(\R)$ è denso in  $L^p(\R)\ \ \forall 1\leq p < +\infty$ \\
		ovvero  $\forall f\in L^p(\R), \forall \e> 0 \ \ \exists g\in C_c(\R)$ tale che  $\|f-g\|_p< \e$
	\end{teo}
	\begin{dimo}
		$f\in L^p(\R), \ \e > 0 $\\
		$f_n = f\chi_{[-n,n]}$ è a supporto compatto\\
		 $f_n(x) \rightarrow f(x)$ quasi ovunque\\
		 $|f_n| \leq |f|$ in  $\R \Rightarrow  f_n \rightarrow f$ in $L^p$\\
		 quindi  $\exists  f_1\in L^p(\R)$, con supporto compatto tale che $\|f-f_1\|_p < \frac \e 3$ \\
		 $T_j(f_1)(x) = \begin{cases}
			 f_1 (x)\ \ \ \text{ se } |f_1| < j\\
			 j \ \ \ \text{ se } f_1(x) > j\\
			 -j \ \ \ \text{ se } f_1(x) < -j
		 \end{cases}$\\
		 $|T_j(f_1)|\leq j \Rightarrow  T_j (f_1)\in L^\infty (\R)\cap L^p(\R)$ \\
		 $T_j(f_1) \rightarrow f_1$ in $L^p(\R)$ convergenza dominata.\\
		 primo passo  $\exists f_1\in L^p(\R)$ supporto compatto tale che $\|f-f_1\|_p < \frac \e 3$\\
		 secondo passo $\exists f_2 \in L^\infty(\R)$ supporto compatto tale che $\|f_2- f_1\|_p \leq \frac\e 3$\\
		 terzo passo: supp$f_2\subset [-k,k]$\\
		 dal teorema di Lusin (e dalla sua dimostrazione)\\
		 $\forall \delta > 0 \exists g_\delta : [-k,k] \rightarrow \R$ continua\\
	 tale che $m(\{f_2\neq g_\delta\}) < \delta$\\
	 $\sup_{[-k,k}|g_\delta|\leq \|f_k\|_\infty$ e  $g_\delta(k) = g_\delta(-k) = 0$\\
	 estenendo a zero $g_\delta$ fuori di  $[-k,k]$ si ottiene  $g\in C_1(\R)$\\
	 tale che $\|f_2 - g\|_p^p = \int_\R |f_2-g|^pd m = \int_{[-k,k]\cap\{f_2\neq g\}}|f_2-g|^pd m \leq (\|g\|_\infty + \|f_2\|_\infty)^p m (\{f_2\neq g\}) \leq 2^p\|f_2\|^p < \left(\frac \e 3 \right)^p$ per $\delta$ sufficientemente piccolo.\\
	 Quindi  $\exists g\in C_0(\R)$ tale che 
	 \[
	 \|f-g\|_p\leq \|f-f_2\|_p + \|f_1-f\|_p + \|f_2-g\|_p < \e
	 .\] 

	\end{dimo}
	\textbf{Domanda:}\\
	Potrebbe esserci un risultato analogo in $L^\infty$?\\
	 \textbf{Osservazione}\\
	 Il risultato di densità si scrive in simboli\\
	 $\overline{C_c(\R)}^{\|\ \|_p} = L^p(\R)$ \  $1\leq p < +\infty$ \\
	 $\overline{C_c(\R)}^{\|\ \|_p} = C_0(\R) = \{f: \R \rightarrow \R: f$ continua e $\lim_{|x| \rightarrow +\infty} f(x) =0 \}$ \\
	 Dimostrazione per esercizio
	
	 \begin{defi}
	 	Uno spazio metrico $Y$\\
		si dice separabile se ammette un sottoinsieme denso numerabile (esempio  $\R$ è separabile, $R^n$ è separabile $\forall n \feq 1$)  \\
	 \end{defi}
	 \begin{teo}
		 $L^p(\R)$ è separabile per  $1\leq p < +\infty$\\
		  $L^\infty(\R)$ non è separabile
	 \end{teo}
	 \begin{dimo}
		 $D = \{s(x) = \sum^{N}_{i=1}q_ix_{(a_i,b_i)}$, $a_i < b_i, a_i,b_i,q_i\in \Q\}$ è numerabile\\
		 Sia $f\in L^p(\R)$ con  $1\leq p < +\infty$ e sia $\e > 0 $\\
		  $\exists g\in C_c(\R) $(continua a supporto compatto) tale che  $\|f-g\|_p < \frac \e 2$\\
		  sia  $k\in \N$ tale che  $supp \ g\subset (-k,k)$\\
		   $g$ è uniformemente continua \hfill(supporto compatto + continua)\\
		   $\forall \eta > 0 \exists \delta_\eta > 0 $ tale che 
		    \[
		   |g(x_1)-g(x_2)| <\eta \ \ \ \forall x_1,x_2\in \R , \ |x_1-x_2| < \delta_\eta
		   .\] 
		   suddividiamo $[-k,k]$ in sottointervalli di ampiezza  $\frac 1j < \delta_\eta$ mediante i punti  $-k + \frac i j$  $0\leq i \leq 2kj$A\\
		    $\forall 0 \leq i \leq 2k_j -1$\\
		    Scegliamo  $x\in (-k + \frac i j, -k + \frac { i + 1}j)$ e $q_i\in \Q$ tale che  $|q-g(x)| < \eta$\\
		    $s(x) = \sum^{2kj - 1}_{i = 0}q_i\chi_{(-k + \frac  i j, -k + \frac {i+1}j)}(x)\in D$\\
		    $\displaystyle\|g-s\|^p_p = \int_{[-k,k]}|g-s|^pd m = \sum^{2kj -1}_{i=0}\int_{[-k+\frac ij, -k + \frac {i+1}j]}|g-q_i|^p$ \\
		    $\exists x\in [2k + \frac ij, -k + \frac {i+1}j]$ tale che  $|g(x) - q_i| < \eta$\\
		    $\forall g\in [-k + \frac ij, -k + \frac {i + 1}j] \ \ |y-x | < \frac 1 j < \delta_\eta$\\
		     $ \Rightarrow  |g(y)-q_i| \leq |g(y)-g(x)| +|g(x)-q_1| < 2\eta$ \\
		     Tornando quindi all'equazione precedente

		    $\displaystyle\|g-s\|^p_p = \int_{[-k,k]}|g-s|^pd m = \\\sum^{2kj -1}_{i=0}\int_{[-k+\frac ij, -k + \frac {i+1}j]}|g-q_i|^p\ \leq (2\eta)^p\frac ij 2kj < \left(\frac e 2 \right)^p$  per $\eta$ sufficientemente piccolo.
			    
	 \end{dimo}
	 \textbf{Osservazione}\\
	 $L^\infty(\R)$ non è separabile, sia  $\{\omega_r\}_r$ famiglia più che numerabile di sottoinsiemi tali che  $\omga_r\subset \R, \ \omega_r\in\eta$ per  $r\neq s$\\
	  $m(\omega_r\setminus \omega_s)$ oppure  $m(\omega_s\setminus \omega_r) > 0 $ \\
	  esempio\\
	   $\omega_r = (-r, r) \ \ r > 0 $\\
	   se  $r < s$  $m(\omega_s\setminus\omega_r) = ([(-s,-r] \cup [r,s)) > 0 $ \\
	   $\{\chi_{\omega_r}\}_{r > 0 }\subset L^\infty(\R)$\\
	   $\|\chi_{\omega_r} - \chi_{\omega_s}\|_\infty  = 1$\\
   $\{B_{\frac 12}(\chi_{\omega_r})\}$ sono disgiunte in  $L^\infty (\R)$\\
   Sia  $D$ denso in $L^\infty(\R)$\\
   $\forall r \ \ \exists f_r\in D\cap B_{\frac 12} (\chi_{\omega_r})$ se  $r\neq s \Rightarrow f_s \neq f_r$ \\
   $ \Rightarrow  D $ è più che misurabile
   $(V_0, \|\ \|_{V_1}), (V_2, \|\ \|_{V_2})$  spazi vettoriali normati\\
   $L: V_1 \rightarrow V_2$ operatore lineare.\\
   \begin{defi}
	   $L$ si dice limitato se $\exists C \geq 0$ tale che  $\| L(v)\|_{V_2} \leq C\|v\|_{V_1} \ \ \ \forall v\in V_1$
   \end{defi}
   \begin{teo}
   	Sia $L$ operatore lineare $L$ è limitata $ \Leftrightarrow$ L è continua.
   \end{teo}
   \begin{dimo}
	   $ ( \Rightarrow )$ Sia $L$ limitato $\forall v_1,v_2\in V_1\ \ \|L(v_1)- L(v_2)\|_{V_2} = \leq L(v_1+v_2)\|_{V_2} \leq C\|v_1 - v_2\|_{V_1} \Rightarrow  L $  è lipschitziano.\\
	   $ ( \Leftarrow )$ sia  $L$ continuo.\\ 
	   Poiché $L(0) = 0\ \ \ \forall \e > 0 \exists\delta > 0$ tale che  $\|L(V)\|_{V_2} < \e$ se $\|v\|_{V_1} < \delta$\\
   $\frac \delta 2 \frac {v}{\|v\|_{V_1}}\in V_1$ e $\|\frac \delta 2\frac {v}{\|v\|_{V_1}} = \frac \delta 2 < \delta$\\
   $ \Rightarrow  \|L(\frac\delta 2 \frac v {\|v\|_{V_1}})\|_{V_2} < \e$ \\
   $\frac \delta {2\|v\|_{V_1}}\|L(V)\|_{V_2} < \e$\\
   $ \Rightarrow  \|L(v)\|_{V_2} < \frac {2\e} \delta\|v\|_{V_1} \ \ \forall v\in V_1$\\
   $ \Rightarrow  L $ è limitato
   \end{dimo}
   $L(V_1,V_2) = \{L:V_1 \rightarrow V_2\ : \ L$ lineare e continuo$\}$\\
   spazio vettoriale\\
   $L\in L(V_1,V_2)$\\
   $ \Rightarrow  \|L(v)\|_{V_2}\leq C\|v\|_{V_1}$ \\
   $ \Rightarrow  \frac {\|L(v)\|_{V_2}}{\|v\|_{V_1}}\leq C \ \ \forall v\in V_1\setminus\{0\}$\\
   $ \Rightarrow  \sup_{v\in V_1\setminus\{0\}}\frac {\|L(v)\|_{V_2}}{\|v\|_{V_1}} < +\infty$ \\
   $\|L\| = \sup_{v\in V_1\setminus\{0\}}\frac {\|L(v)\|_{V_2}}{\|v\|_{V_1}}$\\
   $=\sup_{v\in V_1\setminus\{0\}}\|L(\frac v {\|v\|_{V_1}})\|_{V_2} = \sup_{\substack{v\in V_1\\ \|v\|_{V_1} = 1}}\|L(v)\|_{V_2}$
   \begin{teo}
   $(L(V_1,V_2), \|\ \|)$ è uno spazio vettoriale normato \\
   se $(V_2, \|\ \|_{V_2})$ è di Banach $ \Rightarrow  (L(V_1,V_2), \|\ \|)$ è di Banach.
   \end{teo}
   \begin{dimo}
   	per esercizio
   \end{dimo}
   In particolare $L(V_1, \R) = \{L : V_1 \rightarrow \R$ lineare continua $\}$ è uno spazio di Banach\\
   $V_1' = V_1^*$ spazio dueale di $V_1$\\
   Guarda foto 13 maggio 3 03
\end{document}
