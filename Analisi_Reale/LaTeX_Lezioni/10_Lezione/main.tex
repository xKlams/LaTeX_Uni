\documentclass[12px]{article}

\title{Lezione 10 Analisi Reale}
\date{2025-03-25}
\author{Federico De Sisti}

\usepackage{amsmath}
\usepackage{amsthm}
\usepackage{mdframed}
\usepackage{amssymb}
\usepackage{nicematrix}
\usepackage{amsfonts}
\usepackage{tcolorbox}
\tcbuselibrary{theorems}
\usepackage{xcolor}
\usepackage{cancel}

\newtheoremstyle{break}
  {1px}{1px}%
  {\itshape}{}%
  {\bfseries}{}%
  {\newline}{}%
\theoremstyle{break}
\newtheorem{theo}{Teorema}
\theoremstyle{break}
\newtheorem{lemma}{Lemma}
\theoremstyle{break}
\newtheorem{defin}{Definizione}
\theoremstyle{break}
\newtheorem{propo}{Proposizione}
\theoremstyle{break}
\newtheorem*{dimo}{Dimostrazione}
\theoremstyle{break}
\newtheorem*{es}{Esempio}

\newenvironment{dimo}
  {\begin{dimostrazione}}
  {\hfill\square\end{dimostrazione}}

\newenvironment{teo}
{\begin{mdframed}[linecolor=red, backgroundcolor=red!10]\begin{theo}}
  {\end{theo}\end{mdframed}}

\newenvironment{nome}
{\begin{mdframed}[linecolor=green, backgroundcolor=green!10]\begin{nomen}}
  {\end{nomen}\end{mdframed}}

\newenvironment{prop}
{\begin{mdframed}[linecolor=red, backgroundcolor=red!10]\begin{propo}}
  {\end{propo}\end{mdframed}}

\newenvironment{defi}
{\begin{mdframed}[linecolor=orange, backgroundcolor=orange!10]\begin{defin}}
  {\end{defin}\end{mdframed}}

\newenvironment{lemm}
{\begin{mdframed}[linecolor=red, backgroundcolor=red!10]\begin{lemma}}
  {\end{lemma}\end{mdframed}}

\newcommand{\icol}[1]{% inline column vector
  \left(\begin{smallmatrix}#1\end{smallmatrix}\right)%
}

\newcommand{\irow}[1]{% inline row vector
  \begin{smallmatrix}(#1)\end{smallmatrix}%
}

\newcommand{\matrice}[1]{% inline column vector
  \begin{pmatrix}#1\end{pmatrix}%
}

\newcommand{\C}{\mathbb{C}}
\newcommand{\K}{\mathbb{K}}
\newcommand{\R}{\mathbb{R}}


\begin{document}
	\maketitle
	\newpage
	\subsection{Continuando sulle funzioni misurabili}
	\begin{defi}[Funzione semplice]
	Sia $(X,A,\mu)$ uno spazio di misura. Una funzione semplice in  $X$ è una funzione del tipo
	\[
		s(x) = \sum_{i =1 }^{N}c_i\chi_{E_i}(x), \text { con }c_i\in \R, N\geq 1, E_i\in A
	.\] 
	con $E_i\cap E_j= \emptyset$ se  $i\neq j$ e  $ \bigcup^{N}_{i=1}E_i = X$
\end{defi}
\begin{teo}
	Sia $(X,A,\mu)$ uno spazio di misura, e $f: X \rightarrow [-\infty,+\infty]$\\
	Allora $f $ misurabile $ \Leftrightarrow \exists\{s_m\}$  successione di funzioni semplici tale che $ s_m(x) \xrightarrow{ m \rightarrow +\infty}f(x)\ \ \forall x\in X$\\
	Inoltre
	\begin{enumerate}
		\item se  $f\geq 0 \Rightarrow $ si può scegliere $\{s_m\}$ tale che $s_m(x)\leq s_{m+1}(x) \ \ \ \forall x\in X, \forall m\geq 1$.
		\item se  $f$ è limitata $ \Rightarrow s_m \rightarrow f$ uniformemente in $X$.
	\end{enumerate}
\end{teo}
\begin{dimo}
	$ ( \Leftarrow)$ ovvia, perché $f$ è limite puntuale di una successione di funzioni misurabili\\[10px]
	$ ( \Rightarrow )$ Primo caso: $f\geq 0$, limitata, si può supporre  $0\leq f(x)\leq 1\ \forall x\in X$\\

	%TODO se vuoi disegno 1:33 
	$\forall n\geq 1$ $[0,1) = \bigcup_{k=1}^{2^n}\left[ \frac{k-1}{2^n},\frac {k}{2^n} \right) $\\
	$ E_k^n = \{\frac{k-1}{2^n}\leq f(x) < \frac {k}{2^n}\}\ \ k = 1,\ldots,2^n$\\
	$f$ misurabile  $ \Rightarrow E_k^n$ misurabile $\forall k = 1,\ldots, 2^n, \ \forall n\geq 1$\\
	 $E_k^ n\cap E_j^n = \emptyset$ se  $i\neq j$ e $X = \bigcup^{2^n}_{k=1}E_k^n$\\ 
 \[s_n(x) = \sum^{2^n}_{k=1}\frac{k-1}{2^n}\chi_{E_k^n}(x)\]
 $\forall x\in X, \forall n\geq 1\ \ \exists ! 1\leq k\leq 2^n$ tale che $x\in E_k^n$\\
 $ \Rightarrow  0\leq s_n(x) = \frac{k-1}{2^n}\leq f(x)$ \\
 $x\in E_k^n \Leftrightarrow \frac{k-1}{2^n} f(x) , \frac { k}{2^n}$\\
$\frac{2(k-1)}{2^{n+1}}\leq f(x) < \frac { 2k}{2^{n+1}}$\\
$ \Rightarrow $ sono possibili due casi
\[
	s_n(x) = \frac{ 2k-2}{2^{n+1}}\Leftarrow x\in E_{2k-1}^{n+1}\Leftrightarrow	\frac{2k - 2}{2^{n+1}}\leq f(x) < \frac {2k-1}{2^{n+1}}
.\] 
oppure \[
	s_n(x) = \frac{ 2k-1}{2^{n+1}}\Leftarrow x\in E_{2k-1}^{n+1}\Leftrightarrow\frac{2k-1}{2^{n+1}}\leq f(x) < \frac{2k}{2^{n+1}}
.\] 
nel caso 1\\
$s_n(x) = \frac{k_-1}{2^n} = s_{n+1}(x)$\\
nel caso 2\\
$s_n(x) = \frac {k-1}{2^n} < \frac{2k-1}{2^{n+1}}=s_{n+1}(x)$\\
$\forall x\in X\ \ \exists ! 1\leq k\leq 2^{n+1}$\\
tale che  $x\in E_k^n$\\
$ \Leftrightarrow s_n(x)\leq f(x) < \frac{k}{2^n}$\\
$0\leq f(x) - s_n(x) < \frac 1 {2^n}$\\
$\sup_{x\in X}|f(x) - s(x)| \leq \frac{1}{2^n} $\\
$ \Rightarrow s_m \rightarrow f$ uniformemente in $X$\\[10px]
  $ (\Rightarrow )$ secondo caso: $f\geq 0$\\
   $\forall n\geq 1$
    \[
	    E_I^n = \{f < n\},\ \  \ \ E_{II}^n = \{f\geq n\}
   .\] 
   $E^n_{I,k} =\{\frac{k-1}{2^n}\leq f < \frac {k}{2^n}$ \\
	   $ \bigcup^{2^n}_{k = 1}E^n_{I,k} = E^n_I$
	   \[
		   s_n(x) = n\chi_{E^n_{II}}(x) + \sum^{2^n}_{k = 1}\frac { k + 1}{2^n}\chi_{E_{I,k}^n}(x)
	   .\] 
	   \[
		   E_{II}^n\cup \bigcup^{2^n}_{k=1}E_{I,k}^n = X
	   .\] 
	   $s_n(x) \leq s_{n+1}(x) \ \ \forall x$ (come nel caso 1)\\
	   Se $f(x) = +\infty \Rightarrow f(x) \geq m\ \ \ \forall m$ \\
	   %TODO s eti va altra immaginae 14:01\\
	   $ \Leftrightarrow x\in E_{II}^n$\\
	   $ \Rightarrow s_n(x) = n \rightarrow+\infty = f(x)$ \\
	   Se $f(x) < +\infty \Rightarrow \exists \bar n $ tale che $f(x) \leq \bar n$\\
	   $ \Rightarrow x\in E_I^n = \bigcup^{2^n}_{k=1}E^n_{I,k}\\$
	   $s_n(x) = \frac {k-1}{2^n} \leq f(x) < \frac{ k}{2^n}$ \\
	   $ \Rightarrow  0 \leq f(x) - s_n(x) < \frac 1{2^n}\ \ \ \forall n\geq \bar n$ \\
	   la convergenza non è uniforme perché $\bar n$ dipende da  $f$.\\
	   $ \Rightarrow s_n (x) \rightarrow f(x)$\\[10px]
Terzo caso\\
$f$ di segno variabile\\
$f(x) =f^+(x) - f^-(x)$\\
$f^+(x) = max\{f(x),0\}$\\
$f^-(x) = -min\{f(x),0\}$\\
 $f$ misurabile $ <=> f^-, f^+$ misurabili\\
 Per il secondo caso\\
 $\exists \{s_n\}$ funzioni semplici  $s_n(x) \rightarrow f^+(x) \ \ \forall x$\\
 $\{t_n\}$ funzioni semplici $t_n(x) \rightarrow f^-(x) \ \ \forall x$ \\
 $ \Rightarrow s_n - t_n$ è funzione semplice, $s_n(x) - t_n(x) \rightarrow f(x) \ \ \forall x$
\end{dimo}
\begin{defi}
	Sia $(X,A,\mu)$ spazio di misura e sia $s(x) = \sum_{i=1}^{N}c_i\chi_{E_i}(x), \ c_i\geq 0 \\$	
	si definisce 
	\[
	\int_X s \ d\mu = \sum^{N}_{i=1}c_i\mu(E_i)
	.\] 
	dove si usa la convenzione $0 \cdot (+ \infty )= 0$\\
	e, inoltre, $\forall E \in A$
	 \[
	\int_E s \ f\mu = \int_X s\cdot \chi_E \ d\mu = \sum^{N}_{i=1}c_i\mu (E\cap E_i )
	.\] 
	\[
		\sum^{N}_{i=1}c_i\chi_{E_i\cap E}
	.\] 
	dato che $\chi_{E_i}\cdot \chi_e = \chi_{E_i\cap E}$
\end{defi}
\begin{defi}
	Sia $(X,A,\mu$ spazio di misura e sia  $f:X \rightarrow[0,+\infty]$ misurabile\\
	$ \Rightarrow \int_X f \ d\mu = \sup \{ \int_X s \ d\mu\ s$ funzione semplice $0\leq s\leq f\}$ e  $\forall E\in A$
	 \[
	\int_E f \ d\mu = \int_X \chi_E \ d\mu
	.\] 
\end{defi}
\textbf{Proprietà immediate dell'integrazione}
\begin{enumerate}
	\item $f = 0$ quasi certamente in $X \Rightarrow \int_X f \ d\mu =0$
	\item Se $N\subseteq X, \mu(N) = 0 \Rightarrow \int_N f\ d\mu = 0$ 
	\item $0\leq f\leq g, \ \ f,g$ misurabili $ \Rightarrow \int_X f d\mu\leq\int_X g \ du$ 
	\item Se $E,F \in A\ E\subseteq F\ \ \int_E f \ d\mu =\leq \int_F f\ d\mu$
\end{enumerate}
\textbf{Esempio}\\
$f(x) = \chi_\Q(x) = \chi_\Q(x) = 0\cdot \chi_{\R\setminus\Q} \Rightarrow \int_\R f \ d\mu = 1 \mu(\Q) = 0$ \\
$s$ funzione semplice $0\leq s \leq f$\\
\newpage
\begin{prop}
	Sia $s(x)$ funzione semplice, $\geq 0$, e sia $\mu_s : A \rightarrow [0,+\infty)$\\
	definita da 
	\[
	\mu_s(E) = \int_E s \ d\mu
	.\] 
	$\mu_s$ è una \unferline{misura} su $A$ (cioè  $\mu_S(\emptyset) = 0$ ed è additiva su misurabili disgiunti)
\end{prop}
\begin{dimo}
	$\mu_s(\emptyset) = \int_\emptuset s\ d\mu = 0 $ \\
	Siano $\{E_i\}\subset A$ disgiunti e sia  $s(x) = \sum^{N}_{k=1}c_k\chi_{F_k}(x)$ $F_k\in A\ \ F_k\cap F_j = \emptyset$ per ogni  $k\neq j\ \ \bigcup^{N}_{k=1}F_k=X$ \\
	\[
		\mu_s \left( \bigcup^{K}_{i-1}E_i \right) = \int_{ \bigcup^{K}_{i-1}E_i}s \ d\mu = \int_X s \chi_{ \bigcup^{K}_{i = 1}E_i} \ d\mu
	.\] 
	con
	\[
		\chi_{ \bigcup^{+K}_{i = 1}E_i} = \sum^{K}_{i = 1}\chi_{E_i}
	.\] 
	\[
	x\in \bigcup^{K}_{i= 1}E_i \Leftrightarrow \exists ! i \ x\in E_i
	.\] 
	\[
		\int_X s \sum^{K}_{i= 1}\chi_{E_i}\ d\mu = \int_X \sum^{K}_{i=1}s\chi_{E_i} \ d\mu = \int_X \sum^{K}_{i=1} \sum^{N}_{j=1}c_j \chi_{F_j\cap E_i} \ d\mu
	.\] 
	\[
		\sum^{K}_{i=1} \sum^{N}_{j=1}c_j \mu(E_j \cap E_i)  = \sum^{K}_{i=1}\int_{E_i} s\ d\mu = \sum^{K}_{i=1}\mu_s(E_i)
	.\] 
	\[
	\mu_s( \bigcup^{+\infty}_{i = 1}E_i \geq \mu_S \left( \bigcup^{k}_{i=1}E_i \right) = \sum^{K}_{i=1}\mu_s(E_i) \Rightarrow \mu_s \left( \bigcup^{+\infty}_{i =1}E_i \right) = \sum^{+\infty}_{i=1}\mu_s(E_i)
	.\] 
\end{dimo}
\textbf{Osservazione}\\
Se $N\subseteq X, \mu(N) = 0$\\
 $ \Rightarrow  \int_N S\ d\mu = 0$ \\
 $\mu_s(N) =0 \ \ \forall N: \mu(N) = 0$\\
  $\mu_s << \mu$  $\mu_s$ è assolutamente continua rispetto a $\mu$.
\end{document}
