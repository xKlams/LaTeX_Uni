\documentclass[12px]{article}

\title{Lezione 23 Analisi Reale}
\date{2025-05-23}
\author{Federico De Sisti}

\usepackage{amsmath}
\usepackage{amsthm}
\usepackage{mdframed}
\usepackage{amssymb}
\usepackage{nicematrix}
\usepackage{amsfonts}
\usepackage{tcolorbox}
\tcbuselibrary{theorems}
\usepackage{xcolor}
\usepackage{cancel}

\newtheoremstyle{break}
  {1px}{1px}%
  {\itshape}{}%
  {\bfseries}{}%
  {\newline}{}%
\theoremstyle{break}
\newtheorem{theo}{Teorema}
\theoremstyle{break}
\newtheorem{lemma}{Lemma}
\theoremstyle{break}
\newtheorem{defin}{Definizione}
\theoremstyle{break}
\newtheorem{propo}{Proposizione}
\theoremstyle{break}
\newtheorem*{dimo}{Dimostrazione}
\theoremstyle{break}
\newtheorem*{es}{Esempio}

\newenvironment{dimo}
  {\begin{dimostrazione}}
  {\hfill\square\end{dimostrazione}}

\newenvironment{teo}
{\begin{mdframed}[linecolor=red, backgroundcolor=red!10]\begin{theo}}
  {\end{theo}\end{mdframed}}

\newenvironment{nome}
{\begin{mdframed}[linecolor=green, backgroundcolor=green!10]\begin{nomen}}
  {\end{nomen}\end{mdframed}}

\newenvironment{prop}
{\begin{mdframed}[linecolor=red, backgroundcolor=red!10]\begin{propo}}
  {\end{propo}\end{mdframed}}

\newenvironment{defi}
{\begin{mdframed}[linecolor=orange, backgroundcolor=orange!10]\begin{defin}}
  {\end{defin}\end{mdframed}}

\newenvironment{lemm}
{\begin{mdframed}[linecolor=red, backgroundcolor=red!10]\begin{lemma}}
  {\end{lemma}\end{mdframed}}

\newcommand{\icol}[1]{% inline column vector
  \left(\begin{smallmatrix}#1\end{smallmatrix}\right)%
}

\newcommand{\irow}[1]{% inline row vector
  \begin{smallmatrix}(#1)\end{smallmatrix}%
}

\newcommand{\matrice}[1]{% inline column vector
  \begin{pmatrix}#1\end{pmatrix}%
}

\newcommand{\C}{\mathbb{C}}
\newcommand{\K}{\mathbb{K}}
\newcommand{\R}{\mathbb{R}}


\begin{document}
	\maketitle
	\newpage
	\subsection{Boh}
	$H$ spazio di Hilbert $\{ \varphi_k\}$ sistema ortonormale numerabile in $H$ $ \Rightarrow  \forall f\in H, \sum^{+\infty}_{k = 1}(f, \varphi_k)^2\leq \| f\|^2$\\
	$ \begin{aligned}
		\gamma: H &\rightarrow l^2\\
		f& \rightarrow \{(f, \varphi_k)\}_{l^2}
	\end{aligned}$
	è lineare $\tau(f + g) = \{(f + g, \varphi_k)\}_{l^2} = \{(f, \varphi_k)\} + \{(g, \varphi_k)\}$\\
	$\|\tau(f)\|_{l^2} = \left( \sum^{+\infty}_{k = 1}(f, \varphi_k)^2 \right)^{1/2}\leq \|f\|$\\
	$\tau$ è continua  $\|\tau\|_{(H,l^2)}\leq 1$
	 \begin{prop}
		 $\forall \{ \lambda_k\}\in l^2$ la serie $\sum^{+\infty}_{k = 1} \lambda _k \varphi_k$ è convergente ad un elemento $f$ tale che $(f, \varphi_k) = \lambda_k \ \ \forall k$\\
	\end{prop}
		 \textbf{Osservazione}\\
		 La proposizione dice che $\tau $ è iniettiva\\
		 $\forall \{ \lambda_k\}\in l^2 \ \exists f = \sum^{+\infty}_{k = 1} \lambda_k \varphi_k\in H$ tale che $\tau(f) = \{ \lambda_j\}$ \\
		 \begin{dimo}
			 sia $\{ \lambda_k\}\in l^2 ( \sum^{+\infty}_{k = 1} \lambda_k^2 < +\infty)$\\
			 sia  $f_n = \sum^{n}_{k= 1} \lambda_k \varphi_k$ mostriamo che è di Cauchy, siano $m > n$\\
			  \[
				  \|f_m-f_n\|^2 = \| \sum^{m}_{k = n + 1} \lambda_k \varphi_k\|^2 = \sum^{m}_{k = n+1} \lambda_k^2 \xrightarrow{ n \rightarrow +\infty} 0
			 .\] 
			 $ \Rightarrow  \{f_n\}$ di Cauchy in Hilbert\\
			 $ \Rightarrow  \exists f = \lim_{ n \rightarrow +\infty} f_n = \sum^{ + \infty}_{k =1} \lambda_j \varphi_k$\\
			 $\forall k$ fissato se  $n > k$\\
			  \[
				  (f, \varphi_k) = (f - f_n, \varphi_k) + (f_n, \varphi_k) = (f - f_n, \varphi_k) + \lambda_k
			 .\] 
			 $ \Rightarrow  |(f, \varphi_k), \lambda_k| = |(f-f_n, \varphi_k)| \leq \|f- f_n\| \xrightarrow{ n \rightarrow +\infty} 0 $\\
			 $ \Rightarrow  (f, \varphi_k) = \lambda_k \ \ \ \forall k$ \\
		 \end{dimo}
\textbf{Osservazione}\\
Se $f = \sum^{ + \infty}_{k = 1} \lambda_k \varphi_k$\\
$\displaystyle \Rightarrow  \|f\|^2 = \lim_{n \rightarrow +\infty} \|f_n\|^2 = \lim_{ n \rightarrow +\infty}\| \sum^{n}_{k=1} \lambda_k \varphi_k \|^2 = \lim_{ n \rightarrow +\infty}  \sum^{n}_{k=1} \lambda_k^2 = \sum^{+\infty}_{k = 1} \lambda_k^2 = \| \lambda_k\|^2_{l^2} = \|\tau(f)\|^2_{l^2}$\\
$ \Rightarrow $ \\
$f\in H \rightarrow \{(f, \varphi_k)\}\in l^2$\\
$ \Rightarrow  \sum^{+\infty}_{k = 1}(f, \varphi_k) \varphi_k = \hat f$\
$\hat p \substack{?\\ =} f$ non sempre
 \begin{teo}
	 Sia $H$ uno spazio di Hilbert e $\{ \varphi_k \}$ un sistema ortonormale in $H$ sono equivalenti
	 \begin{enumerate}
		 \item $\{ \varphi_k\}$ è completo
		 \item $\forall f\in H$  $f = \sum^{+\infty}_{k = 1} (f, \varphi_k) \varphi_k$
		 \item $\forall f\in H$ \  $\|f\|^2 = \sum^{+\infty}_{k = 1}(f, \varphi_k)^2 \rightarrow $ Perseval
		 \item $\forall f,g\in H \ \ (f,g) = \sum^{+\infty}_{k = 1}(f, \varphi_k)(g, \varphi_k)$
		 \item $f = 0 \Leftrightarrow (f, \varphi_k) = 0 \ \ \ \forall k \geq 1$
	 \end{enumerate}
\end{teo}
\begin{dimo}
	$1) \Rightarrow  2)$ \\
se $( \varphi_k)$ è completo $ \Rightarrow  $ $d$ è limite di una combinazione lineare finita di elementi di  $\{ \varphi_k\}$ che dista ovunque da $f$ meno di $\e$ \\
$ \Leftrightarrow \forall \e > 0 \exists n, \ \exists \lambda_1, \ldots, \lambda_n\in\R$ tale che 
\[
\|f- \sum^{n}_{k = 1} \lambda_j \varphi_k\| < \e
.\] 
Ma $\forall \lambda_1,\ldots, \lambda_n\in \R$
\[
\|f - \sum^{n}_{k = 1} \lambda_k \varphi_k \| \geq \| f - \sum^{n}_{k  =1} (f, \varphi_k)  \varphi_k \|
.\] 
$ \Rightarrow  \forall\e > 0 \exists n$ tale che $\|f - \sum^{n}_{k=1}(f, \varphi_k) \varphi_k\| < \e$\\
$ \Rightarrow  f = \sum^{+\infty}_{k =1}(f, \varphi_k) \varphi_k$ \\
$ 2) \Rightarrow  1)$ ovvio\\
$ 2) \Leftrightarrow  3)$ $\| f- \sum^{n}_{k=1}(f, \varphi_k) \varphi_k \|^2= \|f\|^2 - \sum^{n}_{k=1}(f, \varphi_k)^2\ \ \ \forall n$\\
$3) \Rightarrow 4)$\\
siano $\displaystyle f,g\in H, \|f+g\|^2 = \sum^{+\infty}_{k=1}(f_g, \varphi_k)^2 = \sum^{+\infty}_{k=1}((f+ \varphi_k)^2 + (g, \varphi_k)^2 + 2(f, \varphi_k)(g, \varphi_k)$\\
$\displaystyle =\sum^{+\infty}_{k = 1}(f, \varphi_k)^2 + \sum^{+\infty}_{k=1}(g, \varphi_k)^2 + 2 \sum^{+\infty}_{k=1}(f , \varphi_k)(f, \varphi_k)$\\
$ = \displaystyle \|f\|^2 + \|g\|^2 + 2 \sum^{+\infty}_{k=1}(f, \varphi_k)( g, \varphi_k)$\\
$4) \Rightarrow 3)$  per $f = g$\\
 $2) \Rightarrow  5)$ \\
 $\forall f\in H, g = \sum^{+\infty}_{k=1} (f, \varphi_k) \varphi_k$ quindi $f = 0 \Leftrightarrow (f, \varphi_k) = 0 \ \ \forall k$\\
 $ 5) \Rightarrow  2)$ $f\in H  \Rightarrow \sum^{+\infty}_{k=1}(f, \varphi_k) \varphi_k = \hat f \in H$ \\
 e $(\hat f, \varphi_k) = (f, \varphi_k) \ \ \forall k$\\
 $ \Rightarrow (f - \hat f, \varphi_k) = 0 \ \ \ \forall k $\\
 $5) \Rightarrow  f = \hat f$
\end{dimo}
\textbf{Osservazione}\\
Il teorema dice che $\{ \varphi_k\}$ completo $ \Leftrightarrow$ $\psi$ isomorfismo (quindi iniettivo poiché suriettivo lo è sempre)\\
\begin{coro}
	Ogni spazio di Hilbert $H$ separabile di dimensione finita è isomorfo a $l^2$
\end{coro}
Un esempio esplicito di sistema ortonormale completo in $L^2$ è il sistema trigonometrico\\
 $L^2((-\pi,\pi))$ che contiene funzioni del tipo  $\cos(kx), \sin(kx)\ \ k\geq 0\ \ x\in (-\pi,\pi)$\\
  \[
	  \int_{-\pi}^\pi\cos(kx)\cos(mx)dx = \begin{cases}
	  	\pi\ \ \ se \ \ k = m\neq 0\\
		2\pi \ \ se \ \ k = m = 0 \\
		0 \ \ \ se \ \ k\neq m
	  \end{cases}
 .\] 
 \[
	 \int^\pi_{-\pi}\cos(kx)\sin(mx)dx = 0 \ \ \ \forall k,m\geq 0
 .\] 
 sistema ortonormale detto sistema trigonometrico
 \[
	 \left\{ \frac {1}{\sqrt{2\pi}}, \frac{\cos(kx)}{\sqrt\pi}, \frac{\sin(kx)}{\sqrt\pi}\right\}_{k\geq 1}
 .\] 
 sia $f\in L^2((-\pi,\pi))$\\
 serie di Fourier di  $f$ associata al sistema trigonometrico:
 \[
	 (f,\frac{1}{\sqrt{2\pi}})\frac 1{\sqrt{2\pi}} + (f,\frac{\cos x}{\sqrt \pi})\frac{\cos x}{\sqrt\pi} + (f,\frac{\sin x}{\sqrt \pi})\frac{\sin x}{\sqrt\pi} + \ldots
 .\] 
 \[
	 \ldots = (f,\frac{\cos(k x)}{\sqrt\pi})\frac{\cos kx}{\sqrt\pi} + (f,\frac{\sin( kx)}{\sqrt\pi})\frac{\sin(kx)}{\sqrt\pi} =
 .\] 
 \[
	 \frac 1{2\pi}\int_{-\pi}^\pi f(x) dx + \left(\frac 1 \pi\int_{-\pi}^\pif(x)\cos xdx \right)\cos x + \left( \frac 1 \pi\int_{-\pi}^\pi f(x)\sin xdx \right)\sin x +\ldots
	 \]
	 \[
	  \ldots  + \frac 1 \pi \left(\int_{-\pi}^\pi f(x)\cos(kx)dx \right)\cos (kx) + \left(\frac 1 \pi\int_{-\pi}^\pi f(X)\sin(kx)dx \right)\sin(kx) + \ldots
	 .\] 
	 Serie di Fourier di $f$
	  \[
		  \frac {a_0}{2} + \sum^{+\infty}_{k = 1} \left(a_k \cos( kx) + b_k \sin(kx) \right)
	 .\]
	 $a_k = \frac 1 \pi \int_{-\pi}^\pi f(x) \cos(kx)dx$\\
	 $b_k = \frac 1 \pi \int_{-\pi}^\pi f(x)\sin(kx) dx$\ \ \ $\forall k$\\
	 Una somma finita del tipo 
	 \[
		 \frac {a_0}2  + \sum^{+\infty}_{k = 1}(a_k\cos(kx) + b_k \sin(kx))
	 .\] 
	 si dice polinomio trigonometrico\\
	 Mostriamo che questo polinomio trigonometrico converge in $L^2$\\
	 dimostriamo il risultato per funzioni continue su $L^2$ e per densità lo è anche in $L^2$\\
	 \begin{teo}
	 	Sia $f\in C(\R)$ periodico di periodo  $2\pi$\\
		$ \Rightarrow  \forall \e > 0 \ \exists p_n$ polinomio trigonometrico tale che $\|p_N - f\|_{\infty} < \e$
	 \end{teo}
	 Se faccio vedere che vicino ad $f$ esiste questo polinomio $p_n$ che dista  $\e$ in norma infinito, allora vicino a  $f$ c'è anche il suo polinomio trigonometrico di Fourier
	 \begin{lemm}
		 $\exists$ successione di polinomi trigonometrici  $\{\Q_n(x)\}$ tale che  
		 \begin{enumerate}
			 \item $\Q_n(x)\geq 0 \ \ \ \forall x\in X$
			 \item  $\int_{-\pi}^\pi Q_n(x)dx = 1$
			 \item  $\forall \delta > 0 \ \ \sup_{\delta < |x| < \pi} \Q_n(x) \xrightarrow{ n \rightarrow +\infty} 0$
		 \end{enumerate}
	 \end{lemm}
	 \begin{dimo}
		 Sia $Q_n(X) = c_n \left( \frac { 1 + \cos x}{2} \right)^n$ \ \ $c_n > 0$\\
		  $Q_n$ è un polinomio trigonometrico di grado  $n$ (si dimostra per induzione su $n$\\
	  $c_n$ viene scelto  $\displaystyle c_n = \frac { 1}{\int_{-\pi}^\pi \left(\frac{1 + \cos x}{2} \right)^ndx}$\\
	  in modo tale che $\int_{-\pi}^\pi Q_n(x) dx = 1$\\
	  quindi per ora sono verificate $1. $ e  $2.$ \\
	   \[
		   \int_{-\pi}^\pi \left(\frac {1 + \cos x}{2} \right)^ndx = 2 \int_0^\pi \frac {1 + \cos x}{2}dx
	  \] 
	  \[
		  \geq 2 \int_0^\pi \left(\frac {1 + \cos x}{2} \right)\sin xdx = \frac{4}{n+1} \left| - \left(\frac { 1 - \cos x}{2} \right)^{n+1} \right|^\pi_0 = \frac{4}{n+1}\frac{1}{2^n}
	  .\] 
	  $\displaystyle c_n \leq \frac{2^n(n+1)}4$
	 \end{dimo}
\end{document}
