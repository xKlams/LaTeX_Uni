\documentclass[12px]{article}

\title{Lezione 24 Analisi Reale}
\date{2025-05-23}
\author{Federico De Sisti}

\usepackage{amsmath}
\usepackage{amsthm}
\usepackage{mdframed}
\usepackage{amssymb}
\usepackage{nicematrix}
\usepackage{amsfonts}
\usepackage{tcolorbox}
\tcbuselibrary{theorems}
\usepackage{xcolor}
\usepackage{cancel}

\newtheoremstyle{break}
  {1px}{1px}%
  {\itshape}{}%
  {\bfseries}{}%
  {\newline}{}%
\theoremstyle{break}
\newtheorem{theo}{Teorema}
\theoremstyle{break}
\newtheorem{lemma}{Lemma}
\theoremstyle{break}
\newtheorem{defin}{Definizione}
\theoremstyle{break}
\newtheorem{propo}{Proposizione}
\theoremstyle{break}
\newtheorem*{dimo}{Dimostrazione}
\theoremstyle{break}
\newtheorem*{es}{Esempio}

\newenvironment{dimo}
  {\begin{dimostrazione}}
  {\hfill\square\end{dimostrazione}}

\newenvironment{teo}
{\begin{mdframed}[linecolor=red, backgroundcolor=red!10]\begin{theo}}
  {\end{theo}\end{mdframed}}

\newenvironment{nome}
{\begin{mdframed}[linecolor=green, backgroundcolor=green!10]\begin{nomen}}
  {\end{nomen}\end{mdframed}}

\newenvironment{prop}
{\begin{mdframed}[linecolor=red, backgroundcolor=red!10]\begin{propo}}
  {\end{propo}\end{mdframed}}

\newenvironment{defi}
{\begin{mdframed}[linecolor=orange, backgroundcolor=orange!10]\begin{defin}}
  {\end{defin}\end{mdframed}}

\newenvironment{lemm}
{\begin{mdframed}[linecolor=red, backgroundcolor=red!10]\begin{lemma}}
  {\end{lemma}\end{mdframed}}

\newcommand{\icol}[1]{% inline column vector
  \left(\begin{smallmatrix}#1\end{smallmatrix}\right)%
}

\newcommand{\irow}[1]{% inline row vector
  \begin{smallmatrix}(#1)\end{smallmatrix}%
}

\newcommand{\matrice}[1]{% inline column vector
  \begin{pmatrix}#1\end{pmatrix}%
}

\newcommand{\C}{\mathbb{C}}
\newcommand{\K}{\mathbb{K}}
\newcommand{\R}{\mathbb{R}}


\begin{document}
	\maketitle
	\newpage
	\subsection{altre cose sui polinomi trigonometrici}
	\begin{teo}[Weierstrass]
		Data $f\in C(\R)$ periodica di periodo  $2\pi$,  $\forall \e > 0 \ \ \exists p_n$ polinomio trigonometrico tale che  $\|f-f_n\|_\infty < \e$
	\end{teo}
	\begin{dimo}
	 \[
		 p_n(x) = \int_{-x}^xQ_n(x-y)f(y)dy
	.\] 
	Osserviamo che $p_n(x)$ è un polinomio trigonometrico
	 \[
		 Q_n(x) = \frac{\alpha_0^n}{2} + \sum^{n}_{k=0} \left(\alpha_k^n \cos( k x) + \beta_k^n \sin(kx) \right)
	.\] 
	\[
		Q_n(x-y) = \frac{\alpha_0^2}{2} + \sum^{n}_{k=0}(\alpjha^n_k \cos(kx -ky) + \beta_k^n\sin(kx-ky))
	.\]  
	è combinazione lineare di $\cos(kx)$ e  $\sin(kx)$\\
	 $ \Rightarrow p_n(x)$ è polinomio trigonometrico\\
	 $p_n(x) = \int_{-\pi}^\pi Q_n(x-y)f(y)dy = -\int^{\pi-x}_{\pi+x}Q_n(t)f(x-t)dt = \int^{\pi + x}_{-\pi+x}Q_nf(x-t)dt = \int^\pi_{-\pi}Q_n(t)f(x-t)dt$ \\
	 Osservo che $p_n$ come regolarità prende il "meglio" delle funzioni  $Q_n$ e  $f$
	  \begin{gather*}
		  |p_n(x)-f(x)| = |\int^\pi_{-\pi}Q_n(t)f(x-t)dt - f(x) \int^\pi_{-\pi}Q_n(t)dt|\\
		 = |\int^\pi_{-\pi}Q_n(t)( f(x-t)-f(x))dt
		 \leq \int_{-\pi}^\pi Q_n(t)|f(x-t)-f(x)|dt\\
		 = \int^\delta_{-\delta}Q_n(t)|f(x-t)-f(x)|dt + \int^{-\delta}_{-\pi}Q_n(t)|f(x-t)-f(x)|dt +\\+ \int^\pi_\delta Q_n(t)|f(x-t)-f(x)|dt
	 \end{gather*}\\
	 $f\in C(\R)$ + $f$ periodica $ \Rightarrow $ uniformemente continua.\\
	 Allora dato $\e > 0 \ \ \ \exists \ \ \ \delta > 0 $ tale che  $|f(x-t)-f(x)| < \frac {\e}{2}$ $x\in \R, \forall t$ tale che  $|t|<\delta$\\
	 Scegliendo questo  $\delta$, sia ha  $|p_n(x)-f(x)| \leq \frac{\e}{2}\int^\pi_{-\pi}Q_n(t)dt + 8\pi\|f\|_\infty\sup_{\delta\leq|t|\leq\pi}Q_n(t) < \frac {\e}3 $ per $n\geq n_\delta$
	\end{dimo}
	\begin{teo}[Completezza in $L^2(-\pi,\pi)$ del sistema trigonometrico]
		Il sistema trigonometrico $\{\frac{1}{\sqrt{2\pi}},\frac{1}{\sqrt pi}\cos(kx), \frac {1}{\sqrt 2}\sin(kx)\}_{k\geq 1}$  è completo in  $L^2((-\pi,\pi))$
	\end{teo}
	\begin{dimo}
		$f\in L^2((-\pi,\pi)); \forall \e > 0 \ \ \exists g\in C_c((-\pi,\pi))$ tale che  $\|f-g\|_2<\e$\\
		 $ \Rightarrow g(-\pi) = g(\pi) = 0 \Rightarrow g$ si estende per periodicità a una funzione $C(\R)$\\
		 Per il teorema di Weierstrass  $\exists p_n$ polinomio trigonometrico tale che  $\|g-p_n\|_\infty < \e$\
		  \[
		  \Rightarrow  \|g-p_n\|_{L^2((-\pi,\pi))} = \sqrt{\int_{-\pi}^\pi|g-p_n|^2dx}\leq \|g-p_n\|_\infty\sqrt{2\pi} < \sqrt{2\pi}\e
		 .\] 
		 In conclusione, $\|f-p_n\|_{L^2((-\pi,\pi))}\leq\|f-g\|_2 + \|g-p_n\|_2 < \e + \sqrt{ 2\pi}\e= (1 + \sqrt{2\pi})\e$ \\
			 \[
			 f(x) = \frac{a_0}{2} + \sum^{+\infty}_{k=1} \left(a_k \cos(kx) + b_k\sin(kx) \right)
			.\] 
		 Dove viene usato il fatto che la convergenza in $L^2$ implica quella puntuale solo a meno di un'estratta, la convergenza quasi ovunque si ha per una sottosuccessione\\
		  \[
		  a_k = \frac 1\pi\int_{-\pi}^\pi f(x)\cos(kx)dx\ \ \ \ b_k = \frac 1 \pi\int_{-\pi}^\pi f(x) \sin(kx) dx
		 .\] 
	\end{dimo}
	\subsection{Svolgimento esercizi foglio 8}
	$(X,\mu)$ spazio di misura
	$R: X \rightarrow [0,+\infty] $ misurabile\\
	\[
		K = \{u\in L^2(X): |u(x)| \leq h(x) \ \ q.o.\ \ \text{in }X\}
	.\] 
	$K\neq \emptyset$ perché  $0\equiv u$ in  $K$\\
	$K$ convesso :  $u_1,u_2\in K, \ \lambda\in[0,1]$\\
	Il resto delle soluzioni te le ha mandate Alberto su wa, nei messaggi importanti
\end{document}
