\documentclass[12px]{article}

\title{Lezione 18 Analisi Reale}
\date{2025-05-06}
\author{Federico De Sisti}

\input{../../../setup.tex}

\begin{document}
	\maketitle
	\newpage
	\subsection{Inclusione sugli spazi $L^p$}
	\begin{prop}
	Sia $(X,\mu)$ uno spazio di misura finito $(\mu(X) < +\infty)$ se  $p,q\geq 1$ con  $p>q$\\
	 $ \Rightarrow  L^p(X)\subsetneq L^q(X)$\\
	 e $\exists c> 0 t.c. \|f\|_q\leq c\|f\|_p \ \ \forall f\in L^p(X)$\\
	 Inoltre se  $f: X \rightarrow [-\infty, +\infty]$ è misurabile ed è tale che $\exists M> 0$ tale che  $|f(x)| \leq M$ per quasi ogni  $x\in X \Rightarrow f\in L^p(X) \ \ \forall p\geq 1 $ 
	\end{prop}
	 \begin{dimo}
		 $f$ misurabile\\
		 $f\in L^p \Leftrightarrow \int_X|f|^p < +\infty$\\
		 $f\in L^q \Leftrightarrow \int_X|f|^q < +\infty$\\
		 $p > q \ \ |f|^q \leq |f|^p \Leftrightarrow 1 \leq |f|^{p-q}$ vero se $|f|\geq 1$\\
		 La dimostrazione segue ma è costruttiva, la hai nella chat con alberto.\\
	 \end{dimo}
	 \textbf{Osservazione}\\
	 $\mu(X) <+\infty$\\
	 $\{f_n\}, f\in L^p(X)\ | \ f_n \rightarrow f $ in $L^p(X) \Rightarrow  f_n \rightarrow f$ in $L^q(X)\ \ \forall 1\leq q < p$\\
	 Usando Holder come nella dimostrazione
	  \[
		  \|f_n-f\|^q_q = \int_X|f_n-f|^qd \mu \leq \left(\int_X|f_n-f|^{q-\frac pq} \right)^{\frac qp}\cdot \left(\int_X (1)^{\frac q{p-q}} \right)^{\frac{p-q}p}
	 \] 
	 \[
	  = \|f_n-f\|^q_p\cdot \mu(X)^{\frac{p-q}p} < +\infty
	 .\] 
	 $f_n \rightarrow f$ in $L^q$ \\
	 \textbf{Esercizio}\\
	 $\mu(X) < +\infty$  $f: X \rightarrow [-\infty, +\infty]$ misurabile tale che $f$ essenzialmente limitata $ \Leftrightarrow \exists M\in \R \ | \ |f|<M$ q.o. in $\X$\\
	 $X = [0,1], \mu =m$\\
	 $f(x) = \begin{cases}
	 	+\infty \ \ se\ \  x = 0\\
		1 \ \ se\ \  x> 0
	 \end{cases} \Rightarrow $ $f$ è essenzialmente limitata\\
	 $f(x) = \begin{cases}
	 	0 \ \ se \ \ x = 0\\
		\frac 1 x\ \ se \ \ x  >0
	 \end{cases} \Rightarrow $ $f$ non è essenzialmente limitata\\
	 $\exists \lim_{p \rightarrow+\infty}\|f\|_p?$
	 \[
		 \left(\int_X|f|^qd\mu \right)^{\frac 1q} \leq \mu(X)^{\frac 1q - \frac 1p} \left(\int_X |f|^pd\mu \right)^{\frac 1p}
	 .\] 
	 \[
		 \left(\frac{\int_X|f|^qd\mu}{\mu(X)} \right)^{\frac 1q}\leq \left(\frac{\int_X|f|^pd\mu}{\mu(X)} \right)^{\frac 1p}\ \ \ con \ \ p> q
	 .\] 
	 la funzione $ \varphi(p) = \left(\frac {1}{\mu(X)} \int_X|f|^pd\mu \right)^{\frac 1p}$\\
	 è monotona crescente quindi $\displaystyle\exists \lim_{p \rightarrow +\infty} \varphi(p) = \lim_{p \rightarrow +\infty} \frac{\|f\|_p}{\mu(X)^{\frac 1p}} = \lim_{p \rightarrow +\infty}\|f\|_p$\\
	 \textbf{Esempio}\\
	 definiamo $\|f\|_{\infty} = \inf\{M > 0 \ | \ |f|\leq M \ \ q.o.\}$
	 \begin{itemize}
		 \item $f = $ costante su  $X\ | \ \mu(X) < +\infty$ \\
			 $ \Rightarrow  \|f\|_\infty = |c| \left(\mu(X) \right)^{\frac 1 p} \xrightarrow{ \rightarrow +\infty} c = \|f\|_\infty$\\
			 se $f$ è limitata in generale
		 \item $|f|\leq M \Rightarrow  \|f\|_p \leq M(\mu(X) )^{\frac 1p} \rightarrow M \Rightarrow  \|f\|_p \leq <\mu(X)^{\frac 1p}$ 
			 \[
				 \limsup_{p \rightarrow +\infty}\|f\|_p \leq \inf\{M\in \R^+\ | \ |f| < M \ \ q.o.\} =\|f\|_\infty
			 .\] 
	 \end{itemize}
Disuguaglianza opposta:\\
sia $\alpha < \|f\|_\infty$ quindi \\
$\mu(\{x\ |\ |f| > \alpha\} ) > 0 $ perché se per assurdo non valesse, quindi  $=0$\\
 $|f(x)| \leq \alpha$ quasi ovunque ma è assurdo perchè  $\alpha < \|f\|_\infty$che è un inf
  \[
	  \|f\|_p = \left(\int_X |f|^pd\mu \right)^{\frac 1p}\geq \left(\int_{|f|>\alpha}|f|^pd\mu \right)^{\frac 1p} \geq \alpha \mu(\{|f|> \alpha\})^{\frac 1p} \xrightarrow
  { p \rightarrow+\infty}\alpha.\] 
  \[
	  \liminf_{p \rightarrow+\infty} \|f\|_p \geq \alpha \ \ \forall \alpha < \|f\|_\infty \Rightarrow  \liminf_{p \rightarrow +\infty}\geq \|f\|_\infty \Rightarrow  \lim_{ p \rightarrow +\infty}\|f\|_p = \|f\|_\infty
  .\] 
  \begin{defi}
  	$(X,\mu)$ spazio di misura
	 \[
		 L^\infty = \{f: X \rightarrow [-\infty, +\infty] \ misur.\ | \ \exists M\geq 0\ | \ |f| \leq M  \ \ q.o.\}
	.\] 
  \end{defi}
  \begin{lemm}
  	$f\in L^\infty(X) \Rightarrow  |f|\leq \|f\|_\infty$ q.o. su $X$
  \end{lemm}
  \begin{dimo}
  	Appunti di Alberto
  \end{dimo}
  \begin{prop}
  	$L^\inftu(X)$ è uno spazio vettoriale,  $\|\cdot\|_\infty$ è una norma su  $L^\infty$
  \end{prop}
  \textbf{Osservazione}\\
  $[a,b]\subset\R$ intervallo chiuso e limitato\\
  $C([a,b])\subset L^\infty((a,b),m)\ \ \ f\in C([a,b])$\\
  $\|f\|_{C([a,b])} = \sup_{[a,b]}|f(x)|$\\
	  $\|f\|_\infty = \inf\{M: \|f\|\leq M \ q.o.\}$\\
	  $|f(x)| \leq \|f\|_{C([a,b])} \ \ \forall x\in [a,b]\\
	  \Rightarrow  \|f\|_\infty \leq \|f\|_{C([a,b])}$\\
  $|f(x)|\leq \|f\|_\infty \ q.o.$ in $[a,b]$\\
  $ \Rightarrow  f(x) \leq \|f\|_\infty \ \ \forall x\in [a,b]$ \\
  $\|f\|_{C([a,b])} = \|f\|_\infty$
   \begin{prop}
  	Siano $f\in L^\infty(X), g\in L^p(X) \ \ 1\leq p > +\infty$\\
	 $ \Rightarrow  fg\in L^p(X)$ \\
	 e $\|fg\|_p \leq \|f\|_\infty\|g\|_p$
  \end{prop}
\end{document}
