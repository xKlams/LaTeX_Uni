\documentclass[12px]{article}

\title{Svolgimento Esercizi Per il primo esonero\\
Analisi II}
\date{2024-11-15}
\author{Federico De Sisti}

\usepackage{amsmath}
\usepackage{amsthm}
\usepackage{mdframed}
\usepackage{amssymb}
\usepackage{nicematrix}
\usepackage{amsfonts}
\usepackage{tcolorbox}
\tcbuselibrary{theorems}
\usepackage{xcolor}
\usepackage{cancel}

\newtheoremstyle{break}
  {1px}{1px}%
  {\itshape}{}%
  {\bfseries}{}%
  {\newline}{}%
\theoremstyle{break}
\newtheorem{theo}{Teorema}
\theoremstyle{break}
\newtheorem{lemma}{Lemma}
\theoremstyle{break}
\newtheorem{defin}{Definizione}
\theoremstyle{break}
\newtheorem{propo}{Proposizione}
\theoremstyle{break}
\newtheorem*{dimo}{Dimostrazione}
\theoremstyle{break}
\newtheorem*{es}{Esempio}

\newenvironment{dimo}
  {\begin{dimostrazione}}
  {\hfill\square\end{dimostrazione}}

\newenvironment{teo}
{\begin{mdframed}[linecolor=red, backgroundcolor=red!10]\begin{theo}}
  {\end{theo}\end{mdframed}}

\newenvironment{nome}
{\begin{mdframed}[linecolor=green, backgroundcolor=green!10]\begin{nomen}}
  {\end{nomen}\end{mdframed}}

\newenvironment{prop}
{\begin{mdframed}[linecolor=red, backgroundcolor=red!10]\begin{propo}}
  {\end{propo}\end{mdframed}}

\newenvironment{defi}
{\begin{mdframed}[linecolor=orange, backgroundcolor=orange!10]\begin{defin}}
  {\end{defin}\end{mdframed}}

\newenvironment{lemm}
{\begin{mdframed}[linecolor=red, backgroundcolor=red!10]\begin{lemma}}
  {\end{lemma}\end{mdframed}}

\newcommand{\icol}[1]{% inline column vector
  \left(\begin{smallmatrix}#1\end{smallmatrix}\right)%
}

\newcommand{\irow}[1]{% inline row vector
  \begin{smallmatrix}(#1)\end{smallmatrix}%
}

\newcommand{\matrice}[1]{% inline column vector
  \begin{pmatrix}#1\end{pmatrix}%
}

\newcommand{\C}{\mathbb{C}}
\newcommand{\K}{\mathbb{K}}
\newcommand{\R}{\mathbb{R}}


\begin{document}
	\maketitle
	\newpage
	\section{Curve}
	\subsection{Verificare che una curva sia semplice}
	Per verificare che una curva sia semplice basta verificare che
	 \[
	\varphi(a) \neq \varphi(b) \ \ \ \forall a\neq b
	.\] 
	Per fare ciò la via più semplice e comune è quella di dimostrare che una delle componenti del vettore in uscita è strettamente monotona.
	\subsection{Rettificabilità di una curva}
	Per verificare che una curva è rettificabile ci basta sia almeno $C^1$.
	Alternativamente possiamo verirficare che sia Lipschitziana, ovvero
	\[
		|| \varphi(t) - \varphi(s)|| \leq L(t-s) \ \ \forall t,s\in[a,b]
	.\]
con $L > 0$.
	Per calcolare la sua lunghezza dobbiamo risolvere.
	\[
	L( \gamma) = \int_a^b ||\varphi(t)'||dt
	.\] 
	\subsection{Verificare che una curva sia regolare}
	Per verificare che una curva sia regolare ci basta verificare che
	\[
	|| \varphi'(t)||\neq 0 \ \ \forall t\in \Omega
	.\] 
	\section{Derivate e ottimizzazione}
	\subsection{Differenziabilità}
	Per verificare che una funzione sia differenziabile in $x_0$ è necessario che esista il limite:
	\[
		\lim_{h \rightarrow 0}\frac{f(x_0 + h) - f(x_0) - \langle v, h \rangle }{||h||} = 0
	.\] 
	Dove se la funzione è derivabile,  $v$ viene sostituito con il gradiente della funzione in $x_0$
	\subsection{Esercizio}
	Esempio
	\[
	f(x,y) = \begin{cases}
		\frac{x^3y-xy^3}{x^2 + y^2} &\text{ se } (x,y)\neq(0,0)\\
		0 &\text{ se } (x,y) = (0,0)
	\end{cases}
	.\] 
	1) Verificare che la funzione sia continua e derivabile in $(0,0)$:\\
	Per verificare questo ci basta calcolare li limite per $||(x,y)|| \rightarrow 0$ della funzione quando è diversa da 0, e successivamente maggiorarla con qualcosa che tende a zero, in questo caso possiamo maggiorarla così
	\[
		\left|\frac{x^3y-xy^3}{x^2 + y^2}\right|\leq x^2\left|\frac{xy}{x^2 + y^2}\right| + y^2\left|\frac{xy}{x^2 + y^2}\right|\leq x^2 + y^2
	.\] 
	quindi per il teorema del confronto possiamo concludere che la funzione è continua in 0.\\
	2)Continuità delle derivate parziali.\\
	Per questo il procedimento è analogo, ci basta calcolare la derivata parziale e verificare che il limite in 0 sia 0\\
	3) Verificare che le derivate seconde miste esistano in $(0,0)$e siano diverse:\\
	Per verificare l'esistenza ci basta risolvere il limite
	\[
		\lim_{k \rightarrow 0}\frac{f_x(0,k)}k = \lim_{k \rightarrow 0} \frac{-k^3}{k^3} = -1 = f_{xy}(0,0)
	.\] 
	Più in generale il limite è della forma 
	\[
		\lim_{k \rightarrow 0}\frac{f(x_0 + ke_i)-f(x_0)}{k}
	.\] 
	che in questo caso risulta essere:\\
	\[
		\lim_{k \rightarrow 0}\frac{f(0,0 + k)-f(0,0)}{k}= 
\lim_{k \rightarrow 0}\frac{f(0,k)}{k}
	
	.\] 
	\section{Ottimizzazione Vincolata}
	\subsection{Metodi comuni}
	Data una funzione $f$ e un vincolo $\Sigma$ Ci sono 2 modi di procedere, il primo è trovare quando il gradiente si annulla per $f$ e vedere se il punto è \underline{interno} a $\Sigma$, se lo è si analizza la matrice Hessiana per capire se è massimo o minimo, successivamente si parametrizza il bordo di $\Sigma$ con delle curve e si cerca quando la derivata della composizione tra  $f$ e la curva si annulla, così si trovano punti critici sul bordo.\\
	Alternativamente, dopo aver cercato tramite il gradiente, si può utilizzare il metodo dei moltiplicatori di Lagrange.\\
	La condizione che si va a cercare diventa quindi
	\[
	Df(x_0,y_0) = \lambda Dg(x_0,y_0)
	.\] 
che caratterizza i punti  stazionari della funzione $f$ soggetti al vincolo $g(x,y) = 0$ che è equivalente, in due dimensioni, alla formulazione:
\[
	\begin{cases}
		
det \matrice{
	f_x(x_0,y_0) & g_x (x_0,y_0)\\
f_y(x_0,y_0) & g_y (x_0,y_0)} = 0\\
g(x_0,y_0) = 0
	\end{cases}
.\] 
È importante fare delle considerazioni preliminari prima di iniziare con questi metodi:\\
\subsection{Discussione preliminare}
Bisogna verificare che l'insieme $\Sigma$ sia chiuso e limitato e che $f$ sia di classe $C^1$, in questo caso possiamo utilizzare il Teorema di Weierstrass, che ci dice che $f$ ammette in $\Sigma$ punti di massimo ed i minimo assoluti, che vanno scelti tra i punti stazionari interni a $\Sigma$, i punti stazionari vincolati alla frontiera di $\Sigma$ e i punti non regolari alla frontiera di  $\Sigma$
\end{document}
